\documentclass[11pt]{article}
\begin{document}

\begin{flushleft}
Ian Davis\\
Economics 8010\\
Problem Set 1 Rewrite\\
\end{flushleft}
\textbf{A.} Decide whether each of the following statements is \textbf{true, false, or uncertain,} and justify your answer.\\
1. "Happy" is not an economically meaningful term but "happier" is.
\\
\textbf{Answer: True} Because "happy" has so many different meanings for nearly every individual, it is useless for the purposes of general/large scale economic analysis. "Happier," however, provides us with originality over time. This allows for comparisons to be made economic research to be done.\\
\\
2. The axiom, "more is preferred to less," implies that rich people are happier than poor people.\\
\textbf{Answer: False} As stated before, the meaning of happiness varies greatly from person to person and, while "happier" was completely usable in studying one individual over time, these varied meaning prevent us from making any extrapolations from happiness measures between people. 
\\
\\
3. A person volunteers to provide blood for hospitals when all blood is acquired by voluntary donations. Then the system changes to allow hospitals to also purchase blood and this person stops donating it. \textbf{TFU:} Such behavior cannot be derived from utility maximization if utility does not directly depend on whether or not blood can be purchased.\\
\textbf{Answer: False} Multiple cases can be presented in which the individual's behavior is completely in line with utility maximization and independent on the price of blood. The first case is simply that the individual donates because there is a need for blood. If, after the price change, the community's blood bank is fully stocked, then the individual would see no need in donating and no longer does so. In the second case, it could be that wait times have grown significantly after the practice of buying blood began. Now, the opportunity cost of selling blood has increased and it good well be that our individual may not be able to take off work long enough or has more valuable things to be doing. In both cases, utility is still being maximized and behavior is independent of prices.\\
\textbf{Note:} While it could be noted that these actions are ramifications of the price increase, it does not necessarily mean that the behavior is wholly dependent on it. The question was specifically looking for cases where the individual is not motivated by real prices and, in both cases, this holds.\\
\\
\textbf{B.} According to Armen Alchain, uncertainty about the future is an important feature of many economic decisions. but difficult to model formally. To explore the issues he raises in the simplest possible way, consider a situation where a candy store places in its window a large glass urn full of jelly beans, along with a sign saying that the person whose guess is closest to the actual number of jelly beans in the urn will win \$100. Only one guess is allowed per person. (What might happen if that were not true?"\\
\\
1. Would you expect everyone's guess to be the same? Why or why not?\\
\textbf{Answer} Because individuals all have different guessing abilities which can be assumed to be randomly distributed, it is highly unlikely for these guesses to all be the same. 
\\
2. If not, how would you expect people's guesses to be distributed relative to the correct number. (A clearly labeled drawing will suffice.) What reasoning lies behind your answer.\\
\textbf{Answer} Because of the randomness assumed above as well as the assumption that a non-trivial number of players will enter the game, we can guess that the distribution will be normal and the mean will be near or at the correct answer through wisdom of the crowd.
\\
3. Suppose that the prize structure for this contest were as follows: Every person must pan an amount of his or her choosing to enter the contest (call the amount paid by person $j$ $A_j"$), and receives a payment $P_j$ equal to $2a_j-.001|N-G_j|$ subject to $P_j \geq 0$, where $N$ represents the actual number of jelly beans in the urn and $G_j$, representing the number submitted by person $j$ as his or her guess about $N$. (The lower bound on the payment is zero-you can't lose more than the amount you wager.) Would you answer question (2) be the same as before? Why or why not? If not, how would it be different?\\
\textbf{Answer} The new competition structure increases the salience of the competitors by increasing the cost of incorrect guesses. This will weed out some of the worse guessers and leave only those relatively confident in their chances of winning. This would lead to a tighter distribution of guesses than that of question 2.\\
\\
4. What things would you take into account in deciding whether or not to enter the contest described in question (3)?\\
\textbf{Answer}It would be imperative for me to consider my own confidence in my guessing ability in tandem with my own risk tolerance. Included in this analysis would be the expected value of entering the game.\\
\\
5. If the candy store offered the same game once a week using the same urn and jelly beans of the same size, what would you expect to happen to the distribution of guesses over time? What factors would determine the rate at which the distribution would change?\\
\textbf{Answer} We can assume that t he distribution of guesses will tighten over time but it will be dependent on how quickly information that the urn does not change is spread. Eventually, the competition would become a race to be the first guess because all competitors will be certain at the accuracy of their guesses.
\end{document}