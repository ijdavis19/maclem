\documentclass[11pt]{article}
\usepackage{amsmath}
\usepackage{amssymb}
\usepackage{centernot}
\begin{document}

\begin{flushleft}
Ian Davis\\
Economics 8010\\
Fall 2019\\
\bigskip
\textbf{Problem Set 2 Rewrite}\\
\end{flushleft}
\textbf{A.} A function $f(x,y)$ is homogeneous (of degree $\rho$) if, for some constants $k$ and $\rho$, $f(kx,ky) = k^\rho f(x,y)$. A function $f(x,y)$ is \textit{homothetic} if the slopes of the level-set boundaries it generates depend only on the ratio $y/x$ (i.e. they have slopes that are the same along any ray through the origin). Which of these functions, if any, are homogeneous? Which, if any, are homoethetic? (In each of these function, all terms except X and Y are constants.)\\
\\
$1. f(X,Y) = AX^\alpha Y^\beta\\
\\
\textbf{Answer: } f(tx,ty) = A(tX)^\alpha (tY)^\beta\\
= At^\alpha X^\alpha t^\beta Y^\beta\\
= t^{\alpha + \beta}X^\alpha Y^\beta = t^{\alpha + \beta}f(x,y) \implies homogenous\\
\&\\
f_X = \alpha AX^{\alpha - 1}y^\beta\\
f_Y = \beta AX^\alpha y^{\beta -1}\\
-\frac{f_x}{f_y} = -\frac{\alpha}{\beta}\frac{Y}{X} \implies homothetic\\
\\
2. f(X,Y) = A + BX^\alpha Y^\beta \\
\\
\textbf{Answer: } f(tx,ty) = A + B(tX)^\alpha (tY)^\beta\\
= A + t^{\alpha + \beta}BX^\alpha Y^\beta \centernot\implies homogenous\\
\&\\
f_X = \alpha BX^{\alpha -1}Y^\beta\\
f_Y = \beta BX^\alpha Y^{\beta -1}\\
-\frac{f_X}{f_Y} = -\frac{\alpha}{\beta}\frac{Y}{X} \implies homothetic\\
\\
3. f(X,Y) = A[\alpha X^\beta + (1-\alpha )Y^\beta ]^\frac{\gamma}{\beta}\\
\textbf{Answer: }f(tX,tY) - A[\alpha t^\beta X^\beta + (1 - \alpha)t^\beta Y^\beta]^{\gamma /\beta}\\
= At^\gamma [\alpha X^\beta + (1-\alpha )Y^\beta ]^\frac{\gamma}{\beta}\\
= t^\gamma f(X,Y) \implies homogenous\\
\&\\
-\frac{f_X}{f_Y} = -(\frac{\alpha}{1 - \alpha})(\frac{X}{Y})^{\beta - 1}\\
\\
4. f(X,Y) = A(X-B)^\alpha (Y-C)^\beta\\
\textbf{Answer: }f(tX,tY) = A(tX - B)^\alpha (tY - C)^\beta\\
= At^{\alpha + \beta}(X - \frac{B}{t})^\alpha tY - \frac{C}{t})^\beta \centernot\implies homogenous\\
\&\\
-\frac{f_X}{f_Y} = -(\frac{\alpha}{\beta})(\frac{Y-C}{X_B}) \centernot\implies homothetic\\
\\
5. f(X,Y) = \alpha Y + \beta X - (\gamma /2)X^2\\
\textbf{Answer: }f(tX,tY) = \alpha tY + \beta tX - \frac{\gamma}{2}t^2X^2\\
= t(\alpha Y + \beta X - (\gamma /2)tX^2) \centernot\implies homogenous\\
\&\\
-\frac{f_X}{f_Y} = -\frac{\beta - \gamma X}{\alpha} \centernot\implies homothetic$\\
\\
\textbf{B.} Draw the indifference curves (and, if necessary, budget constraints) representing the following statements:\\
\\
1. Max has all the shoes he'll ever need.\\
\textbf{Answer: }Max already having all of the shoes he needs implies he is endowed with so many shoes that there is now way and additional shoe can give him anymore utility. Hence, his indifference curves, with shoes on the x-axis, will be horizontal parallel lines.\\
\\
2. Donald can't afford caviar.\\
\textbf{Answer: }Donald's utility curves can follow the standard assumptions but his budget curve will not allow for a single unit of caviar to be bought. With caviar on the x-axis and a composit good on the y-axis, the budget constrain could be illustrated as a vertical line from the orgin to the point $I/P_0$ on the y-axis.\\
\\
3. Ellen hates Brussels sprouts.\\
\textbf{Answer: }Assuming Ellen hates brussel sprouts so much that an additional unit of Brussels sprouts will lower her utility and Brussels sprouts are represented on the x-axis with a composit good on the y-axis, the utitilty curves can be modelled as straight, upward sloping lines with utility decreasing as one moves across the x-axis.\\
\\
4. A wise person consumes all things in moderation.\\
\textbf{Answer: }The above statement can have two meanings. Either a wise person will never consume in a corner or a wise person will, at some point, stop getting utility from additional units of a good. The former would have a standard budget constraint-utility curve look and the latter would look like the topographical map of a hill. Additionally, the former is the case believed by the TA while the latter is the one that's correct.\\
\\
5. Felicia needs more vitamin D.\\
\textbf{Answer: }Another case were there has been some mild disagreement between myself and the powers at be, the case can be illustrated using a standing budget constraint-utility curve model with the amount of vitamin D needed to live a healthy life is greater than the optimal amount of vitamin D consumed. The graders believe that it must be that the healthy amount of vitamin D should be too far out on its respective axis to be able to be afforded by Felicia. I, however, believe that it could very well be the case that Felicia is not consuming the appropriate amount of vitamin D because she is simply ignorant of the vitamin's importance and her indifference curves are those that give her an optimal amount below the healthy amount.\\
\\
\textbf{C.} Decide whether each of the following statements is \textbf{true, false, or uncertain,} and justify your answer.\\
\\
1. Restricting the quantity of any good to be non negative means that utility must be a positive number.\\
\textbf{Answer: False }It could very well be a situation where any quantity of a good is worse than a quantity of 0 (cancer, bad food). Because there is no restriction that utility functions must have positive outputs, all $q > 0$ could produce negative utility in certain cases.\\
\\
2. Dieter has a specific amount of money he dedicates to the purchase of beer and pretzels. The more beer Dieter drinks, the greater his marginal utility from beer. The more pretzels he eats, the greater his marginal utility from pretzels. \textbf{TFU:} Depending on the relative prices, Dieter will either drink beer or eat pretzels, but he will not consume both.\\
\textbf{Answer: Uncertain }While it is possible for the two good sto be perfect substitiutes (which would in fact lead to a corner solution), increasing marginal utilities is not a necessary not sufficient condition for a corner solution. This is because the slope of the dindifference curve is defined as $-\frac{MU_x}{MU_y}$. In the cases of i) increasing marginal utilities and ii) decreasing marginal utilities, the slope of the indifference curve is still negative.\\
i) $-\frac{(+)}{(+)} < 0$\\
ii) $-\frac{(-)}{(-)} < 0 = -\frac{(-1)(+)}{(-1)(+)} = -\frac{(+)}{(+)} < 0$\\
This gives us the usual convex indifference curves which could lead to any number of solutions in and between the corners.\\
\\
3. The invention of a drug that will make it easier to quit smoking would increase smoking rates.\\
\textbf{Answer: Uncertain} Certain groups of people (people who are only smoking because they are addicted) would likely take the drug in order to finally quit. This could be offset by individuals who now start soking because they no longer fear getting addicted. We don't know how the sizes of each group relate to one another so we do not know if it would be an increase or decrease in total smokers. Another possibility would be for the price of the druge to be so high that it is not worth purchasing for many smokers.\\
\\
\textbf{D.} Consider an endowment economy with two people, Ann and Bob. Let Ann's preferences be represented by the function $U_A = X_A^\alpha Y_A^\beta$, while Bob's are represented by the function $U_B = X_B^\gamma Y_A^\mu$. At the initial endowment, Ann has all the X and Bob has all the Y.\\
\\
1. Show that the contract curve is linear if $(\alpha /\beta) = (\gamma /\mu)$\\
\textbf{Answer: }The slope of the level curves are\\
$-(\frac{MU_AX}{MU_AY}) \& -(\frac{MU_BX}{MU_BY})$ respectively, giving us\\
slope of $IC_A = -(\frac{MU_AX}{MU_AY}) = -\frac{\alpha X^{\alpha - 1}_AY^\beta _A}{\beta X^{\alpha}_AY^{\beta - 1} _A}$\\
slope of $IC_A = -(\frac{MU_AX}{MU_AY}) = -\frac{\gamma X^{\gamma - 1}_BY^\mu _B}{\mu X^{\gamma}_BY^{\mu - 1} _B}$\\
\& the contract curve is the points where the slopes are equal, i.e.\\
$-(\frac{MU_AX}{MU_AY}) = -(\frac{MU_BX}{MU_BY})\\
\implies -(\frac{\alpha X^{\alpha - 1}_AY^\beta _A}{\beta X^{\alpha}_AY^{\beta - 1} _A}) = -(\frac{\gamma X^{\gamma - 1}_BY^\mu _B}{\mu X^{\gamma}_BY^{\mu - 1} _B})\\
\implies \frac{\alpha X^{\alpha - 1}_AY^\beta _A}{\beta X^{\alpha}_AY^{\beta - 1} _A} = \frac{\gamma X^{\gamma - 1}_BY^\mu _B}{\mu X^{\gamma}_BY^{\mu - 1} _B}$\\
\\
Now, we replace $X_B = X_T - X_A \& Y_B = Y_T - Y_A$ giving us\\
$\frac{\alpha X^{\alpha - 1}_AY^\beta _A}{\beta X^{\alpha}_AY^{\beta - 1} _A} = \frac{\gamma (X_T - X_A)^{\gamma - 1}(Y_T - Y_X)^\mu}{\mu (X_T - X_A)^{\gamma}(Y_T - Y_X)^{\mu - 1}}$\\
\\
dividing out the like quantities and recalling $(\alpha /\beta) = (\gamma /\mu)$,
$\frac{Y_A}{X_A} = \frac{Y_T - Y_A}{X_T - X_A}\\
\implies Y_A = (\frac{Y_T - Y_A}{X_T - X_A})X_A\\
\implies Y_A(X_T - X_A) = (Y_T - Y_A)\\
\implies Y_AX_T - X_AY_A = Y_TX_A - Y_AX_A\\
\implies Y_A = (\frac{Y_T}{X_T})X_A$\\
\\
2. How will the shape of the contract curve differ from a straight line if $(\alpha /\beta) > (\gamma /\mu)$? (That is, if Ann likes X relatively more than Bob does.)\\
\textbf{Answer: }In the case of $(\alpha /\beta) > (\gamma /\mu)$, the contract curve is still the set of all point such that\\
$\frac{\alpha}{\beta}\frac{Y_A}{X_A} = {\gamma}{\mu}\frac{Y_B}{X_B}$\\
\\
which directly implies\\
$\frac{Y_A}{X_A} < \frac{Y_B}{X_B}$\\
\\
giving us a convex contract curve. Intuitively, we can say this is because Ann favors $X$ relatively more than Bob, she will be willing to give up more of $Y$ in order to gain $X$.\\
\\
3. What is the allocation of $X$ and $Y$ under Nash bargaining solution if $0 < \alpha + \beta < 1$ and $0 < \gamma + \mu < 1$\\
\textbf{Answer: }See document regarding Nash-Bargaining Solutions
\end{document}