\documentclass[11pt]{article}
\usepackage{amsmath}
\usepackage{amssymb}
\usepackage{centernot}
\begin{document}

\begin{flushleft}
Ian Davis\\
Economics 8010\\
Fall 2019\\
\bigskip
\textbf{Problem Set 5 Rewrite}\\
\end{flushleft}
\textbf{A. } For a consumer with the utility function $U(X_1,X_2,X_3) = X_1^\alpha ,X_2^\beta , X_3^\gamma$\\
\\
1. Derive the compensated demand functions for $X_1$, $X_2$, and $X_3$, and verify that they satify the relevant adding-up and symmetry properties.\\
\\
2. Derive algebraic expressions for both the compensating variation and equivalent variation measures of the net welfare effect of a 10\% fall in the price of $X_1$ coupled with a 20\% rise in the price of $X_2$ and a 15\% fall in the price of $X_3$.\\
\\
3. Calculate the values of the Laspeyres and Paashe price indices representing the change in the cost of living for the price changes in part 2 above.\\
\\
\\
\\
\textbf{B. }Jane is a customer of Getflix, which offers movies on demand via streaming video. Under her current contract, each movie she watches costs her \$5. Getflix has recently introduced the option of an alternative contract, which offers the opportunity to view up to 20 movies per month for a monthly fee of \$75 with no per-movie charge.\\
\\
Jane's utility function is $U = X^.1Y^.9$, where $X$ represents movies watched per month and Y represents a bundle of all the other goods that she consumes. the untis of the items in the bundle "all other goods" are defined in such a way to make the price of one unit of this bundle equal to \$10. Jane's current monthly expenditures are \$500.\\
\\
1. Will Jane switch to the new contract? Explain.\\
\\
2. What is the dollar value of the gain or loss to Jane from switching to the new contract as measured by both the compensating surplus and the equivalent surplus?\\
\end{document}