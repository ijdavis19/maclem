\documentclass[11pt]{article}
\usepackage{amsmath}
\usepackage{amssymb}
\usepackage{centernot}
\begin{document}

\begin{flushleft}
Ian Davis\\
Economics 8010\\
Fall 2019\\
\bigskip
\textbf{Problem Set 6 Rewrite}\\
\end{flushleft}
\textbf{A. } Demand Aggregation\\
\\
Consider an economy comprising of $n$ households and two goods ($X$ and $Y$), in which household $i$'s preferences are of the form $U_i=X_i^{\alpha_i} Y_i^{\beta_i}$, where $X_i$ and $Y_i$ refer to the quantities of $X$ and $Y$ consumed by household $i$, and household $i$'s income is $M_i$.\\
\\
1. Derive the market demand function for $X$, expressed as a function of prices and aggregate income. Under what conditions(s) is this demand function independent of the distribution of income in the economy?\\
\\
2. How would your answer change, if at all, if each household's preferences were of the form $U_i = (X_i-x_i)^\alpha (Y_i-y_i)^\beta$?\\
\\
3. What function form(s) of the "representative household's" utility from $X$ and $Y$ would be consistent with the data generated by a set of N households that each allocate their expenditures randomly between X and Y according to the uniform distribution and subject to the condition that they spend all of their income in each period, as N $\rightarrow \infty$?\\
\\
\\
\\
\textbf{B: }Consider and economy in which different households receive endowments of a completely nonstorable commodity at different times. Specifically, a fraction $\theta$ of the population receives $W_1$ units of the commodity in period 1, while the remaining $(1-\theta)$ fraction of the population receives $W_2$ units in period 2. Each household survives through two periods with certainty, and no longer than that.\\
\\
Each household's intertemporal utility function is $U(C_1,C_2) = C_1C_2^\beta$, where $C_t$ represents consumption in period $t (t=1,2)$.\\
\\
If people are able to make fully and costlessly enforceable loan contracts with each other, what will be the competitive equilibrium price of a unit of period-2 consumption in terms of period-1 consumption, expressed as a function of the relevant parameters? (Explain why the specific assumtions made about individuals' preferences make this problem tractable.)\\
\\
Here is some terminology: The preference parameter $\beta$ is called the "discount factor". The "pure rate of time preference" is defined as $\delta$ such that $\beta = 1/(1 + r)$, where $r$ is the equilibrium one-period rate of interest.\\
\end{document}