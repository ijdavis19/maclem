\documentclass[11pt]{article}
\begin{document}

\begin{flushleft}
Ian Davis\\
Economics 8010\\
Problem Set 2 Rewrite\\
\end{flushleft}
\textbf{A.} A function $f(x,y)$ is homogeneous (of degree $\rho$) if, for some constants $k$ and $\rho$, $f(kx,ky) = k^\rho f(x,y)$. A function $f(x,y)$ is \textit{homothetic} if the slopes of the level-set boundaries it generates depend only on the ratio $y/x$ (i.e. they have slopes that are the same along any ray through the origin). Which of these functions, if any, are homogeneous? Which, if any, are homoethetic? (In each of these function, all terms except X and Y are constants.)\\
\\
$1. f(X,Y) = AX^\alpha Y^\beta\\
2. f(X,Y) = A + BX^\alpha Y^\beta \\
3. f(X,Y) = A[\alpha X^\beta + (1-\alpha )Y^\beta ]^{\gamma /\beta}\\
4. f(X,Y) = A(X-B)^\alpha (Y-C)^\beta\\
5. f(X,Y) = \alpha Y + \beta X - (\gamma /2)X^2\\\
$\\
\textbf{B.} Draw the indifference curves (and, if necessary, budget constraints) representing the following statements:\\
\\
1. Max has all the shoes he'll ever need.\\
2. Donald can't afford caviar.\\
3. Ellen hates Brussels sprouts.\\
4. A wise person consumes all things in moderation.\\
5. Felicia needs more vitamin D.\\
\\
\textbf{C.} Decide whether each of the following statements is \textbf{true, false, or uncertain,} and justify your answer.\\
\\
1. Restricting the quantity of any good to be non negative means that utility must be a positive number.\\
\\
2. Dieter has a specific amount of money he dedicates to the purchase of beer and pretzels. The more beer Dieter drinks, the greater his marginal utility from beer. The more pretzels he eats, the greater his marginal utility from pretzels. \textbf{TFU:} Depending on the relative prices, Dieter will either drink beer or eat pretzels, but he will not consume both.\\
\\
3. The invention of a drug that will make it easier to quit smoking would increase smoking rates.\\
\\
\textbf{D.} Consider an endowment economy with two people, Ann and Bob. Let Ann's preferences be represented by the function $U_A = X_A^\alpha Y_A^\beta$, while Bob's are represented by the function $U_B = X_B^\gamma Y_A^\mu$.\\
\\
1. Show that the contract curve is linear if $(\alpha /\beta) = (\gamma /\mu)$\\
\\
2. How will the shape of the contract curve differ from a straight line if $(\alpha /\beta) > (\gamma /\mu)$? (That is, if Ann likes X relatively more than Bob does.)\\
\\
3. What is the allocation of X and Y under the Nash bargaining solution if $0 < \alpha + \beta < 1$ and $0 < \gamma + \mu < 1$?
\end{document}