\documentclass[11pt]{article}
\usepackage{amsmath}
\usepackage{amssymb}
\usepackage{centernot}
\begin{document}

\begin{flushleft}
Ian Davis\\
Economics 8010\\
Fall 2019\\
\bigskip
\textbf{Problem Set 3 Rewrite}\\
\end{flushleft}
\textbf{A. } Tom's only asset is his car, which is worth \$19,600. He lives in an area full of bad drivers and estimates that over the coming year, although he is a perfectly safe driver, he faces a 5\% chance that his car will be totally destroyed in an accident, a 10\% chance that his car will incure \$5,200 in damages, and a 15\% chance that his car will incur \$2,700 in damages. Tom's utilty-of-wealth function is $U(W)=W^.5$.\\
\\
1. What is the expected value of Tom's wealth?\\
\textbf{Answer: }$(W) = (.05)(0) + (.1)(14,00) + (.15)(16,900) (.7)(19,600) = 17,695\\
\implies$ The expected value of Tom's wealth is \$17,695\\
\\
2. What is the maximum amount that Tom would be willing to pay per year for insurance that would completely reimburse him for any damage to his car?\\
\textbf{Answer: }Tom's maximum willingness to pay per year for insurance is $w_i = E(W) - W_k$ where $E(U) = U(W_k)\\
E(U) = .05(0) + .1(120) + .15(130) + .7(140) = 129.5\\
\implies W_k = (129.5)^2 = 16770.25$\\
so\\
$E(W) - W_k = \$17,695 - \$16770.25 = \$924.75\\$
\\
\textbf{B. } Determine whether each of the following statements is \textbf{true, false, or uncertain,} and justify your answer.\\
\\
1. If an individual prefers lottery L to lottery l' and lottery L to lottery L'' .But prefers a compound lottery that give probability $\alpha$ to L' and probability $1 - \alpha$ to L'' to the original lottery L, one can design a series of deals leading an agent to a sure loss of money.\\
\textbf{Answer: True} If we present an agent with the option of either L or the compount lottery, we know the compount lottery will be chosen. But because $0 < \alpha < 1$, either L' or L'' must be preffered to the compund lottery. Because of this we know the agent will pay to get out of the compound lottery and into either L' or L'', both of which L is prefferable to. Now, we can offer the agent an opportunity to buy themselves into L. We then can make an offer for the agent to buy into the original compound lottery an repeat the process until it is guaranteed that the agent will lose money.\\
\\
2. Suppose Eric buys a bottle of wine when cheap and the wine is young, and hold the bottle until it is worth \$200. He would never pay that much to buy a bottle of wine, yet he refuses an offer of \$200. \textbf{TFU: }His behavior violates the axioms of rational choice.\\
\textbf{Answer: False} Eric may get some sort of use value from the process of aging wine. This would make him value cheaper, younger wines more because of the opportunity to age them and then consuming these wines that he knows he aged can give him additional utility. If this additional utility is worth more than \$200, his actions are completely rational.\\
\\
3. At any given moment, there will be greater inequality of consumption in an endowment economy where individuals are risk seekers than one with the same initial endowment where individuals are risk-averse.\\
\textbf{Answer: False} Consider the endowment where agent A gets there entire stock of each good. In this case, no matter the risk preferences of the individuals, there will be no bargaining and the maximum amount of inequality. 
\end{document}