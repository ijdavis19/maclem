\documentclass[11pt]{article}
\usepackage{amsmath}
\usepackage{amssymb}
\usepackage{centernot}
\begin{document}

\begin{flushleft}
Ian Davis\\
Economics 8010\\
Fall 2019\\
\bigskip
\textbf{Problem Set 3 Rewrite}\\
\end{flushleft}
\textbf{A. } Tom's only asset is his car, which is worth \$19,600. He lives in an area full of bad drivers and estimates that over the coming year, although he is a perfectly safe driver, he faces a 5\% chance that his car will be totally destroyed in an accident, a 10\% chance that his car will incure \$5,200 in damages, and a 15\% chance that his car will incur \$2,700 in damages. Tom's utilty-of-wealth function is $U(W)=W^.5$.\\
\\
1. What is the expected value of Tom's wealth?\\
\\
2. What is the maximum amount that Tom would be willing to pay per year for insurance that would completely reimburse him for any damage to his car?\\
\\
\\
\\
\textbf{B. } Determine whether each of the following statements is \textbf{true, false, or uncertain,} and justify your answer.\\
\\
1. If an individual prefers lottery L to lottery l' and lottery L to lottery L'' .But prefers a compound lottery that give probability $\alpha$ to L' and probability $1 - \alpha$ to L'' to the original lottery L, one can design a series of deals leading an agent to a sure loss of money.\\
\\
2. Suppose Eric buys a bottle of wine when cheap and the wine is young, and hold the bottle until it is worth \$200. He would never pay that much to buy a bottle of wine, yet he refuses an offer of \$200. \textbf{TFU: }His behavior violates the axioms of rational choice.\\
\\
3. At any given moment, there will be greater inequality of consumption in an endowment economy where individuals are risk seekers than one with the same initial endowment where individuals are risk-averse.\\
\end{document}