\documentclass{beamer}
\usetheme{Boadilla}

%%Packages
\usepackage{multirow}
\usepackage{tabularx}
\usepackage[english]{babel}
%\usepackage{Romannum}
\usepackage[comma, sort&compress]{natbib}
\usepackage[para, flushleft]{threeparttablex}
\usepackage{rotating}
\usepackage{booktabs}
\usepackage{caption}
\usepackage{subcaption}
\usepackage{bm}
\usepackage{amsmath}
\usepackage[capposition=top]{floatrow}
%\usepackage[plainpages=false,pdfpagelabels]{hyperref}
%\usepackage[toc,title,page]{appendix}
%\usepackage{tocloft}
%\usepackage{indentfirst}
%\usepackage{fancyhdr}
\usepackage{graphicx}
%\usepackage{setspace}

%Commands
\newcommand\fnote[1]{\captionsetup{font=small}\caption*{#1}}


\title{The Effect of Entry of Generic Antibiotics on Prescriptions of Antibiotics in the Ambulatory Care Settings}
\subtitle{The Case Study of Sulfamethozaxole-Trimethoprim}
\author{Ian Davis}
\institute{Clemson University}
\date{March 23, 2020}

\begin{document}

\begin{frame}
\titlepage
\end{frame}

\begin{frame}
\frametitle{Outline}
\tableofcontents
\end{frame}

\section{Purpose of Study}
\begin{frame}{Research Questions}
\begin{enumerate}
\item Build an empirical understanding of the extent to which  universities increase productivity in the manufacturing and agricultural sector.
\item Determine whether the foundation of colleges from 1870-1940 impacted productivity of the manufacturing and agricultural sectors differently.
\end{enumerate}
\end{frame}

\begin{frame}{Contribution to Literature}
\begin{enumerate}
\item I use a comprehensive data set on US colleges that allows me to better estimate the effect of universities by exploiting variation in the size of student populations as a proxy for college size.
\item I examine the effects of colleges on both manufacturing and agricultural productivity
\end{enumerate}
\end{frame}

\begin{frame}{TestFrame}

    %\begin{table}[ht]
    \centering
    %\subfloat[Decay Channels]{
     %\rule{4cm}{3cm}
     % \newcommand{\minitab}[2][l]{\begin{tabular}{#1}#2\end{tabular}}
     %\renewcommand{\multirowsetup}{\centering}
      \begin{tabular}{|c|c|} \hline
        Description & Basic cuts  \\ \hline
       \multirow{4}{*}{Jet} &${\bf p}_T >$ 25 GeV   \\
           &$|\eta| < $ 2.5   \\
           &$\Delta R(j,l) >$ 0.2  \\
           &$\Delta \phi(j,{\bf p}_T^{miss}) >$ 0.6  \\ \hline
        3 Leading jets & ${\bf p}_T >$ 40 GeV   \\ \hline
        b-tagging & $\ge$ 2 \\ \hline
        \multirow{3}{*}{Lepton} & ${\bf p}_T >$ 20 GeV \\
               & $|\eta| <$ 2.5 \\
               & $\Delta R(l,j) >$ 0.4 \\ \hline
        {${\bf p}_T^{miss}$} & $\Delta \phi({\bf p}_T^{miss},j) >$ 0.8 \\ \hline
        & {\bf Advanced cuts} \\ \hline
        $E_T^{miss}$ & $>$ 100, 120, 140, 160, 180, 200 (GeV)\\ \hline
        $H_T$ & $>$ 400, 450, 500, 550, 600 (GeV)\\ \hline
        $m_T$ & $>$ 100, 120, 140, 160, 180, 200 (GeV)\\ \hline
        $N_j$ & $\ge$ 4, 5, 6 \\ \hline
        $N_{bj}$ & $\ge$ 2, 3, 4 \\ \hline
      \end{tabular}
    %}
  %  \caption{Summary of event selection cuts}
  % \end{table}

\end{frame}

\begin{frame}{Test 2}
\begin{tabular}{l*{4}{c}}
\hline\hline
            Variable&\multicolumn{1}{c}{Time frame}&\multicolumn{1}{c}{Total}&\multicolumn{1}{c}{Weighted Share of Sample}\\
\hline
\textbf{OffLabel}                                                &     Entire Study&             387264&      .967\\
(=1 if no diagnoses made were FDA approved          &     Before Entry of Generic&    223267&      .966\\
indications of sulfamethoxazole-trimethoprim)  &     After Entry of Generic&      163997&      .968\\
[1em]
\textbf{GovInsurance}                                            &     Entire Study&             105273&      .26 \\
(=1 if patient is on either Medicare or Medicaid)       &     Before Entry of Generic&     30480 &      .248\\
                                                        &     After Entry of Generic&      44793 &      .28\\
[1em]
\textbf{NonWhite}                                                &     Entire Study&             61442&      .164\\
(=1 if patient is a race other than white)                &     Before Entry of Generic&     37733&      .16\\
                                                        &     After Entry of Generic&      23709&      .171\\
\hline
$\text{Sample Size for Years 2006-2016} = 399245$\\
$\text{Before Entry of Generic} = 230182$\\
$\text{After Entry of Generic} = 169063$\\
\hline\hline
\multicolumn{4}{l}{\footnotesize All observations after August 2012 are considered to be after entry of generic.}\\
\end{tabular}

%multicolumn{5}{l}{"Share of Sample" and "Proportion Prescribed ST" are both weighted proportions}\\
%\multicolumn{5}{l}{offLabel(=1) indicates no diagnoses made were on label indicators, govInsurance(=1) indicates patient is on Medicare or Medicaid,}\\
%\multicolumn{5}{l}{nonWhite(=1) indicates patient is race other than white}\\
%\end{tabular}
%\label{tab:Table4.2}
%\end{table}

\end{frame}


\begin{frame}{College Expansion}
\begin{itemize}
\item From 1870-1940 the size of the university system increased significantly. 
\begin{itemize}
\item The Morrill Act of 1862 promoted the establishment of colleges specializing in agriculture and the mechanic arts by granting states land for education. 
\item The number of colleges in the Reports of the Commissioner of Education more than doubled from 332 to 700 with a peak of 906 colleges in 1926.
\item Total student enrollment in universities increased nearly twenty-fold from 54,033 to 1,009,945.
\end{itemize}
\item I assume colleges are plausibly located randomly when controlling for county and county-by-year.
\end{itemize}
\end{frame}

\section{Empirical Results}
\begin{frame}{Productivity Measures}
\begin{equation*}
y_{ct}=\beta_0+\beta_1students_{ct}+\beta_2population_{ct}+\gamma_c+\gamma_{ct}+\varepsilon_{ct}
\end{equation*}
\begin{itemize}
\item c=county; t=year; y=outcome variable (either agricultural revenue per acre or manufacturing revenue per employee); students=average number of students enrolled in universities over the decennial and 9 previous years; $\gamma_c$=county fixed effects; $\gamma_{ct}$=county-by-year fixed effects
\item The coefficient $\beta_1$ reports the effect of average number of students on outcome, y, comparing changes in counties with relative increases in average number of students to other counties controlling for population, county fixed effects, and county-by-year fixed effects.
\end{itemize}
\end{frame}

\section{Conclusion}

\begin{frame}{Conclusion}
\begin{itemize}
\item I find that every one unit increase in the average number of students enrolled in a university in a county over the previous decade increases 
\begin{itemize}
\item Agricultural output per acre by $\$.01$. 
\item Manufacturing output per employee by $\$2.13$.
\end{itemize}
\item I find that the effect of universities on agricultural productivity occurs as early as 1880.
\item I find that the effect on manufacturing productivity does not appear significant until 1920.
\end{itemize}
\end{frame}

\begin{frame}{Limitations of Study}
\begin{itemize}
\item Results are undermined if colleges are not located randomly and are instead influenced by the presence of highly productive firms.
\item Lack of consistent data throughout the period of study limited the types of studies that could be conducted and required me to make possibly inaccurate.
\end{itemize}
\end{frame}

\end{document}
