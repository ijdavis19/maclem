\documentclass{beamer}
\usetheme{Boadilla}

%%Packages
\usepackage{multirow}
\usepackage{tabularx}
\usepackage[english]{babel}
%\usepackage{Romannum}
\usepackage[comma, sort&compress]{natbib}
\usepackage[para, flushleft]{threeparttablex}
\usepackage{rotating}
\usepackage{booktabs}
\usepackage{caption}
\usepackage{subcaption}
\usepackage{bm}
\usepackage{amsmath}
\usepackage[capposition=top]{floatrow}
%\usepackage[plainpages=false,pdfpagelabels]{hyperref}
%\usepackage[toc,title,page]{appendix}
%\usepackage{tocloft}
%\usepackage{indentfirst}
%\usepackage{fancyhdr}
\usepackage{graphicx}
%\usepackage{setspace}

%Commands
\newcommand\fnote[1]{\captionsetup{font=small}\caption*{#1}}


\title{The Effect of Entry of Generic Antibiotics on Prescriptions of Antibiotics in the Ambulatory Care Settings}
\subtitle{The Case Study of Sulfamethozaxole-Trimethoprim}
\author{Ian Davis}
\institute{Clemson University}
\date{August 4, 2020}

\begin{document}

\begin{frame}
\titlepage
\end{frame}

\begin{frame}
\frametitle{Outline}
\tableofcontents
\end{frame}

\section{Introduction and Background}
\begin{frame}{Introduction}
\begin{itemize}
\item The effectiveness of an antibiotic is an exhaustible resource and efficient use eventually renders the ineffective.
\item Over 2.8 million Americans per year are sickened by antibiotic resistant bacteria leading to at least 35,000 deaths. \textbf{CITE}
\item Through increasing the amount of treatment options available to a patient and raising aggregate demand for treatments using a given antibiotic, entry of generic antibiotics may accelerate the spread of antibiotic resistant bacteria.
\end{itemize}
\end{frame}

\begin{frame}{Research Question}
\begin{itemize}
\item How does entry of a generic form of an antibiotic affect the demand for treatments which rely on the use of the antibiotic? 
\item How does this affect differ among patient populations?
\end{itemize}
\end{frame}

\begin{frame}{Generic Medications}
\begin{itemize}
\item Hatch-Waxely
\item Now only need to show medications are Bioequivalent
\item Bioequivalent means this
\end{itemize}
\end{frame}

\begin{frame}{Antibiotic Usage}
\begin{itemize}
\item Much research focuses on the ambulatory setting.
\item Over 50\% of Fluoroquinole antibiotics are prescribed for patients who do not have a diagnosis associated with an Food and Drug Administration (FDA) approved usage. 
\begin{itemize}
  \item These non FDA approved uses are known as "off-label" uses
\end{itemize}
\item Usage rates for adults and the elderly have remained constant since the early 2000s
\item Antibiotics behave similarly to other goods in terms of own price and cross price elasticities.
\end{itemize}
\end{frame}

\begin{frame}{Sulfamethozaxole-Trimethoprim}
\begin{itemize}
\item Analysis focuses on Sulfamethozaxole-Trimethoprim due to its generic appearing in the middle of the observational period and having a relatively large amount of prescriptions in the data compared to other antibiotics with similar dates of entry of generics.
\item The drug is a combination antibiotic from the antimetabolite/sulfonamide class first introduced in 1968.
\item Brand name versions of the antibiotic include Bactrim, Bactrim DS, Septra and Septra DS.
\item Popular antibiotic due to its low cost and high familiarity among physicians.
\end{itemize}
\end{frame}

\begin{frame}{Sulfamethozaxole-Trimethoprim}
\begin{itemize}
\item The drug has FDA approval to fight urinary tract infections, ear infections (acute otitis media), acute exacerbations of chronic bronchitis in adults, Shigellosis, treatment and prophylaxis of \textit{Pneumocystis jirovecii} pneumonia, and Traveler's diarrhea in adults.
\item The antibiotic is also approved for use against infections due to \textit{Listeria, Nocardia, Salmonella, Brucella, Paracoccidioides,} melioidosis, \textit{Burkholderia, Stenotrophomonas,} cyclospora, isospora, Whipple's disease, and alternative therapy for toxoplasmosis and community-acquired MRSA skin infections.
\item Evidence has been found to indicate effectiveness in curing uncomplicated skin infections.
\end{itemize}
\end{frame}

\section{Economic Framework}
\begin{frame}{Theoretical Model}
\begin{itemize}
\item Patient's, with assistance of their physician, seek to maximize their expected utility from treatment of their medical condition.
\item A patient considers both the cost and effectiveness of the treatment.
  \begin{itemize}
    \item Probability condition will be cured
    \item Minimizing treatment convalescence
  \end{itemize}
\item Entry of the generic lowers the efficiency templeton phrase which would increase demand under these assumptions
\end{itemize}
\end{frame}

\begin{frame}{Theoretical Model}
\begin{itemize}
\item Build an empirical understanding of the extent to which  universities increase productivity in the manufacturing and agricultural sector.
\item Here's a point where I use $\zeta$ to see what it looks like
\end{itemize}
\end{frame}


\begin{frame}{Econometric Model and Estimation Procedures}
\begin{itemize}
\item Build an empirical understanding of the extent to which  universities increase productivity in the manufacturing and agricultural sector.
\item Determine whether the foundation of colleges from 1870-1940 impacted productivity of the manufacturing and agricultural sectors differently.
\end{itemize}
\end{frame}


\begin{frame}{Econometric Model and Estimation Procedures}
\begin{itemize}
\item Build an empirical understanding of the extent to which  universities increase productivity in the manufacturing and agricultural sector.
\item Determine whether the foundation of colleges from 1870-1940 impacted productivity of the manufacturing and agricultural sectors differently.
\end{itemize}
\end{frame}

\section{Data and Variables}
\begin{frame}{National Ambulatory Medical Care Survey}
\begin{itemize}
\item Build an empirical understanding of the extent to which  universities increase productivity in the manufacturing and agricultural sector.
\item Determine whether the foundation of colleges from 1870-1940 impacted productivity of the manufacturing and agricultural sectors differently.
\end{itemize}
\end{frame}

\begin{frame}{Variables}
\begin{itemize}
\item Build an empirical understanding of the extent to which  universities increase productivity in the manufacturing and agricultural sector.
\item Determine whether the foundation of colleges from 1870-1940 impacted productivity of the manufacturing and agricultural sectors differently.
\end{itemize}
\end{frame}

\begin{frame}{Statistical Descriptions of Continuous Variables}
\scalebox{.475}{\begin{tabular}{l*{6}{c}}
\hline\hline
            Variable&\multicolumn{1}{c}{Time frame}&\multicolumn{1}{c}{Weighted Mean}&\multicolumn{1}{c}{Weighted Median}&\multicolumn{1}{c}{Standard Deviation}&\multicolumn{1}{c}{Minimum}&\multicolumn{1}{c}{Maximum}\\
\hline
\textbf{TimeSinceGeneric}                    &     2006-2026&             -15.316&    -17&   37.167&     -79&  52\\
(Time in months since entry &     Before Entry of Generic&     -39.703&    -40&    22.497 &     -98&  -1\\
 of generic)   &     After Entry of Generic&       25.013 &    25&      14.792&     0&  52\\
[1em]
\textbf{Age}                                 &     2006-2026&             45.917&    50&    25.09 &     0&  100\\
(Age of patient in years)           &     Before Entry of Generic&     45.221&    49&    25.207&     0&  100\\
                                    &     After Entry of Generic&      47.069&    51&    24.853&     0&  92\\
[1em]
\textbf{AgeSQ}                               &     2006-2026&             2737.892&    2500&  2218.939&     0&  10000\\
(Age of patient squared)   &     Before Entry of Generic&    2680.279&    2401&  2216.963 &     0&  10000\\
                                    &     After Entry of Generic&      2833.167&    2601&  2218.929 &     0&  8464\\
\hline
$\text{Sample Size for Years 2006-2016} = 399245$\\
$\text{Before Entry of Generic} = 230182$\\
$\text{After Entry of Generic} = 169063$\\
\hline\hline
\multicolumn{4}{l}{\footnotesize All observations after July 2012 are considered to be after entry of generic.}\\
\end{tabular}
}
\end{frame}

\begin{frame}{Statistical Descriptions of Categorical Variables}
\scalebox{.65}{\begin{tabular}{l*{4}{c}}
\hline\hline
            Variable&\multicolumn{1}{c}{Time frame}&\multicolumn{1}{c}{Total}&\multicolumn{1}{c}{Weighted Share of Sample}\\
\hline
\textbf{OffLabel}                                                &     Entire Study&             387264&      .967\\
(=1 if no diagnoses made were FDA approved          &     Before Entry of Generic&    223267&      .966\\
indications of sulfamethoxazole-trimethoprim)  &     After Entry of Generic&      163997&      .968\\
[1em]
\textbf{GovInsurance}                                            &     Entire Study&             105273&      .26 \\
(=1 if patient is on either Medicare or Medicaid)       &     Before Entry of Generic&     30480 &      .248\\
                                                        &     After Entry of Generic&      44793 &      .28\\
[1em]
\textbf{NonWhite}                                                &     Entire Study&             61442&      .164\\
(=1 if patient is a race other than white)                &     Before Entry of Generic&     37733&      .16\\
                                                        &     After Entry of Generic&      23709&      .171\\
\hline
$\text{Sample Size for Years 2006-2016} = 399245$\\
$\text{Before Entry of Generic} = 230182$\\
$\text{After Entry of Generic} = 169063$\\
\hline\hline
\multicolumn{4}{l}{\footnotesize All observations after August 2012 are considered to be after entry of generic.}\\
\end{tabular}

%multicolumn{5}{l}{"Share of Sample" and "Proportion Prescribed ST" are both weighted proportions}\\
%\multicolumn{5}{l}{offLabel(=1) indicates no diagnoses made were on label indicators, govInsurance(=1) indicates patient is on Medicare or Medicaid,}\\
%\multicolumn{5}{l}{nonWhite(=1) indicates patient is race other than white}\\
%\end{tabular}
%\label{tab:Table4.2}
%\end{table}
}
\end{frame}

% Probs split into two
\begin{frame}{Proportions of Patients Prescribed Sulfamethozaxole-Trimethoprim by Patient Characteristics}
\begin{center}
\scalebox{.4}{\begin{table}[htbp]\centering
\def\sym#1{\ifmmode^{#1}\else\(^{#1}\)\fi}
\caption{Proportions of Patients Prescribed Sulfamethoxazole-Trimethoprim (SXT) by Group\label{tab1}}
\begin{tabular}{l*{3}{c}}
\hline\hline
            Variable&\multicolumn{1}{c}{Timeframe}&\multicolumn{1}{c}{Total Prescriptions of SXT}&\multicolumn{1}{c}{Weighted Proportion Prescribed SXT}\\
\hline
\textbf{Total Sample}                                   &     Entire Study&             3340&     .00861\\
                                                        &     Before Generic Entry&    2072&     .00851\\
                                                        &     After Generic Entry&      1268&     .0088\\
[1em]
\textbf{offLabel}                                       &     Entire Study&             2736&     .00718\\
(=1 if no diagnoses made were FDA approved         &     Before Generic Entry&    1663&     .00696\\
indications of Sulfamethoxazole-Trimethoprim)  &     After Generic Entry&      1073&     .0078\\
[1em]
(=0 if at least one diagnosis made during               &     Entire Study&             604&     .0507\\
visit is an FDA approved indication of                  &     Before Generic Entry&    409&     .0535\\
Sulfamethoxazole-Trimethoprim)                          &     After Generic Entry&      195&     .0454\\
[1em]
\textbf{govInsurance}                                   &     Entire Study&             869&     .00871\\
(=1 if patient is on either Medicare of Medicaid)       &     Before Generic Entry&     533 &     .00806\\
                                                        &     After Generic Entry&      336 &     .00975\\
[1em]
(=0 if patient is on neither Medicare nor Medicaid)     &     Entire Study&             2471&     .00858\\
                                                        &     Before Generic Entry&     1539 &     .00865\\
                                                        &     After Generic Entry&      932 &     .00843\\
[1em]
\textbf{nonWhite}                                       &     Entire Study&             534&      .00933\\
(=1 if patient is race other than white)                &     Before Generic Entry&     354&      .00846\\
                                                        &     After Generic Entry&      180&      .0108\\
[1em]
(=0 if patient is white)                                &     Entire Study&             2806&      .00847\\
                                                        &     Before Generic Entry&     1718&      .00852\\
                                                        &     After Generic Entry&      1088&      .00838\\
[1em]
\textbf{UnspecCellAbscess}                              &     Entire Study&             349 &      .175\\
(=1 if patient was diagnosed with an unspecified        &     Before Generic Entry&     224 &      .170\\
 skin abscess or cellulitis)                            &     After Generic Entry&      125  &      .184\\
[1em]
(=0 if patient was not diagnosed with an unspecified    &     Entire Study&             2291 &      .00761\\
skin abscess or cellulitis)                             &     Before Generic Entry&     1848 &      .0075\\
                                                        &     After Generic Entry&      1143  &      .0078\\
\hline
$n(\text{Entire Study}) = 399245$\\
$n(\text{Before Generic Entry}) = 249345$\\
$n(\text{After Generic Entry}) = 149900$\\
\hline\hline
%\multicolumn{5}{l}{"Share of Sample" and "Proportion Prescribed ST" are both weighted proportions}\\
%\multicolumn{5}{l}{offLabel(=1) indicates no diagnoses made were on label indicators, govInsurance(=1) indicates patient is on Medicare or Medicaid, nonWhite(=1) indicates}\\
%\multicolumn{5}{l}{patient is race other than white, UnspecCellAbscess(=1) indicates patient was diagnosed with and unspecified skin abscess or cellulitis}\\
\end{tabular}
\label{tab:Table4.4}
\end{table}

%multicolumn{5}{l}{"Share of Sample" and "Proportion Prescribed ST" are both weighted proportions}\\
%\multicolumn{5}{l}{offLabel(=1) indicates no diagnoses made were on label indicators, govInsurance(=1) indicates patient is on Medicare or Medicaid,}\\
%\multicolumn{5}{l}{nonWhite(=1) indicates patient is race other than white}\\
%\end{tabular}
%\label{tab:Table4.2}
%\end{table}
}
\end{center}
\end{frame}

\begin{frame}{12 Month Moving Average Probability of Prescription of Sulfamethozaxole-Trimethoprim}
\includegraphics[width=\textwidth, height=.75\textheight]{figs/twelveMonthMA.png}
\end{frame}

\begin{frame}{Seperated 12 Month Moving Average Probability of Prescription of Sulfamethozaxole-Trimethoprim}
\includegraphics[width=\textwidth, height=.75\textheight]{figs/twelveMonthMAonoff.png}
\end{frame}

\section{Empirical Analysis and Results}
\begin{frame}{Empirical Analysis and Results}
\begin{itemize}
\item Build an empirical understanding of the extent to which  universities increase productivity in the manufacturing and agricultural sector.
\item Determine whether the foundation of colleges from 1870-1940 impacted productivity of the manufacturing and agricultural sectors differently.
\end{itemize}
\end{frame}

\begin{frame}{Estimated Effects on Probability of Prescription Before and After Generic Entry}
\begin{center}
\scalebox{.45}{\def\sym#1{\ifmmode^{#1}\else\(^{#1}\)\fi}
\begin{tabular}{l*{3}{c}}
\hline\hline
Variable            &\multicolumn{1}{c}{Before Generic Entry}&\multicolumn{1}{c}{After Generic Entry}&\multicolumn{1}{c}{Difference}\\
\hline
\textbf{timeSinceGeneric}&                           -0.00032\sym{***}&   -0.000592\sym{***}&   -.000272\\
(Time in months since generic entry            &     (-6.51)         &     (-6.72)         &     [0.368]         \\
of Sulfamethoxazole-Trimethoprim)\\
[.5em]
\textbf{offLabel}    &                                     -0.0326\sym{***}&     -0.0513\sym{***}&   -.0188\\
(=1 if no diagnoses made were FDA approved            &    (-14.65)         &    (-20.05)         &    [0.0676]         \\
indications of Sulfamethoxazole-Trimethoprim)\\
[.5em]
\textbf{offLabel}$\times$\textbf{timeSinceGeneric} &    0.000372\sym{***}&    0.00047\sym{***}&   .0000984\\
            &                                           (7.7)         &      (5.64)         &    [0.757]         \\
[.5em]
\textbf{age}         &                      0.000172\sym{*}  &    0.0000696\sym{*}  &   -.000103\\
(Age of patient in years)            &      (3.08)         &      (1.06)         &    [0.378]         \\
[.5em]
\textbf{ageSQ}       &                             -0.00000215\sym{**} & -0.00000128\sym{*}  &   0.000000863\\
(Age of patient in years squared)            &     (-3.36)         &     (-1.74)         &    [0.495]         \\
[.5em]
\textbf{age}$\times$\textbf{timeSinceGeneric}      & -0.00000229\sym{**} &  0.00000419         &   0.000000648\\
            &                                        (-1.88)         &      (1.88)         &    [0.107]         \\
[.5em]
\textbf{ageSQ}$\times$\textbf{timeSinceGeneric}    &    .0000000274\sym{**} &   -0.000000031         &   -0.000000059\\
            &                                           (1.98)         &     (-1.23)         &    [0.184]         \\
[.5em]
\textbf{govInsurance}&                                    0.000176         &     0.00185\sym{**} &   0.00168\\
(=1 if patient is on either Medicare or            &      (0.38)         &      (3.54)         &    [0.161]         \\
Medicaid)\\
[.5em]
\textbf{nonwhite}    &                                    -0.00000675         &     0.00219\sym{***}&   0.00222\\
(=1 if patient is race other than white)            &     (-0.01)         &      (3.65)         &    [0.164]         \\
[.5em]
\textbf{\_cons}      &      0.0383\sym{***}&      0.0586\sym{***}&   0.0203\sym{*}\\
            &               (16.83)         &     (21.83)         &    [0.0403]         \\
\hline
r2          &     0.009         &     0.0058\\
N           &      230182         &      169063\\
\hline\hline
\multicolumn{3}{l}{\footnotesize \textit{t} statistics in parentheses, \scalebox{1.25}{$\text{Pr}(\frac{\hat{\beta}^\text{before}_i - \hat{\beta}^\text{after}_i}{[\hat{\sigma}^2\{\hat{\beta}^\text{before}_i\} + \hat{\sigma}^2\{\hat{\beta}^\text{after}_i\}]^\frac{1}{2}} > X^2)$} in brackets}\\
\multicolumn{3}{l}{\footnotesize \sym{*} \(p<0.05\), \sym{**} \(p<0.01\), \sym{***} \(p<0.001\)}\\
\multicolumn{4}{l}{\footnotesize "Probability of prescription" refers to probability that medical visit will have a prescription of Sulfamethoxazole-Trimethoprim}
\end{tabular}}
\end{center}
\end{frame}

\begin{frame}{Empirical Analysis and Results}
\begin{itemize}
\item From 1870-1940 the size of the university system increased significantly. 
\begin{itemize}
\item The Morrill Act of 1862 promoted the establishment of colleges specializing in agriculture and the mechanic arts by granting states land for education. 
\item The number of colleges in the Reports of the Commissioner of Education more than doubled from 332 to 700 with a peak of 906 colleges in 1926.
\item Total student enrollment in universities increased nearly twenty-fold from 54,033 to 1,009,945.
\end{itemize}
\item I assume colleges are plausibly located randomly when controlling for county and county-by-year.
\end{itemize}
\end{frame}

\begin{frame}{Estimated Effects on Probability of Prescription Before and After Generic Entry for Patients with an Uncomplicated Skin Abscess or Cellulitis}
\begin{center}
\scalebox{.45}{\begin{table}[htbp]\centering
\def\sym#1{\ifmmode^{#1}\else\(^{#1}\)\fi}
\caption{Estimated Effects on Probability of Prescription Before and After Generic Entry for Patients with Uncomplicated Skin Abscess or Cellulitis\label{tab1}}
\begin{tabular}{l*{3}{c}}
\hline\hline
            &\multicolumn{1}{c}{Before Generic Entry}&\multicolumn{1}{c}{After Generic Entry}&\multicolumn{1}{c}{Difference}\\
\hline
\textbf{timeSinceGeneric}&     0.00181         &     -0.0107\sym{**} & -.0125\\
(Time in months since generic entry            &      (1.47)         &     (-3.12) & [.0052]\\
of Sulfamethoxazole-Trimethoprim)\\
[1em]
\textbf{age}         &     0.00189         &    -0.00582 & -.0077\\
(Age of patient in years)            &      (0.63)         &     (-1.50) & [.212]\\
[1em]
\textbf{ageSQ}       &  -0.0000445         &   0.0000238 & .0000682\\
(Age of patient in years squared)            &     (-1.33)         &      (0.57) & [.267]\\
[1em]
\textbf{age}$\times$\textbf{timeSinceGeneric}      &  -0.0000630         &    0.000379\sym{*}& .000442 \\
            &     (-1.00)         &      (2.48)         & [.0274]\\
[1em]
\textbf{ageSQ}$\times$\textbf{timeSinceGeneric}    & 0.000000653         & -0.00000324\sym{*}  & -.0000039\\
            &      (0.91)         &     (-2.08)         & [.0449]\\
[1em]
\textbf{govInsurance}&      0.0156         &      0.0256         & .01\\
(=1 if patient is on either Medicare or            &      (0.65)         &      (0.77)         & [.896]\\
Medicaid)\\
[1em]
\textbf{nonwhite}    &   0.0000720         &      0.0911\sym{*}  & .091\\
(=1 if patient is race other than white)            &      (0.00)         &      (2.46)         & [.364]\\
[1em]
\textbf{\_cons}      &       0.228\sym{***}&       0.413\sym{***} & .185\\
            &      (3.72)         &      (4.90)         & [.207] \\
\hline
r2          &      0.0268         &      0.0439         & (.)\\
N           &        1359         &         779         & (.)\\
\hline\hline
\multicolumn{3}{l}{\footnotesize \textit{t} statistics in parentheses, \scalebox{1.25}{$\text{Pr}(\frac{\hat{\beta}^\text{before}_i - \hat{\beta}^\text{after}_i}{[\hat{\sigma}^2\{\hat{\beta}^\text{before}_i\} + \hat{\sigma}^2\{\hat{\beta}^\text{after}_i\}]^\frac{1}{2}} > X^2)$} in brackets}\\
\multicolumn{3}{l}{\footnotesize \sym{*} \(p<0.05\), \sym{**} \(p<0.01\), \sym{***} \(p<0.001\)}\\
\end{tabular}
\end{table}
}
\end{center}
\end{frame}

\begin{frame}{Estimated Probability of Prescription Before and After Generic Entry}
\begin{center}
\scalebox{.55}{\begin{table}[htbp]\centering
\def\sym#1{\ifmmode^{#1}\else\(^{#1}\)\fi}
\caption{Estimated Probability of Prescription of Sulfamethoxazole-Trimethoprim Immediately Before and After Generic Entry \label{tab1}}
\begin{tabular}{l*{3}{c}}
\hline\hline
Patient Group  &\multicolumn{1}{c}{(Before Generic Entry)}&\multicolumn{1}{c}{(After Generic Entry)}&\multicolumn{1}{c}{Difference}\\
\hline
\textbf{onLabel}                                               &   0.0405\sym{***}   &   0.0594\sym{***}   &   .019\\
(=1 if at least one diagnosis made during visit is an          &     [0.000]         &     [0.000]         &     [0.0736]       \\
FDA approved indication of Sulfamethoxazole-Trimethoprim)\\
[1em]
\textbf{offLabel}                                     &     0.00756\sym{***}&     .00824\sym{***}  &   .000679\\
(=1 if no diagnoses made were FDA approved            &    [0.000]          &    [.00195]         &    [0.389]         \\
indications of Sulfamethoxazole-Trimethoprim)\\
[1em]
\textbf{govInsurance}$\times$\textbf{onLabel}&    0.0405\sym{***}         &     0.0607\sym{***} &   0.0202\\
(Patient on Medicare or Medicaid        &      [0.000]         &      [0.000]        &    [0.0600]         \\
and \textbf{onLabel}=1)\\
[1em]
\textbf{nonwhite}$\times$\textbf{onLabel}    &  0.0405\sym{***}         &     0.0614\sym{***}&   0.0209\sym{*}\\
(Patient race other than white            &     [0.000]        &      [0.000]         &    [0.0484]         \\
and \textbf{onLabel}=1)\\
[1em]
\textbf{UnspecCellAbscess}                   &  0.228\sym{***}         &     0.435&   0.207\\
(=1 if patient was diagnosed with an unspecified            &     [0.004]        &      [0.000]         &    [0.162]         \\
 skin abscess or cellulitis)\\
\hline\hline
\multicolumn{3}{l}{\footnotesize \scalebox{1.25}{$\text{Pr}(\frac{\hat{\beta}^\text{before}_i - \hat{\beta}^\text{after}_i}{[\hat{\sigma}^2\{\hat{\beta}^\text{before}_i\} + \hat{\sigma}^2\{\hat{\beta}^\text{after}_i\}]^\frac{1}{2}} > X^2)$} in brackets}\\
\multicolumn{3}{l}{\footnotesize \sym{*} \(p<0.05\), \sym{**} \(p<0.01\), \sym{***} \(p<0.001\)}\\
\end{tabular}
\end{table}
}
\end{center}
\end{frame}

\section{Discussion}
\begin{frame}{Results}
\begin{itemize}
\item I find that every one unit increase in the average number of students enrolled in a university in a county over the previous decade increases 
\begin{itemize}
\item Agricultural output per acre by $\$.01$. 
\item Manufacturing output per employee by $\$2.13$.
\end{itemize}
\item I find that the effect of universities on agricultural productivity occurs as early as 1880.
\item I find that the effect on manufacturing productivity does not appear significant until 1920.
\end{itemize}
\end{frame}

\begin{frame}{Visits with Diagnoses Associated with On-Label Indications}
\begin{itemize}
\item I find that every one unit increase in the average number of students enrolled in a university in a county over the previous decade increases 
\begin{itemize}
\item Agricultural output per acre by $\$.01$. 
\item Manufacturing output per employee by $\$2.13$.
\end{itemize}
\item I find that the effect of universities on agricultural productivity occurs as early as 1880.
\item I find that the effect on manufacturing productivity does not appear significant until 1920.
\end{itemize}
\end{frame}

\begin{frame}{Visits with Diagnoses Not Associated with On-Label Indications}
\begin{itemize}
\item I find that every one unit increase in the average number of students enrolled in a university in a county over the previous decade increases 
\begin{itemize}
\item Agricultural output per acre by $\$.01$. 
\item Manufacturing output per employee by $\$2.13$.
\end{itemize}
\item I find that the effect of universities on agricultural productivity occurs as early as 1880.
\item I find that the effect on manufacturing productivity does not appear significant until 1920.
\end{itemize}
\end{frame}

\section{Conclusions}
\begin{frame}{Conclusions}
\begin{itemize}
\item I find that every one unit increase in the average number of students enrolled in a university in a county over the previous decade increases 
\begin{itemize}
\item Agricultural output per acre by $\$.01$. 
\item Manufacturing output per employee by $\$2.13$.
\end{itemize}
\item I find that the effect of universities on agricultural productivity occurs as early as 1880.
\item I find that the effect on manufacturing productivity does not appear significant until 1920.
\end{itemize}
\end{frame}

\begin{frame}{Limitations of Study}
\begin{itemize}
\item Results are undermined if colleges are not located randomly and are instead influenced by the presence of highly productive firms.
\item Lack of consistent data throughout the period of study limited the types of studies that could be conducted and required me to make possibly inaccurate.
\end{itemize}
\end{frame}

\begin{frame}{Recommendations for Future Research}
\begin{itemize}
\item Results are undermined if colleges are not located randomly and are instead influenced by the presence of highly productive firms.
\item Lack of consistent data throughout the period of study limited the types of studies that could be conducted and required me to make possibly inaccurate.
\end{itemize}
\end{frame}

\end{document}
