\chapter{Empirical Analysis and Results}
\indent \autoref{tab:Table5.1} illustrates the results of the two binary linear probability models. In both cases, the probability of a white patient with an ailment associated with an on-label use of sulfamethozaxole-trimethoprim who is not on government insurance has a statistically nonzero probability of demanding treatment with the antibiotic. The change in this probability of prescription after entry of the generic is 1.8 percentage points and significant at the 90\% confidence level. Both before and after generic entry the \textbf{timeSinceGeneric} coefficient is negative which indicates a decreasing probability of being prescribed sulfamethozaxole-trimethoprim over the years 2006-2016. The negative slope estimate after the generic has entered is more than twice that of before generic entry (.0608 percentage points and .0283 percentage points respectively) but the difference fails to be significant. Deviation from the base probability due to an off-label visit increases at the 90\% confidence level from -3.29\% to -5.12\%. The postive coefficients for the interactions between \textbf{timeSinceGeneric} and \textbf{offLabel} are greater than the negative coefficient attached to \textbf{timeSinceGeneric} in absolute value indicates that the probability of prescription for offLabel visits was increasing with time. Significant positive coefficents on \textbf{age} and significant negative coefficients on \textbf{ageSQ} indicate that a patient's probability of prescription is increasing with age but this increase decreases as the patient ages. The interactions between \textbf{age} and \textbf{timeSinceGeneric} as well as \textbf{ageSQ} and \textbf{timeSinceGeneric} show how this age effect changes over the course of time. During the period of time before the generic had entered, there was a small but significant decrease in the probability of prescription as patient's age increased over time which decreased further as a patient increased in age. This effect loses significance once the generic enters the market.\\
\indent Both patients on Medicare or Medicaid and patients of a race other than white did not have a probability significantly different from a white patient not on government insurance for visits with an on-label usage before entry of the generic. After entry of the generic, however, patients on government insurance became .182 percentage points more likely to be prescribed sulfamethozaxole-trimethoprim than those not on Medicare or Medicaid. Similarly, patients of a race other than white became .246 percentage points more likely to be prescribed sulfamethozaxole-trimethoprim than their white counterparts. Both of these estimates are significant at the 99\% confidence level. 
\def\sym#1{\ifmmode^{#1}\else\(^{#1}\)\fi}
\begin{tabular}{l*{3}{c}}
\hline\hline
Variable            &\multicolumn{1}{c}{Before Generic Entry}&\multicolumn{1}{c}{After Generic Entry}&\multicolumn{1}{c}{Difference}\\
\hline
\textbf{timeSinceGeneric}&                           -0.00032\sym{***}&   -0.000592\sym{***}&   -.000272\\
(Time in months since generic entry            &     (-6.51)         &     (-6.72)         &     [0.368]         \\
of Sulfamethoxazole-Trimethoprim)\\
[.5em]
\textbf{offLabel}    &                                     -0.0326\sym{***}&     -0.0513\sym{***}&   -.0188\\
(=1 if no diagnoses made were FDA approved            &    (-14.65)         &    (-20.05)         &    [0.0676]         \\
indications of Sulfamethoxazole-Trimethoprim)\\
[.5em]
\textbf{offLabel}$\times$\textbf{timeSinceGeneric} &    0.000372\sym{***}&    0.00047\sym{***}&   .0000984\\
            &                                           (7.7)         &      (5.64)         &    [0.757]         \\
[.5em]
\textbf{age}         &                      0.000172\sym{*}  &    0.0000696\sym{*}  &   -.000103\\
(Age of patient in years)            &      (3.08)         &      (1.06)         &    [0.378]         \\
[.5em]
\textbf{ageSQ}       &                             -0.00000215\sym{**} & -0.00000128\sym{*}  &   0.000000863\\
(Age of patient in years squared)            &     (-3.36)         &     (-1.74)         &    [0.495]         \\
[.5em]
\textbf{age}$\times$\textbf{timeSinceGeneric}      & -0.00000229\sym{**} &  0.00000419         &   0.000000648\\
            &                                        (-1.88)         &      (1.88)         &    [0.107]         \\
[.5em]
\textbf{ageSQ}$\times$\textbf{timeSinceGeneric}    &    .0000000274\sym{**} &   -0.000000031         &   -0.000000059\\
            &                                           (1.98)         &     (-1.23)         &    [0.184]         \\
[.5em]
\textbf{govInsurance}&                                    0.000176         &     0.00185\sym{**} &   0.00168\\
(=1 if patient is on either Medicare or            &      (0.38)         &      (3.54)         &    [0.161]         \\
Medicaid)\\
[.5em]
\textbf{nonwhite}    &                                    -0.00000675         &     0.00219\sym{***}&   0.00222\\
(=1 if patient is race other than white)            &     (-0.01)         &      (3.65)         &    [0.164]         \\
[.5em]
\textbf{\_cons}      &      0.0383\sym{***}&      0.0586\sym{***}&   0.0203\sym{*}\\
            &               (16.83)         &     (21.83)         &    [0.0403]         \\
\hline
r2          &     0.009         &     0.0058\\
N           &      230182         &      169063\\
\hline\hline
\multicolumn{3}{l}{\footnotesize \textit{t} statistics in parentheses, \scalebox{1.25}{$\text{Pr}(\frac{\hat{\beta}^\text{before}_i - \hat{\beta}^\text{after}_i}{[\hat{\sigma}^2\{\hat{\beta}^\text{before}_i\} + \hat{\sigma}^2\{\hat{\beta}^\text{after}_i\}]^\frac{1}{2}} > X^2)$} in brackets}\\
\multicolumn{3}{l}{\footnotesize \sym{*} \(p<0.05\), \sym{**} \(p<0.01\), \sym{***} \(p<0.001\)}\\
\multicolumn{4}{l}{\footnotesize "Probability of prescription" refers to probability that medical visit will have a prescription of Sulfamethoxazole-Trimethoprim}
\end{tabular}
\newpage
I use a similar estimation technique to look only at patients who were diagnosed with an unspecified skin abscess or cellulitis. Variables regarding the off-label distinction of a patient's ailment are removed because of redundancy as very few of these patients were also diagnosed with an ailment associated with an on-label use of sulfamethozaxole-trimethoprim. The results of these regressions are shown in \autoref{tab:Table5.2}.\\
\indent There are fewer significant variables and probability of being prescribed the antibiotic for patients with this condition is much higher. White patients with this condition who were not on Medicare or Medicaid had a probability of demanding treatment with the antibiotic of 22.8\% before entry of the generic and 41.3\% after entry. This 18.5 percentage point increase is not statistically significant, however. The coefficient for \textbf{timeSinceGeneric} is significant and negative after generic entry and the interactions between the age related variables and \textbf{timeSinceGeneric} indicate that, after the generic enters the market, the effect on probability of prescription to due to increases in age increases as more time passes since generic entry but this change decreases with age. Finally, patients of a race other than white are 9.11 percentage points more likely to be prescribed sulfamethozaxole-trimethoprim in this context than white patients after the generic enters the market at the 95\% confidence level. This is an increase from an insignificant .0072 percentage points before generic entry.

\begin{table}[htbp]\centering
\def\sym#1{\ifmmode^{#1}\else\(^{#1}\)\fi}
\caption{Estimated Effects on Probability of Prescription Before and After Generic Entry for Patients with Uncomplicated Skin Abscess or Cellulitis\label{tab1}}
\begin{tabular}{l*{3}{c}}
\hline\hline
            &\multicolumn{1}{c}{Before Generic Entry}&\multicolumn{1}{c}{After Generic Entry}&\multicolumn{1}{c}{Difference}\\
\hline
\textbf{timeSinceGeneric}&     0.00181         &     -0.0107\sym{**} & -.0125\\
(Time in months since generic entry            &      (1.47)         &     (-3.12) & [.0052]\\
of Sulfamethoxazole-Trimethoprim)\\
[1em]
\textbf{age}         &     0.00189         &    -0.00582 & -.0077\\
(Age of patient in years)            &      (0.63)         &     (-1.50) & [.212]\\
[1em]
\textbf{ageSQ}       &  -0.0000445         &   0.0000238 & .0000682\\
(Age of patient in years squared)            &     (-1.33)         &      (0.57) & [.267]\\
[1em]
\textbf{age}$\times$\textbf{timeSinceGeneric}      &  -0.0000630         &    0.000379\sym{*}& .000442 \\
            &     (-1.00)         &      (2.48)         & [.0274]\\
[1em]
\textbf{ageSQ}$\times$\textbf{timeSinceGeneric}    & 0.000000653         & -0.00000324\sym{*}  & -.0000039\\
            &      (0.91)         &     (-2.08)         & [.0449]\\
[1em]
\textbf{govInsurance}&      0.0156         &      0.0256         & .01\\
(=1 if patient is on either Medicare or            &      (0.65)         &      (0.77)         & [.896]\\
Medicaid)\\
[1em]
\textbf{nonwhite}    &   0.0000720         &      0.0911\sym{*}  & .091\\
(=1 if patient is race other than white)            &      (0.00)         &      (2.46)         & [.364]\\
[1em]
\textbf{\_cons}      &       0.228\sym{***}&       0.413\sym{***} & .185\\
            &      (3.72)         &      (4.90)         & [.207] \\
\hline
r2          &      0.0268         &      0.0439         & (.)\\
N           &        1359         &         779         & (.)\\
\hline\hline
\multicolumn{3}{l}{\footnotesize \textit{t} statistics in parentheses, \scalebox{1.25}{$\text{Pr}(\frac{\hat{\beta}^\text{before}_i - \hat{\beta}^\text{after}_i}{[\hat{\sigma}^2\{\hat{\beta}^\text{before}_i\} + \hat{\sigma}^2\{\hat{\beta}^\text{after}_i\}]^\frac{1}{2}} > X^2)$} in brackets}\\
\multicolumn{3}{l}{\footnotesize \sym{*} \(p<0.05\), \sym{**} \(p<0.01\), \sym{***} \(p<0.001\)}\\
\end{tabular}
\end{table}

\newpage
The previous tables provide the changes only in the context of deviations from the probability of a white, non government insured patient with a diagnosis associated with an on-label use. To better understand the expected changes between before and after entry of the generic, I use the expected value of the categorical variables and their coefficients. I calculate the mean effects of the variables to hold them constant while allowing for other variables in the regression to change. The resulting estimates can be interpreted as the expected probability an individual is prescribed sulfamethozaxole-trimethoprim. The results are shown in \autoref{tab:Table5.3} along with the results of the same hypothesis test used previously.\\
\indent As before, the probability of a patient with a diagnosis associated with an on-label use of sulfamethozaxole-trimethoprim will demand a prescription of the antibiotic increases by 1.88 percentage points from 4.06\% to 5.94\% and this change is significant at the 90\% confidence level. The change in probability for a patient without a diagnosis associated with an on-label use is a much lower .0627\% and is indistinguishable from zero. Visits associated with on-label uses with patients who were on Medicare or Medicaid have an increase in probability of prescription of 2\% from 4.07\% to 6.07\% and the change is significant at the 90\% confidence level. Most significant is the probability increase of 2.1 percentage points for on-label visits with a patient of a race other than white. This increase from 4.05\% to 6.14\% is significant at the 95\% confidence level.\\
\indent Although \autoref{tab:Table5.2} had a constant change of over 20 percentage points, it can be seen that the expected change in probability due to generic entry for patients diagnosed with an unspecified skin abscess or cellulitis is only 1.8 percentage points and not significant. This lack of change is driven primarily by the large negative value of $\hat{\beta}_2^{after}\EX[age] = -.00582\EX[age] = -.292$ percentage points.
\begin{landscape}
\begin{table}[htbp]\centering
\def\sym#1{\ifmmode^{#1}\else\(^{#1}\)\fi}
\caption{Estimated Probability of Prescription of Sulfamethoxazole-Trimethoprim Immediately Before and After Generic Entry \label{tab1}}
\begin{tabular}{l*{3}{c}}
\hline\hline
Patient Group  &\multicolumn{1}{c}{(Before Generic Entry)}&\multicolumn{1}{c}{(After Generic Entry)}&\multicolumn{1}{c}{Difference}\\
\hline
\textbf{onLabel}                                               &   0.0405\sym{***}   &   0.0594\sym{***}   &   .019\\
(=1 if at least one diagnosis made during visit is an          &     [0.000]         &     [0.000]         &     [0.0736]       \\
FDA approved indication of Sulfamethoxazole-Trimethoprim)\\
[1em]
\textbf{offLabel}                                     &     0.00756\sym{***}&     .00824\sym{***}  &   .000679\\
(=1 if no diagnoses made were FDA approved            &    [0.000]          &    [.00195]         &    [0.389]         \\
indications of Sulfamethoxazole-Trimethoprim)\\
[1em]
\textbf{govInsurance}$\times$\textbf{onLabel}&    0.0405\sym{***}         &     0.0607\sym{***} &   0.0202\\
(Patient on Medicare or Medicaid        &      [0.000]         &      [0.000]        &    [0.0600]         \\
and \textbf{onLabel}=1)\\
[1em]
\textbf{nonwhite}$\times$\textbf{onLabel}    &  0.0405\sym{***}         &     0.0614\sym{***}&   0.0209\sym{*}\\
(Patient race other than white            &     [0.000]        &      [0.000]         &    [0.0484]         \\
and \textbf{onLabel}=1)\\
[1em]
\textbf{UnspecCellAbscess}                   &  0.228\sym{***}         &     0.435&   0.207\\
(=1 if patient was diagnosed with an unspecified            &     [0.004]        &      [0.000]         &    [0.162]         \\
 skin abscess or cellulitis)\\
\hline\hline
\multicolumn{3}{l}{\footnotesize \scalebox{1.25}{$\text{Pr}(\frac{\hat{\beta}^\text{before}_i - \hat{\beta}^\text{after}_i}{[\hat{\sigma}^2\{\hat{\beta}^\text{before}_i\} + \hat{\sigma}^2\{\hat{\beta}^\text{after}_i\}]^\frac{1}{2}} > X^2)$} in brackets}\\
\multicolumn{3}{l}{\footnotesize \sym{*} \(p<0.05\), \sym{**} \(p<0.01\), \sym{***} \(p<0.001\)}\\
\end{tabular}
\end{table}

\end{landscape}
