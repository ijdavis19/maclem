\chapter{Empirical Analysis and Results}
For a more complete illustration of the effect of generic entry on the probability of being prescribed sulfamethozaxole-trimethoprim, I run two pairs of regressions. These regressions serve to present this effect, how it interacts with other characteristics of each patient, and how these effects change over time. Next, I use the expected values of each category and their respective coefficients to predict the change in probability during the months immediately after the generic enters the market. \\
\indent The first regressions compares on and off label visits while controlling for the time since generic entry and patient characteristics. These regressions can be expressed as
\begin{equation}
\begin{split}
    \text{Pr}(prescription)^t & = \beta^t_0 + \beta^t_1\cdot(timeSinceGeneric) + \beta_2^t\cdot(offLabel) + \beta_3^t\cdot(offLabel\times timeSinceGeneric)\\
    & + \beta_4^t\cdot(age) + \beta_5^t\cdot(ageSQ)  + \beta_6^t\cdot(age\times timeSinceGeneric)\\
    & + \beta_7^t\cdot(ageSQ\times timeSinceGeneric)  + \beta_8^t\cdot(govInsurance)\\
    & + \beta_9^t\cdot(nonWhite)
\end{split}
\end{equation}
where $t$ represents the timeframe of either before or after generic entry. This gives us a $\beta_0^t$ which can be interpretted as the probability of a white patient with an on label diagnosis who is not on government insurance being prescribed sulfamethozaxole-trimethoprim during timeframe $t$. The results of these regressions along with the differences of each coefficient are provided in \autoref{tab:Table5.1}.\\
\indent To test the significance of the differences among coefficients, I use a cross model hypothesis test consistent with Clogg, Petkova, and Haritou (1995)\cite{clogg_statistical_1995}. The test rejects the null hypothesis that $\beta_i^{before} = \beta_i^{after}$ if 
\begin{equation}
\text{Pr}(\frac{\hat{\beta}^\text{before}_i - \hat{\beta}^\text{after}_i}{[\hat{\sigma}^2\{\hat{\beta}^\text{before}_i\} + \hat{\sigma}^2\{\hat{\beta}^\text{after}_i\}]^\frac{1}{2}})
\end{equation}
exceeds the chi sqaured threshold.\\
\indent It is important to note that the differences of the categorical coefficients do not represent the actual differences in probability of prescription across the two timeframes. Instead, they represent the differences in the differences between the patient group indicated and the base case of a white patient with an on label diagnosis who is not on government insurance. Hence, these changes must be contextualized as relative to the base case.\\
\indent From the table, it is clear many coefficients which were significant on their own are not significantly different than their counterpart from the opposite timeframe. Both base probabilities are significantly different from zero and the change in said probabilities of prescription due to generic entry is 1.8 percentage points and significant at the 90\% confidence interval. Both before and after generic entry the \textbf{timeSinceGeneric} coefficient is negative which indicates a decreasing probability of being prescribed sulfamethozaxole-trimethoprim over the entire study. The negative slope estimate after the generic has entered is more than twice that of before generic entry (.0608 percentage points and .0283 percentage points respectively) but the difference fails to be significant. Deviation from the base probability due to an off label visit increases at the 90\% confidence interval from -3.29\% to -5.12\%. The postive coefficients for the interactions between \textbf{timeSinceGeneric} and \textbf{offLabel} are greater than the negative coefficient attached to \textbf{timeSinceGeneric} in absolute value which indicates that the probability of prescription for offLabel visits was increasing with time. Significant positive coefficents on \textbf{age} and significant negative coefficients on \textbf{ageSQ} indicate that a patient's probability of prescription is increasing with age but this increase decreases as the patient ages. The interactions between \textbf{age} and \textbf{timeSinceGeneric} as well as \textbf{ageSQ} and \textbf{timeSinceGeneric} show how this age effect changes over the course of time. We see that, before entry of the generic, there was a small but significant decrease in the probability of prescription due to a patient's age over time which decreased further as a patient increased in age. This effect loses significance once the generic enters the market.\\
\indent Both patients on Medicare or Medicaid and patients of a race other than white did not have a probability significantly different from the base case for on label visits before generic entry. After generic entry, however, patients on governement insurance became .182 percentage points more likely to be prescribed sulfamethozaxole-trimethoprim than those not on Medicare or Medicaid. Simlarly, patients of a race other than white became .246 percentage points more likely to be prescribed sulfamethozaxole-trimethoprim than their white counterparts. Both of these estimates are significant at the 99\% confidence interval. 
\def\sym#1{\ifmmode^{#1}\else\(^{#1}\)\fi}
\begin{tabular}{l*{3}{c}}
\hline\hline
Variable            &\multicolumn{1}{c}{Before Generic Entry}&\multicolumn{1}{c}{After Generic Entry}&\multicolumn{1}{c}{Difference}\\
\hline
\textbf{timeSinceGeneric}&                           -0.00032\sym{***}&   -0.000592\sym{***}&   -.000272\\
(Time in months since generic entry            &     (-6.51)         &     (-6.72)         &     [0.368]         \\
of Sulfamethoxazole-Trimethoprim)\\
[.5em]
\textbf{offLabel}    &                                     -0.0326\sym{***}&     -0.0513\sym{***}&   -.0188\\
(=1 if no diagnoses made were FDA approved            &    (-14.65)         &    (-20.05)         &    [0.0676]         \\
indications of Sulfamethoxazole-Trimethoprim)\\
[.5em]
\textbf{offLabel}$\times$\textbf{timeSinceGeneric} &    0.000372\sym{***}&    0.00047\sym{***}&   .0000984\\
            &                                           (7.7)         &      (5.64)         &    [0.757]         \\
[.5em]
\textbf{age}         &                      0.000172\sym{*}  &    0.0000696\sym{*}  &   -.000103\\
(Age of patient in years)            &      (3.08)         &      (1.06)         &    [0.378]         \\
[.5em]
\textbf{ageSQ}       &                             -0.00000215\sym{**} & -0.00000128\sym{*}  &   0.000000863\\
(Age of patient in years squared)            &     (-3.36)         &     (-1.74)         &    [0.495]         \\
[.5em]
\textbf{age}$\times$\textbf{timeSinceGeneric}      & -0.00000229\sym{**} &  0.00000419         &   0.000000648\\
            &                                        (-1.88)         &      (1.88)         &    [0.107]         \\
[.5em]
\textbf{ageSQ}$\times$\textbf{timeSinceGeneric}    &    .0000000274\sym{**} &   -0.000000031         &   -0.000000059\\
            &                                           (1.98)         &     (-1.23)         &    [0.184]         \\
[.5em]
\textbf{govInsurance}&                                    0.000176         &     0.00185\sym{**} &   0.00168\\
(=1 if patient is on either Medicare or            &      (0.38)         &      (3.54)         &    [0.161]         \\
Medicaid)\\
[.5em]
\textbf{nonwhite}    &                                    -0.00000675         &     0.00219\sym{***}&   0.00222\\
(=1 if patient is race other than white)            &     (-0.01)         &      (3.65)         &    [0.164]         \\
[.5em]
\textbf{\_cons}      &      0.0383\sym{***}&      0.0586\sym{***}&   0.0203\sym{*}\\
            &               (16.83)         &     (21.83)         &    [0.0403]         \\
\hline
r2          &     0.009         &     0.0058\\
N           &      230182         &      169063\\
\hline\hline
\multicolumn{3}{l}{\footnotesize \textit{t} statistics in parentheses, \scalebox{1.25}{$\text{Pr}(\frac{\hat{\beta}^\text{before}_i - \hat{\beta}^\text{after}_i}{[\hat{\sigma}^2\{\hat{\beta}^\text{before}_i\} + \hat{\sigma}^2\{\hat{\beta}^\text{after}_i\}]^\frac{1}{2}} > X^2)$} in brackets}\\
\multicolumn{3}{l}{\footnotesize \sym{*} \(p<0.05\), \sym{**} \(p<0.01\), \sym{***} \(p<0.001\)}\\
\multicolumn{4}{l}{\footnotesize "Probability of prescription" refers to probability that medical visit will have a prescription of Sulfamethoxazole-Trimethoprim}
\end{tabular}

\newpage
I use a similar estimation technique to look only at patients who were diagnosed with an unspecified skin abscess or cellulitis. Using only these observations, I preform a regression of the following form
\begin{equation}
\begin{split}
    \text{Pr}(prescription)^t & = \gamma^t_0 + \gamma^t_1\cdot(timeSinceGeneric)+ \gamma_2^t\cdot(age) + \gamma_3^t\cdot(ageSQ)\\
    &  + \gamma_4^t\cdot(age\times timeSinceGeneric)+ \gamma_5^t\cdot(ageSQ\times timeSinceGeneric)\\
    & + \gamma_6^t\cdot(govInsurance) + \gamma_7^t\cdot(nonWhite)
\end{split}
\end{equation}
where $t$ represents the timeframe of either before or after generic entry. The same test for coefficient equality is used as before. The results of this regression are shown in \autoref{tab:Table5.2}. The offLabel distinction was removed because an insignifcant number of observations (15) had an on label diagnosis after restricting the sample.
\indent It can be seen from \autoref{tab:Table5.2} that the coefficients in this context behave differently than in the previous regressions. There are fewer significant variables and the base probabilities are much greater. The base probability of prescription is 22.8\% before generic entry and 41.3\% after but this 18.5\% jump is not statistically significant. The coefficient for \textbf{timeSinceGeneric} is significant and negative after generic entry and the interactions between the age related variables and \textbf{timeSinceGeneric} indicate that, after the generic enters the market, the effect of probability to due to increases in age increases as more time passes since generic entry but this change decreases with age. Finally, patients of a race other than white are 9.11 percentage points more likely to be prescribed sulfamethozaxole-trimethoprim in this context than white patients after the generic enters the market at the 95\% confidence interval. This is an increase from an insignificant .0072 percentage points before generic entry.

\begin{table}[htbp]\centering
\def\sym#1{\ifmmode^{#1}\else\(^{#1}\)\fi}
\caption{Estimated Effects on Probability of Prescription Before and After Generic Entry for Patients with Uncomplicated Skin Abscess or Cellulitis\label{tab1}}
\begin{tabular}{l*{3}{c}}
\hline\hline
            &\multicolumn{1}{c}{Before Generic Entry}&\multicolumn{1}{c}{After Generic Entry}&\multicolumn{1}{c}{Difference}\\
\hline
\textbf{timeSinceGeneric}&     0.00181         &     -0.0107\sym{**} & -.0125\\
(Time in months since generic entry            &      (1.47)         &     (-3.12) & [.0052]\\
of Sulfamethoxazole-Trimethoprim)\\
[1em]
\textbf{age}         &     0.00189         &    -0.00582 & -.0077\\
(Age of patient in years)            &      (0.63)         &     (-1.50) & [.212]\\
[1em]
\textbf{ageSQ}       &  -0.0000445         &   0.0000238 & .0000682\\
(Age of patient in years squared)            &     (-1.33)         &      (0.57) & [.267]\\
[1em]
\textbf{age}$\times$\textbf{timeSinceGeneric}      &  -0.0000630         &    0.000379\sym{*}& .000442 \\
            &     (-1.00)         &      (2.48)         & [.0274]\\
[1em]
\textbf{ageSQ}$\times$\textbf{timeSinceGeneric}    & 0.000000653         & -0.00000324\sym{*}  & -.0000039\\
            &      (0.91)         &     (-2.08)         & [.0449]\\
[1em]
\textbf{govInsurance}&      0.0156         &      0.0256         & .01\\
(=1 if patient is on either Medicare or            &      (0.65)         &      (0.77)         & [.896]\\
Medicaid)\\
[1em]
\textbf{nonwhite}    &   0.0000720         &      0.0911\sym{*}  & .091\\
(=1 if patient is race other than white)            &      (0.00)         &      (2.46)         & [.364]\\
[1em]
\textbf{\_cons}      &       0.228\sym{***}&       0.413\sym{***} & .185\\
            &      (3.72)         &      (4.90)         & [.207] \\
\hline
r2          &      0.0268         &      0.0439         & (.)\\
N           &        1359         &         779         & (.)\\
\hline\hline
\multicolumn{3}{l}{\footnotesize \textit{t} statistics in parentheses, \scalebox{1.25}{$\text{Pr}(\frac{\hat{\beta}^\text{before}_i - \hat{\beta}^\text{after}_i}{[\hat{\sigma}^2\{\hat{\beta}^\text{before}_i\} + \hat{\sigma}^2\{\hat{\beta}^\text{after}_i\}]^\frac{1}{2}} > X^2)$} in brackets}\\
\multicolumn{3}{l}{\footnotesize \sym{*} \(p<0.05\), \sym{**} \(p<0.01\), \sym{***} \(p<0.001\)}\\
\end{tabular}
\end{table}

\newpage
Because the above tables provide the changes only in the context of changes to a base case of a white, non government insured patient with an on label diagnosis, the estimates need to be refined in order to provide an expected change in probability for each group. To do so, I use expected value of the categorical variables and their coefficents as shown below
\begin{eqnarray}
    \EX[\hat{\beta}_4^t\cdot(age)] &=& \hat{\beta}_4^t\EX[age]\\
    \EX[\hat{\beta}_5^t\cdot(ageSQ)] &=& \hat{\beta}_5^t\EX[ageSQ]\\
    \EX[\hat{\beta}_8^t\cdot(govInsurance)] &=& \hat{\beta}_8^t\EX[govInsurance]\\
    \EX[\hat{\beta}_9^t\cdot(nonWhite)] &=& \hat{\beta}_9^t\EX[nonWhite]\\
    \EX[\hat{\gamma}_2^t\cdot(age)] &=& \hat{\beta}_2^t\EX[age]\\
    \EX[\hat{\gamma}_3^t\cdot(ageSQ)] &=& \hat{\beta}_3^t\EX[ageSQ]\\
    \EX[\hat{\gamma}_6^t\cdot(govInsurance)] &=& \hat{\gamma}_6^t\EX[govInsurance]\\
    \EX[\hat{\gamma}_7^t\cdot(nonWhite)] &=& \hat{\gamma}_7^t\EX[nonWhite]
\end{eqnarray}
where $t$ represents the timeframe of either before or after generic entry and equations 5.4 through 5.7 give the expected value the effect on probability of prescription from a patient's age and squared age, a patient being on government insurance, and being a race other than white respectively. Equations 5.8 and 5.11 give the same estimates but restricts the sample to those visits where a diagnosis of an unspecified skin abscess or cellulitis was made. I use the expected values of these variables as a means to hold them constant while allowing for other variables in the regression to change. The resulting estimates can be interpretted as the expected probability an individual is prescribed sulfamethozaxole-trimethoprim. The results are shown in \autoref{tab:Table5.3} along with the results of the same hypothesis test used previously.\\
\indent By presenting the data using these expected values, a much clearer picture is presented of how each probability of prescription changes due to the introduction of the generic. The new probability of an on label visit leading to a prescription of sulfamethozaxole-trimethoprim increases by 1.88 percentage points from 4.06\% to 5.94\% and is significant at the 90\% confidence interval. The change in probability for off label visits is a much lower .0627\% and is indistinguishable from zero. On label visits with patients who on Medicare or Medicaid have an increase in probability by 2\% from 4.07\% to 6.07\% and the change is significant at the 90\% confidence interval. Most significant is the probability increase of 2.1 percentage points for on label visits with a patient of a race other than white. This increase from 4.05\% to 6.14\% is significant at the 95\% confidence interval.\\
\indent Although \autoref{tab:Table5.2} had a constant change of over 20 percentage points, it can be seen that the expected change in probability due to generic entry for patients diagnosed with an unspecified skin abscess or cellulitis is only 1.8 percentage points and not significant. This lack of change is driven primarily by the large negative value of $\hat{\beta}_2^t\EX[age] = -.00582\EX[age] = -.292$ percentage points.
\begin{landscape}
\begin{table}[htbp]\centering
\def\sym#1{\ifmmode^{#1}\else\(^{#1}\)\fi}
\caption{Estimated Probability of Prescription of Sulfamethoxazole-Trimethoprim Immediately Before and After Generic Entry \label{tab1}}
\begin{tabular}{l*{3}{c}}
\hline\hline
Patient Group  &\multicolumn{1}{c}{(Before Generic Entry)}&\multicolumn{1}{c}{(After Generic Entry)}&\multicolumn{1}{c}{Difference}\\
\hline
\textbf{onLabel}                                               &   0.0405\sym{***}   &   0.0594\sym{***}   &   .019\\
(=1 if at least one diagnosis made during visit is an          &     [0.000]         &     [0.000]         &     [0.0736]       \\
FDA approved indication of Sulfamethoxazole-Trimethoprim)\\
[1em]
\textbf{offLabel}                                     &     0.00756\sym{***}&     .00824\sym{***}  &   .000679\\
(=1 if no diagnoses made were FDA approved            &    [0.000]          &    [.00195]         &    [0.389]         \\
indications of Sulfamethoxazole-Trimethoprim)\\
[1em]
\textbf{govInsurance}$\times$\textbf{onLabel}&    0.0405\sym{***}         &     0.0607\sym{***} &   0.0202\\
(Patient on Medicare or Medicaid        &      [0.000]         &      [0.000]        &    [0.0600]         \\
and \textbf{onLabel}=1)\\
[1em]
\textbf{nonwhite}$\times$\textbf{onLabel}    &  0.0405\sym{***}         &     0.0614\sym{***}&   0.0209\sym{*}\\
(Patient race other than white            &     [0.000]        &      [0.000]         &    [0.0484]         \\
and \textbf{onLabel}=1)\\
[1em]
\textbf{UnspecCellAbscess}                   &  0.228\sym{***}         &     0.435&   0.207\\
(=1 if patient was diagnosed with an unspecified            &     [0.004]        &      [0.000]         &    [0.162]         \\
 skin abscess or cellulitis)\\
\hline\hline
\multicolumn{3}{l}{\footnotesize \scalebox{1.25}{$\text{Pr}(\frac{\hat{\beta}^\text{before}_i - \hat{\beta}^\text{after}_i}{[\hat{\sigma}^2\{\hat{\beta}^\text{before}_i\} + \hat{\sigma}^2\{\hat{\beta}^\text{after}_i\}]^\frac{1}{2}} > X^2)$} in brackets}\\
\multicolumn{3}{l}{\footnotesize \sym{*} \(p<0.05\), \sym{**} \(p<0.01\), \sym{***} \(p<0.001\)}\\
\end{tabular}
\end{table}

\end{landscape}
