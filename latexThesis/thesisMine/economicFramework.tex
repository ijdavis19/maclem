\chapter{Economic Framework}
\section{Theoretical Model}
To illustrate how entry of generics may lead to an increase in demand for antibiotics, I first posit the following model. The model assumes that, with assistance from their physician, an individual patient, seeks to maximize their utility from the treatment of their specific medical condition. To do so, a patient considers both the cost of a treatment option and how effective this treatment will be. Weighing these two characteristics, a patient will choose treatment with a given antibiotic if it is the utility maximizing treatment option. A patient may be willing to trade a higher probability of treatment success (effectiveness) for a less expensive treatment. Conversely, a patient may be willing to pay more for treatment if it shortens convalescence or has a higher probability of effectiveness. I assume, holding effectiveness constant, a patient will choose the lower priced treatment option. This logic indicates that a reduction in price caused by entry of manufacturers of generic antibiotics would increase demand for treatments in which the active ingredient of said antibiotic is used.

\indent More formally, a patient's utility $(U)$ decreases with the cost of the treatment $(C)$ and increases with the treatment's efficacy $(E)$. Hence, a patient's utility can be expressed as $U(C,E)$ where $U_C < 0$ and $U_E > 0$.\\
\indent Simply put, the patient can either demand a treatment that leverages the active ingredient in antibiotic $\alpha$ or not. The case of choosing a treatment which does not use $\alpha$ will be denoted by the $0$ superscript. The patient will demand a treatment that uses the active ingredient of antibiotic $\alpha$ if the expected utility of a treatment with antibiotic $\alpha$, $\EX[U(C^\alpha,E^\alpha)]$, is greater than the expected value of the utility maximizing treatment for the patient's condition that does not include the active ingredient of antibiotic $\alpha$, $\text{max}(\EX[U(C^0,E^0)])$.\\
Extending this notation, define the patient's decision, $D$, to be $D=1$ if the patient demands a treatment that uses the active ingredient in antibiotic $\alpha$ and $D=0$ otherwise. Hence, the entire decision can be described symbolically as
\begin{eqnarray}
  D =
  \begin{cases}
    1, & \text{if }\EX[U(C^\alpha,E^\alpha)] > \text{max}(\EX[U(C^0,E^0)]) \\
    0, & \text{if }\text{max}(\EX[U(C^0,E^0)]) > \EX[U(C^\alpha,E^\alpha)] \\
  \end{cases}
\end{eqnarray}
\indent Theory says that the entry of generic antibiotic manufacturers would increase the supply of antibiotic $\alpha$ and lower the price of treatments using its active ingredient. Holding efficacy constant, we can anticipate expected utility of treatment to increase in response to a decrease in prices caused by generic entry. At the margin, this increase in expected utility causes consumers to substitute treatments that do not use the active ingredient of antibiotic $\alpha$ with treatments that do, raising the number of total prescriptions of antibiotic $\alpha$.
\section{Econometric Model and Estimation Procedures}
\indent It is important to distinguish that, although these expected utilities are known to the patient, they cannot be observed by a researcher. Instead, the binary decision must be transformed into a probabilistic one \cite{train_discrete_nodate, templeton_household_2008}. To do so, define the transformed expected utility for patient $i$ of a given treatment, $\EX[\tilde{U}(C^s,E^s)]$ where $s = \{\alpha,0\}$ into two parts
\begin{eqnarray}
\EX[\tilde{U}(C^s,E^s)] = \tilde{U}_i^s + \tilde{\nu}_i^s
\end{eqnarray}
where $\tilde{U}_i^s$ is the observable portion of the patient's utility from treatment $s$ and $\tilde{\nu}_i^s$ is the unobservable portion. $\tilde{U}_i^s$ is a function of characteristics of the treatment, the visit, and the patient. The function is also conditioned on the ailment which the patient is seeking to treat. The variable $\tilde{\nu}_i^s$ is an independently and identically distributed random variable. This makes the decision to demand treatment with antibiotic $\alpha$ from the perspective of the researcher, to be
\begin{eqnarray}
P_i^\alpha =\text{Pr}(D = 1)_i = \text{Pr}(\tilde{U}^\alpha_i + \tilde{\nu}^\alpha_i > \text{max}(\tilde{U}^0_i + \tilde{\nu}^0_i))
\end{eqnarray}
which is the probability a patient $i$ chooses a treatment that utilizes antibiotic $\alpha$.\\
\indent To estimate this equation, I use a binary linear probability model. For the patient characteristics, I consider whether the patient is on medicare or medicaid (\textbf{govInsurance}=1), the time in months since the entry of the generic (\textbf{timeSinceGeneric}), whether the patient has no diagnoses associated with FDA approved reasons for using sulfamethozaxole-trimethoprim (\textbf{offLabel}), and the age and squared age of the patient. Interactions between the off-label distinction and the time since generic entry as well as interaction between the age and age squared variables with time since generic entry are also included. With these variables, the equation above becomes
\begin{equation}
\begin{split}
    P_i^\alpha =\text{Pr}(D = 1)_i & = \beta^t_0 + \beta^t_1\cdot(timeSinceGeneric_i) + \beta_2^t\cdot(offLabel_i) + \beta_3^t\cdot(offLabel_i\times timeSinceGeneric_i)\\
    & + \beta_4^t\cdot(age_i) + \beta_5^t\cdot(ageSQ_i)  + \beta_6^t\cdot(age_i\times timeSinceGeneric_i)\\
    & + \beta_7^t\cdot(ageSQ_i\times timeSinceGeneric_i)  + \beta_8^t\cdot(govInsurance_i)\\
    & + \beta_9^t\cdot(nonWhite_i) + \varepsilon_i
\end{split}
\end{equation}\
where $t$ represents if the treatment is occurring before or after entry of the generic and the variable $\varepsilon_i$ is the normally distributed error term.\\
\indent To test the significance of the differences among coefficients for the two possible treatment times, I use a cross model hypothesis test which rejects the null hypothesis that $\beta_i^{before} = \beta_i^{after}$ if 
\begin{equation}
\text{Pr}(\frac{\hat{\beta}^\text{before}_i - \hat{\beta}^\text{after}_i}{[\hat{\sigma}^2\{\hat{\beta}^\text{before}_i\} + \hat{\sigma}^2\{\hat{\beta}^\text{after}_i\}]^\frac{1}{2}})
\end{equation}
exceeds the chi sqaured threshold.\\