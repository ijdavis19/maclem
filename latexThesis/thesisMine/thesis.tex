%
% thesis.tex
%
% Master's Thesis/Ph.D. Dissertation Template
% Clemson University
%

%
% The document guidelines say the font can be between 10pt and 12pt.
% Specify whatever you want it to be here.
%
\documentclass[10pt]{ClemsonThesis}

%
% Use any additional packages you might need
%
%% \usepackage{listings}
%% \usepackage{comment}
\usepackage{amsmath,amssymb}
\usepackage{mathtools}
\usepackage{lscape}
\DeclareRobustCommand{\bbone}{\text{\usefont{U}{bbold}{m}{n}1}}

\DeclareMathOperator{\EX}{\mathbb{E}}% expected value
%
% Make the document your own -- fill in these values to reflect the type of
% document you are writing.
%
\title{The Effect of Entry of Generic Antibiotics on Prescriptions of Antiboitics in the Ambulatory Care Settings The Case Study of Sulfamethozaxole-Trimethoprim}
\department{John E. Walker Department of Economics}
\documentType{Dissertation}
\major{Economics}
\degree{Masters of Art}
\graduationMonth{August}
\graduationYear{2020}
\author{Ian James Davis}
\committeeChair{Dr. Scott Templeton}
\committeeMemberOne{Reed Watson}
\committeeMemberTwo{Dr. Michael Makowsky}
%% optional (for Master's) \committeeMemberThree{Dr. John Doe}
%% optional \committeeMemberFour{Dr. Jane Doe}
%% optional \committeeMemberFive{Dr. Mary Doe}
%% optional \committeeMemberSix{Dr. Mark Doe}

%
% PDF Setup -- most of this you do not need to touch
%
\hypersetup{
    colorlinks,
    linkcolor={black},
    citecolor={black},
    filecolor={black},
    urlcolor={black},
    pdftitle={\theTitle},
    pdfauthor={\theAuthor},
    pdfsubject={\theDocumentType},
    pdfkeywords={Clemson University, \theDepartment, \theDocumentType, \theMajor, \theDegree},
    pdfstartpage={1},
}


%
% User-specified command definitions/redefinitions
%
%% \newcommand{\cplusplus}{{\rm C\raise.5ex\hbox{\small ++}}}
%% \newcommand{\num}[1]{\mbox{(\textit{#1})}}
%% \renewcommand{\ttdefault}{pcr}
%% \renewcommand\lstlistlistingname{List of Listings}


\begin{document}
%  ============================================================================
    \frontmatter % Begin front matter (pages are numbered with Roman numerals)
%  ============================================================================

    \addtotoc{Title Page}{\maketitle}          % Generate the title page
    \doublespacing                             % Text should be double spaced
    \setcounter{page}{2}                       % Abstract begins on page 2
    \addtotoc{Abstract}{\chapter*{Abstract}
Due to the increased emphasis on antibiotic stewardship, careful attention must be paid to the study of antibiotic demand and factors which affect it. The purpose of this paper is the estimate the effect of generic entry on antibiotic demand through changes in the probability of prescription using Sulfamethoxazole-Trimethoprim as a case study. I pool data of outpatient ambulatory care visits from the National Ambulatory Medical Care Surveys from 2006-2016 to estimate probability of prescription of Sulfamethoxazole-Trimethoprim for various groups of patients both before and after generic entry. I estimate and increase of 1.88 percentage points (CI 90\%) in the probability of prescription due to generic entry for patients with FDA approved reasons for prescription of Sulfamethoxazole-Trimethoprim. Visits where patients were not diagnosed with FDA approved reasons for prescription of Sulfamethoxazole-Trimethoprim saw no significant change in probability of prescription.}  % Generate the abstract

    %
    % The dedication page is optional.  Comment out this line if you do not
    % want to include this page.
    %
    %\addtotoc{Dedication}{\chapter*{Dedication}
Lorem ipsum dolor sit amet, consetetur sadipscing elitr,  sed diam nonumy eirmod
tempor invidunt ut labore et dolore magna aliquyam erat, sed diam voluptua. At
vero eos et accusam et justo duo dolores et ea rebum. Stet clita kasd gubergren,
no sea takimata sanctus est Lorem ipsum dolor sit amet. Lorem ipsum dolor sit
amet, consetetur sadipscing elitr,  sed diam nonumy eirmod tempor invidunt ut
labore et dolore magna aliquyam erat, sed diam voluptua. At vero eos et accusam
et justo duo dolores et ea rebum. Stet clita kasd gubergren, no sea takimata
sanctus est Lorem ipsum dolor sit amet. Lorem ipsum dolor sit amet, consetetur
sadipscing elitr,  sed diam nonumy eirmod tempor invidunt ut labore et dolore
magna aliquyam erat, sed diam voluptua. At vero eos et accusam et justo duo
dolores et ea rebum. Stet clita kasd gubergren, no sea takimata sanctus est
Lorem ipsum dolor sit amet.
}

    %
    % The acknowledgment page is optional.  Comment out this line if you do
    % not want to include this page.
    %
    %\addtotoc{Acknowledgments}{\chapter*{Acknowledgments}
Lorem ipsum dolor sit amet, consetetur sadipscing elitr,  sed diam nonumy eirmod
tempor invidunt ut labore et dolore magna aliquyam erat, sed diam voluptua. At
vero eos et accusam et justo duo dolores et ea rebum. Stet clita kasd gubergren,
no sea takimata sanctus est Lorem ipsum dolor sit amet. Lorem ipsum dolor sit
amet, consetetur sadipscing elitr,  sed diam nonumy eirmod tempor invidunt ut
labore et dolore magna aliquyam erat, sed diam voluptua. At vero eos et accusam
et justo duo dolores et ea rebum. Stet clita kasd gubergren, no sea takimata
sanctus est Lorem ipsum dolor sit amet. Lorem ipsum dolor sit amet, consetetur
sadipscing elitr,  sed diam nonumy eirmod tempor invidunt ut labore et dolore
magna aliquyam erat, sed diam voluptua. At vero eos et accusam et justo duo
dolores et ea rebum. Stet clita kasd gubergren, no sea takimata sanctus est
Lorem ipsum dolor sit amet.

Duis autem vel eum iriure dolor in hendrerit in vulputate velit esse molestie
consequat, vel illum dolore eu feugiat nulla facilisis at vero eros et accumsan
et iusto odio dignissim qui blandit praesent luptatum zzril delenit augue duis
dolore te feugait nulla facilisi. Lorem ipsum dolor sit amet, consectetuer
adipiscing elit, sed diam nonummy nibh euismod tincidunt ut laoreet dolore
magna aliquam erat volutpat.

Ut wisi enim ad minim veniam, quis nostrud exerci tation ullamcorper suscipit
lobortis nisl ut aliquip ex ea commodo consequat. Duis autem vel eum iriure
dolor in hendrerit in vulputate velit esse molestie consequat, vel illum dolore
eu feugiat nulla facilisis at vero eros et accumsan et iusto odio dignissim qui
blandit praesent luptatum zzril delenit augue duis dolore te feugait nulla
facilisi.

Nam liber tempor cum soluta nobis eleifend option congue nihil imperdiet doming
id quod mazim placerat facer possim assum. Lorem ipsum dolor sit amet,
consectetuer adipiscing elit, sed diam nonummy nibh euismod tincidunt ut laoreet
dolore magna aliquam erat volutpat. Ut wisi enim ad minim veniam, quis nostrud
exerci tation ullamcorper suscipit lobortis nisl ut aliquip ex ea commodo
consequat.

Duis autem vel eum iriure dolor in hendrerit in vulputate velit esse molestie
consequat, vel illum dolore eu feugiat nulla facilisis.

At vero eos et accusam et justo duo dolores et ea rebum. Stet clita kasd
gubergren, no sea takimata sanctus est Lorem ipsum dolor sit amet. Lorem ipsum
dolor sit amet, consetetur sadipscing elitr,  sed diam nonumy eirmod tempor
invidunt ut labore et dolore magna aliquyam erat, sed diam voluptua. At vero
eos et accusam et justo duo dolores et ea rebum. Stet clita kasd gubergren, no
sea takimata sanctus est Lorem ipsum dolor sit amet. Lorem ipsum dolor sit
amet, consetetur sadipscing elitr,  At accusam aliquyam diam diam dolore
dolores duo eirmod eos erat, et nonumy sed tempor et et invidunt justo labore
Stet clita ea et gubergren, kasd magna no rebum. sanctus sea sed takimata ut
vero voluptua. est Lorem ipsum dolor sit amet. Lorem ipsum dolor sit amet,
consetetur sadipscing elitr,  sed diam nonumy eirmod tempor invidunt ut labore
et dolore magna aliquyam erat. 
}

    \singlespacing                             % Single space the lists
    \tableofcontents \clearpage                % Generate the Table of Contents

    %
    % REMEMBER: Review your caption listings in the genrated lists
    %           and make sure they include '\newline' commands as necessary.
    %           See the README for further information.
    %
    \addtotoc{List of Tables}{\listoftables}   % Generate the List of Tables
    \addtotoc{List of Figures}{\listoffigures} % Generate the List of Figures

    %
    % Include other optional lists.  Computer science, for example, would
    % likely include a 'List of Listings' (and would \usepackage{listings}
    % and \renewcommand\lstlistlistingname{List of Listings}).
    %
    %% \addtotoc{List of Listings}{\lstlistoflistings}



%  ===========================================================================
    \mainmatter % Begin main matter (pages are numbered with Arabic numerals)
%  ===========================================================================
    \doublespacing % Text should be double spaced

    %
    % Here we have each chapter in a separate file.  Name these as you choose,
    % and include them in the order you want them to appear.  Be sure to use
    % the \inputfile command.
    %
    \inputfile{introduction.tex}
    \inputfile{background.tex}
    \inputfile{economicFramework.tex}
    \inputfile{data.tex}
    \inputfile{analysis.tex}
    \inputfile{discussion.tex}
    \inputfile{conclusions.tex}

    %
    % The appendices are optional.  This is the format for two or more.
    % If you do not wish to include an appendix, comment out these lines.
    % If you want just one, see the formatting guidelines.
    %
    %\begin{appendices}
    %    \begin{subappendices}
    %        \inputfile{appendixA.tex}
    %        \inputfile{appendixB.tex}
    %        \inputfile{appendixC.tex}
    %    \end{subappendices}
    %\end{appendices}



    \singlespacing                             % Single space the Bibliography

    %
    % The bibliography style.  Set this to whatever matches you discipline.
    % For example, Computer Science would likely use 'plain'.  You might
    % also want to change the name from 'Bibliography' to 'References'
    % or 'Work Cited'.
    %
    % 'plain'   gets you numbered references and citations (e.g., [1] Dyson).
    %
    % 'alpha'   gets you labels formed from an abbreviation of the authors'
    %           names and the year of publication.  If there is more than
    %           one author, it will use the first letter of up to the first
    %           three authors' last names.
    %
    %           Some examples:
    %               [DED01] F.W. Dyson, A.G. Edgar, and D.B. Denny ... 2001
    %               [DE01] F.W. Dyson, A.G. Edgar ... 2001
    %               [Dys01] F.W. Dyson ... 2001
    %
    % 'apalike' gets you labels formed from the authors' names and year of
    %           publication.
    %
    %           Some examples:
    %               [Dyson et al., 2001] F.W. Dyson, A.G. Edgar, and
    %                 D.B. Denny ... 2001
    %               [Dyson and Edgar, 2001] F.W. Dyson, A.G. Edgar ... 2001
    %               [Dyson, 2001] F.W. Dyson ... 2001
    %
    \bibliographystyle{amsplain}
    \addtotoc{Bibliography}{\bibliography{bibliography}}
\end{document}
