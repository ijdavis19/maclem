\chapter{Data and Variables}

\section{Sulfamethoxazole-Trimethoprim}
\indent For simplification, we focus our analysis to prescriptions of Sulfamethoxazole/Trimethoprim. Sulfamethoxazole/Trimethoprim is a combination antibiotic from the class antimetabolite/sulfonamide and was first introduced in 1968. The brand versions of the drug include Bactrim, Bactrim DS, Septra, and Septra DS and the generic form entered the American market in July of 2012. The antibiotic was popular even before generic entry due its high familiarity among physicians and low cost \cite{noauthor_sulfamethoxazole_nodate,ho_considerations_2011}. The drug has FDA approval to fight urinary tract infections, ear infections (acute otitis media), acute exacerbations of chronic bronchitis in adults, Shigellosis, treatment and prophylaxis of \textit{Pneumocystis jirovecii} pneumonia, and Traveler's diarrhea in a adults. The antibiotic is also used for infections due to \textit{Listeria, Nocardia, Salmonella, Brucella, Paracoccidioides,} melioidosis, \textit{Burkholderia, Stenotrophomonas,} cyclospora, isospora, Whipple's disease, and alterntive therapy for toxoplasmosis and community-acquired MRSA skin infections \cite{schlossberg_antibiotics_2017}. While not specifically named by the FDA, evidence has been found which indicates Sulfamethoxazole-Trimethoprim is successful in treating uncomplicated skin abcesses \cite{noauthor_trimethoprimsulfamethoxazole_nodate}\\

\section{National Ambulatory Medical Care Survey}
Data used are from the National Ambulatory Medical Care Survey (NAMCS) which is a nationally representative survey of outpatient medical visits. Included in the scope of the survey are freestanding clinics/urgicenters, community health centers, mental health centers, health maintenance organizations,  non-federal government clinics, family practice plans, and private solo or group practices. Not included are hospital emergency or outpatient departments, ambulatory surgicenters, institutional settings such as schools or prisons, industrial outpatient facilities, clinics operated by the federal government, and laser vision surgery centers. \cite{hing_basic_nodate}. The surveys include patient, visit, and provider characterists and weights are provided in order to create national estimates.\\
\indent I pool results from the years 2006 to 2016 and drop variables which are not consistently tracked across this time frame or have more than 30\% missing values as instructed in documentation \cite{myrick_understanding_nodate}. For the specific cases of diagnoses and prescriptions, the maximum amount of available entries increased during the study period. The 2006 NAMCS survey allowed for collection of three unique diagnoses and eight unique prescriptions. In 2012, the maximum number of drugs collected was raised to twelve and 2014 saw that number increased again to thirty. The maximum number of diagnoses recorded was raised from 3 to 5 in 2014 as well. In order to accurately measure trends across the study period, only diagnoses 1-3 and prescriptions 1-8 were considered as explicitely instructed in documentation \cite{schappert_analyzing_nodate}.\\
\indent In order to ensure analsysis is of relevant medical visits, I use daignoses which led to a prescription of ST to restrict the sample. For years 2006-2015 diagnoses are label using ICD-9-CM codes and used ICD-10-CM codes for the year 2016. To allow for comparability across all years of the study, each ICD-10-CM code was recoded as its exact or closest ICD-9-CM counterpart. I designate each diagnosis to be a relevant diagnosis if a visit which lead to the diagnosis also lead to a prescription of Sulfamethoxazole-Trimethoprim. All visits where none of these relevant diagnoses were made are dropped from the sample. Because all relevant diagnoses are given equal importance regardless of whether it was the specific one which led to the prescription, it is possible that this strategy does not fully rid the sample of all non relevant visits which would negatively bias our estimates.\\
\indent To further control for nonrelevant visits, we create an indicator for on label diagnoses. Reasons for prescribing antibiotics can be categorized as on label and off label. On label indications are the FDA approved reasons for precribing mentioned above while off label indications are non FDA approved indications. Each on label indication is mapped to one or more ICD-9-CM codes illustrated in \autoref{tab:Table4.1}. We consider a visit where one or on label diagnoses were made to be an on label visit and visits where no on label diagnoses were made to be off label visits. 
\begin{table}[htbp]\centering
\def\sym#1{\ifmmode^{#1}\else\(^{#1}\)\fi}
\caption{On Label Indications\label{tab1}}
\begin{tabular}{l*{2}{c}}
\hline\hline
            Indicator&\multicolumn{1}{c}{ICD-9-CM Code}&\multicolumn{1}{c}{ICD-9-CM Description}\\
\hline
Travelers Diarrhea    &     78791&       Diarhhea\\
[1em]
Urinary Tract Infection    &     5990&       Urinary tract infection, site unspecified\\
[1em]
Ear Infection    &     382&       Otitis media\\
[1em]
Chronic Bronchitis    &     491&       Chronic Brnchitis\\
[1em]
Shigellosis    &     004&       Shigellosis\\
[1em]
Pneumonia    &     480-488&       Pneumonia of Various Classifications\\
[1em]
Brucella    &     023&       Brucellosis\\
[1em]
Nocardia    &     039&       Actinomycotic infections\\
[1em]
Salmonella    &     003&       Other Salmonella Infections\\
[1em]
Paracoccidioides    &     1161&       Paracoccidiodomycosis\\
[1em]
Melioidoisis    &     025&       Melioidosis\\
[1em]
Burkholderia    &     2002&       Burckett's Tumors of Lymphatic Tissue\\
[1em]
Stenotrophomonas    &     n.a.&       n.a.\\
[1em]
Cyclospora    &     0075&       Cyclosporiasis\\
[1em]
Isospora    &     0072&       Coccidiosis\\
[1em]
Whipple's Disease    &     0402&       Whipple's Disease\\
[1em]
Toxoplasmosis    &     130&       Toxoplasmosis\\
[1em]
MRSA Related Skin Infection    &     0412&       Pneumococcus infection\\
\hline
idk\\
idk\\
\hline\hline
\end{tabular}
\label{tab:Table4.1}
\end{table}

\section{Variables}
The independent variables described in \autoref{tab:Table4.2} are as follows. We consider timeSinceGeneric which is a continuous variable from -82 to 49 indicating the number months since the generic version of Sulfamethoxazole-Trimethoprim entered the market (November of 2012). We include age and ageSQ which are the ages and squared ages of the patient at the time of the visit respectively. Lastly, we include the indicators offLabel, govInsurance, and nonWhite. OffLabel indicates that none of the diagnoses which resulted from the visit were FDA approved indications for Sulfamethoxazole-Trimethoprim. GovInsurance and nonWhite indicate whether a patient was Medicare or Medicaid and if the patient was a race other than white respectively.\\
\begin{landscape}
\begin{table}[htbp]\centering
\def\sym#1{\ifmmode^{#1}\else\(^{#1}\)\fi}
\caption{Summary of Independent Variables\label{tab1}}
\begin{tabular}{l*{6}{c}}
\hline\hline
            Continuous Variables&\multicolumn{1}{c}{Timeframe}&\multicolumn{1}{c}{Weighted Mean}&\multicolumn{1}{c}{Median}&\multicolumn{1}{c}{Standard Deviation}&\multicolumn{1}{c}{Minimum}&\multicolumn{1}{c}{Maximum}\\
\hline
timeSinceGeneric    &     Entire Study&             -18.316&    -20&   37.167&     -82&  49\\
                    &     Before Generic Entry&     -41.283&    -41&    23.33 &     -82&  -1\\
                    &     After Generic Entry&       23.54 &    24&      13.917&     0&  49\\
[1em]
age                 &     Entire Study&             45.917&    50&    25.09 &     0&  100\\
                    &     Before Generic Entry&     45.236&    49&    25.202&     0&  100\\
                    &     After Generic Entry&      47.159&    51&    24.836&     0&  92\\
[1em]
ageSQ               &     Entire Study&             2737.892&    2500&  2218.939&     0&  10000\\
                    &     Before Generic Entryt&    2681.421&    2401&  2217.38 &     0&  10000\\
                    &     After Generic Entry&      2840.828&    2601&  2218.09 &     0&  8464\\
\hline\hline
            Categorical Variables&\multicolumn{1}{c}{Timeframe}&\multicolumn{1}{c}{Total}&\multicolumn{1}{c}{Share of Sample}&\multicolumn{1}{c}{Proportion Prescribed ST}\\
\hline
offLabel                  &     Entire Study&             387264&      .178&     .00718\\
                          &     Before Generic Entryt&    241872&      .967&     .00696\\
                          &     After Generic Entry&      145392&      .968&     .0868\\
[1em]
govInsurance              &     Entire Study&             105273&      .26 &     .0929\\
                          &     Before Generic Entry&     65470 &      .249&     .0894\\
                          &     After Generic Entry&      39803 &      .281&     .0982\\
[1em]
nonWhite                  &     Entire Study&             61442&      .164&     .0961\\
                          &     Before Generic Entry&     40458&      .160&     .0916\\
                          &     After Generic Entry&      20984&      .173&     .103\\
[1em]
unspecCellAbscess         &     Entire Study&             2138 &      .054&     .175\\
                          &     Before Generic Entry&     1359 &      .055&     .170\\
                          &     After Generic Entry&      779  &      .052&     .184\\
\hline
$n(\text{Entire Study}) = 399245$\\
$n(\text{Before Generic Entry}) = 249345$\\
$n(\text{After Generic Entry}) = 149900$\\
\hline\hline
\multicolumn{7}{l}{"Share of Sample" and "Proportion Prescribed ST" are both weighted proportions}\\
\multicolumn{7}{l}{offLabel(=1) indicates no diagnoses made were on label indicators, govInsurance(=1) indicates patient is on Medicare or Medicaid, nonWhite(=1) indicates}\\
\multicolumn{7}{l}{patient is race other than white, UnspecCellAbscess(=1) indicates patient was diagnosed with and unspecified skin abscess or cellulitis}\\
\end{tabular}
\label{tab:Table4.2}
\end{table}

\begin{tabular}{l*{6}{c}}
\hline\hline
            Variable&\multicolumn{1}{c}{Time frame}&\multicolumn{1}{c}{Weighted Mean}&\multicolumn{1}{c}{Weighted Median}&\multicolumn{1}{c}{Standard Deviation}&\multicolumn{1}{c}{Minimum}&\multicolumn{1}{c}{Maximum}\\
\hline
\textbf{TimeSinceGeneric}                    &     2006-2026&             -15.316&    -17&   37.167&     -79&  52\\
(Time in months since entry &     Before Entry of Generic&     -39.703&    -40&    22.497 &     -98&  -1\\
 of generic)   &     After Entry of Generic&       25.013 &    25&      14.792&     0&  52\\
[1em]
\textbf{Age}                                 &     2006-2026&             45.917&    50&    25.09 &     0&  100\\
(Age of patient in years)           &     Before Entry of Generic&     45.221&    49&    25.207&     0&  100\\
                                    &     After Entry of Generic&      47.069&    51&    24.853&     0&  92\\
[1em]
\textbf{AgeSQ}                               &     2006-2026&             2737.892&    2500&  2218.939&     0&  10000\\
(Age of patient squared)   &     Before Entry of Generic&    2680.279&    2401&  2216.963 &     0&  10000\\
                                    &     After Entry of Generic&      2833.167&    2601&  2218.929 &     0&  8464\\
\hline
$\text{Sample Size for Years 2006-2016} = 399245$\\
$\text{Before Entry of Generic} = 230182$\\
$\text{After Entry of Generic} = 169063$\\
\hline\hline
\multicolumn{4}{l}{\footnotesize All observations after July 2012 are considered to be after entry of generic.}\\
\end{tabular}

\begin{tabular}{l*{4}{c}}
\hline\hline
            Variable&\multicolumn{1}{c}{Time frame}&\multicolumn{1}{c}{Total}&\multicolumn{1}{c}{Weighted Share of Sample}\\
\hline
\textbf{OffLabel}                                                &     Entire Study&             387264&      .967\\
(=1 if no diagnoses made were FDA approved          &     Before Entry of Generic&    223267&      .966\\
indications of sulfamethoxazole-trimethoprim)  &     After Entry of Generic&      163997&      .968\\
[1em]
\textbf{GovInsurance}                                            &     Entire Study&             105273&      .26 \\
(=1 if patient is on either Medicare or Medicaid)       &     Before Entry of Generic&     30480 &      .248\\
                                                        &     After Entry of Generic&      44793 &      .28\\
[1em]
\textbf{NonWhite}                                                &     Entire Study&             61442&      .164\\
(=1 if patient is a race other than white)                &     Before Entry of Generic&     37733&      .16\\
                                                        &     After Entry of Generic&      23709&      .171\\
\hline
$\text{Sample Size for Years 2006-2016} = 399245$\\
$\text{Before Entry of Generic} = 230182$\\
$\text{After Entry of Generic} = 169063$\\
\hline\hline
\multicolumn{4}{l}{\footnotesize All observations after August 2012 are considered to be after entry of generic.}\\
\end{tabular}

%multicolumn{5}{l}{"Share of Sample" and "Proportion Prescribed ST" are both weighted proportions}\\
%\multicolumn{5}{l}{offLabel(=1) indicates no diagnoses made were on label indicators, govInsurance(=1) indicates patient is on Medicare or Medicaid,}\\
%\multicolumn{5}{l}{nonWhite(=1) indicates patient is race other than white}\\
%\end{tabular}
%\label{tab:Table4.2}
%\end{table}

\end{landscape}
