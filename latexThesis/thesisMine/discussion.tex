\chapter{Discussion}
Based on fundamental economic theory, entry of the generic sulfamethoxazole-trimethoprim increases the supply of the antibiotic which decreases the price faced by the consumer. Because the patient is looking to maximize their utility, this decrease in the price should lead to an increase in the demand for sulfamethoxazole-trimethoprim which would be reflected in an increase in the probability a given medical visit will lead to a prescription of the antibiotic. I find this prediction is accurate only for certain groups of patients within the data and varies by patient and visit characteristics.\\
\section{Visits Associate with On-Label Uses}
\indent For patients with at least one diagnosis associated with an on-label use of sulfamethoxazole-trimethoprim, a patient is more likely to demand a treatment that uses sulfamethoxazole-trimethoprim by 1.88 percentage points and this change to be significant at the 90\% confidence interval. I attribute this increase in probability to the decrease in price caused by entry of the generic. This price effect is consistent with a negative own price elasticity found previously in antibiotics \cite{kaier_impact_2013}.\\
\indent When I fix the race of the patient to be other than white or fix the patient's insurance to be either Medicare or Medicaid, I estimate larger increases in probability. Patients on Medicare or Medicaid saw an increase in probability of prescription of 2 percentage points and this increase was also significant at the 90\% confidence interval. Additionally, the difference between the probability of a patient not on Medicare or Medicaid and a patient who is on either one to be indistinguishable from zero before the entrance of generic sulfamethoxazole-trimethoprim. Once the generic has entered the market, there is a small but significant increase in probability of prescription for patients on Medicare or Medicaid compared to patients who are not. Because these patient's choice sets are further restricted by their insurance policies than other patients, this increase could be an amplification of the price effect seen earlier.\\
\indent Fixing the race of the patient to a race other than white generates a similar change. Again, the probability a patient demands treatment with sulfamethoxazole-trimethoprim increased after the introduction of the generic. In this case, probability of prescription increased by 2.1 percentage points and is significant at the 95\% confidence interval. Additionally, there was originally no distinguishable difference between the probability of prescription for white and non white patients. After the generic enters the market, the increase in probability for non white patients was greater than white patients by a small but significant. I posit this difference, similar to that of the patients on Medicare or Medicaid, comes from an amplification of the price effect. These patients may be more reliant on generics and more likely to be benefited by the introduction of the generic as they have less treatment choices available.\\
\indent More generally, these increases are interesting because they occur during decreasing trends of prescription probability. This could be due to rising instances of resistant bacteria. High rates of sulfamethoxazole-trimethoprim resistant urinary tract infection were observed in Latin America as early as 1997 \cite{gales_urinary_2002}. Other studies find a significant increase in the prevalence of resistant staph infection dating back to before 2010 both abroad \cite{noauthor_resistance_nodate} and in the United States \cite{khamash_increasing_2019}. This trend of increased resistance was not uniform across all on-label indications however. The minimum inhibitory concentration, the minimum dosage required to prevent bacterial growth, of sulfamethoxazole-trimethoprim for strains of bacteria causing Whooping Cough saw no change from 1967-2015 \cite{jakubu_trends_2017}.
\section{Visits with Off-Label Diagnoses}
\indent Patients who did not have an ailment associated with the FDA approved uses for prescription of sulfamethoxazole-trimethoprim did not display behavior consistent with the model. My theoretical claims indicate that there should be an increase in probability of prescription of the antibiotic even though the prescriptions are not for FDA approved uses. I found that entry of the generic sulfamethoxazole-trimethoprim had no significant effect on the probability of prescription for off-label visits. This may be due to the cases which warrant the use of sulfamethoxazole-trimethoprim for off-label uses having less treatments which can serve as substitutes for treatments with the antibiotic. This would indicate there were less patients who were at the margin and would be affected by a price affect.\\
\indent Patients diagnosed with an unspecified skin abscess or cellulitis showed similar behavior to the other off-label visits. A 1.8 percentage point increase in probability of prescription was estimated due to entry of the generic but this estimate was not significant. Additionally, the coefficients associated with time and age both shift from positive to negative. Unlike the patients with ailments associated with on-label uses of the antibiotic, the probability a patient was going to demand treatment using the antibiotic was increasing up until the entry of the generic and then began to decrease after. Because this diagnosis had the largest share of prescriptions of sulfamethoxazole-trimethoprim, the implications of these estimates are important to consider. As supply of the antibiotic increases and the price faced by the consumer decreases, another variable could be changing which counteracts this price effect. In the specific context of skin infections, fear of creating more resistant strains of MRSA could have arisen. This would lead patients and doctors to become more cautious with their usage as other types of patients increase their usage. Additionally, an insignificant but negative correlation arose between patient age and probability of prescription for this diagnosis. This change may indicate the aforementioned preference among the elderly for newer, more expensive drugs \cite{kianmehr_system_2020} is stronger in this context than others.