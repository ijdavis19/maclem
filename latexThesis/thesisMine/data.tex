\chapter{Data and Variables}

\section{National Ambulatory Medical Care Survey}
Data used are from the National Ambulatory Medical Care Survey (NAMCS) which is a nationally representative survey of outpatient medical visits. Included in the scope of the survey are freestanding clinics/urgicenters, community health centers, mental health centers, health maintenance organizations,  non-federal government clinics, family practice plans, and private solo or group practices. Not included are hospital emergency or outpatient departments, ambulatory surgicenters, institutional settings such as schools or prisons, industrial outpatient facilities, clinics operated by the federal government, and laser vision surgery centers \cite{hing_basic_nodate}. The surveys include information about the patient, the visit itself, and the provider seen. Weights are provided in order to create national estimates.\\
\indent I pool observations from the years 2006 to 2016 and drop variables which are not consistently tracked across this time or have more than 30\% missing values as instructed in the survey documentation \cite{myrick_understanding_nodate}. For the specific cases of diagnoses and prescriptions, the maximum amount of available entries increased during the study period. The 2006 NAMCS survey provided three slots to record diagnoses and eight slots to record prescriptions. This set up means that even if more than three diagnoses were made during a medical visit, only three of them would be recorded as there was no option in the survey to add additional diagnoses. The same restriction applies in the case where more than eight medications were prescribed. In 2012, the maximum number of medications recorded was raised to twelve and rose again in 2014 to thirty. The maximum number of diagnoses recorded was raised from 3 to 5 in 2014 as well. In order to accurately measure trends across the study period, I only use the first three diagnoses and the first eight prescriptions as explicitly instructed in the survey documentation \cite{schappert_analyzing_nodate}.\\
\indent In order to restrict observations to only those which may have led to a prescription of sulfamethoxazole-trimethoprim, I track all diagnoses which occurred during visits where the antibiotic was prescribed. For the sake of analysis, these diagnoses are considered relevant diagnoses. Then, all visits where one of these relevant diagnoses are made are then marked as relevant visits. This characterization indicates that at least one of the diagnoses made during this visit could have led to the prescription of sulfamethoxazole-trimethoprim based on the behavior of other prescribing physicians. Hence, this visit is considered relevant because it could have led to a prescription of sulfamethoxazole-trimethoprim.\\
\indent Visits where no diagnosis made ever leads to a prescription of the antibiotic are dropped from the sample. For years 2006-2015 diagnoses are labeled using ICD-9-CM codes and ICD-10-CM codes are used for the year 2016. To allow for comparability across all years of the study, each ICD-10-CM code was recoded as its exact or closest ICD-9-CM counterpart. Because all relevant diagnoses are given equal importance regardless of whether it was the specific one which led to the prescription, it is possible that this strategy does not fully rid the sample of all non relevant visits which would negatively bias estimates.\\
\indent To further control for nonrelevant visits, I create an indicator for visits where at least one diagnosis is associated with an FDA approved use of sulfamethoxazole-trimethoprim. The reasons for prescribing antibiotics can be categorized as on-label and off-label. On-label uses of the antibiotic are the FDA approved reasons for prescribing sulfamethoxazole-trimethoprim mentioned previously while off-label indications are non FDA approved uses. Each on-label indication is mapped to one or more ICD-9-CM codes illustrated in \autoref{tab:Table4.1}.
\begin{table}[htbp]\centering
\def\sym#1{\ifmmode^{#1}\else\(^{#1}\)\fi}
\caption{On Label Indications\label{tab1}}
\begin{tabular}{l*{2}{c}}
\hline\hline
            Indicator&\multicolumn{1}{c}{ICD-9-CM Code}&\multicolumn{1}{c}{ICD-9-CM Description}\\
\hline
Travelers Diarrhea    &     78791&       Diarhhea\\
[1em]
Urinary Tract Infection    &     5990&       Urinary tract infection, site unspecified\\
[1em]
Ear Infection    &     382&       Otitis media\\
[1em]
Chronic Bronchitis    &     491&       Chronic Brnchitis\\
[1em]
Shigellosis    &     004&       Shigellosis\\
[1em]
Pneumonia    &     480-488&       Pneumonia of Various Classifications\\
[1em]
Brucella    &     023&       Brucellosis\\
[1em]
Nocardia    &     039&       Actinomycotic infections\\
[1em]
Salmonella    &     003&       Other Salmonella Infections\\
[1em]
Paracoccidioides    &     1161&       Paracoccidiodomycosis\\
[1em]
Melioidoisis    &     025&       Melioidosis\\
[1em]
Burkholderia    &     2002&       Burckett's Tumors of Lymphatic Tissue\\
[1em]
Stenotrophomonas    &     n.a.&       n.a.\\
[1em]
Cyclospora    &     0075&       Cyclosporiasis\\
[1em]
Isospora    &     0072&       Coccidiosis\\
[1em]
Whipple's Disease    &     0402&       Whipple's Disease\\
[1em]
Toxoplasmosis    &     130&       Toxoplasmosis\\
[1em]
MRSA Related Skin Infection    &     0412&       Pneumococcus infection\\
\hline
idk\\
idk\\
\hline\hline
\end{tabular}
\label{tab:Table4.1}
\end{table}

\section{Variables}
The independent variables considered in the analysis are as follows. \textbf{TimeSinceGeneric} is a continuous variable from -82 to 49 and indicates the number months since the generic version of sulfamethoxazole-trimethoprim entered the market in July of 2012. Defining the timeline this way means that the months before July of 2012 take negative values. For example, May of 2012 (2 months before generic entry) would be coded as \textbf{timeSinceGeneric}$=-2$. I include \textbf{age} and \textbf{ageSQ} which are the ages and squared ages of the patient at the time of the visit. OffLabel indicates that none of the diagnoses which resulted from the visit were associated with an FDA approved usage of sulfamethoxazole-trimethoprim. GovInsurance and nonWhite indicate whether a patient was Medicare or Medicaid and if the patient was of an ethnicity other than white. Controlling for Medicare and Medicaid help to control for differences in the price faced by consumers due to insurance and is used as an additional income proxy. Lastly, \textbf{unspecCellAbscess} indicates if a patient was diagnosed with an unspecified skin abscess or cellulitis. I include this diagnosis because, although it is not associated with an on-label use of sulfamethoxazole-trimethoprim, it was the diagnosis associated with the most prescriptions of sulfamethoxazole-trimethoprim. Visits where this condition was diagnoses acount for 12.2\% of all prescriptions of sulfamethoxazole-trimethoprim and sulfamethoxazole-trimethoprim was prescribed for 17.5\% of visits where this condition was diagnosed.\\
\indent \autoref{tab:Table4.1} is a statistical summary of the continuous variables \textbf{timeSinceGeneric}, \textbf{age}, and \textbf{ageSQ}. I present the variables in the context of the entire study followed by summaries for before and after generic entry. The study goes across 131 months from January of 2006 (\textbf{timeSinceGeneric}$=-82$) to December of 2016 (\textbf{timeSinceGeneric}$=49$) with \textbf{timeSinceGeneric}$=0$ indicating July of 2012 when the generic sulfamethoxazole-trimethoprim entered the market. It is important to note that \textbf{timeSinceGeneric}$=0$ is included in "After Generic Entry". The final item of note from the table is the average age of the patient during the study period increased in the months after the generic entry by 1 year from 45.2 years old to 47.2 years old.\\
\begin{landscape}
\begin{tabular}{l*{6}{c}}
\hline\hline
            Variable&\multicolumn{1}{c}{Time frame}&\multicolumn{1}{c}{Weighted Mean}&\multicolumn{1}{c}{Weighted Median}&\multicolumn{1}{c}{Standard Deviation}&\multicolumn{1}{c}{Minimum}&\multicolumn{1}{c}{Maximum}\\
\hline
\textbf{TimeSinceGeneric}                    &     2006-2026&             -15.316&    -17&   37.167&     -79&  52\\
(Time in months since entry &     Before Entry of Generic&     -39.703&    -40&    22.497 &     -98&  -1\\
 of generic)   &     After Entry of Generic&       25.013 &    25&      14.792&     0&  52\\
[1em]
\textbf{Age}                                 &     2006-2026&             45.917&    50&    25.09 &     0&  100\\
(Age of patient in years)           &     Before Entry of Generic&     45.221&    49&    25.207&     0&  100\\
                                    &     After Entry of Generic&      47.069&    51&    24.853&     0&  92\\
[1em]
\textbf{AgeSQ}                               &     2006-2026&             2737.892&    2500&  2218.939&     0&  10000\\
(Age of patient squared)   &     Before Entry of Generic&    2680.279&    2401&  2216.963 &     0&  10000\\
                                    &     After Entry of Generic&      2833.167&    2601&  2218.929 &     0&  8464\\
\hline
$\text{Sample Size for Years 2006-2016} = 399245$\\
$\text{Before Entry of Generic} = 230182$\\
$\text{After Entry of Generic} = 169063$\\
\hline\hline
\multicolumn{4}{l}{\footnotesize All observations after July 2012 are considered to be after entry of generic.}\\
\end{tabular}

\end{landscape} 
\indent \autoref{tab:Table4.2} is a statistical summary of the categorical variables \textbf{offLabel}, \textbf{govInsurance}, and \textbf{nonWhite}. The vast proportion of visits in the sample did not have a diagnosis associated with an on-label use of sulfamethoxazole-trimethoprim. These visits accounted for over 96\% of the weighted sample across each time period. Patients on government insurance (Medicare or Medicaid) made up over a quarter of all visits over the entire time period and their weighted share of the sample increased from 24.9\% to 28.1\% during the time after the generic was introduced. Nonwhite patients make up less of the sample with a weighted average of 16.4\% over the entire study. Similar to the government insurance group, this category saw an increase in their weighted proportion of the sample after the generic came on from 16\% to 17.3\%. The final category, those diagnosed with an unspecific skin abscess or cellulitis, made up 5.4\% of the weighted sample across the entire study and decreased from 5.5\% to 5.2\% once the generic entered the market.
\begin{landscape}
\begin{tabular}{l*{4}{c}}
\hline\hline
            Variable&\multicolumn{1}{c}{Time frame}&\multicolumn{1}{c}{Total}&\multicolumn{1}{c}{Weighted Share of Sample}\\
\hline
\textbf{OffLabel}                                                &     Entire Study&             387264&      .967\\
(=1 if no diagnoses made were FDA approved          &     Before Entry of Generic&    223267&      .966\\
indications of sulfamethoxazole-trimethoprim)  &     After Entry of Generic&      163997&      .968\\
[1em]
\textbf{GovInsurance}                                            &     Entire Study&             105273&      .26 \\
(=1 if patient is on either Medicare or Medicaid)       &     Before Entry of Generic&     30480 &      .248\\
                                                        &     After Entry of Generic&      44793 &      .28\\
[1em]
\textbf{NonWhite}                                                &     Entire Study&             61442&      .164\\
(=1 if patient is a race other than white)                &     Before Entry of Generic&     37733&      .16\\
                                                        &     After Entry of Generic&      23709&      .171\\
\hline
$\text{Sample Size for Years 2006-2016} = 399245$\\
$\text{Before Entry of Generic} = 230182$\\
$\text{After Entry of Generic} = 169063$\\
\hline\hline
\multicolumn{4}{l}{\footnotesize All observations after August 2012 are considered to be after entry of generic.}\\
\end{tabular}

%multicolumn{5}{l}{"Share of Sample" and "Proportion Prescribed ST" are both weighted proportions}\\
%\multicolumn{5}{l}{offLabel(=1) indicates no diagnoses made were on label indicators, govInsurance(=1) indicates patient is on Medicare or Medicaid,}\\
%\multicolumn{5}{l}{nonWhite(=1) indicates patient is race other than white}\\
%\end{tabular}
%\label{tab:Table4.2}
%\end{table}

\end{landscape}
\autoref{tab:Table4.4} provides a look at the weighted proportion of patients of each type which were prescribed sulfamethoxazole-trimethoprim. Overall, .861\% of all visits lead to a prescription of sulfamethoxazole-trimethoprim over the study period with an increase of .029 percentage points post entry of the generic. Looking only at visits that did not have a diagnosis associated with an on-label use of sulfamethoxazole-trimethoprim, the numbers do not change a great deal. The weighted proportion prescribed the drug was .718\% across the study and increased from .696\% to .78\% between the time before the generic entered and after. The story does change for the compliment of the off-label category, however. Visits that saw at least one diagnosis associated with an on-label use of sulfamethoxazole-trimethoprim led to a prescription of the antibiotic 5.07\% of the time. However, this proportion decreased nearly 1 percentage point from 5.35\% to 4.53\% between the two time periods.\\
\indent The other categorical variables tell similar stories to the entire sample. Patients on government insurance were prescribed sulfamethoxazole-trimethoprim for .871\% of there total visits and saw an increase from .806\% to .975\%. The proportion of patients not on any form of government insurance saw their proportion of visits which lead to a prescription of the antibiotic fall slightly from .865\% to .843\%. For nonwhite patients, the weighted proportion of visits leading to a prescription of the drug was .933\% with an increase from .846\% to 1.08\% between between before entry of the generic and after. White patients did not see a similar increase as their proportion decreased from .852\% to .838\%. Finally, patients with an unspecified skin abscess or cellulitis had the largest proportion of visits lead to a prescription of the drug with 17.5\% across the years 2006-2016. This group's probability of prescription increased 1.4\% percentage points from the period before entry of the generic to the period after. 
\begin{landscape}
\begin{table}[htbp]\centering
\def\sym#1{\ifmmode^{#1}\else\(^{#1}\)\fi}
\caption{Proportions of Patients Prescribed Sulfamethoxazole-Trimethoprim (SXT) by Group\label{tab1}}
\begin{tabular}{l*{3}{c}}
\hline\hline
            Variable&\multicolumn{1}{c}{Timeframe}&\multicolumn{1}{c}{Total Prescriptions of SXT}&\multicolumn{1}{c}{Weighted Proportion Prescribed SXT}\\
\hline
\textbf{Total Sample}                                   &     Entire Study&             3340&     .00861\\
                                                        &     Before Generic Entry&    2072&     .00851\\
                                                        &     After Generic Entry&      1268&     .0088\\
[1em]
\textbf{offLabel}                                       &     Entire Study&             2736&     .00718\\
(=1 if no diagnoses made were FDA approved         &     Before Generic Entry&    1663&     .00696\\
indications of Sulfamethoxazole-Trimethoprim)  &     After Generic Entry&      1073&     .0078\\
[1em]
(=0 if at least one diagnosis made during               &     Entire Study&             604&     .0507\\
visit is an FDA approved indication of                  &     Before Generic Entry&    409&     .0535\\
Sulfamethoxazole-Trimethoprim)                          &     After Generic Entry&      195&     .0454\\
[1em]
\textbf{govInsurance}                                   &     Entire Study&             869&     .00871\\
(=1 if patient is on either Medicare of Medicaid)       &     Before Generic Entry&     533 &     .00806\\
                                                        &     After Generic Entry&      336 &     .00975\\
[1em]
(=0 if patient is on neither Medicare nor Medicaid)     &     Entire Study&             2471&     .00858\\
                                                        &     Before Generic Entry&     1539 &     .00865\\
                                                        &     After Generic Entry&      932 &     .00843\\
[1em]
\textbf{nonWhite}                                       &     Entire Study&             534&      .00933\\
(=1 if patient is race other than white)                &     Before Generic Entry&     354&      .00846\\
                                                        &     After Generic Entry&      180&      .0108\\
[1em]
(=0 if patient is white)                                &     Entire Study&             2806&      .00847\\
                                                        &     Before Generic Entry&     1718&      .00852\\
                                                        &     After Generic Entry&      1088&      .00838\\
[1em]
\textbf{UnspecCellAbscess}                              &     Entire Study&             349 &      .175\\
(=1 if patient was diagnosed with an unspecified        &     Before Generic Entry&     224 &      .170\\
 skin abscess or cellulitis)                            &     After Generic Entry&      125  &      .184\\
[1em]
(=0 if patient was not diagnosed with an unspecified    &     Entire Study&             2291 &      .00761\\
skin abscess or cellulitis)                             &     Before Generic Entry&     1848 &      .0075\\
                                                        &     After Generic Entry&      1143  &      .0078\\
\hline
$n(\text{Entire Study}) = 399245$\\
$n(\text{Before Generic Entry}) = 249345$\\
$n(\text{After Generic Entry}) = 149900$\\
\hline\hline
%\multicolumn{5}{l}{"Share of Sample" and "Proportion Prescribed ST" are both weighted proportions}\\
%\multicolumn{5}{l}{offLabel(=1) indicates no diagnoses made were on label indicators, govInsurance(=1) indicates patient is on Medicare or Medicaid, nonWhite(=1) indicates}\\
%\multicolumn{5}{l}{patient is race other than white, UnspecCellAbscess(=1) indicates patient was diagnosed with and unspecified skin abscess or cellulitis}\\
\end{tabular}
\label{tab:Table4.4}
\end{table}

%multicolumn{5}{l}{"Share of Sample" and "Proportion Prescribed ST" are both weighted proportions}\\
%\multicolumn{5}{l}{offLabel(=1) indicates no diagnoses made were on label indicators, govInsurance(=1) indicates patient is on Medicare or Medicaid,}\\
%\multicolumn{5}{l}{nonWhite(=1) indicates patient is race other than white}\\
%\end{tabular}
%\label{tab:Table4.2}
%\end{table}

\end{landscape}