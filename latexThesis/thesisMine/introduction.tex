\chapter{Introduction}
An antibiotic's effectiveness is an exhaustible resource the nature of which is not fully internalized in the private cost of consumption. Because users do not face the full social cost of consumption, the rate at which an antibiotic is utilizaed may be higher than optimal. This inefficiency leads to overusage and acceleration of the evolution and spread of antibiotic resistant bacterial strains. In the United States alone, over 2.8 million Americans per year are sickened by these resistant bacteria which cause at least 35,000 deaths \cite{centers_for_disease_control_and_prevention_us_antibiotic_2019}. Overusage of antibiotics amplify these adverse effects \cite{gerber_outpatient_2019}.\\
\indent In order to progress in our understanding of how to combat antibiotic overusage, more foundational questions about antibiotic markets and consumer demand must be investigated. Among other needs, we must know how price sensitive consumer's demand for antibiotics is, what characteristics of a consumer make them more likely to demand antibiotic therepy over other forms of treatment, how preceived differences in quality affect if consumers view generic antibotics as perfect substitutes, and how generic entry in antibiotic markets affects demand for the antibiotic's active ingredient.\\
\indent In this paper, I aim to answer the final question by analyzing prescrition trends of sulfamethoxazole-trimethoprim before and after entry of its generic counterpart in Novemeber of 2012. I use data from the National Ambulatory Medical Care Survey, a nationally representative survey of medical visits, to track prescriptions of sulfamethoxazole-trimethoprim from January of 2006 to December of 2016. I then analyze differences in multiple linear probability models to determine the effect of this generic entry. Specific attention is given to patients with Food and Drug Administration (FDA) approved reasons for prescription, also known as on label indications, of sulfamethoxazole-trimethoprim. Patients diagnosed with an unspecified skin abscess or cellulitis are given additional attention as well.\\
\indent I find that, despite negative trends in probability of prescription over time, a small but significant (90\% CI) increase is present in the probability of prescription of sulfamethoxazole-trimethoprim for individuals diagnosed with FDA approved indications. This trend becomes larger and more signficant for patients on Medqwficare or Medicaid and patients who are a race other than white. These changes were not present in individuals diagnosed solely with non FDA approved indications of sulfamethoxazole-trimethoprim although these visits made up a majority of the drug's prescriptions.\\
\indent The rest of the thesis proceeds as follows. I lay out background information on generic medicines, sulfamethoxazole-trimethoprim, and present relevant literature on demand for prescription drugs. I then present the economic framework behind how generic entry increases demand for sulfamethoxazole-trimethoprim followed by a description of the my data and variables. Chapter 5 describes my empirical model along with my findings and is followed by a brief discussion. I close my paper with concluding remarks on my findings, limitations of my study, and ideas for future research.
