\chapter{Conclusions}
While not uniform across all patient groups, I find a small but significant increase in the probability of prescription of Sulfamethoxazole-Trimethoprim due to generic entry for patients with FDA approved indications. On label patient's expected probability of prescription increased by 1.88 percentage points (90\% CI) immediately after generic entry. On label patients who were on Medicare or Medicaid saw increases in expected probability of perscription of 2 percentage points (90\% CI) and 2.1 percentage points (95\% CI) respectively. I attribute these changes to a decrease in price caused by generic entry.\\
\indent Patients with non FDA approved indications did not see significant changes in probability of prescription. This finding is important because these cases made up more than half of all prescriptions of Sulfamethoxazole-Trimethoprim. I attribute this result to the reported preference of elderly individuals to be prescribed newer drugs and a fear of spreading MRSA resistant bacteria.

\section{Limitations of the Study}
These results may be limited by the specific setting in which they come from. Because emergency departments are not included in the analysis, population more likely to leverage those services as well as diagnoses which are more likely to occur in that setting will be underrepresented. Second, Sulfamethoxazole-Trimethoprim is only one antibiotic which had already seen resistance forming before generic entry. Because of this, consumers may have been more reluctant to demand the drug as time progressed. Generic entry of antibotics with less reported resistance may have a higher proportion of patients adopting generic treatment. This study is additionally limited by its selection of variables. Consideration of a patient's gender, region, and characteristics of the physician may reveal some omitted variable bias albiet at the cost of increased complexity within the model. Finally, a more refined process could be implemented in order to determine what visits can be considered relevant in the study of a given antibiotic. Including all diagnoses which lead to prescription of Sulfamethoxazole-Trimethoprim may lead to inclusion of irrevelent visits. One possible example would be and individual who is diagnosed with an on lable indication as well as hypertension. The method employed in this study would go on to count all visits where a diagnosis of hypertension was made to be relevent visits even though Sulfamethoxazole-Trimethoprim would not be diagnoses for that condition which could negatively bias empirical results. 

\section{Recommendations for Future Research}
The limitations mentioned above serve as directions for future research. First, expanding the scope of the study to include emergency department and non ambulatory care would help provide a more complete understanding of Sulfamethoxazole-Trimethoprim before and after generic entry. Second, subjecting different drugs to a similar methodology is needed to determine how many results from this study are products of specific characteristics of Sulfamethoxazole-Trimethoprim. Third, inclusion of additional variables about the patient and physician may help lead to additional findings not seen in this study. Fourth, development of a more sophisticated method to control for relevant diagnoses could further ensure unbiased results. Finally, controlling for prescription trends of close substitutes would provide the researcher with an idea of what the opportunity cost of adoption of generics may specifically be. 