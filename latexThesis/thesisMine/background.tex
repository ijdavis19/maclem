\chapter{Background}

\section{Generic Medicines}
Generic medication markets as they are known in the United States are far more nascent than their brand name counterparts. Prior to 1984, FDA rules required generic drug companies to repeat clinical trials which had already been undertaken by its on brand predecessor. Hence, even after a formula had proven to be safe and effective, generic companies would have to subject a similar formula to the same level of scrutiny as it had already overcome. This costly redundancy served as a barrier to entry preventing generic drug companies from competing with larger brands \cite{eban_bottle_2019}.\\
\indent Senators Orrin Hatch and Henry Waxman authored the Drug Price Competition and Patent Term Restoration Act (known now as the Hatch-Waxman Act) which served to lower the cost and expediate the process for FDA approval of generic drugs. The Hatch-Waxman Act, passed in 1984, removed the requirements that a generic needed to undergo all of the clinical trials and safety procedures first undertaken by the original product. Instead, generics simply need to prove bioequivalence to the brand name and demonstrate the drugs exhibit similar behavior inside of the body. Proving bioequivalence requires determining that, in addition to having the same active ingredient, the rate and extent at which the active ingredient becomes available to the body is not significantly different between the branded drug and the generic version being tested \cite{fda_primer}. This new standard greatly lowered the fixed cost of bringing a generic to market and opened the door for the generic drug industry as it is known today.\\
\indent Extensive research has been put forth on both patient and doctor preceptions of generic medicines, as well as the consequences of generic usage or lackthereof both in America and abroad. One study of Turkish healthcare costs estimated the country's total healthcare expenditure could be dimished by 31\% by adopting more widespread use of generic antibiotics \cite{mercanoglu_evaluation_2018}. In the United States, further adoption of generic antiretrovial therepies would save consumers an estimated \$920 Million \cite{walensky_economic_2013} Additionally, a case study of Ireland found generic medicines to be 20-90\% cheaper than their on brand competition \cite{dunne_review_2013}. Studies have found treatment outcomes using generics and brand name medicines to be comparable or equivalent both abroad \cite{lin_comparative_2017} and in the United States \cite{desai_comparative_2019}\\
\indent Where the conlusions about generic medicines are less uniform are in patient and physicians preceptions of their effectiveness. A 2015 meta analysis of generic medicince usage found patients have strong opinions that cheaper drugs are of lower quality \cite{dunne_what_2015}. This bias is exacerbated by direct to consumer advertising done by patented and brand name drugs \cite{morgan_economics_2003}. The meta analysis does go on to show, however, that more educated patients are significantly more likely to accept generic treatment and overall trust of generics has improved over time. A patient's trust in their doctor's judgement also tends to overrule biases that a patient has about cheaper generics.

\section{Antibiotic Usage}
Research on the subject of antibiotic and more general prescription practices when conducted by medical and pharmaceutical researchers has primarily focused on teasing out how antibiotics are used and places less of an emphasis on providing hypotheses which delineate the underlying causes of these trends. Many studies have focused on the widespread antibiotic usage in ambulatory medical care settings across the United States. Findings from these studies include high usage of antibiotics to fight purulent skin and soft tissue infections in which over 70\% of patients displaying these symptoms were prescribed an antibiotic in the years 2000 to 2015 \cite{fritz_national_2020}. Further research has shown physicians in this setting to be unafraid to prescribe antibiotics for non FDA approved indications. An analysis of outpatient prescriptions of Fluoroquinole antibiotics found over 50\% of visits leading to a prescription of this class of antibiotics were for ailments which these antibiotics had not yet established efficacy against \cite{almalki_off-label_2016}. Finally, an analysis of antibiotic usage among elderly populations found usage to stable among adults \cite{roumie_trends_2005} and eldely populations \cite{kabbani_outpatient_2018}. The latter of which averaging a prescriptions per thousand visits measure greater than 1 across the early 2010s.\\
\indent Economic research in this area has yielded more focused empirical conclusions about demand for prescription medications and antibiotics. Research from outpatient facilities in the United States from 1993-2015 found high elasticity of demand in both new and more expensive antbiotic classes as well as classic, cheaper penicillins. Additional findings indicate a positive correlation between the proportion of elderly patients and demand for newer antibiotics \cite{kianmehr_system_2020}. A German study on the demand for broad spectrum antimicrobials found significant negative own price elasticities for all antibiotics studied in the outpatient care setting. This trend did not persist in the hospital setting \cite{kaier_impact_2013}.\\

\section{Sulfamethoxazole-trimethoprim}
\indent For simplification, I focus my analysis to prescriptions of sulfamethoxazole-trimethoprim. Sulfamethoxazole-trimethoprim is a combination antibiotic from the class antimetabolite/sulfonamide and was first introduced in 1968. The brand versions of the drug include Bactrim, Bactrim DS, Septra, and Septra DS and the generic form entered the American market in July of 2012. The antibiotic was popular even before generic entry due its high familiarity among physicians and low cost \cite{noauthor_sulfamethoxazole_nodate,ho_considerations_2011}. The drug has FDA approval to fight urinary tract infections, ear infections (acute otitis media), acute exacerbations of chronic bronchitis in adults, Shigellosis, treatment and prophylaxis of \textit{Pneumocystis jirovecii} pneumonia, and Traveler's diarrhea in a adults. The antibiotic is also used for infections due to \textit{Listeria, Nocardia, Salmonella, Brucella, Paracoccidioides,} melioidosis, \textit{Burkholderia, Stenotrophomonas,} cyclospora, isospora, Whipple's disease, and alterntive therapy for toxoplasmosis and community-acquired MRSA skin infections \cite{schlossberg_antibiotics_2017}. While not specifically named by the FDA, evidence has been found which indicates sulfamethoxazole-trimethoprim is successful in treating uncomplicated skin abcesses \cite{noauthor_trimethoprimsulfamethoxazole_nodate}\\
