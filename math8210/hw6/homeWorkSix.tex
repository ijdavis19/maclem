% SampleProject.tex -- main LaTeX file for sample LaTeX article
%
%\documentclass[12pt]{article}
\documentclass[11pt]{SelfArxOneColBMN}
% add the pgf and tikz support.  This automatically loads
% xcolor so no need to load color
\usepackage{pgf}
\usepackage{tikz}
\usetikzlibrary{matrix}
\usetikzlibrary{calc}
\usepackage{xstring}
\usepackage{pbox}
\usepackage{etoolbox}
\usepackage{marginfix}
\usepackage{xparse}
\setlength{\parskip}{0pt}% fix as marginfix inserts a 1pt ghost parskip
% standard graphics support
\usepackage{graphicx,xcolor}
\usepackage{wrapfig}
%
\definecolor{color1}{RGB}{0,0,90} % Color of the article title and sections
\definecolor{color2}{RGB}{0,20,20} % Color of the boxes behind the abstract and headings
%----------------------------------------------------------------------------------------
%	HYPERLINKS
%----------------------------------------------------------------------------------------
\usepackage[pdftex]{hyperref} % Required for hyperlinks
\hypersetup{hidelinks,
colorlinks,
breaklinks=true,%
urlcolor=color2,%
citecolor=color1,%
linkcolor=color1,%
bookmarksopen=false%
,pdftitle={HomeWorkSix},%
pdfauthor={Ian Davis}}
%\usepackage[round,numbers]{natbib}
\usepackage[numbers]{natbib}
\usepackage{lmodern}
\usepackage{setspace}
\usepackage{xspace}
%
\usepackage{subfigure}
\newcommand{\goodgap}{
  \hspace{\subfigtopskip}
  \hspace{\subfigbottomskip}}
%
\usepackage{atbegshi}
%
\usepackage[hyper]{listings}
%
% use ams math packages
\usepackage{amsmath,amsthm,amssymb,amsfonts}
\usepackage{mathrsfs}
%
% use new improved Verbatim
\usepackage{fancyvrb}
%
\usepackage[titletoc,title]{appendix}
%
\usepackage{url}
%
% Create length for the baselineskip of text in footnotesize
\newdimen\footnotesizebaselineskip
\newcommand{\test}[1]{%
 \setbox0=\vbox{\footnotesize\strut Test \strut}
 \global\footnotesizebaselineskip=\ht0 \global\advance\footnotesizebaselineskip by \dp0
}
%
\usepackage{listings}

\DeclareGraphicsExtensions{.pdf, .jpg, .tif,.png}

% make sure we don't get orphaned words if at top of page
% or orphans if at bottom of page
\clubpenalty=9999
\widowpenalty=9999
\renewcommand{\textfraction}{0.15}
\renewcommand{\topfraction}{0.85}
\renewcommand{\bottomfraction}{0.85}
\renewcommand{\floatpagefraction}{0.66}
%
\DeclareMathOperator{\sech}{sech}

\newcommand{\mycite}[1]{%
(\citeauthor{#1} \citep{#1} \citeyear{#1})\xspace
}

\newcommand{\mycitetwo}[2]{%
(\citeauthor{#2} \citep[#1]{#2} \citeyear{#2})\xspace
}

\newcommand{\mycitethree}[3]{%
(\citeauthor{#3} \citep[#1][#2]{#3} \citeyear{#3})\xspace
}

\newcommand{\myincludegraphics}[3]{% file name, width, height
\includegraphics[width=#2,height=#3]{#1}
}

\newcommand{\myincludegraphicstwo}[2]{% file name, width, height
\includegraphics[scale=#1]{#2}
}

\newcommand{\mysimplegraphics}[1]{% file name, width, height
\includegraphics{#1}
}

\newcommand{\MB}[1]{
\boldsymbol{#1}
}

\newcommand{\myquotetwo}[1]{%
\small
%\singlespacing
\begin{quotation}
#1
\end{quotation}
\normalsize
%\onehalfspacing  
}

\newcommand{\jimquote}[1]{%
\small
%\singlespacing
\begin{quotation}
#1
\end{quotation}
\normalsize
%\onehalfspacing
}

\newcommand{\myquote}[1]{%
\small
%\singlespacing
\begin{quotation}
#1
\end{quotation}
\normalsize
%\onehalfspacing  
}

%A =
%
%[2 r_1 	     r_1]
%[-2r_1 + r_2  r_2 - r_1]
%
%has eigenvalues r_1 neq r_2.
% #1 = 2 r_1, #2 = r_1, #3 = -2r_1+r_2, #4 = r_2 - r_1
\newcommand{\myrealdiffA}[4]{
\left [
\begin{array}{rr}
#1  & #2\\
#3  & #4
\end{array}
\right ]
}

% args:
% 1, 2 ,3, 4, 5 = caption, label, width, height, file name
%\mysubfigure{}{}{}{}{}
\newcommand{\mysubfigure}[5]{%
\subfigure[#1]{\label{#2}\includegraphics[width=#3,height=#4]{#5}}
}

\newcommand{\mysubfiguretwo}[3]{%
\subfigure[#1]{\label{#2}\includegraphics{#3}}
}

\newcommand{\mysubfigurethree}[4]{%
\subfigure[#1]{\label{#2}\includegraphics[scale=#3]{#4}}
}

\newcommand{\myputimage}[5]{% file name, width, height
\centering
\includegraphics[width=#3,height=#4]{#5}
\caption{#1}
\label{#2}
}

\newcommand{\myputimagetwo}[4]{% caption, label, scale, file name
\centering
\includegraphics[scale=#3]{#4}
\caption{#1}
\label{#2}
}

\newcommand{\myrotateimage}[5]{% file name, width, height
\centering
\includegraphics[scale=#3,angle=#4]{#5}
\caption{#1}
\label{#2}
}

\newcommand{\myurl}[2]{%
\href{#1}{\bf #2}
}

\RecustomVerbatimEnvironment%
{Verbatim}{Verbatim}  
  {fillcolor=\color{black!20}}
  
  \DefineVerbatimEnvironment%
{MyVerbatim}{Verbatim}  
  {frame=single,
   framerule=2pt,
   fillcolor=\color{black!20},
   fontsize=\small}
   
\newcommand{\myfvset}[1]{%  
\fvset{frame=single,
       framerule=2pt,
       fontsize=\small,
       xleftmargin=#1in}}
       
\newcommand{\mylistverbatim}{%
\lstset{%
  fancyvrb, 
  basicstyle=\small,
  breaklines=true}
}  

\newcommand{\mylstinlinebf}[1]{%
{\bf #1}
}

\newcommand{\mylstinline}{%
\lstset{%
  basicstyle=\color{black!80}\bfseries\ttfamily,
  showstringspaces=false,
  showspaces=false,showtabs=false,
  breaklines=true}
\lstinline
}

\newcommand{\mylstinlinetwo}[1]{%
\lstset{%
  basicstyle=\color{black!80}\bfseries\ttfamily,
  showstringspaces=false,
  showspaces=false,showtabs=false,
  breaklines=true}
\lstinline!#1! 
}

%fontfamily=tt
%fontfamily=courier
%fontfamily=helvetica
%frame=topline,
%frame=single,
 %frame=lines,
 %framesep=10pt,
 %fontshape=it,
 %fontseries=b,
 %fontsize=\relsize{-1},
 %fillcolor=\color{black!20},
 %rulecolor=\color{yellow},
 %fillcolor=\color{red}
 %label=\fbox{\Large\emph{The code}}
\DefineVerbatimEnvironment%
{MyListVerbatim}{Verbatim}  
{
fillcolor=\color{black!10},
fontfamily=courier,
frame=single,
%formatcom=\color{white},
framesep=5mm,
labelposition=topline,
fontshape=it,
fontseries=b,
fontsize=\small,
label=\fbox{\large\emph{The code}\normalsize}
} 

%  caption={[#1] \large\bf{#1}}, 
%\centering \framebox[.6\textwidth][c]{\Large\bf{#1}}
\newcommand{\myfancyverbatim}[1]{%
\lstset{%
  fancyvrb=true, 
  %fvcmdparams= fillcolor 1,
  %morefvcmdparams = \textcolor 2,
  frame=shadowbox,framerule=2pt, 
  basicstyle=\small\bfseries,
  backgroundcolor=\color{black!08},
  showstringspaces=false,
  showspaces=false,showtabs=false,
  keywordstyle=\color{black}\bfseries,
  %numbers=left,numberstyle=\tiny,stepnumber=5,numbersep=5pt,
  stringstyle=\ttfamily,
  caption={[\quad #1] \mbox{}\\ \vspace{0.1in} \framebox{\large \bf{#1} \small} },  
  belowcaptionskip=20 pt,  
  label={},
  xleftmargin=17pt,
  framexleftmargin=17pt,
  framexrightmargin=5pt,
  framexbottommargin=4pt,
  nolol=false,
  breaklines=true}
}

\newcommand{\mylistcode}[3]{%
\lstset{%
  language=#1, 
  frame=shadowbox,framerule=2pt, 
  basicstyle=\small\bfseries,
  backgroundcolor=\color{black!16},
  showstringspaces=false,
  showspaces=false,showtabs=false,
  keywordstyle=\color{black!40}\bfseries,
  numbers=left,numberstyle=\tiny,stepnumber=5,numbersep=5pt,
  stringstyle=\ttfamily,
  caption={[\quad#2] \mbox{}\\ \vspace{0.1in} \framebox{\large \bf{#2} \small} },
  belowcaptionskip=20 pt,
  breaklines=true,
  xleftmargin=17pt,
  framexleftmargin=17pt,
  framexrightmargin=5pt,
  framexbottommargin=4pt,  
  label=#3,
  breaklines=true} 
}

  %caption={[#2] #3},
  %caption={[#2]{\mbox{}\\ \vspace{0.1in} \framebox{\large \bf{#3} \small}},
  %caption={[#2] \mbox{}\\ \bf{#3} },

% frame=single,
% caption={[Code Fragment] {\bf Code Fragment} },
% caption={[Code Fragment] \mbox{}\\ \vspace{0.1in} \framebox{\large \bf{Code Fragment} \small} },
\newcommand{\mylistcodequick}[1]{%
\lstset{%
  language=#1, 
  frame=shadowbox,framerule=2pt, 
  basicstyle=\small\bfseries,
  backgroundcolor=\color{black!16},
  showstringspaces=false,
  showspaces=false,showtabs=false,
  keywordstyle=\color{black!40}\bfseries,
  numbers=left,numberstyle=\tiny,stepnumber=5,numbersep=5pt,
  stringstyle=\ttfamily,
  caption={[\quad Code Fragment] \large \bf{Code Fragment} \small},   
  belowcaptionskip=20 pt,  
  label={},
  xleftmargin=17pt,
  framexleftmargin=17pt,
  framexrightmargin=5pt,
  framexbottommargin=4pt,
  breaklines=true} 
}

%  caption={[#2] \mbox{}\\ \vspace{0.1in} \framebox{\large \bf{#2} \small} },
\newcommand{\mylistcodequicktwo}[2]{%
\lstset{%
  language=#1, 
  frame=shadowbox,framerule=2pt, 
  basicstyle=\small\bfseries,
  extendedchars=true,
  backgroundcolor=\color{black!16},
  showstringspaces=false,
  showspaces=false,
  showtabs=false,
  keywordstyle=\color{black!40}\bfseries,
  numbers=left,numberstyle=\tiny,stepnumber=5,numbersep=5pt,
  stringstyle=\ttfamily,
  caption={[\quad#2] \large \bf{#2} \small},
  belowcaptionskip=20 pt,
  label={},
  xleftmargin=17pt,
  framexleftmargin=17pt,
  framexrightmargin=5pt,
  framexbottommargin=4pt,
  breaklines=true} 
}

%  caption={[#2] \mbox{}\\ \vspace{0.1in} \framebox{\large \bf{#2} \small} },
\newcommand{\mylistcodequickthree}[2]{%
\lstset{%
  language=#1, 
  frame=shadowbox,framerule=2pt, 
  basicstyle=\small\bfseries,
  extendedchars=true,
  backgroundcolor=\color{black!16},
  showstringspaces=false,
  showspaces=false,
  showtabs=false,
  keywordstyle=\color{black!40}\bfseries,
  numbers=left,numberstyle=\tiny,stepnumber=5,numbersep=5pt,
  stringstyle=\ttfamily,
  caption={[\quad#2] \large\bf{#2}\small},
  belowcaptionskip=20 pt,
  label={},
  xleftmargin=17pt,
  framexleftmargin=17pt,
  framexrightmargin=5pt,
  framexbottommargin=4pt,
  breaklines=true} 
}

%  frame=single,
\newcommand{\mylistset}[4]{%
\lstset{language=#1,
  basicstyle=\small,
  showstringspaces=false,
  showspaces=false,showtabs=false,
  keywordstyle=\color{black!40}\bfseries,
  numbers=left,numberstyle=\tiny,stepnumber=5,numbersep=5pt,
  stringstyle=\ttfamily,
  caption={[\quad#2]#3},
  label=#4}
}

\newcommand{\mylstinlineset}{%
\lstset{%
  basicstyle=\color{blue}\bfseries\ttfamily,
  showstringspaces=false,
  showspaces=false,showtabs=false,
  breaklines=true}
}

\newcommand{\myframedtext}[1]{%
\centering
\noindent
%\fbox{\parbox[c]{.9\textwidth}{\color{black!40} \small \singlespacing #1\onehalfspacing \normalsize \\}}
\fbox{\parbox[c]{.9\textwidth}{\color{black!40} \small  #1 \normalsize \\}}
}

\newcommand{\myemptybox}[2]{% from , to
\fbox{\begin{minipage}[t][#1in][c]{#2in}\hspace{#2in}\end{minipage}}
}

\newcommand{\myemptyboxtwo}[2]{% from , to
\centering\fbox{
\begin{minipage}{#1in}
\hfill\vspace{#2in}
\end{minipage}
}
}

\newcommand{\boldvector}[1]{
\boldsymbol{#1}
}

\newcommand{\dEdY}[2]{\frac{d E}{d Y_{#1}^{#2}}}
\newcommand{\dEdy}[2]{\frac{d E}{d y_{#1}^{#2}}}
\newcommand{\dEdT}[2]{\frac{\partial E}{\partial T_{{#1} \rightarrow {#2}}}}
\newcommand{\dEdo}[1]{\frac{\partial E}{\partial o^{#1}}}
\newcommand{\dEdg}[1]{\frac{\partial E}{\partial g^{#1}}}
\newcommand{\dYdY}[4]{\frac{\partial Y_{#1}^{#2}}{\partial Y_{#3}^{#4}}}
\newcommand{\dYdy}[4]{\frac{\partial Y_{#1}^{#2}}{\partial y_{#3}^{#4}}}
\newcommand{\dydY}[4]{\frac{\partial y_{#1}^{#2}}{\partial Y_{#3}^{#4}}}
\newcommand{\dydy}[4]{\frac{\partial y_{#1}^{#2}}{\partial y_{#3}^{#4}}}
\newcommand{\dydT}[4]{\frac{\partial y_{#1}^{#2}}{\partial T_{{#3} \rightarrow {#4}}}}
\newcommand{\dYdT}[4]{\frac{\partial Y_{#1}^{#2}}{\partial T_{{#3} \rightarrow {#4}}}}
\newcommand{\dTdT}[4]{\frac{\partial T_{{#1} \rightarrow {#2}}}{\partial T_{{#3} \rightarrow {#4}}}}
\newcommand{\ssum}[2]{\sum_{#1}^{#2}}

\newcommand{\ssty}[1]{\scriptscriptstyle #1}

\newcommand{\myparbox}[2]{%
\parbox{#1}{\color{black!20} #2}
}

\newcommand{\bs}[1]{
\boldsymbol{#1}
}

\newcommand{\parone}[2]{%
\frac{\partial #1 }{ \partial #2 }
}
\newcommand{\partwo}[2]{%
\frac{ \partial^2 {#1} }{ \partial {#2}^2 }
}

\newcommand{\twodvectorvarfun}[2]{
\left [
\begin{array}{r}
{{#1_{\ssty{1}}}(#2)} \\
{{#1_{\ssty{2}}}(#2)}
\end{array}
\right ]
}
\newcommand{\twodvectorvarprimed}[1]{
\left [
\begin{array}{r}
{{#1_{\ssty{1}}}'(t)} \\
{{#1_{\ssty{2}}}'(t)}
\end{array}
\right ]
}

\newcommand{\complex}[2]{#1 \: #2 \: \boldsymbol{i}}
\newcommand{\complexmag}[2]%
{
\sqrt{(#1)^2 \: + \: (#2)^2}
}
\newcommand{\threenorm}[3]%
{
\sqrt{(#1)^2 \: + \: (#2)^2 \: + \: (#3)^2}
}
\newcommand{\norm}[1]{\mid \mid #1 \mid \mid}

\newcommand{\myderiv}[2]{\frac{d #1}{d #2}}
\newcommand{\myderivb}[2]{\frac{d}{d #2} \left ( #1 \right )}
\newcommand{\myrate}[3]%
{#1^\prime(#2) &=& #3 \: #1(#2)
}
\newcommand{\myrateexter}[4]%
{#1^\prime(#2) &=& #3 \: #1(#2) \: + \: #4
}
\newcommand{\myrateic}[3]%
{#1( \: #2 \:) &=& #3 
}

\newcommand{\mytwodsystemeqn}[6]{
#1 \: x    #2 \: y &=& #3\\
#4 \: x    #5 \: y &=& #6\\
}

\newcommand{\mytwodsystem}[8]{
#3 \: #1 \: + \: #4 \: #2 &=& #5\\
#6 \: #1 \: + \: #7 \: #2 &=& #8\\
}  

\newcommand{\mytwodarray}[4]{
\left [
\begin{array}{rr}
#1 & #2\\
#3 & #4
\end{array}
\right ]
}

\newcommand{\mytwoid}{
\left [
\begin{array}{rr}
1 & 0\\
0 & 1
\end{array}
\right ]
}

\newcommand{\myxprime}[2]{
\left [
\begin{array}{r}
#1^\prime(t)\\
#2^\prime(t)
\end{array}
\right ]
}

\newcommand{\myxprimepacked}[2]{
\left [
\begin{array}{r}
#1^\prime\\
#2^\prime
\end{array}
\right ]
}

\newcommand{\myx}[2]{
\left [
\begin{array}{r}
#1(t)\\
#2(t)
\end{array}
\right ]
}

\newcommand{\myxonly}[2]{
\left [
\begin{array}{r}
#1\\
#2
\end{array}
\right ]
}

\newcommand{\myv}[2]{
\left [
\begin{array}{r}
#1\\
#2
\end{array}
\right ]
}

\newcommand{\myxinitial}[2]{
\left [
\begin{array}{r}
#1(0)\\
#2(0)
\end{array}
\right ]
}

\newcommand{\twodboldv}[1]{
\boldsymbol{#1}
}

\newcommand{\mytwodvector}[2]{
\left [
\begin{array}{r}
#1\\
#2
\end{array}
\right ]
}

\newcommand{\mythreedvector}[3]{
\left [
\begin{array}{r}
#1\\
#2\\
#3
\end{array}
\right ]
}

\newcommand{\mytwodsystemvector}[6]{
\left [
\begin{array}{rr}
#1 & #2\\
#4 & #5
\end{array}
\right ]
\:
\left [
\begin{array}{r}
x \\
y 
\end{array}
\right ]
&=&
\left [
\begin{array}{r}
#3\\
#6
\end{array}
\right ]
}

\newcommand{\mythreedarray}[9]{
\left [
\begin{array}{rrr}
#1 & #2 & #3\\
#4 & #5 & #6\\
#7 & #8 & #9
\end{array}
\right ]
}

\newcommand{\myodetwo}[6]{
#1 \: #6^{\prime \prime}(t) \: #2 \: #6^{\prime}(t) \: #3 \: #6(t) &=& 0\\
#6(0)                                           &=& #4\\
#6^{\prime}(0)                                  &=& #5
}

\newcommand{\myodetwoNoIC}[4]{
#1 \: #4^{\prime \prime}(t) \: #2 \: #4^{\prime}(t) \: #3 \: #4(t) &=& 0
}

\newcommand{\myodetwopacked}[5]{
\hspace{-0.3in}& & #1 u^{\prime \prime} #2 u^{\prime} #3 u \: = \: 0\\
\hspace{-0.3in}& & u(0) \: = \: #4, \: \: u^{\prime}(0)    \: = \:  #5
}

\newcommand{\myodetwoforced}[6]{
#1\: u^{\prime \prime}(t) \: #2 \: u^{\prime}(t) \: #3 \: u(t) &=& #6\\
u(0)                                           &=& #4\\
u^{\prime}(0)                                  &=& #5\\
}

\newcommand{\myodesystemtwo}[8]{
#1 \: x^\prime(t) \: #2 \: y^\prime(t) \: #3 \: x(t) \: #4 \: y(t) &=& 0\\
#5 \: x^\prime(t) \: #6 \: y^\prime(t) \: #7 \: x(t) \: #8 \: y(t) &=& 0\\
}

\newcommand{\myodesystemtwoic}[2]{
x(0)                                       &=& #1\\ 
y(0)                                       &=& #2
}

\newcommand{\mypredprey}[4]{
x^\prime(t) &=& #1 \: x(t) \: - \: #2 \: x(t) \: y(t)\\
y^\prime(t) &=& -#3 \: y(t) \: + \: #4 \: x(t) \: y(t)
}

\newcommand{\mypredpreypacked}[4]{
x^\prime &=& #1 \: x - #2 \: x \: y\\
y^\prime &=& -#3 \: y + #4 \: x \: y
}

\newcommand{\mypredpreyself}[6]{
x^\prime(t) &=&  #1 \: x(t) \: - \: #2 \: x(t) \: y(t) \: - \: #3 \: x(t)^2\\
y^\prime(t) &=& -#4 \: y(t) \: + \: #5 \: x(t) \: y(t) \: - \: #6 \: y(t)^2
}

\newcommand{\mypredpreyfish}[5]{
x^\prime(t) &=&  #1 \: x(t) \: - \: #2 \: x(t) \: y(t) \: - \: #5 \: x(t)\\
y^\prime(t) &=& -#3 \: y(t) \: + \: #4 \: x(t) \: y(t) \: - \: #5 \: y(t)
}

\newcommand{\myepidemic}[4]{
S^\prime(t) &=& - #1 \: S(t) \: I(t)\\
I^\prime(t) &=&   #1 \: S(t) \: I(t) \: - \: #2 \: I(t)\\
S(0)        &=&   #3\\
I(0)        &=&   #4\\
}

\newcommand{\bsred}[1]{%
\textcolor{red}{\boldsymbol{#1}}
}

\newcommand{\bsblue}[1]{%
\textcolor{blue}{\boldsymbol{#1}}
}


\newcommand{\myfloor}[1]{%
\lfloor{#1}\rfloor
}

\newcommand{\cubeface}[7]{%
\begin{bmatrix}
\bs{#3}          & \longrightarrow & \bs{#4}\\
\uparrow          &                         &  \uparrow  \\
\bs{#1} & \longrightarrow & \bs{#2}\\
              & \text{ \bfseries #5:} \: \bs{#6} \: \text{\bfseries  #7 } & 
\end{bmatrix}
}

\newcommand{\cubefacetwo}[5]{%
\begin{bmatrix}
\bs{#3}          & \longrightarrow & \bs{#4}\\
\uparrow          &                         &  \uparrow  \\
\bs{#1} & \longrightarrow & \bs{#2}\\
              & \text{ \bfseries #5} & 
\end{bmatrix}
}

\newcommand{\cubefacethree}[9]{%
\begin{bmatrix}
\bs{#3}                  & \overset{#9}{\longrightarrow} & \bs{#4}\\
\uparrow \: #7         &                                             &  \uparrow  \: #8 \\
\bs{#1}                  & \overset{#6}{\longrightarrow} & \bs{#2}\\
                               & \text{ \bfseries #5} & 
\end{bmatrix}
}

\renewcommand{\qedsymbol}{\hfill \blacksquare}
\newcommand{\subqedsymbol}{\hfill \Box}
%\theoremstyle{plain}

\newtheoremstyle{mystyle}% name
  {6pt}%      Space above
  {6pt}%      Space below
  {\itshape}%         Body font
  {}%         Indent amount (empty = no indent, \parindent = para indent)
  {\bfseries}% Thm head font
  {}%        Punctuation after thm head
  { }%     Space after thm head: " " = normal interword space; \newline = linebreak
  {}%         Thm head spec (can be left empty, meaning `normal')
\theoremstyle{mystyle}
 
\newtheorem{axiom}{Axiom}
%\newtheorem{solution}{Solution}[section]
\newtheorem*{solution}{Solution}
\newtheorem{exercise}{Exercise}[section]
\newtheorem{theorem}{Theorem}[section]
\newtheorem{proposition}[theorem]{Proposition}
\newtheorem{prop}[theorem]{Proposition}
\newtheorem{assumption}{Assumption}[section]
\newtheorem{definition}{Definition}[section]
\newtheorem{comment}{Comment}[section]
\newtheorem*{question}{Question}
\newtheorem{program}{Program}[section]
%\newtheorem{myproof}{Proof}
%\newtheorem*{myproof}{Proof}[section]
\newtheorem{myproof}{Proof}[section]
\newtheorem{hint}{Hint}[section]
\newtheorem*{phint}{Hint}
\newtheorem{lemma}[theorem]{Lemma}
\newtheorem{example}{Example}[section]
      
\newenvironment{myassumption}[4]
{
\centering
\begin{assumption}[{\textbf{#1}\nopunct}]%
\index{#2}
\mbox{}\\  \vskip6pt \colorbox{black!15}{\fbox{\parbox{.9\textwidth}{#3}}}
\label{#4}
\end{assumption}
%\renewcommand{\theproposition}{\arabic{chapter}.\arabic{section}.\arabic{assumption}} 
}%
{}

\newenvironment{myproposition}[4]
{
\centering
\begin{proposition}[{\textbf{#1}\nopunct}]%
\index{#2} 
\mbox{}\\  \vskip6pt \colorbox{black!15}{\fbox{\parbox{.9\textwidth}{#3}}}
\label{#4}
\end{proposition}
%\renewcommand{\theproposition}{\arabic{chapter}.\arabic{section}.\arabic{proposition}} 
}%
{}

\newenvironment{mytheorem}[4]
{
\centering
\begin{theorem}[{\textbf{#1}\nopunct}]%
\index{#2} 
\mbox{}\\ \vskip6pt \colorbox{black!15}{\fbox{\parbox{.9\textwidth}{#3}}}
\label{#4}
\end{theorem}
%\renewcommand{\thetheorem}{\arabic{chapter}.\arabic{section}.\arabic{theorem}} 
}%
{}

\newenvironment{mydefinition}[4]
{
\centering
\begin{definition}[{\textbf{#1}\nopunct}]%
\index{#2} 
\mbox{}\\  \vskip6pt \colorbox{black!15}{\fbox{\parbox{.9\textwidth}{#3}}}
\label{#4}
\end{definition}
%\renewcommand{\thedefinitio{n}{\arabic{chapter}.\arabic{section}.\arabic{definition}} 
}%
{}

\newenvironment{myaxiom}[4]
{
\centering
\begin{axiom}[{\textbf{#1}\nopunct}]%
\index{#2} 
\mbox{}\\  \vskip6pt \colorbox{black!15}{\fbox{\parbox{.9\textwidth}{#3}}}
\label{#4}
\end{axiom}
%\renewcommand{\theaxiom}{\arabic{chapter}.\arabic{section}.\arabic{axiom}} 
}%
{}

\newenvironment{mylemma}[4]
{
\centering
\begin{lemma}[{\textbf{#1}\nopunct}]%
\index{#2} 
\mbox{}\\  \vskip6pt \colorbox{black!15}{\fbox{\parbox{.9\textwidth}{#3}}}
\label{#4}
\end{lemma}
%\renewcommand{\thelemma}{\arabic{chapter}.\arabic{section}.\arabic{lemma}} 
}%
{}
   
\newenvironment{reason}[1]
{
 \vskip0.05in
 \begin{myproof}
 \mbox{}\\#1
 $\qedsymbol$
 \end{myproof}  
 \vskip0.05in
}%
{}

\newenvironment{reasontwo}[1]
{
 \vskip+.05in
 \begin{myproof}
 \mbox{}\\#1
 \end{myproof}  
 \vskip+0.05in
}%
{}

\newenvironment{subreason}[1]
{
 \vskip0.05in
 \renewcommand{\themyproof}{}
 \begin{myproof}
 #1
 $\subqedsymbol$
 \end{myproof}
 \vskip0.05in
 \renewcommand{\themyproof}{\thetheorem}
 %\renewcommand{\themyproof}{\arabic{chapter}.\arabic{section}.\arabic{myproof}}   
 %
}%
{}  

\newenvironment{myhint}[1]
{
 \vskip0.05in
 \begin{hint}
 #1
 $\subqedsymbol$ 
 \end{hint}  
 \vskip0.05in
}%
{} 

\newenvironment{myeqn}[3]
{
 \renewcommand{\theequation}{$\boldsymbol{#1}$}
 \begin{eqnarray}
 \label{equation:#2}
 #3 
 \end{eqnarray}
 \renewcommand{\theequation}{\arabic{chapter}.\arabic{eqnarray}}   
}%
{} 


\JournalInfo{MATH 8210:  Homework six, 1-\pageref{LastPage}, 2020} % Journal information
\Archive{Draft Version \today} % Additional notes (e.g. copyright, DOI, review/research article)

\PaperTitle{MATH 8210 Homework Six}
\Authors{Ian Davis\textsuperscript{1}}
\affiliation{\textsuperscript{1}\textit{John E. Walker Department of Economics,
Clemson University,Clemson, SC: email ijdavis@g.clemson.edu}}
%\affiliation{*\textbf{Corresponding author}: yournamehere@clemson.edu} % Corresponding author

\date{\small{Version ~\today}}
\Abstract{Operators, Completeness and ODEs}
\Keywords{}
\newcommand{\keywordname}{Keywords}
%
\onehalfspacing
\begin{document}

\flushbottom

\addcontentsline{toc}{section}{Title}
\maketitle

\renewcommand{\theexercise}{\arabic{exercise}}%

\section{The Existence of a Solution to a Nonlinear IVP ODE}

Let $f: [t_0,t_1] \times [a,b] \rightarrow \Re$ be locally Lipschitz on its
domain.  This means at each
$(t_0,x_0) \in [0,1] \times [a,b]$, there is a positve $\delta$ and a positive constant $L$
so that $|f(t,x) - f(t_0,x_0)| \leq L|s-t_0| + L|x - x_0|$ if $(t,x) \in B(t_0,x_0); \delta)$.
Note $L$ and $\delta$ depending on $(t_0,x_0$ means they
could be written $L^{t_0,x_0}$ and $\delta^{t_0,x_0}$ but we won't
do that as it is too messy notationally.  Just remember they depend on
the location in the domain.  Consider the IVP
\begin{eqnarray*}
x^\prime(t) &=& f(t,x(t)), \quad x(t_0) = x_0
\end{eqnarray*}
\noindent
This is clearly equivalent to the integral equation
\begin{eqnarray*}
x(t) &=& x_0 + \int_{t_0}^t f(s,x(s))ds
\end{eqnarray*}
\noindent

\subsection{Preliminaries}

\begin{exercise}
Prove such a locally Lipschitz function is also continuous.\
\end{exercise}

\begin{solution}
consider $\delta  = \frac{\varepsilon}{2L}$, then $(t,x) \in B((t_0,x_0);\delta) \implies |t - t_0| < \frac{\varepsilon}{2L}$ and $|x - x_0| < \frac{\varepsilon}{2L}$. Hence,
\begin{eqnarray*}
  (t,x) &\in& B((t_0,x_0);\delta)\\
  &\implies& |f(t,x) - f(t_0,x_0)| \leq L|t - t_0| + L|x - x_0|\\
  &<& L(\frac{\varepsilon}{2L}) + L(\frac{\varepsilon}{2L}) = \varepsilon
\end{eqnarray*}

\end{solution}

\begin{exercise}
Prove any such function which has continuous partial derivatives
is locally Lipschitz.
\end{exercise}

\begin{solution}
Consider first the partial derivative occuring when $x$ is held constant. Based on the continuity proven above and implied from the continuous partial derivate, we know the assumptions required for the Mean Value Theorem hold and $\exists s \in (t,t_0) \ni$
\begin{eqnarray*}
  \frac{|f(t,x) - f(t_0,x)|}{|t - t_0|} &=& f'(s)\\
  |f(t,x) - f(t_0,x)| = f'(s)|t - t_0|
\end{eqnarray*}
Now, if we set $L_t = \{\max\{f'(s)\} \:|\: s \in (t,t_0) \}$ we get
\begin{eqnarray*}
  |f(t,x) - f(t_0,x)| \leq L_t|t - t_0|
\end{eqnarray*}
We use the same process with $t$ being held constant to retrieve $L_x$ such that
\begin{eqnarray*}
  |f(t,x) - f(t,x_0)| \leq L_t|x - x_0|
\end{eqnarray*}
We then set $M = \max\{L_t,L_x\}$ then $(t,x) \in B((t_0,x_0);\delta) \implies$
\begin{eqnarray*}
  |f(t,x) - f(t_0,x_0)| \leq L|t - t_0| + L|x - x_0|
\end{eqnarray*}
Satisfying the Lipschitz condition
\end{solution}

\begin{exercise}
\item Define $T$ acting on $C([t_0,t_1])$ by 
$(T(x))(t) = x_0 + \int_{t_0}^t  f(s,x(s))ds$
for all $t \in [t_0,t_1]$.  Prove $T(x)$ is continuous on $[t_0,t_1]$
and hence $T : C([t_0,t_1]) \rightarrow C([t_0,t_1])$ is well defined.
\end{exercise}

\begin{solution}
We want to show that $\forall \varepsilon > 0 \exists \delta > 0 \ni |t - t'| < \delta \implies$
\begin{eqnarray*}
  |(T(x))(t) - (T(x))(t')| < \varepsilon
\end{eqnarray*}
So now lets look futher into $|(T(x))(t) - (T(x))(t')|$ where we assuming $t' > t$
\begin{eqnarray*}
  |(T(x))(t) - (T(x))(t')| &=& |x_0 + \int_{t_0}^{t}f(s,x(s))ds - x_0 - \int_{t_0}^{t'}f(s,x(s))ds|\\
  &=& |\int_{t_0}^{t}f(s,x(s))ds - \int_{t_0}^{t'}f(s,x(s))ds| 
  \\
  &=& |\int_{t}^{t'}f(s,x(s))ds|
\end{eqnarray*}
Becuase f is integrable, we know it must be bounded. Hence, $\exists M \geq |f(s,x(s))| \forall s \in [t,t']$ which means
\begin{eqnarray*}
  \int_t^{t'}f(s,x(s))ds \leq \int_t^{t'}Mds = M|t' - t|
\end{eqnarray*}
So, if we set $\delta = \frac{\varepsilon}{M}$ then 
\begin{eqnarray*}
  \int_t^{t'}f(s,x(s))ds \leq \int_t^{t'}Mds = M|t' - t| < M(\frac{\varepsilon}{M}) = \varepsilon
\end{eqnarray*}
The arguement is almost identical if $t' < t$ and we can conclude $(T(x))(t)$ is continuous.
\end{solution}


\begin{exercise}
Define $F$ on $[t_0,t_1]$ by $F(t) = f(t,x(t))$.  Prove
$F$ is continuous on $C([t_0,t_1])$.
\end{exercise}

\begin{solution}
We want to show that $\forall \varepsilon > 0 \exists \delta > 0 \ni |t - t_0| \implies$
\begin{eqnarray*}
  |F(t) - F(t_0)| < \varepsilon
\end{eqnarray*}
Consider now what $|F(t) - F(t_0)|$ is and recall the Lipschitzity of $f$
\begin{eqnarray*}
  |F(t) - F(t_0)| &=& |f(t,x(t) - f(t_0,x(t_0)|\\
  \leq L|t - t_0| + L|x(t) - x(t_0)|
\end{eqnarray*}
Because we are only looking at the ball $B(t_0,x(t_0);\delta)$, we are still able to preserve $|t - t_0| < \delta$ and $|x(t) - x(t_0)| < \delta$. We define a new $d_t \ni |t - t_0| < \delta_t \implies |x(t) - x(t_0)| < \delta$. We then set $\frac{\varepsilon}{2L} = max\{\delta,\delta_t\}$ and we can use a similar proof to that used in exercise 1 to get $|t - t'| < \delta_t \implies |x(t) - x(t)_0| < \delta_f$ and 
\begin{eqnarray*}
  |F(t) - F(t_0)| &\leq& L|t - t_0| + L|x(t) - x(t_0)|\\
  &<& L(\frac{\varepsilon}{2L}) + L(\frac{\varepsilon}{2L}) = \varepsilon
\end{eqnarray*}
so $F(t)$ is continuous.
\end{solution}

\begin{exercise}
Let $y = T(x)$.  Use the Fundamental Theorem of Calculus
to prove $y^\prime = f(t,x(t))$ on $[t_0,t_1]$.
\end{exercise}

\begin{solution}
First, we recall the implications of the Fundamental Theorem of Calculus. Specifically, with $f \in RI[a,b]$ and $f:[a,b] \rightarrow R$ by $F(x) = \int_a^x\:f(t)dt$ then
 \begin{enumerate}
   \item $F \in C[a,b]$
   \item $F'(c) = f(c)$
 \end{enumerate}
Contextualizing this theorem within our problem, we get, becuase we have already established the continuity of $(T(x))(t)$ on $[t_0,t_1]$ so the assumptions needed for the theorem are satisfied. Hence,
\begin{eqnarray*}
  y &=& (T(x))(t) = x_0 + \int_{t_0}^t\:f(s,x(s))ds\\
  y' &=& (T(x))'(t) = 0 + f(t,x(t)) = f(t,x(t))
\end{eqnarray*}
by FTOC.
\end{solution}

\begin{exercise}
Prove T is a compact operator. T is compact if it maps a closed bounded subset of its domain into a sequentially compact set in the range. This is a problem involving equicontinuity.
\end{exercise}

\begin{solution}
Becuase we have proven $(T(x))(t)$ is well definined and continuous on $C([t_0,t_1])$, fo a general $x(t)$, we can conlude $(T(x))(t)$ is bounded and equicontinuous on the finite $[a,b]$. Hence, we can use Azrela - Ascoli to say $\exists (x_n) \subset T$ that converges uniformly to a continuous function $x$ on $[t_0,t_1]$. In other words, it maps a closed and bounded subset of its domain into a sequentially compact set in the range, and T is a compact operator
.
\end{solution}

\subsection{Proving Local Existence for a IVP}

To prove the IVP has a local solution on some interval
$[t_0,t_0+\Delta] \subset [t_0,t_1]$, it is enough to show
the operator $I-T$ has a fixed point on the space $C([t_0,t_0+\Delta])$.
Take some time to process this.  We need a solution $x$ whose 
ordered pairs $(t,x(t)) \in [t_0,t_1] \times [a,b]$ always.  This means
$x(t)$ lies in $[a,b]$ always.  We need this fact to be able to use the
local Lipschitzity of $f$.\\

\noindent
Choose $x_0(t) = x_0$ on $[t_0,t_1]$.
Assume $x_0$ is an interior point of $[a,b]$
although it is easy enough to alter our arguments to be one
sided.  Then there is a positive $r$ so that $B(x_0;r) \subset (a,b)$.
Define $x_1 = T(x_0)$, $x_2 = T(x_1)$
and in general, $x_{n+1} = T(x_n)$.  Note at each
iteration, we must make sure the function we put in $f$
has a range that fits.   So this means we must be careful
how much of the interval $[t_0,t_1]$ we use at each step.
We will control this using the local Lipschitz property of $f$.
So let's focus on the interval $[t_0, t_0 + \Delta]$ and figure out
how to choose $\Delta$.  Now let $\|f\|^{x_0}_\infty = \max_{t_0 \leq s \leq t_1} |f(s,x_0)|$.

\begin{exercise}
Prove
\begin{itemize}
\item 
\begin{eqnarray*}
|x_1(t) - x_0(t)| &\leq& \|f\|^{x_0}_\infty \Delta
\Longrightarrow
\|x_1 - x_0\|_\infty \leq \|f\|^{x_0}_\infty \Delta
\end{eqnarray*}
\noindent
This implies we must have $\|f\|^{x_0}_\infty \Delta < r$.
\item
\begin{eqnarray*}
|x_2(t) - x_1(t)| &\leq& (L \Delta) \|x_1 - x_0\|_\infty
\Longrightarrow
\|x_2 - x_1\|_\infty \leq  ( L \Delta)  \|x_1 - x_0\|_\infty \leq  ( L \Delta) (\|f\|^{x_0}_\infty \Delta)
\end{eqnarray*}
\noindent
We need to have $x_2$ in the ball where $f$ is locally Lipschitz about $x_0$.
By the triangle inequality,
\begin{eqnarray*}
|x_2(t) - x_0(t)| &\leq& |x_2(t) - x_1(t)| + |x_1(t) - x_0(t)| 
                           \leq ( L \Delta) (\|f\|^{x_0}_\infty \Delta) (1 + (L \Delta) )
\end{eqnarray*}
\noindent
Now if we assume $L \Delta < 1$, we know $1 + L \Delta < \frac{1}{1 - (L \Delta)}$
because we know how to sum a geometric series.  Hence,
\begin{eqnarray*}
\|x_2 - x_0\|_\infty & \leq & ( L \Delta) (\|f\|^{x_0}_\infty \Delta) \frac{1}{1 - (L \Delta)}
 =L  \|f\|^{x_0}_\infty \frac{ \Delta}{1 - (L \Delta)}
\end{eqnarray*}
\noindent
Since $\frac{ \Delta}{1 - (L \Delta)} \rightarrow 0$ as $\Delta \rightarrow 0$,
we see we can choose $\Delta$ so that $\|x_2 - x_0\|_\infty < \delta$.
Hence, $x_2$ is in the ball about $x_0$ where the local Lipschitz property holds.
\item Since $x_2$ is in the ball about $x_0$, it follows
\begin{eqnarray*}
|x_3(t) - x_2(t)| &\leq& (L \Delta) \|x_2 - x_1\|_\infty
\Longrightarrow
\|x_3 - x_2\|_\infty \leq (L \Delta)^2  \Delta  \|x_1 - x_0\|_\infty
\end{eqnarray*}
\noindent
and we can show
\begin{eqnarray*}
\|x_2 - x_0\|_\infty & \leq & (L  \|f\|^{x_0}_\infty) \frac{ \Delta}{1 - (L \Delta)} < \delta
\end{eqnarray*}
\end{itemize}
\end{exercise}

\begin{solution}
Assuming $|x_1(t) - x_0(t)| \leq \|f\|_\infty^{x_0}$, consider the implications
\begin{eqnarray*}
  |x_1(t) - x_0(t)| &=& |T(x_0) - x_0|\\
  &=& |x_0 + \int_{t_0}^t\:f(s,x_0)ds - x_0|\\
  &=& \int_{t_0}^t\:|f(s,x_0)|ds\\
  &\leq& \int_{t_0}^t\:\|f(s,x_0)\|_{\infty}ds\\
  &=& \|f(s,x_0)\|_{\infty}\\
  &\leq& \|f\|_{\infty}^{x_0} \Delta
\end{eqnarray*}
So now, we consider
\begin{eqnarray*}
  \|x_1 - x_0\|_{\infty} &=& \|T(x_0) - x_0\|_{\infty}\\
  &=& \|x_0 + \int_{t_0}^t\:f(s,x_0)ds - x_0\|_{\infty}\\
  &=& \|\int_{t_0}^t\:f(s,x_0)ds\|_{\infty}\\
  &\leq& \|f\|_{\infty}^{x_0} \Delta
\end{eqnarray*}
For the second bullter, we start from
\begin{eqnarray*}
  |x_2(t) - x_1(t)| \leq \|x_1 - x_0\|_\infty(l\Delta)
\end{eqnarray*}
and get
\begin{eqnarray*}
  |x_2(t) - x_1(t)| = |t(x_1(t)) - T(x_0)|\\
  &=& |x_0 + \int_{t_0}^t\:f(s,x_1(s))ds - x_0 - \int_{t_0}^t\;f(s,x_0(s)ds)|\\
  &=& |\int_{t_0}^t\:f(s,x_1(s))ds - \int_{t_0}^t\:f(s,x_0(s))ds |\\
  &\leq& \|x_2 - x_1\|_\infty(t - t_0) \leq \|x_1 - x_0\|_\infty(L\Delta)\\
  &\leq& (L\Delta)(\|f\|_\infty^{x_0}\Delta) \; \text{By the first argument}
\end{eqnarray*}
Finally, we consider starting from
\begin{eqnarray*}
  |x_3(t) - x_2(t)| &\leq& (L\Delta)\|x_2 - x_1\|_\infty
\end{eqnarray*}
We get
\begin{eqnarray*}
  |x_3(t) - x_2(t)| &\leq& \|x_3 - x_2\|_\infty\\
  &\leq& (L\Delta)\|x_2 - x_1\|_\infty\\
  &\leq& (L\Delta)(L\Delta)\|x_1 - x_0\|_\infty\\
  &\leq& (L\Delta)^2 \Delta \|f\|_\infty^{x_0}
\end{eqnarray*}
\end{solution}

\begin{exercise}
Now prove
\begin{eqnarray*}
\|x_{n+1} - x_n\|_\infty \leq (L \Delta)^n  \Delta  \|x_1 - x_0\|_\infty
\end{eqnarray*}
\noindent
and $x_{n+1}$ is in the ball about $x_0$ where local Lipschitzity holds.
\end{exercise}

\begin{solution}
We want to use induction for this proof.
\begin{enumerate}
  \item We have proven the base case in the above exercise.
  \item Assume
  \begin{eqnarray*}
    \|x_{k+1} - x_{k} \|_\infty \leq (L\Delta)^k\Delta\|x_1 - x_0\|_\infty
  \end{eqnarray*}
  When the Lipschitzity conditions hold
  \item Now, we want to prove
  \begin{eqnarray*}
    \|x_{k+2} - x_{k+1} \|_\infty \leq (L\Delta)^{k+1}\Delta\|x_1 - x_0\|_\infty
  \end{eqnarray*}
  So consider
  \begin{eqnarray*}
    \|x_{k+2}(t) - x_{k+1}(t) \| &\leq& (L\Delta)\|x_{k+1} - x_k\|_\infty \; \text{by Lipschitz requirements}\\
    &\leq& (L\Delta)(L\Delta)^k\|x_1 - x_0\|_\infty\\
    &=& (L\Delta)^{k+1}\|x_1 - x_0\|_\infty
  \end{eqnarray*}
\end{enumerate}
And we have proven $\|x_{n+1} - x_n\|_\infty \leq (L \Delta)^n  \Delta  \|x_1 - x_0\|_\infty$ by induction.
\end{solution}

\begin{exercise}
Now prove for any positive $p$
\begin{eqnarray*}
\|x_{n+p} - x_n\|_\infty \leq (L \Delta)^p  \biggl ( \frac{1}{1 - (L \Delta)} \biggr)
\end{eqnarray*}
\end{exercise}

\begin{solution}
Again, we want to use induction.
\begin{enumerate}
  \item The base case
  \begin{eqnarray*}
    \|x_{2} - x_0\|_\infty \leq (L \Delta)^2  \biggl ( \frac{1}{1 - (L \Delta)} \biggr) 
  \end{eqnarray*}
  By the logic of exercise 7.
  \item assume
  \begin{eqnarray*}
    \|x_{k+p} - x_k\|_\infty \leq (L \Delta)^p  \biggl ( \frac{1}{1 - (L \Delta)} \biggr)
  \end{eqnarray*}
  \item No we consider the $k + 1$ case
  \begin{eqnarray*}
    \|x_{k+p + 1} - x_{k + 1}\|_\infty &\leq& (L \Delta)  \biggl \|x_{k + p} - x_k\|_\infty\\
    &\leq& (L \Delta) (L \Delta)^p  \biggl ( \frac{1}{1 - (L \Delta)} \biggr)\\
    &\leq& (L \Delta)^{p+1}  \biggl ( \frac{1}{1 - (L \Delta)} \biggr)
  \end{eqnarray*}
\end{enumerate}
and we have proven $\|x_{n+p} - x_n\|_\infty \leq (L \Delta)^p  \biggl ( \frac{1}{1 - (L \Delta)} \biggr)$ by induction.
\end{solution}

\begin{exercise}
Prove the sequence $(x_n)$ is a Cauchy Sequence in $(C([t_0,t_0+\Delta]), \| \cdot \|_\infty)$
and so there is a function $x \in C([t_0,t_0+\Delta])$ so that $x_n$ converges
uniformly to $x$.
\end{exercise}

\begin{solution}
If we pick a $\Delta$ such that, given $\varepsilon$ and $L$,
\begin{eqnarray*}
  \frac{L\Delta}{1 - L\Delta} < \varepsilon
\end{eqnarray*}
and we get
\begin{eqnarray*}
  \|x_n - x_m\|_\infty &\leq& (L\Delta)^{|n - m|}(\frac{1}{1 - L\Delta})\\
  &\leq& \frac{L\Delta}{1 - L\Delta}\\
  &<& \varepsilon
\end{eqnarray*}
and $x_n$ is Cauchy in $(C([t_0,t_0+\Delta])$ and then there is a function $x \in C([t_0,t_0+\Delta])$ so that $x_n$ converges
uniformly to $x$.
\end{solution}

\begin{exercise}
Using standard add and subtract tricks show that
$\|x - T(x)\|_\infty < \epsilon$ for all $\epsilon > 0$.
Hence $\|x - T(x)\|_\infty = 0$ and so $x = T(x)$.
Of course, this $x$ solves the IVP as well.
So we have found there is a solution to the IVP
that exists locally at $t_0$.
\end{exercise}

\begin{solution}
Your solution here.
\end{solution}

\noindent
Here is a thought exercise.  Would it be hard to extend
this result to $x^\prime(t) = f(t,x(t))$ with $x(t_0) = x_0$ to the
$n$ - dimensional case?
Of course, once you have this local solution, you can go to its
endpoint and extend again.  Essentially, you can extend
until you reach a boundary point where the behavior of $f$
becomes bad; i.e. $f$ {\em blows} up.  This solution
would then be called the maximal interval of existence.
If $f$ is nice always, then we would be able to extend
to $[t_0, \infty)$.

\section{Normed Spaces}

\begin{exercise}
Prove if the normed linear space $(X, \| \cdot\|)$ has a
Schauder Basis it is separable.
\end{exercise}

\begin{solution}
Let $(X,\|\cdot\|)$ be a NLS with Schauder Basis $E = (e_n)_{n=1}^\infty$ so, given $x \in X \; \exists \; N \ni n > N \implies$
\begin{eqnarray*}
  \|\sum_{i = 1}^\infty\alpha_ie_i - x \| < \frac{\varepsilon}{2}
\end{eqnarray*}
Next, we define the countable set
\begin{eqnarray*}
  Q = \{\sum_{i = 1}^nq_ie_i \; | \; q_i \in \mathbb{Q}\}
\end{eqnarray*}
Now, we know, due to $\mathbb{Q}$'s density in $\mathbb{R}$, $\exists \; q \in \mathbb{Q} \ni |q_i - \alpha_i| < \frac{\varepsilon}{2}$.\\
\\
Finally, setting $y = \sum_{i = 1}^\infty q_ie_i \in Q$ we get
\begin{eqnarray*}
  \|x - y\| &\leq& \|x - \sum_{i = 1}^\infty \alpha_i e_i \| + \|\sum_{i = 1}^\infty \alpha_i e_i  - \sum_{i = 1}^\infty q_i e_i \|\\
  &<& \frac{\varepsilon}{2} + \frac{\varepsilon}{2} = \varepsilon
\end{eqnarray*}
Hence, X is countable and dense meaning X is separable.
\end{solution}

\begin{exercise}
Explain why $\ell^1$ is not reflexive.
\end{exercise}

\begin{solution}
Recall the implication of reflexivitiy being $x \cong X'$ or, more specifically, $\exists \; \alpha_i: X \rightarrow X''$ that is linear, 1 - 1, onto, and norm preserving. Also remember that we know
\begin{enumerate}
  \item $\ell^1$ is separable
  \item $(\ell^1)1 \cong \ell^infty$
  \item $\ell^\infty$ is not separable
\end{enumerate}
and $\ell^1$ being reflexive would imply $\ell^1 \cong (\ell^1)'' = ((\ell^1)')'$. But because we know $X'$ being separable implies $X$ is separablethis would mean $(\ell^1)'$ is also separable which we know to be not true. 
\end{solution}

\section{The Completion of $C([0,1])$ Using $d_2$}

\begin{exercise}
Find a Cauchy Sequence of Riemann Integrable functions which gives a representative for $f(t) = t^{-\frac{3}{4}}$
\end{exercise}

\begin{solution}
Conider the function 
\begin{eqnarray*}
x_n(t) &=&
\left \{
\begin{array}{ll}
0, & 0 \leq t \leq \frac{1}{n}\\
t^{-\frac{3}{4}}, & \frac{1}{n} < t \leq 1
\end{array}
\right .
\end{eqnarray*}
which is clearly Riemann Integrable. Now we consider, for $n > m$
\begin{eqnarray*}
  \int_0^1\:|x_n(t) - x_m(t)|^2dt\\
  &=& \int_\frac{1}{n}^\frac{1}{m}\:|t^{-\frac{2}{3}} - 0|^2dt\\
  &=& |-3(\frac{1}{t^\frac{1}{3}})||_\frac{1}{n}^\frac{1}{m}\\
  &=& 3|m^\frac{1}{3} - n^\frac{1}{3}|\\
  &=& C_{nm}|\frac{1}{m^\frac{1}{3}} - \frac{1}{n^\frac{1}{3}}| \;\; \text{(By MVT)}\\
  &<& \frac{1}{m}|\frac{n^\frac{4}{3} - m^\frac{4}{3}}{(mn)^\frac{4}{3}}|\\
  &<& \frac{1}{m^\frac{8}{3}}
\end{eqnarray*}
\end{solution}

\end{document}