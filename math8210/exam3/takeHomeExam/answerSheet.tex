\documentclass[11pt]{article}
% use ams math packages
\usepackage{amsmath,amsthm,amssymb,amsfonts}
\usepackage{mathrsfs}
%
\usepackage{xcolor}
\usepackage{fancyvrb}
\usepackage{listings}
%
\usepackage{geometry}

\renewcommand{\qedsymbol}{\hfill \blacksquare}
\newcommand{\subqedsymbol}{\hfill \Box}
%\theoremstyle{plain}

\newtheoremstyle{mystyle}% name
  {6pt}%      Space above
  {6pt}%      Space below
  {\itshape}%         Body font
  {}%         Indent amount (empty = no indent, \parindent = para indent)
  {\bfseries}% Thm head font
  {}%        Punctuation after thm head
  { }%     Space after thm head: " " = normal interword space; \newline = linebreak
  {}%         Thm head spec (can be left empty, meaning `normal')
\theoremstyle{mystyle}
 
\newtheorem*{solution}{solution}
\newtheorem*{theorem}{Theorem}

\newenvironment{mysolution}[1]
{
\centering
\begin{solution}%
\mbox{}\\ \vskip6pt \colorbox{black!15}{\fbox{\parbox{.9\textwidth}{#1}}}
 $\qedsymbol$
\end{solution}
}%
{}
      
\newenvironment{mytheorem}[1]
{
\centering
\begin{theorem}%
\mbox{}\\ \vskip6pt \colorbox{black!15}{\fbox{\parbox{.9\textwidth}{#1}}}
\end{theorem}
}%
{}

\newenvironment{reason}[1]
{
 \vskip0.05in
 \begin{myproof}
 \mbox{}\\#1
 $\qedsymbol$
 \end{myproof}  
 \vskip0.05in
}%
{}

\newcommand{\bs}[1]{
\boldsymbol{#1}
}

\pagestyle{plain}
\geometry{lmargin=1.0in,rmargin=1.0in,top=1.0in,bottom=1.0in}
\pagestyle{plain}

\title{Take Home Exam 3: MATH 8210: Spring 2020}
\author{Dr. Peterson}
\date{\today}

\begin{document}
\maketitle

\noindent
\textbf{Ian Davis}\\
This is a closed book and closed notes test.  Make sure
to give me all the details of your arguments!

\begin{description} 
\item[Part 1: Definitions (28 Points)] \mbox{}\\
  \begin{enumerate}
  \item(4 Points) Define what an inner product on the set $X$ is carefully.
                           Do this for both the real and complex field.
       \begin{mysolution}
       {
       \textbf{Real Inner Product Space: }\\
       Let $V$ be a vector space with reals as the scalar field. Then a mapping $\omega$ which assigns a pair of objects in $V$ to a real number is called an inner produc space if
       \begin{enumerate}
        \item $\omega(\mathbf{u},\mathbf{v}) = \omega(\mathbf{v},\mathbf{u})$; the order is important and switching the order will conjugate the result.
        \item $\omega(c\odot\mathbf{u},\mathbf{v}) = c\omega(\mathbf{u},\mathbf{v})$; scalars in the first slot can be pulled out. 
        \item $\omega(\mathbf{u} \oplus \mathbf{v},\mathbf{w}) = \omega(\mathbf{u},\mathbf{w}) + \omega(\mathbf{v},\mathbf{w})$; for any three objects in V
        \item $\omega(\mathbf{u},\mathbf{u}) \geq 0$ and $\omega(\mathbf{u},\mathbf{v}) = 0 \iff \mathbf{u} = 0$
       \end{enumerate}
       Additionally, these properties imply that
       \begin{eqnarray*}
            \omega(\mathbf{u},c\odot\mathbf{v}) = c\omega(\mathbf{u},\mathbf{v})
       \end{eqnarray*}
       A vector space V with a real inner product is called \textbf{Real Inner Product Space}.\\
       \textbf{Complex Inner Product Space: }\\
       \\
       Let $V$ be a vector space with the complex numbers as the scalar field. Then a mapping $\omega$ which assigns a pair of objects in $V$ to a complex number is an inner product space if
       \begin{enumerate}
        \item $\omega(\mathbf{u},\mathbf{v}) = \overline{\omega(\mathbf{v},\mathbf{u})}$; the order of the two objects is unimportnat
        \item $\omega(c\odot\mathbf{u},\mathbf{v}) = c\omega(\mathbf{u},\mathbf{v})$; scalars in the first slot can be pulled out. These two rules imply
        \begin{eqnarray*}
            \omega(\mathbf{u},c\odot\mathbf{v}) = \overline{\omega(c\odot\mathbf{v},\mathbf{u})} = c\overline{\omega(\mathbf{v},\mathbf{u})} = \overline{c}\omega(\mathbf{u},\mathbf{v})
        \end{eqnarray*}
        \item $\omega(\mathbf{u} \oplus \mathbf{v},\mathbf{w}) = \omega(\mathbf{u},\mathbf{w}) + \omega(\mathbf{v},\mathbf{w})$; for any three objects in V
        \item $\omega(\mathbf{u},\mathbf{u}) \geq 0$ and $\omega(\mathbf{u},\mathbf{v}) = 0 \iff \mathbf{u} = 0$
       \end{enumerate}
       A vector space $V$ with a complex inner product is called a \textbf{Complex Inner Product Space}.
       }
       \end{mysolution}
       
  \item(4 Points) Define a bounded linear operator $T$ between the normed linear spaces
                          $(X,\| \cdot \|_X)$ and $(Y,\| \cdot \|_Y)$.
			  
       \begin{mysolution}
       {
       Let $T:dom(T) \in (X,\| \cdot \|_X) \rightarrow (Y,\| \cdot \|_Y)$ be a linear operator (i.e. $T(\alpha x + \beta y) = \alpha T(x) + \beta T(y)$). Then, we say $T$ is a bounded linear operator if
       \begin{eqnarray*}
        B(T) = sup_{x \in dom(T), \|x\|_X \neq 0}\frac{\|T(x)\|_Y}{\|x\|_X} < \infty
       \end{eqnarray*}
       And, because of the linearity of $T$, we can rewrite
       \begin{eqnarray*}
        \frac{T(x)}{\|x\|_X} = T(\frac{x}{\|x\|_X})
       \end{eqnarray*}
       Which gives us
       \begin{eqnarray*}
        B(T) = sup_{x \in dom(T), \|x\|_X = 1}\|T(x)\|_Y < \infty
       \end{eqnarray*}
       }
       \end{mysolution}
       
  \item(4 Points) Define what it means for a set $D$ in a metric space to be dense.
  
  	\begin{mysolution}
       {
       Let $(X,d)$ be a metrc space and $D \in X$. We say $D$ is dense in $X$ if, $\forall \epsilon > 0 \; \exists \; y \in D \ni d(x,y) < \epsilon$. This implies, given $x \in X$, $\exists \; y_n \in D \ni y_n \rightarrow x$ in $d$.
       }
        \end{mysolution}
	
  \item(4 Points) Define what it means for a metric space $D$ to be separable.
  
  	\begin{mysolution}
       {
       We say $D$ is a separable subspace of the metric space $(X,d)$ if D is countable and dense in $X$.
       }
        \end{mysolution}
	
  \item(4 Points) Define what a Schauder Basis means for a normed linear space.
  
  	\begin{mysolution}
       {
       Let $(X,\|\cdot\|)$ be a nonempty normed linear space. Let $M = (m_n)_{n=1}\infty$ be a countable subset of $X$ so that given $x \in X$, there is a unique sequence of scalars $(a_n)_{n=1}^\infty$ so that $x = \lim_{N\rightarrow\infty}\sum_{i=1}^N\:a_im_i$. This says that $x =\sum_{i=1}^\infty\:a_im_i$ is an infinite series in $X$, whose partial sums converge in norm to $x$. This set $M$ is calleed the Schauder Basis for $X$.
       }
        \end{mysolution}
	
  \item(4 Points) Define the dual space of the normed linear space $(X,\| \cdot \|_X)$.
  
  	\begin{mysolution}
       {
       The Duel Space of the Normed Linear Space $(X,\|\cdot\|)$ is $X' = B(X,F)$ where $F$ is the underlying field and
       \begin{eqnarray*}
        B(X,F) &=& \{f:X \rightarrow F\:|\:f\; \text{is continuous and linear} \}\\
        &=& \{f:X \rightarrow F\:|\:f\; \text{is bounded and linear} \}
       \end{eqnarray*}
       }
        \end{mysolution}
	
  \item(4 Points) Define the double dual space of the normed linear space $(X,\| \cdot \|_X)$
       and the canonical injection $j: X \rightarrow X^{\prime\prime}$.
       
  	\begin{mysolution}
       {
       The Double Duel of the Normed Linear Space $(X,\|\cdot\|)$ is $X' = B(B(X,F),F)$ where $F$ is the underlying field and
       \begin{eqnarray*}
        B(B(X,F),F) &=& \{g:B(X,F) \rightarrow F\:|\:g\; \text{is continuous and linear} \}\\
        &=& \{f:B(X,F) \rightarrow F\:|\:g\; \text{is bounded and linear} \}
       \end{eqnarray*}
       }
        \end{mysolution}
	
  \end{enumerate}
%28+8 = 36
\item[Part 2:  Theorems and Lemmas (8 Points)] \mbox{}\\  
   \begin{enumerate}
   \item(4 Points) State the Parallelogram Equality
   
	\begin{mysolution}
       {
       Let $(X,<\cdot,\cdot>)$ be an inner product space. Then the norm induced by the inner product space satisfies the following
       \begin{eqnarray*}
        \|\textbf{x} + \textbf{y}\|^2 + \|\textbf{x} - \textbf{y}\|^2 = 2(\|\textbf{x}\|^2 + \|\textbf{y}\|^2)
       \end{eqnarray*}
       $\forall \; \mathbf{x},\; \mathbf{y} \in X.$
       }
    \end{mysolution}
	
   \item(4 Points) State the Minimizing Vector Theorem.
   
	\begin{mysolution}
       {
       Let $(X,<\cdot,\cdot>)$ be a nonempty inner product space and let $M \neq 0$ be a be a convex subset of $X$ $(t\mathbf{x} + (t -1)\mathbf{y}) \in X \; \forall \; t \in [0,1]$ and $\textbf{x}, \; \textbf{y} \in M$. This says the line segment $[\mathbf{x},\mathbf{y}]$ lies in $M$ if $\mathbf{x}$ and $\mathbf{y}$ are in $M$ Assume M is complete in the metric induced by $<\cdot,\cdot>$. Then, $\forall \; \textbf{x} \in X, \exists \; y_x \in M$ so that $inf_{\mathbf{y} \in M}\|\mathbf{x} - \mathbf{y}\| = \|\mathbf{x} - y_x\|$
       }
        \end{mysolution}
	
   \end{enumerate}
%36+20 = 56
\item[Part 3:  Is it Possible and Short Answer (20 Points)] \mbox{}\\ 
  \begin{enumerate}
  \item(4 Points) Give an example of a metric that cannot be induced by a norm.
  
  	\begin{mysolution}
       {
       The metric
       \begin{eqnarray*}
        d(x,y) = \sum_{n=1}^\infty\:\frac{1}{2^n}(\frac{|x - y|}{1 + |x - y|})
       \end{eqnarray*}
       cannot be induced by a norm becuase it fails to satisfy the condition $\rho(\alpha x) = \alpha\rho(x)$
       }
        \end{mysolution}
	
  \item(4 Points) Give an example of a norm that cannot be induced by an inner product.
  
  	\begin{mysolution}
       {
       For all $p$ such that $1 \leq p < \infty$, $p \neq 2 \implies \|\cdot\|_p$ cannot be induced by an inner product becuase it would fail the Parallelogram Equality.
       }
    \end{mysolution}
	
  \item(4 Points) If $(f_n)$ is a Cauchy Sequence in $X^\prime$
        where $X = (C([a,b]), \| \cdot \|_1)$, is it possible for
	this sequence to fail to converge?
			 
	\begin{mysolution}
       {
       No it is not possible. In this case, $X' = (C([a,b]), \| \cdot \|_\infty)$ and all sequences in this space must converge under the $\|\cdot\|_\infty$ norm and, by extension, any $\|\cdot\|_p$ with $1 \leq p < \infty$
       }
        \end{mysolution}
	
  \item(4 Points) If we identify $\Re^n$ with column vectors,
       is it true multiplication by a row vector  defines an element of
       $(\Re^n)^\prime$?
			 
	\begin{mysolution}
       {
       No, consider the case where we have $\Re^n$ split into column vectors all of dimensionality $1\times n$. Then multiplication with a row vector of $n \times 1$ yields a $1 \times 1$ object which is defined in $\Re^1$ not $\Re^n$
       }
        \end{mysolution}
	
  \item(4 Points) If $A$ is a $n \times n$ Hermetian matrix, is it true
                         the eigenvectors of $A$ fail to be a basis of $C^n$
			 ($C$ denotes the complex field)?
			 
	\begin{mysolution}
       {
       No, eigenvectors of Hermetian matricies form bases which span the complex field.
       }
        \end{mysolution}
	
  \end{enumerate}
%56 + 18 = 74
\item[Part 3: Computations (18 Points)] \mbox{}\\
\begin{enumerate}
\item(4 Points) Describe how to complete $(C([a,b]), d_2)$ in reasonable detail.

	\begin{mysolution}
       {
       1. Define
       \begin{eqnarray*}
        S = \{(x_n)\;|\;x_n \in (C([a,b]), d_2), \; (x_n) \text{ is a Cauchy Sequence in }(C([a,b]), d_2)\}
       \end{eqnarray*}
       2. Define an equivalence relation on S by
       \begin{eqnarray*}
        (x_n) \sim (y_n) \iff lim_{n\rightarrow\infty}d_2(x_n,y_n) = 0
       \end{eqnarray*}
       3. Let $\widetilde(C([a,b]) = S/\sim$, the set of all equivalence classes in $S$ udner $\sim$. Let the elements of $\widetilde(C([a,b])$ be denoted by $[(\widetilde{x})]$.\\
       4. Extend the metric $d_2$ on $X$ to the metrix $\widetilde{d}_2$ by defining
       \begin{eqnarray*}
        \widetilde{d}_2(\widetilde{x},\widetilde{y}) = lim_{n\rightarrow\infty}(x_n,y_n)
       \end{eqnarray*}
       for any equivalence class $\widetilde{x}$ and $\widetilde{y}$ in $\widetilde{X}$ and any choice of representatives $(x_n)\in \widetilde{x}$ and $\widetilde{y}$\\
       5. Define the mapping $T:C([a,b]) \rightarrow \widetilde{C}([a,b])$ by $T(x) = [(\widetilde{x})]$ where $\widetilde{x} = (x,x,x,...)$ is the constant sequence. Note, each entry $x$ is actually a continuous function on [a,b]. Then $T$ is 1 - 1, onto, and isometric.\\
       6. $T(C([a,b]))$ is dense (as previously defined) in $(\widetilde{C}([a,b]),\widetilde{d}_2)$\\
       7. $(\widetilde{C}([a,b]),\widetilde{d}_2)$ is complete.
       }
        \end{mysolution}
	
\item(6 Points)  Find a Cauchy sequence of Riemann Integrable functions
       which gives a representative for $f(t) = t^{-4/5}$ on $[0,1]$.
       
	\begin{mysolution}
       {
       Conider the function 
        \begin{eqnarray*}
            x_n(t) &=&
            \left \{
            \begin{array}{ll}
            0, & 0 \leq t \leq \frac{1}{n}\\
            t^{-\frac{4}{5}}, & \frac{1}{n} < t \leq 1
            \end{array}
            \right .
        \end{eqnarray*}
        which is clearly Riemann Integrable. Now we consider, for $n > m$
        \begin{eqnarray*}
            \int_0^1\:|x_n(t) - x_m(t)|^2dt\\
            &=& \int_\frac{1}{n}^\frac{1}{m}\:|t^{-\frac{4}{5}} - 0|^2dt\\
            &=& \int_\frac{1}{n}^\frac{1}{m}\:t^{-\frac{8}{3}}dt\\
            &=& |-5(\frac{1}{t^\frac{1}{5}})||_\frac{1}{n}^\frac{1}{m}\\
            &=& 5|m^\frac{1}{5} - n^\frac{1}{5}|\\
            &=& C_{nm}|\frac{1}{m^\frac{1}{5}} - \frac{1}{n^\frac{1}{5}}| \;\; \text{(By MVT)}\\
            &<& \frac{1}{m}|\frac{n^\frac{6}{5} - m^\frac{6}{5}}{(mn)^\frac{6}{5}}|\\
            &<& \frac{1}{m^\frac{12}{5}}
        \end{eqnarray*}
        Which is a Cauchy sequence that is able to serve as an upper bound of the integral.
        }
    \end{mysolution}
	
\item(8 Points) Let 
         \begin{eqnarray*}
	 A = &=& 
	 \begin{bmatrix}
	 1 & 2 & 3\\
	 2 & 4 & -1\\
	 3 & -1 & -2
	 \end{bmatrix}
	 \end{eqnarray*}
	 \noindent
	 Find $\|A \|_{op}$ for the three cases: using $\| \cdot \|_1$ on
	 for both the domain and the range of $A$,  using $\| \cdot \|_\infty$ on
	 for both the domain and the range of $A$ and
	  using $\| \cdot \|_2$ on
	 for both the domain and the range of $A$.  You can use \textregistered{MATLAB}
	 to get the eigenvalues of $A$.
       
	\begin{mysolution}
       {
       Using matlab/octave, we are able to get
       \begin{eqnarray*}
        \lambda_1 = 4.3004\\
        \lambda_2 = 2.2611\\
        \lambda_3 = 5.0393
       \end{eqnarray*}
       }
    \end{mysolution}
	
\end{enumerate}
%74 + 26 = 100
\item[Part 4: Proofs (26 Points)] \mbox{}\\
Provide careful proofs of the following propositions.  You will
be graded on the mathematical correctness of your arguments as well
as your use of language, syntax and organization in the proof.
  \begin{enumerate}
  \item(12 Points)  If $A$ is a $n \times n$ Hermetian matrix, we know
           $f(x) = | <A(x), x>|$ is a continuous function of $x$.	
       \begin{enumerate}
       \item(6 Points)    Prove $f$ is not a linear mapping. HINT: what does $f$ do to $cx$
           for any scalar $c$.	    
	\begin{mysolution}
       {
       Your work here.
       }
	\end{mysolution}
	\item(6 Points)   Prove $f$ is not a bounded mapping. HINT: what does $f$ do to $cE_i$
           for any scalar $c$ and eigenvector $E_i$?
	\end{enumerate}	 
  \item(14 Points)  Let $P$ be the set of all polynomials on $[0,1]$.
         \begin{enumerate}
         \item(8 Points) Prove the set of all polynomials with rational coefficients is dense in $P$
	 using $\| \cdot \|_\infty$.
	   
         \begin{mysolution}
        {
        Your work here.
        }
	 \end{mysolution}
	 
         \item(6 Points) There is theorem called the Weierstrass Approximation Theorem
	          that says given any $f \in (C([a,b], \| \cdot\|_\infty)$, and given any $\epsilon>0$,
		  there is a polynomial $p_\epsilon$ (called the Bernstein Polynomial) 
		  with $\| f - p_\epsilon\|_\infty < \epsilon$.  Using the first part of this problem
		  prove the set of all polynomials with rational coefficients is dense
	          in $(C([a,b], \| \cdot\|_\infty)$.
		  
         \begin{mysolution}
        {
        Your work here.
        }
	 \end{mysolution}
	 \end{enumerate}
	 
  \end{enumerate}
\end{description}
             
\end{document}
