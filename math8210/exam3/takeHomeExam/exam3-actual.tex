\documentclass[11pt]{article}
% use ams math packages
\usepackage{amsmath,amsthm,amssymb,amsfonts}
\usepackage{mathrsfs}
%
\usepackage{xcolor}
\usepackage{fancyvrb}
\usepackage{listings}
%
\usepackage{geometry}

\renewcommand{\qedsymbol}{\hfill \blacksquare}
\newcommand{\subqedsymbol}{\hfill \Box}
%\theoremstyle{plain}

\newtheoremstyle{mystyle}% name
  {6pt}%      Space above
  {6pt}%      Space below
  {\itshape}%         Body font
  {}%         Indent amount (empty = no indent, \parindent = para indent)
  {\bfseries}% Thm head font
  {}%        Punctuation after thm head
  { }%     Space after thm head: " " = normal interword space; \newline = linebreak
  {}%         Thm head spec (can be left empty, meaning `normal')
\theoremstyle{mystyle}
 
\newtheorem*{solution}{solution}
\newtheorem*{theorem}{Theorem}

\newenvironment{mysolution}[1]
{
\centering
\begin{solution}%
\mbox{}\\ \vskip6pt \colorbox{black!15}{\fbox{\parbox{.9\textwidth}{#1}}}
 $\qedsymbol$
\end{solution}
}%
{}
      
\newenvironment{mytheorem}[1]
{
\centering
\begin{theorem}%
\mbox{}\\ \vskip6pt \colorbox{black!15}{\fbox{\parbox{.9\textwidth}{#1}}}
\end{theorem}
}%
{}

\newenvironment{reason}[1]
{
 \vskip0.05in
 \begin{myproof}
 \mbox{}\\#1
 $\qedsymbol$
 \end{myproof}  
 \vskip0.05in
}%
{}

\newcommand{\bs}[1]{
\boldsymbol{#1}
}

\pagestyle{plain}
\geometry{lmargin=1.0in,rmargin=1.0in,top=1.0in,bottom=1.0in}
\pagestyle{plain}

\title{Take Home Exam 3: MATH 8210: Spring 2020}
\author{Dr. Peterson}
\date{\today}

\begin{document}
\maketitle

\noindent
This is a closed book and closed notes test.  Make sure
to give me all the details of your arguments!

\begin{description} 
\item[Part 1: Definitions (28 Points)] \mbox{}\\
  \begin{enumerate}
  \item(4 Points) Define what an inner product on the set $X$ is carefully.
                           Do this for both the real and complex field.
       \begin{mysolution}
       {
       Your work here.
       }
       \end{mysolution}
       
  \item(4 Points) Define a bounded linear operator $T$ between the normed linear spaces
                          $(X,\| \cdot \|_X)$ and $(Y,\| \cdot \|_Y)$.
			  
       \begin{mysolution}
       {
       Your work here.
       }
       \end{mysolution}
       
  \item(4 Points) Define what it means for a set $D$ in a metric space to be dense.
  
  	\begin{mysolution}
       {
       Your work here.
       }
        \end{mysolution}
	
  \item(4 Points) Define what it means for a metric space $D$ to be separable.
  
  	\begin{mysolution}
       {
       Your work here.
       }
        \end{mysolution}
	
  \item(4 Points) Define what a Schauder Basis means for a normed linear space.
  
  	\begin{mysolution}
       {
       Your work here.
       }
        \end{mysolution}
	
  \item(4 Points) Define the dual space of the normed linear space $(X,\| \cdot \|_X)$.
  
  	\begin{mysolution}
       {
       Your work here.
       }
        \end{mysolution}
	
  \item(4 Points) Define the double dual space of the normed linear space $(X,\| \cdot \|_X)$
       and the canonical injection $j: X \rightarrow X^{\prime\prime}$.
       
  	\begin{mysolution}
       {
       Your work here.
       }
        \end{mysolution}
	
  \end{enumerate}
%28+8 = 36
\item[Part 2:  Theorems and Lemmas (8 Points)] \mbox{}\\  
   \begin{enumerate}
   \item(4 Points) State the Parallelogram Equality
   
	\begin{mysolution}
       {
       Your work here.
       }
        \end{mysolution}
	
   \item(4 Points) State the Minimizing Vector Theorem.
   
	\begin{mysolution}
       {
       Your work here.
       }
        \end{mysolution}
	
   \end{enumerate}
%36+20 = 56
\item[Part 3:  Is it Possible and Short Answer (20 Points)] \mbox{}\\ 
  \begin{enumerate}
  \item(4 Points) Give an example of a metric that cannot be induced by a norm.
  
  	\begin{mysolution}
       {
       Your work here.
       }
        \end{mysolution}
	
  \item(4 Points) Give an example of a norm that cannot be induced by an inner product.
  
  	\begin{mysolution}
       {
       Your work here.
       }
        \end{mysolution}
	
  \item(4 Points) If $(f_n)$ is a Cauchy Sequence in $X^\prime$
        where $X = (C([a,b]), \| \cdot \|_1)$, is it possible for
	this sequence to fail to converge?
			 
	\begin{mysolution}
       {
       Your work here.
       }
        \end{mysolution}
	
  \item(4 Points) If we identify $\Re^n$ with column vectors,
       is it true multiplication by a row vector  defines an element of
       $(\Re^n)^\prime$?
			 
	\begin{mysolution}
       {
       Your work here.
       }
        \end{mysolution}
	
  \item(4 Points) If $A$ is a $n \times n$ Hermetian matrix, is it true
                         the eigenvectors of $A$ fail to be a basis of $C^n$
			 ($C$ denotes the complex field)?
			 
	\begin{mysolution}
       {
       Your work here.
       }
        \end{mysolution}
	
  \end{enumerate}
%56 + 18 = 74
\item[Part 3: Computations (18 Points)] \mbox{}\\
\begin{enumerate}
\item(4 Points) Describe how to complete $(C([a,b]), d_2$ in reasonable detail.

	\begin{mysolution}
       {
       Your work here.
       }
        \end{mysolution}
	
\item(6 Points)  Find a Cauchy sequence of Riemann Integrable functions
       which gives a representative for $f(t) = t^{-4/5}$ on $[0,1]$.
       
	\begin{mysolution}
       {
       Your work here.
       }
        \end{mysolution}
	
\item(8 Points) Let 
         \begin{eqnarray*}
	 A = &=& 
	 \begin{bmatrix}
	 1 & 2 & 3\\
	 2 & 4 & -1\\
	 3 & -1 & -2
	 \end{bmatrix}
	 \end{eqnarray*}
	 \noindent
	 Find $\|A \|_{op}$ for the three cases: using $\| \cdot \|_1$ on
	 for both the domain and the range of $A$,  using $\| \cdot \|_\infty$ on
	 for both the domain and the range of $A$ and
	  using $\| \cdot \|_2$ on
	 for both the domain and the range of $A$.  You can use \textregistered{MATLAB}
	 to get the eigenvalues of $A$.
       
	\begin{mysolution}
       {
       Your work here.
       }
        \end{mysolution}
	
\end{enumerate}
%74 + 26 = 100
\item[Part 4: Proofs (26 Points)] \mbox{}\\
Provide careful proofs of the following propositions.  You will
be graded on the mathematical correctness of your arguments as well
as your use of language, syntax and organization in the proof.
  \begin{enumerate}
  \item(12 Points)  If $A$ is a $n \times n$ Hermetian matrix, we know
           $f(x) = | <A(x), x>|$ is a continuous function of $x$.	
       \begin{enumerate}
       \item(6 Points)    Prove $f$ is not a linear mapping. HINT: what does $f$ do to $cx$
           for any scalar $c$.	    
	\begin{mysolution}
       {
       Your work here.
       }
	\end{mysolution}
	\item(6 Points)   Prove $f$ is not a bounded mapping. HINT: what does $f$ do to $cE_i$
           for any scalar $c$ and eigenvector $E_i$?
	\end{enumerate}	 
  \item(14 Points)  Let $P$ be the set of all polynomials on $[0,1]$.
         \begin{enumerate}
         \item(8 Points) Prove the set of all polynomials with rational coefficients is dense in $P$
	 using $\| \cdot \|_\infty$.
	   
         \begin{mysolution}
        {
        Your work here.
        }
	 \end{mysolution}
	 
         \item(6 Points) There is theorem called the Weierstrass Approximation Theorem
	          that says given any $f \in (C([a,b], \| \cdot\|_\infty)$, and given any $\epsilon>0$,
		  there is a polynomial $p_\epsilon$ (called the Bernstein Polynomial) 
		  with $\| f - p_\epsilon\|_\infty < \epsilon$.  Using the first part of this problem
		  prove the set of all polynomials with rational coefficients is dense
	          in $(C([a,b], \| \cdot\|_\infty)$.
		  
         \begin{mysolution}
        {
        Your work here.
        }
	 \end{mysolution}
	 \end{enumerate}
	 
  \end{enumerate}
\end{description}
             
\end{document}
