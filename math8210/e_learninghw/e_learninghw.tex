% SampleProject.tex -- main LaTeX file for sample LaTeX article
%
%\documentclass[12pt]{article}
\documentclass[11pt]{SelfArxOneColBMN}
% add the pgf and tikz support.  This automatically loads
% xcolor so no need to load color
\usepackage{pgf}
\usepackage{tikz}
\usetikzlibrary{matrix}
\usetikzlibrary{calc}
\usepackage{xstring}
\usepackage{pbox}
\usepackage{etoolbox}
\usepackage{marginfix}
\usepackage{xparse}
\setlength{\parskip}{0pt}% fix as marginfix inserts a 1pt ghost parskip
% standard graphics support
\usepackage{graphicx,xcolor}
\usepackage{wrapfig}
%
\definecolor{color1}{RGB}{0,0,90} % Color of the article title and sections
\definecolor{color2}{RGB}{0,20,20} % Color of the boxes behind the abstract and headings
%----------------------------------------------------------------------------------------
% HYPERLINKS
%----------------------------------------------------------------------------------------
\usepackage[pdftex]{hyperref} % Required for hyperlinks
\hypersetup{hidelinks,
colorlinks,
breaklinks=true,%
urlcolor=color2,%
citecolor=color1,%
linkcolor=color1,%
bookmarksopen=false%
,pdftitle={SampleProject},%
pdfauthor={Peterson}}
%\usepackage[round,numbers]{natbib}
\usepackage[numbers]{natbib}
\usepackage{lmodern}
\usepackage{setspace}
\usepackage{xspace}
%
\usepackage{subfigure}
\newcommand{\goodgap}{
  \hspace{\subfigtopskip}
  \hspace{\subfigbottomskip}}
%
\usepackage{atbegshi}
%
\usepackage[hyper]{listings}
%
% use ams math packages
\usepackage{amsmath,amsthm,amssymb,amsfonts}
\usepackage{mathrsfs}
%
% use new improved Verbatim
\usepackage{fancyvrb}
%
\usepackage[titletoc,title]{appendix}
%
\usepackage{url}
%
% Create length for the baselineskip of text in footnotesize
\newdimen\footnotesizebaselineskip
\newcommand{\test}[1]{%
 \setbox0=\vbox{\footnotesize\strut Test \strut}
 \global\footnotesizebaselineskip=\ht0 \global\advance\footnotesizebaselineskip by \dp0
}
%
\usepackage{listings}

\DeclareGraphicsExtensions{.pdf, .jpg, .tif,.png}

% make sure we don't get orphaned words if at top of page
% or orphans if at bottom of page
\clubpenalty=9999
\widowpenalty=9999
\renewcommand{\textfraction}{0.15}
\renewcommand{\topfraction}{0.85}
\renewcommand{\bottomfraction}{0.85}
\renewcommand{\floatpagefraction}{0.66}
%
\DeclareMathOperator{\sech}{sech}

\newcommand{\mycite}[1]{%
(\citeauthor{#1} \citep{#1} \citeyear{#1})\xspace
}

\newcommand{\mycitetwo}[2]{%
(\citeauthor{#2} \citep[#1]{#2} \citeyear{#2})\xspace
}

\newcommand{\mycitethree}[3]{%
(\citeauthor{#3} \citep[#1][#2]{#3} \citeyear{#3})\xspace
}

\newcommand{\myincludegraphics}[3]{% file name, width, height
\includegraphics[width=#2,height=#3]{#1}
}

\newcommand{\myincludegraphicstwo}[2]{% file name, width, height
\includegraphics[scale=#1]{#2}
}

\newcommand{\mysimplegraphics}[1]{% file name, width, height
\includegraphics{#1}
}

\newcommand{\MB}[1]{
\boldsymbol{#1}
}

\newcommand{\myquotetwo}[1]{%
\small
%\singlespacing
\begin{quotation}
#1
\end{quotation}
\normalsize
%\onehalfspacing  
}

\newcommand{\jimquote}[1]{%
\small
%\singlespacing
\begin{quotation}
#1
\end{quotation}
\normalsize
%\onehalfspacing
}

\newcommand{\myquote}[1]{%
\small
%\singlespacing
\begin{quotation}
#1
\end{quotation}
\normalsize
%\onehalfspacing  
}

%A =
%
%[2 r_1 	     r_1]
%[-2r_1 + r_2  r_2 - r_1]
%
%has eigenvalues r_1 neq r_2.
% #1 = 2 r_1, #2 = r_1, #3 = -2r_1+r_2, #4 = r_2 - r_1
\newcommand{\myrealdiffA}[4]{
\left [
\begin{array}{rr}
#1  & #2\\
#3  & #4
\end{array}
\right ]
}

% args:
% 1, 2 ,3, 4, 5 = caption, label, width, height, file name
%\mysubfigure{}{}{}{}{}
\newcommand{\mysubfigure}[5]{%
\subfigure[#1]{\label{#2}\includegraphics[width=#3,height=#4]{#5}}
}

\newcommand{\mysubfiguretwo}[3]{%
\subfigure[#1]{\label{#2}\includegraphics{#3}}
}

\newcommand{\mysubfigurethree}[4]{%
\subfigure[#1]{\label{#2}\includegraphics[scale=#3]{#4}}
}

\newcommand{\myputimage}[5]{% file name, width, height
\centering
\includegraphics[width=#3,height=#4]{#5}
\caption{#1}
\label{#2}
}

\newcommand{\myputimagetwo}[4]{% caption, label, scale, file name
\centering
\includegraphics[scale=#3]{#4}
\caption{#1}
\label{#2}
}

\newcommand{\myrotateimage}[5]{% file name, width, height
\centering
\includegraphics[scale=#3,angle=#4]{#5}
\caption{#1}
\label{#2}
}

\newcommand{\myurl}[2]{%
\href{#1}{\bf #2}
}

\RecustomVerbatimEnvironment%
{Verbatim}{Verbatim}  
  {fillcolor=\color{black!20}}
  
  \DefineVerbatimEnvironment%
{MyVerbatim}{Verbatim}  
  {frame=single,
   framerule=2pt,
   fillcolor=\color{black!20},
   fontsize=\small}
   
\newcommand{\myfvset}[1]{%  
\fvset{frame=single,
       framerule=2pt,
       fontsize=\small,
       xleftmargin=#1in}}
       
\newcommand{\mylistverbatim}{%
\lstset{%
  fancyvrb, 
  basicstyle=\small,
  breaklines=true}
}  

\newcommand{\mylstinlinebf}[1]{%
{\bf #1}
}

\newcommand{\mylstinline}{%
\lstset{%
  basicstyle=\color{black!80}\bfseries\ttfamily,
  showstringspaces=false,
  showspaces=false,showtabs=false,
  breaklines=true}
\lstinline
}

\newcommand{\mylstinlinetwo}[1]{%
\lstset{%
  basicstyle=\color{black!80}\bfseries\ttfamily,
  showstringspaces=false,
  showspaces=false,showtabs=false,
  breaklines=true}
\lstinline!#1! 
}

%fontfamily=tt
%fontfamily=courier
%fontfamily=helvetica
%frame=topline,
%frame=single,
 %frame=lines,
 %framesep=10pt,
 %fontshape=it,
 %fontseries=b,
 %fontsize=\relsize{-1},
 %fillcolor=\color{black!20},
 %rulecolor=\color{yellow},
 %fillcolor=\color{red}
 %label=\fbox{\Large\emph{The code}}
\DefineVerbatimEnvironment%
{MyListVerbatim}{Verbatim}  
{
fillcolor=\color{black!10},
fontfamily=courier,
frame=single,
%formatcom=\color{white},
framesep=5mm,
labelposition=topline,
fontshape=it,
fontseries=b,
fontsize=\small,
label=\fbox{\large\emph{The code}\normalsize}
} 

%  caption={[#1] \large\bf{#1}}, 
%\centering \framebox[.6\textwidth][c]{\Large\bf{#1}}
\newcommand{\myfancyverbatim}[1]{%
\lstset{%
  fancyvrb=true, 
  %fvcmdparams= fillcolor 1,
  %morefvcmdparams = \textcolor 2,
  frame=shadowbox,framerule=2pt, 
  basicstyle=\small\bfseries,
  backgroundcolor=\color{black!08},
  showstringspaces=false,
  showspaces=false,showtabs=false,
  keywordstyle=\color{black}\bfseries,
  %numbers=left,numberstyle=\tiny,stepnumber=5,numbersep=5pt,
  stringstyle=\ttfamily,
  caption={[\quad #1] \mbox{}\\ \vspace{0.1in} \framebox{\large \bf{#1} \small} },  
  belowcaptionskip=20 pt,  
  label={},
  xleftmargin=17pt,
  framexleftmargin=17pt,
  framexrightmargin=5pt,
  framexbottommargin=4pt,
  nolol=false,
  breaklines=true}
}

\newcommand{\mylistcode}[3]{%
\lstset{%
  language=#1, 
  frame=shadowbox,framerule=2pt, 
  basicstyle=\small\bfseries,
  backgroundcolor=\color{black!16},
  showstringspaces=false,
  showspaces=false,showtabs=false,
  keywordstyle=\color{black!40}\bfseries,
  numbers=left,numberstyle=\tiny,stepnumber=5,numbersep=5pt,
  stringstyle=\ttfamily,
  caption={[\quad#2] \mbox{}\\ \vspace{0.1in} \framebox{\large \bf{#2} \small} },
  belowcaptionskip=20 pt,
  breaklines=true,
  xleftmargin=17pt,
  framexleftmargin=17pt,
  framexrightmargin=5pt,
  framexbottommargin=4pt,  
  label=#3,
  breaklines=true} 
}

  %caption={[#2] #3},
  %caption={[#2]{\mbox{}\\ \vspace{0.1in} \framebox{\large \bf{#3} \small}},
  %caption={[#2] \mbox{}\\ \bf{#3} },

% frame=single,
% caption={[Code Fragment] {\bf Code Fragment} },
% caption={[Code Fragment] \mbox{}\\ \vspace{0.1in} \framebox{\large \bf{Code Fragment} \small} },
\newcommand{\mylistcodequick}[1]{%
\lstset{%
  language=#1, 
  frame=shadowbox,framerule=2pt, 
  basicstyle=\small\bfseries,
  backgroundcolor=\color{black!16},
  showstringspaces=false,
  showspaces=false,showtabs=false,
  keywordstyle=\color{black!40}\bfseries,
  numbers=left,numberstyle=\tiny,stepnumber=5,numbersep=5pt,
  stringstyle=\ttfamily,
  caption={[\quad Code Fragment] \large \bf{Code Fragment} \small},   
  belowcaptionskip=20 pt,  
  label={},
  xleftmargin=17pt,
  framexleftmargin=17pt,
  framexrightmargin=5pt,
  framexbottommargin=4pt,
  breaklines=true} 
}

%  caption={[#2] \mbox{}\\ \vspace{0.1in} \framebox{\large \bf{#2} \small} },
\newcommand{\mylistcodequicktwo}[2]{%
\lstset{%
  language=#1, 
  frame=shadowbox,framerule=2pt, 
  basicstyle=\small\bfseries,
  extendedchars=true,
  backgroundcolor=\color{black!16},
  showstringspaces=false,
  showspaces=false,
  showtabs=false,
  keywordstyle=\color{black!40}\bfseries,
  numbers=left,numberstyle=\tiny,stepnumber=5,numbersep=5pt,
  stringstyle=\ttfamily,
  caption={[\quad#2] \large \bf{#2} \small},
  belowcaptionskip=20 pt,
  label={},
  xleftmargin=17pt,
  framexleftmargin=17pt,
  framexrightmargin=5pt,
  framexbottommargin=4pt,
  breaklines=true} 
}

%  caption={[#2] \mbox{}\\ \vspace{0.1in} \framebox{\large \bf{#2} \small} },
\newcommand{\mylistcodequickthree}[2]{%
\lstset{%
  language=#1, 
  frame=shadowbox,framerule=2pt, 
  basicstyle=\small\bfseries,
  extendedchars=true,
  backgroundcolor=\color{black!16},
  showstringspaces=false,
  showspaces=false,
  showtabs=false,
  keywordstyle=\color{black!40}\bfseries,
  numbers=left,numberstyle=\tiny,stepnumber=5,numbersep=5pt,
  stringstyle=\ttfamily,
  caption={[\quad#2] \large\bf{#2}\small},
  belowcaptionskip=20 pt,
  label={},
  xleftmargin=17pt,
  framexleftmargin=17pt,
  framexrightmargin=5pt,
  framexbottommargin=4pt,
  breaklines=true} 
}

%  frame=single,
\newcommand{\mylistset}[4]{%
\lstset{language=#1,
  basicstyle=\small,
  showstringspaces=false,
  showspaces=false,showtabs=false,
  keywordstyle=\color{black!40}\bfseries,
  numbers=left,numberstyle=\tiny,stepnumber=5,numbersep=5pt,
  stringstyle=\ttfamily,
  caption={[\quad#2]#3},
  label=#4}
}

\newcommand{\mylstinlineset}{%
\lstset{%
  basicstyle=\color{blue}\bfseries\ttfamily,
  showstringspaces=false,
  showspaces=false,showtabs=false,
  breaklines=true}
}

\newcommand{\myframedtext}[1]{%
\centering
\noindent
%\fbox{\parbox[c]{.9\textwidth}{\color{black!40} \small \singlespacing #1\onehalfspacing \normalsize \\}}
\fbox{\parbox[c]{.9\textwidth}{\color{black!40} \small  #1 \normalsize \\}}
}

\newcommand{\myemptybox}[2]{% from , to
\fbox{\begin{minipage}[t][#1in][c]{#2in}\hspace{#2in}\end{minipage}}
}

\newcommand{\myemptyboxtwo}[2]{% from , to
\centering\fbox{
\begin{minipage}{#1in}
\hfill\vspace{#2in}
\end{minipage}
}
}

\newcommand{\boldvector}[1]{
\boldsymbol{#1}
}

\newcommand{\dEdY}[2]{\frac{d E}{d Y_{#1}^{#2}}}
\newcommand{\dEdy}[2]{\frac{d E}{d y_{#1}^{#2}}}
\newcommand{\dEdT}[2]{\frac{\partial E}{\partial T_{{#1} \rightarrow {#2}}}}
\newcommand{\dEdo}[1]{\frac{\partial E}{\partial o^{#1}}}
\newcommand{\dEdg}[1]{\frac{\partial E}{\partial g^{#1}}}
\newcommand{\dYdY}[4]{\frac{\partial Y_{#1}^{#2}}{\partial Y_{#3}^{#4}}}
\newcommand{\dYdy}[4]{\frac{\partial Y_{#1}^{#2}}{\partial y_{#3}^{#4}}}
\newcommand{\dydY}[4]{\frac{\partial y_{#1}^{#2}}{\partial Y_{#3}^{#4}}}
\newcommand{\dydy}[4]{\frac{\partial y_{#1}^{#2}}{\partial y_{#3}^{#4}}}
\newcommand{\dydT}[4]{\frac{\partial y_{#1}^{#2}}{\partial T_{{#3} \rightarrow {#4}}}}
\newcommand{\dYdT}[4]{\frac{\partial Y_{#1}^{#2}}{\partial T_{{#3} \rightarrow {#4}}}}
\newcommand{\dTdT}[4]{\frac{\partial T_{{#1} \rightarrow {#2}}}{\partial T_{{#3} \rightarrow {#4}}}}
\newcommand{\ssum}[2]{\sum_{#1}^{#2}}

\newcommand{\ssty}[1]{\scriptscriptstyle #1}

\newcommand{\myparbox}[2]{%
\parbox{#1}{\color{black!20} #2}
}

\newcommand{\bs}[1]{
\boldsymbol{#1}
}

\newcommand{\parone}[2]{%
\frac{\partial #1 }{ \partial #2 }
}
\newcommand{\partwo}[2]{%
\frac{ \partial^2 {#1} }{ \partial {#2}^2 }
}

\newcommand{\twodvectorvarfun}[2]{
\left [
\begin{array}{r}
{{#1_{\ssty{1}}}(#2)} \\
{{#1_{\ssty{2}}}(#2)}
\end{array}
\right ]
}
\newcommand{\twodvectorvarprimed}[1]{
\left [
\begin{array}{r}
{{#1_{\ssty{1}}}'(t)} \\
{{#1_{\ssty{2}}}'(t)}
\end{array}
\right ]
}

\newcommand{\complex}[2]{#1 \: #2 \: \boldsymbol{i}}
\newcommand{\complexmag}[2]%
{
\sqrt{(#1)^2 \: + \: (#2)^2}
}
\newcommand{\threenorm}[3]%
{
\sqrt{(#1)^2 \: + \: (#2)^2 \: + \: (#3)^2}
}
\newcommand{\norm}[1]{\mid \mid #1 \mid \mid}

\newcommand{\myderiv}[2]{\frac{d #1}{d #2}}
\newcommand{\myderivb}[2]{\frac{d}{d #2} \left ( #1 \right )}
\newcommand{\myrate}[3]%
{#1^\prime(#2) &=& #3 \: #1(#2)
}
\newcommand{\myrateexter}[4]%
{#1^\prime(#2) &=& #3 \: #1(#2) \: + \: #4
}
\newcommand{\myrateic}[3]%
{#1( \: #2 \:) &=& #3 
}

\newcommand{\mytwodsystemeqn}[6]{
#1 \: x    #2 \: y &=& #3\\
#4 \: x    #5 \: y &=& #6\\
}

\newcommand{\mytwodsystem}[8]{
#3 \: #1 \: + \: #4 \: #2 &=& #5\\
#6 \: #1 \: + \: #7 \: #2 &=& #8\\
}  

\newcommand{\mytwodarray}[4]{
\left [
\begin{array}{rr}
#1 & #2\\
#3 & #4
\end{array}
\right ]
}

\newcommand{\mytwoid}{
\left [
\begin{array}{rr}
1 & 0\\
0 & 1
\end{array}
\right ]
}

\newcommand{\myxprime}[2]{
\left [
\begin{array}{r}
#1^\prime(t)\\
#2^\prime(t)
\end{array}
\right ]
}

\newcommand{\myxprimepacked}[2]{
\left [
\begin{array}{r}
#1^\prime\\
#2^\prime
\end{array}
\right ]
}

\newcommand{\myx}[2]{
\left [
\begin{array}{r}
#1(t)\\
#2(t)
\end{array}
\right ]
}

\newcommand{\myxonly}[2]{
\left [
\begin{array}{r}
#1\\
#2
\end{array}
\right ]
}

\newcommand{\myv}[2]{
\left [
\begin{array}{r}
#1\\
#2
\end{array}
\right ]
}

\newcommand{\myxinitial}[2]{
\left [
\begin{array}{r}
#1(0)\\
#2(0)
\end{array}
\right ]
}

\newcommand{\twodboldv}[1]{
\boldsymbol{#1}
}

\newcommand{\mytwodvector}[2]{
\left [
\begin{array}{r}
#1\\
#2
\end{array}
\right ]
}

\newcommand{\mythreedvector}[3]{
\left [
\begin{array}{r}
#1\\
#2\\
#3
\end{array}
\right ]
}

\newcommand{\mytwodsystemvector}[6]{
\left [
\begin{array}{rr}
#1 & #2\\
#4 & #5
\end{array}
\right ]
\:
\left [
\begin{array}{r}
x \\
y 
\end{array}
\right ]
&=&
\left [
\begin{array}{r}
#3\\
#6
\end{array}
\right ]
}

\newcommand{\mythreedarray}[9]{
\left [
\begin{array}{rrr}
#1 & #2 & #3\\
#4 & #5 & #6\\
#7 & #8 & #9
\end{array}
\right ]
}

\newcommand{\myodetwo}[6]{
#1 \: #6^{\prime \prime}(t) \: #2 \: #6^{\prime}(t) \: #3 \: #6(t) &=& 0\\
#6(0)                                           &=& #4\\
#6^{\prime}(0)                                  &=& #5
}

\newcommand{\myodetwoNoIC}[4]{
#1 \: #4^{\prime \prime}(t) \: #2 \: #4^{\prime}(t) \: #3 \: #4(t) &=& 0
}

\newcommand{\myodetwopacked}[5]{
\hspace{-0.3in}& & #1 u^{\prime \prime} #2 u^{\prime} #3 u \: = \: 0\\
\hspace{-0.3in}& & u(0) \: = \: #4, \: \: u^{\prime}(0)    \: = \:  #5
}

\newcommand{\myodetwoforced}[6]{
#1\: u^{\prime \prime}(t) \: #2 \: u^{\prime}(t) \: #3 \: u(t) &=& #6\\
u(0)                                           &=& #4\\
u^{\prime}(0)                                  &=& #5\\
}

\newcommand{\myodesystemtwo}[8]{
#1 \: x^\prime(t) \: #2 \: y^\prime(t) \: #3 \: x(t) \: #4 \: y(t) &=& 0\\
#5 \: x^\prime(t) \: #6 \: y^\prime(t) \: #7 \: x(t) \: #8 \: y(t) &=& 0\\
}

\newcommand{\myodesystemtwoic}[2]{
x(0)                                       &=& #1\\ 
y(0)                                       &=& #2
}

\newcommand{\mypredprey}[4]{
x^\prime(t) &=& #1 \: x(t) \: - \: #2 \: x(t) \: y(t)\\
y^\prime(t) &=& -#3 \: y(t) \: + \: #4 \: x(t) \: y(t)
}

\newcommand{\mypredpreypacked}[4]{
x^\prime &=& #1 \: x - #2 \: x \: y\\
y^\prime &=& -#3 \: y + #4 \: x \: y
}

\newcommand{\mypredpreyself}[6]{
x^\prime(t) &=&  #1 \: x(t) \: - \: #2 \: x(t) \: y(t) \: - \: #3 \: x(t)^2\\
y^\prime(t) &=& -#4 \: y(t) \: + \: #5 \: x(t) \: y(t) \: - \: #6 \: y(t)^2
}

\newcommand{\mypredpreyfish}[5]{
x^\prime(t) &=&  #1 \: x(t) \: - \: #2 \: x(t) \: y(t) \: - \: #5 \: x(t)\\
y^\prime(t) &=& -#3 \: y(t) \: + \: #4 \: x(t) \: y(t) \: - \: #5 \: y(t)
}

\newcommand{\myepidemic}[4]{
S^\prime(t) &=& - #1 \: S(t) \: I(t)\\
I^\prime(t) &=&   #1 \: S(t) \: I(t) \: - \: #2 \: I(t)\\
S(0)        &=&   #3\\
I(0)        &=&   #4\\
}

\newcommand{\bsred}[1]{%
\textcolor{red}{\boldsymbol{#1}}
}

\newcommand{\bsblue}[1]{%
\textcolor{blue}{\boldsymbol{#1}}
}


\newcommand{\myfloor}[1]{%
\lfloor{#1}\rfloor
}

\newcommand{\cubeface}[7]{%
\begin{bmatrix}
\bs{#3}          & \longrightarrow & \bs{#4}\\
\uparrow          &                         &  \uparrow  \\
\bs{#1} & \longrightarrow & \bs{#2}\\
              & \text{ \bfseries #5:} \: \bs{#6} \: \text{\bfseries  #7 } & 
\end{bmatrix}
}

\newcommand{\cubefacetwo}[5]{%
\begin{bmatrix}
\bs{#3}          & \longrightarrow & \bs{#4}\\
\uparrow          &                         &  \uparrow  \\
\bs{#1} & \longrightarrow & \bs{#2}\\
              & \text{ \bfseries #5} & 
\end{bmatrix}
}

\newcommand{\cubefacethree}[9]{%
\begin{bmatrix}
\bs{#3}                  & \overset{#9}{\longrightarrow} & \bs{#4}\\
\uparrow \: #7         &                                             &  \uparrow  \: #8 \\
\bs{#1}                  & \overset{#6}{\longrightarrow} & \bs{#2}\\
                               & \text{ \bfseries #5} & 
\end{bmatrix}
}

\renewcommand{\qedsymbol}{\hfill \blacksquare}
\newcommand{\subqedsymbol}{\hfill \Box}
%\theoremstyle{plain}

\newtheoremstyle{mystyle}% name
  {6pt}%      Space above
  {6pt}%      Space below
  {\itshape}%         Body font
  {}%         Indent amount (empty = no indent, \parindent = para indent)
  {\bfseries}% Thm head font
  {}%        Punctuation after thm head
  { }%     Space after thm head: " " = normal interword space; \newline = linebreak
  {}%         Thm head spec (can be left empty, meaning `normal')
\theoremstyle{mystyle}
 
\newtheorem{axiom}{Axiom}
%\newtheorem{solution}{Solution}[section]
\newtheorem*{solution}{Solution}
\newtheorem{exercise}{Exercise}[section]
\newtheorem{theorem}{Theorem}[section]
\newtheorem{proposition}[theorem]{Proposition}
\newtheorem{prop}[theorem]{Proposition}
\newtheorem{assumption}{Assumption}[section]
\newtheorem{definition}{Definition}[section]
\newtheorem{comment}{Comment}[section]
\newtheorem*{question}{Question}
\newtheorem{program}{Program}[section]
%\newtheorem{myproof}{Proof}
%\newtheorem*{myproof}{Proof}[section]
\newtheorem{myproof}{Proof}[section]
\newtheorem{hint}{Hint}[section]
\newtheorem*{phint}{Hint}
\newtheorem{lemma}[theorem]{Lemma}
\newtheorem{example}{Example}[section]
      
\newenvironment{myassumption}[4]
{
\centering
\begin{assumption}[{\textbf{#1}\nopunct}]%
\index{#2}
\mbox{}\\  \vskip6pt \colorbox{black!15}{\fbox{\parbox{.9\textwidth}{#3}}}
\label{#4}
\end{assumption}
%\renewcommand{\theproposition}{\arabic{chapter}.\arabic{section}.\arabic{assumption}} 
}%
{}

\newenvironment{myproposition}[4]
{
\centering
\begin{proposition}[{\textbf{#1}\nopunct}]%
\index{#2} 
\mbox{}\\  \vskip6pt \colorbox{black!15}{\fbox{\parbox{.9\textwidth}{#3}}}
\label{#4}
\end{proposition}
%\renewcommand{\theproposition}{\arabic{chapter}.\arabic{section}.\arabic{proposition}} 
}%
{}

\newenvironment{mytheorem}[4]
{
\centering
\begin{theorem}[{\textbf{#1}\nopunct}]%
\index{#2} 
\mbox{}\\ \vskip6pt \colorbox{black!15}{\fbox{\parbox{.9\textwidth}{#3}}}
\label{#4}
\end{theorem}
%\renewcommand{\thetheorem}{\arabic{chapter}.\arabic{section}.\arabic{theorem}} 
}%
{}

\newenvironment{mydefinition}[4]
{
\centering
\begin{definition}[{\textbf{#1}\nopunct}]%
\index{#2} 
\mbox{}\\  \vskip6pt \colorbox{black!15}{\fbox{\parbox{.9\textwidth}{#3}}}
\label{#4}
\end{definition}
%\renewcommand{\thedefinitio{n}{\arabic{chapter}.\arabic{section}.\arabic{definition}} 
}%
{}

\newenvironment{myaxiom}[4]
{
\centering
\begin{axiom}[{\textbf{#1}\nopunct}]%
\index{#2} 
\mbox{}\\  \vskip6pt \colorbox{black!15}{\fbox{\parbox{.9\textwidth}{#3}}}
\label{#4}
\end{axiom}
%\renewcommand{\theaxiom}{\arabic{chapter}.\arabic{section}.\arabic{axiom}} 
}%
{}

\newenvironment{mylemma}[4]
{
\centering
\begin{lemma}[{\textbf{#1}\nopunct}]%
\index{#2} 
\mbox{}\\  \vskip6pt \colorbox{black!15}{\fbox{\parbox{.9\textwidth}{#3}}}
\label{#4}
\end{lemma}
%\renewcommand{\thelemma}{\arabic{chapter}.\arabic{section}.\arabic{lemma}} 
}%
{}
   
\newenvironment{reason}[1]
{
 \vskip0.05in
 \begin{myproof}
 \mbox{}\\#1
 $\qedsymbol$
 \end{myproof}  
 \vskip0.05in
}%
{}

\newenvironment{reasontwo}[1]
{
 \vskip+.05in
 \begin{myproof}
 \mbox{}\\#1
 \end{myproof}  
 \vskip+0.05in
}%
{}

\newenvironment{subreason}[1]
{
 \vskip0.05in
 \renewcommand{\themyproof}{}
 \begin{myproof}
 #1
 $\subqedsymbol$
 \end{myproof}
 \vskip0.05in
 \renewcommand{\themyproof}{\thetheorem}
 %\renewcommand{\themyproof}{\arabic{chapter}.\arabic{section}.\arabic{myproof}}   
 %
}%
{}  

\newenvironment{myhint}[1]
{
 \vskip0.05in
 \begin{hint}
 #1
 $\subqedsymbol$ 
 \end{hint}  
 \vskip0.05in
}%
{} 

\newenvironment{myeqn}[3]
{
 \renewcommand{\theequation}{$\boldsymbol{#1}$}
 \begin{eqnarray}
 \label{equation:#2}
 #3 
 \end{eqnarray}
 \renewcommand{\theequation}{\arabic{chapter}.\arabic{eqnarray}}   
}%
{} 


\JournalInfo{MATH 8210:  E-Learning Homework, 1-\pageref{LastPage}, 2020} % Journal information
\Archive{Draft Version \today} % Additional notes (e.g. copyright, DOI, review/research article)

\PaperTitle{MATH 8210 E-Learning Homework}
\Authors{Ian Davis\textsuperscript{1}}
\affiliation{\textsuperscript{1}\textit{John E. Walker Department of Economics,
Clemson University,Clemson, SC: email ijdavis@g.clemson.edu}}
%\affiliation{*\textbf{Corresponding author}: yournamehere@clemson.edu} % Corresponding author

\date{\small{Version ~\today}}
\Abstract{Variation of Parameters and Boundry Value Problems}
\Keywords{}
\newcommand{\keywordname}{Keywords}
%
\onehalfspacing
\begin{document}

\flushbottom
\addcontentsline{toc}{section}{Title}
\maketitle

\renewcommand{\theexercise}{\arabic{exercise}}%

\noindent Use VoP to solve

\begin{exercise}
  Consider
  \begin{eqnarray*}
    u''(t) - 4u(t) &=& f(t)\\
    u(0) &=& 1\\
    u(8) &=& -1\\
  \end{eqnarray*}
  where f is a continuous function
  \begin{solution}
    In this case, we want to note that we can follow the usual VoP steps with $\beta^2 = 1$. So this means that
    \begin{eqnarray*}
      4x_p(t) &=& v_1(t)e^{-t} + v_2(t)e^{t}\\
      4x_p'(t) &=& v_1'(t)e^{-t} - v_1(t)e^{-t} + v_2(t)e^{t} + v_2(t)e^{-t}\\
      &=& (v_1'(t)e^{-t} + v_2'(t)e^{t}) + (-v_1(t)e^{-t} + v_2(t)e^{-t})
    \end{eqnarray*}
    Now, we consider the related system
    \begin{eqnarray*}
      \frac{1}{4}
      \begin{bmatrix}
        e^{-t} & e^t \\
        -e^{-t} & e^t
      \end{bmatrix}
      \begin{bmatrix}
        \phi(t) \\
        \psi(t)
      \end{bmatrix}
      &=&
      \begin{bmatrix}
        0 \\
        f(t)
      \end{bmatrix}
      \\
      \implies \frac{1}{4}(v_1'(t)e^{-t} + v_2'(t)e^t) &=& 0\\
    \end{eqnarray*}
    as well as
    \begin{eqnarray*}
      x'p(t) = \frac{1}{4}(-v_1(t)e^{-t} + v_2(t)e^t) = 0\\
      \implies x_p''(t) = \frac{1}{4}(-v_1'(t)e^{-t} + v_2'(t)e^t + v_1(t)e^{-t} + v_2(t)e^t)
    \end{eqnarray*}
    Plugging this in to the nonhomogenous equation we get
    \begin{eqnarray*}
      f(t) &=& \frac{1}{4}(v_2'(t)e^t - v_1'(t)e^{-t} + v_2(t)e^t + v_1(t)e^{-t} - (v_2(t)e^t + v_1(t)e^{-t}))\\
      \implies f(t) &=& \frac{1}{4}(v_2'(t)e^t - v_1'(t)e^t)
    \end{eqnarray*}
    Then, by Cramer's rule, we get
    \begin{eqnarray*}
      v_1'(t) &=& -2f(t)e^t = -2f(t)e^t\\
      v_2'(t) &=& 2f(t)e^{-t} = 2f(t)e^{-t}\\
    \end{eqnarray*}
    which imply
    \begin{eqnarray*}
      v_1(t) &=& -2\int_0^t\:f(u)e^udu\\
      v_2(t) &=& 2\int_0^t\:f(u)e^{-u}du\\
      \implies x_p(t) &=& v_1(t)e^{-t} + v_2(t)e^t\\
      &=& (-2\int_0^t\:f(u)e^udu)e^{-t} + (2\int_0^t\:f(u)e^{-u}du)e^t
    \end{eqnarray*}
    And our general solution is
    \begin{eqnarray*}
      x(t) &=& x_h(t) + x_p(t)\\
      &=& A_1e^{-t} + A_2e^t - 2e^{-t}\int_0^t\:f(u)e^udu + 2e^t\int_0^t\:f(u)e^{-u}du\\
      &=& A_1e^{-t} + A_2e^t - 2\int_0^t\:f(u)e^{u -t}du + 2\int_0^t\:f(u)e^{-(u -t)}du\\
      &=& A_1e^{-t} + A_2e^t + 2\int_0^t\:f(u)(e^{u - t} - e^{-(u - t)})\\
      &=& A_1cosh(t) + A_2sinh(t) + 2\int_0^1\:f(u)sinh(t - u)
    \end{eqnarray*}
    Now, we consider the two initial conditions
    \begin{eqnarray*}
      u(0) &=& 1\\
      \implies A_1 &=& 1
    \end{eqnarray*}
    and
    \begin{eqnarray*}
      U(8) &=& - 1\\
      \implies -1 &=& cosh(8) + A_2sinh(8) + 2\int_0^8\:f(u)sinh(8 - u)du\\
      \implies A_2 &=& -\frac{1}{sinh(8)}(cosh(8) + 2\int_0^8\:f(u)sinh(8 - u)du + 1)
    \end{eqnarray*}
  \end{solution}
\end{exercise}

\begin{exercise}
 Consider
  \begin{eqnarray*}
    u''(t) + 4u(t) &=& f(t)\\
    u(0) &=& 1\\
    u(6) &=& -1\\
  \end{eqnarray*}
  where f is a continuous function
  \begin{solution}
    The characterstic equation in this case is $r^2 + 4 = 0$ which has complex roots of $\pm 2i$. Hence $u_1(t) = cos(3t)$ and $u_2(t) = sin(3t)$ so we set the solution to be
    \begin{eqnarray*}
      u_p(t) = A(t)cos(2t) + B(t)sin(2t)
    \end{eqnarray*}
    and the nonhomogenous solution to be of the form
    \begin{eqnarray*}
      u_p(t) = \phi(t)cos(2t) + \psi(t)sin(2t)
    \end{eqnarray*}
    satisfying
    \begin{eqnarray*}
      \begin{bmatrix}
        cos(2t) & sin(2t) \\
        -2sin(2t) & 2cos(2t)
      \end{bmatrix}
      \begin{bmatrix}
        \phi'(t) \\
        \psi'(t)
      \end{bmatrix}
      &=&
      \begin{bmatrix}
        0 \\
        f(t)
      \end{bmatrix}
      \\
    \end{eqnarray*}
    Applying Cramer's rule we get
    \begin{eqnarray*}
      \phi'(t) = \frac{1}{2}
      \begin{vmatrix}
        0 & sin(2t) \\
        f(t) & 2cos(2t)
      \end{vmatrix}
      = -\frac{1}{2}f(t)sin(2t)
    \end{eqnarray*}
    and
    \begin{eqnarray*}
      \psi'(t) = \frac{1}{2}
      \begin{vmatrix}
        cos(2t) & 0 \\
        -2sin(2t) & f(t)
      \end{vmatrix}
      = \frac{1}{2}f(t)cos(2t)
    \end{eqnarray*}
    integrating gives us
    \begin{eqnarray*}
      \phi(t) &=& -\frac{1}{2}\int_0^t\: f(u)sin(2u)du\\
      \psi(t) &=& \frac{1}{2}\int_0^t\: f(u)cos(2u)du
    \end{eqnarray*}
    and the general solution is
    \begin{eqnarray*}
      u(t) &=& Acos(2t) + Bsin(2t) - \frac{1}{2}\int_0^t\: f(u)sin(2u)ducos(2t) + \frac{1}{2}\int_0^t\: cos(2u)dusin(2t)\\
      &=& Acos(2t) + Bsin(2t) + \frac{1}{2}\int_0^t\: f(u)sin(2t - 2u)du
    \end{eqnarray*}
    Now, considering the initial conditions we get
    \begin{eqnarray*}
      u(0) &=& 1\\
      \implies A &=& 1
    \end{eqnarray*}
    and
    \begin{eqnarray*}
      u(4) &=& -1\\
      \implies B &=& -\frac{1}{sin(8)}(cos(8) + \frac{1}{2}\int_0^4\: f(u)sin(8 - 2u)du + 1)
    \end{eqnarray*}
  \end{solution}
\end{exercise}

\noindent Do BVP analysis for
\begin{exercise}
 Consider
  \begin{eqnarray*}
    u'' + 9\omega^2u &=& f, \: 0 \leq x \leq 3,\\
    u'(0) &=& 0\\
    u'(3) &=& 0\\
  \end{eqnarray*}
\end{exercise}

\begin{solution}
  The general solution to the homogenous equaiton is given by
  \begin{eqnarray*}
    u_h(x) = Acos(3\omega x) + Bsin(3\omega x)
  \end{eqnarray*}
  Applying VoP gives
  \begin{eqnarray*}
    u_p(x) = \phi(x)cos(3\omega x) + \psi(x)sin(3\omega x)
  \end{eqnarray*}
  we want $\phi(x)$ and $\psi(x)$ which simplify
  \begin{eqnarray*}
    \begin{vmatrix}
      cos(3\omega x) & sin(3\omega x)\\
      -3\omega sin(3\omega x) & 3\omega cos(3\omega x)
    \end{vmatrix}
    \begin{vmatrix}
      \phi'(x)\\
      \psi'(x)
    \end{vmatrix}
    =
    \begin{vmatrix}
      0\\
      f(x)
    \end{vmatrix}
  \end{eqnarray*}
  The det of the 2x2 matrix above is $3\omega$ so we can use Cramer's rule to get
  \begin{eqnarray*}
    \phi'(x) = -\frac{f(x)sin(3\omega x)}{3\omega}\\
    \psi'(x) = \frac{f(x)cos(3\omega x)}{3\omega}
  \end{eqnarray*}
  Hence
  \begin{eqnarray*}
    \phi(x) = -\frac{1}{3\omega}\int_0^x\: f(s)sin(3\omega s)ds\\
    \psi(x) = \frac{1}{3\omega}\int_0^x\: f(s)cos(3\omega s)ds\\
  \end{eqnarray*}
  Because the general solution is
  \begin{eqnarray*}
    u(x) = u_n(x) + u_p(x)
  \end{eqnarray*}
  we get
  \begin{eqnarray*}
    u(x) = Acos(3\omega x) + Bsin(3\omega x) + \frac{1}{w}\int_0^x\: f(s)sin(3\omega(x - s))ds
  \end{eqnarray*}
  Next, we apply the boundary conditions. First, consider leibniez rule for derivatives which gives us
  \begin{eqnarray*}
    u'(x) = -3\omega Asin(3\omega x) + 3\omega cos(3\omega x) + \int_0^3\: f(s)cos(3\omega(3 - s))ds
  \end{eqnarray*}
  Hence, $u'(0) = 0 = 3\omega B$ and 
  \begin{eqnarray*}
    u'(3) &=& -3\omega Asin(9\omega) + 3\omega cos(9\omega) + \int_0^3\: f(s)cos(\omega(3 - s))\\
    \implies B &=& 0 
  \end{eqnarray*}
  But we cannot find A from 
  \begin{eqnarray*}
    0 = -3\omega Asin(9\omega) + 3\omega cos(9\omega) + \int_0^3\: f(s)cos(\omega(3 - s))\\
  \end{eqnarray*}
  So consider the cases.\\
  Case 1:
  \begin{eqnarray*}
    3\omega = n\pi \implies \int_0^3\: f(s)cos(n\pi(3 - s))ds = 0
  \end{eqnarray*}
  Case 2:
  \begin{eqnarray*}
    3\omega \neq n\pi\\
    \implies \frac{1}{3\omega sin(9\omega)}\int_0^3 \:f(s)cos(3\omega(3- s))ds = A
  \end{eqnarray*}
  giving us
  \begin{eqnarray*}
    u(x) = \frac{cos(3\omega x)}{3\omega sin(9\omega)}\int_0^3\: f(s)cos(3\omega(3 - s))ds + \frac{1}{3\omega}\int_0^x\: f(s)sin(3\omega(x - s))ds
  \end{eqnarray*}
  which can be manipulated to give us
  \begin{eqnarray*}
    u(x) = \int_0^3\: f(s)k_w(x,s)ds
  \end{eqnarray*}
  with
  \begin{eqnarray*}
    k_w(x,s) &=& \frac{1}{3\omega sin(9\omega)}
    \left \{
    \begin{array}{ll}
      cos(3\omega s)cos(3\omega(3 - x)), & 0 \leq s \leq x\\
      cos(3\omega x)cos(3\omega(3 - s)), & x < s \leq 3
    \end{array}
    \right
  \end{eqnarray*}
\end{solution}

\noindent Do the cable problem analysis for
\begin{exercise}
  \begin{eqnarray*}
    4\frac{\delta^2 \Phi}{\delta x^2} - \Phi - 0.7\frac{\delta\Phi}{\delta t} &=& 0, \: 0 \leq x \leq 6, \: t \geq 0,\\
    \frac{\delta\Phi}{\delta t}(0,t) &=& 0,\\
    \frac{\delta\Phi}{\delta t}(6,t) &=& 0,\\
    \Phi(x,0) &=& f(x)
  \end{eqnarray*}
  for $f(x) = 5x(6 - x)$. Find the first four terms of the solution $\Phi$.

  \begin{solution}
    Seperating the variables we get
    \begin{eqnarray*}
      \Phi(x,t) = u(x)w(t)
    \end{eqnarray*}
    giving us
    \begin{eqnarray*}
      4\frac{d^2u}{dx^2}w(t) - u(x)w(t) - .7u(x)\frac{dw}{dt} = 0 
    \end{eqnarray*}
    Rewriting we find, for all x,t we need
    \begin{eqnarray*}
      w(t)(4\frac{d^2u}{dx^2} - u(x)) = .7u(x)\frac{dw}{dt}
    \end{eqnarray*}
    Rewriting again, we have
    \begin{eqnarray*}
      \frac{4\frac{d^2u}{dx^2} - u(x)}{u(x)} = \frac{.7\frac{dw}{dt}}{w(t)}, 0 \leq x \leq 6, t > 0
    \end{eqnarray*}
    And get the following two conditions
    \begin{eqnarray*}
      .7\frac{dw}{dt} = \Theta w(t), t > 0\\
      4\frac{d^2u}{dx^2} = (1 + \Theta)u(x), 0 \leq x \leq 6
    \end{eqnarray*}
    We also get boundry conditions
    \begin{eqnarray*}
      \frac{du}{dx}(0)w(t) = 0, t > 0\\
      \frac{du}{dx}(6)w(t) = 0, t > 0
    \end{eqnarray*}
    Since these questions must hold for all t, it forces
    \begin{eqnarray*}
      \frac{du}{dx}(0) = 0\\
      \frac{du}{dx}(6) = 0
    \end{eqnarray*}
    The model now becomes
    \begin{eqnarray*}
      u'' - \frac{1 + \Theta}{4}u = 0\\
      \frac{du}{dx}(0) = 0\\
      \frac{du}{dx}(6) = 0
    \end{eqnarray*}
    We are only looking for nonzero solutions so any $\Theta$ leading to a zero gets rejected\\
    Case 1: $1 + \Theta = \omega^2, \omega \neq 0$
    \begin{eqnarray*}
      u'' - \frac{\omega^2}{4}u = 0\\
      u'(0) = 0\\
      u'(6) = 0
    \end{eqnarray*}
    with characteristic equation $r^2 - \frac{\omega^2}{4}$ with real roots $\pm \frac{\omega}{4}$. The general solution is given by $u(x) = Acosh(\frac{\omega}{2}x) + Bsinh(\frac{\omega}{2}x)$ which implies
    \begin{eqnarray*}
     u'(x) = A\frac{\omega}{2}sinh(\frac{\omega}{2}x) +  b\frac{\omega}{2}cosh(\frac{\omega}{2}x)
    \end{eqnarray*}
    Then apply the boundry conditions and get
    \begin{eqnarray*}
      u'(0) = B = 0\\
      u'(6) = 0 = Asinh(3\omega)
    \end{eqnarray*}
    But since $sinh(3\omega)$ is never 0 when $\omega \neq 0$, we see A = 0 also. So the only solution if u is trivial and thats boring so we throw it out.\\
    Case 2: $1 + \Theta = 0$\\
    The model to solve now is
    \begin{eqnarray*}
      u'' = 0\\
      u'(0) = 0\\
      u'(6) = 0
    \end{eqnarray*}
    with characteristic equation $r = 0$ with double root $r = 0$. So the general solution is
    \begin{eqnarray*}
      u(x) = A + Bx
    \end{eqnarray*}
    Applying boundry conditions gives us
    \begin{eqnarray*}
      u(0) = 0\\
      u(6) = 0
    \end{eqnarray*}
    Hence, since $u'(x) = B$, we have
    \begin{eqnarray*}
      u'(0) = 0 = B\\
      u'(6) = 0 = B6
    \end{eqnarray*}
    Hence, B = 0. But the value of A can't be determined. Hence any non zero A gives a nonzero solution.\\
    Choosing A = 1, let $u_0(x) = 1$ be our chosen nonzero. We now need to solve for w in this case. Becuase $\Theta = -1$, the model to solve is\\
    \begin{eqnarray*}
      \frac{dw}{dt} = -\frac{1}{.7}w(t), 0 \leq t
   \end{eqnarray*}
   An the general solution is $w(t) = Ce^{-\frac{1}{.7}t} \; \forall C$. Choosing C = 1 we get 
   \begin{eqnarray*}
    w_0(t) = e^{-\frac{1}{.7}t}
   \end{eqnarray*}
   Hence, the product $\phi_0(x,t) = u_0(x)w_0(t)$ solves the boundary conditions. That is
   \begin{eqnarray*}
    \phi_0(x,t) = e^{-\frac{1}{.7}t}
   \end{eqnarray*}
   is a solution\\
   Case 3: $1 + \Theta = -\omega^2, \omega \neq 0$
   \begin{eqnarray*}
    u(x) = Acos(\frac{\omega}{2}x) +Bsin(\frac{\omega}{2}x)
   \end{eqnarray*}
   and hence
   \begin{eqnarray*}
    u'(x) = -A\frac{\omega}{2}sin(\frac{\omega}{2}x) +B\frac{\omega}{2}cos(\frac{\omega}{2}x)
   \end{eqnarray*}
   Next, apply boundary conditions to find
   \begin{eqnarray*}
    u'(0) = 0 = B\\
    u'(6) = 0 = Asin(3\omega)
   \end{eqnarray*}
   Hence, B = 0 and $Asin(3\omega) = 0$. Thus, we can determine a unieq value for A only if $sin(3\omega) \neq 0$. If $\omega = \frac{n\pi}{3}$ we can solve for A and find A = 0 but other A's cannot be determined. So the only solutions are the trivial or zero solutions unless $\omega 6 = \pi n^2$\\
   \\
   Letting $\omega_n = \frac{n\pi}{3}$, we find a non-zero solution for each nonzero value of A of the form
   \begin{eqnarray*}
    u_n(x) = Acos(\frac{\omega_n}{4}x) = Acos(\frac{n\pi}{6}x)
   \end{eqnarray*}
   For convenience, let's choose all the constants A = 1. Then we have an infinite family of nonzero solutions $u_n(x) = cos(\frac{n\pi}{6}x)$ and an infinite family of separation constants $\Theta_n = -1 - \omega^2_n = -1 0 \frac{n^2\pi^2 4}{36}$\\
   We can solve the w equation. We must solve
   \begin{eqnarray*}
    \frac{dw}{dt} = - \frac{1 + \omega_n^2}{.7}w(t), t \geq 0
   \end{eqnarray*}
   The general solution is
   \begin{eqnarray*}
    w(t) = B_ne^{-\frac{1 + \omega^2_n}{.7}} = B_ne^{-(\frac{1}{.7} + \frac{n^2 \pi^2 4}{(.7)36})t}
   \end{eqnarray*}
   Choosing the constants $B_n = 1$, we obtain the w function\\
   \begin{eqnarray*}
    w_n(t) = e^{-(\frac{1}{.7} + \frac{n^2 \pi^2 4}{(.7)36})t}
   \end{eqnarray*}
   Hence, any product
   \begin{eqnarray*}
    \phi_n(x,t) = u_n(x)w_n(t)
   \end{eqnarray*}
   Will solve the model with the x boundry conditions.\\
   Further, any infinite sum of the form for arbitrary constants $A_n$
   \begin{eqnarray*}
    \Psi_n(x,t) = \sum_{n=1}^N\: A_n\phi_n(x,t) = \sum_{n=1}^N\: A_nu_n(x)w_n(t) = sum_{n=1}^N\: A_ncos(\frac{n\pi}{6})e^{-(\frac{1}{.7} + \frac{n^2 \pi^2 4}{(.7)36})t}
   \end{eqnarray*}
   Adding the $1 + \Theta = 0$ case, we find the most general finite term solution has the form
   \begin{eqnarray*}
    \phi_n(x,t) &=& A_0\Phi_0(x,t) + \sum_{n=1}^\infty\: A_n\phi_n(x,t)\\
    &=& A_0u_0w_0 + \sum_{n=1}^\infty\: A_nu_n(x)w_n(t)\\
    &=& A_0e^{-\frac{1}{.7}} + \sum_{n=1}^\infty\: A_ncos(\frac{n\pi}{6})e^{-(\frac{1}{.7} + \frac{n^2 \pi^2 4}{(.7)36})t} 
   \end{eqnarray*}
   The finite solutions solve the boundry conditions $\frac{\delta \Phi}{\delta x}(0,t) = 0$ and $\frac{\delta \Phi}{\delta x}(6,t) = 0$\\
   For $\Phi(x,0) = f(x)$, we let $n \rightarrow \infty$ and set
   \begin{eqnarray*}
    \Phi(x,t) &=& A_0\Phi_0(x,t) + \sum_{n=1}^\infty\: A_n\phi_n(x,t)\\
    &=& A_0e^{-\frac{1}{.7}t} + \sum_{n=1}^\infty\: A_ncos(\frac{n\pi}{6})e^{-(\frac{1}{.7} + \frac{n^2 \pi^2 4}{(.7)36})t} 
   \end{eqnarray*}
   We take the partial derivative with respect to x of the infinite term and get
   \begin{eqnarray*}
     \sum_{n=1}^\infty\: A_n\frac{n\pi}{6}sin(\frac{n\pi}{6})e^{-(\frac{1}{.7} + \frac{n^2 \pi^2 4}{(.7)36})t} 
   \end{eqnarray*}
   Which satisfies the derivative boundary conditions. So $\Phi(x,t)$ is given by
   \begin{eqnarray*}
    \Phi(x,t) = A_0e^{-\frac{1}{.7}t} + \sum_{n=1}^\infty\: A_ncos(\frac{n\pi}{6}x)e^{-\frac{36 + n^2\pi^2 4}{(.7)36}t}
   \end{eqnarray*}
   For appropriate constants $(A_n)$. We find the right choice of $(A_n)$ comes from
   \begin{eqnarray*}
    \Phi(x,t) = f(x) for 0 \leq x \leq 6
   \end{eqnarray*}
   We know
   \begin{eqnarray*}
    \Phi(x,0) = A_0 + \sum_{n=1}^\infty\: A_ncos(\frac{n\pi}{6}x)
   \end{eqnarray*}
   and so rewriting these in terms of the series solution for $0 \leq x \leq 6$ we find
   \begin{eqnarray*}
    A_0 + \sum_{n=1}^\infty\: A_ncos(\frac{n\pi}{6}x) = f(x)
   \end{eqnarray*}
   The fourier series for the appropriate f is given by
   \begin{eqnarray*}
    f(x) = B + \sum_{n=1}^\infty\: B_ncos(\frac{n\pi}{6}x)
   \end{eqnarray*}
   with
   \begin{eqnarray*}
    B_0 &=& \frac{1}{6}\int_0^6 \: f(x)dx\\
    B_n &=& \frac{1}{3}\int_0^6f(x)cos(\frac{n\pi}{6}x)dx
   \end{eqnarray*}
   Thus setting these series equal, we find a solution given by $A_n = B_n$ for all $n \geq 0$. Thus, we know
   \begin{eqnarray*}
    \Phi(x,t) = B_0 + \sum_{n = 1}^\infty \: B_ncos(\frac{n\pi}{6}x)e^{-\frac{36 + n^2\pi^2 4}{.7(36)}t}
   \end{eqnarray*}
  \end{solution}
\end{exercise}

\end{document}