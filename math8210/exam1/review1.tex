% $Header$

\documentclass{beamer}

% This file is a solution template for:

% - Talk at a conference/colloquium.
% - Talk length is about 20min.
% - Style is ornate.



% Copyright 2004 by Till Tantau <tantau@users.sourceforge.net>.
%
% In principle, this file can be redistributed and/or modified under
% the terms of the GNU Public License, version 2.
%
% However, this file is supposed to be a template to be modified
% for your own needs. For this reason, if you use this file as a
% template and not specifically distribute it as part of a another
% package/program, I grant the extra permission to freely copy and
% modify this file as you see fit and even to delete this copyright
% notice. 


\mode<presentation>
{
  \usetheme{Warsaw}
  % or ...

  \setbeamercovered{transparent}
  % or whatever (possibly just delete it)
}


\usepackage[english]{babel}
% or whatever

\usepackage[latin1]{inputenc}
% or whatever
\usepackage{listings}
\usepackage{times}
\usepackage[T1]{fontenc}
% Or whatever. Note that the encoding and the font should match. If T1
% does not look nice, try deleting the line with the fontenc.


\title[Math 8210: Review 1] % (optional, use only with long paper titles)
{Mapping of Inifinite Dimensional Spaces}

\subtitle
{Exam 1 Review}

\author[Davis, 2020] % (optional, use only with lots of authors)
{Ian Davis\inst{1}}
% - Give the names in the same order as the appear in the paper.
% - Use the \inst{?} command only if the authors have different
%   affiliation.

\institute[Clemson University] % (optional, but mostly needed)
{
  \inst{1}%
  John E. Walker Department of Economics\\
  Clemson University
  }
 
% - Use the \inst command only if there are several affiliations.
% - Keep it simple, no one is interested in your street address.

\date[Spring 2020] % (optional, should be abbreviation of conference name)
{Spring 2020}
\subject{Linear Analysis}
% This is only inserted into the PDF information catalog. Can be left
% out. 



% If you have a file called "university-logo-filename.xxx", where xxx
% is a graphic format that can be processed by latex or pdflatex,
% resp., then you can add a logo as follows:

% \pgfdeclareimage[height=0.5cm]{university-logo}{university-logo-filename}
% \logo{\pgfuseimage{university-logo}}



% Delete this, if you do not want the table of contents to pop up at
% the beginning of each subsection:
\AtBeginSubsection[]
{
  \begin{frame}<beamer>{Outline}
    \tableofcontents[currentsection,currentsubsection]
  \end{frame}
}


% If you wish to uncover everything in a step-wise fashion, uncomment
% the following command: 

%\beamerdefaultoverlayspecification{<+->}


\begin{document}

\begin{frame}
  \titlepage
\end{frame}

\begin{frame}{Outline}
  \tableofcontents
  % You might wish to add the option [pausesections]
\end{frame}


% Structuring a talk is a difficult task and the following structure
% may not be suitable. Here are some rules that apply for this
% solution: 

% - Exactly two or three sections (other than the summary).
% - At *most* three subsections per section.
% - Talk about 30s to 2min per frame. So there should be between about
%   15 and 30 frames, all told.

% - A conference audience is likely to know very little of what you
%   are going to talk about. So *simplify*!
% - In a 20min talk, getting the main ideas across is hard
%   enough. Leave out details, even if it means being less precise than
%   you think necessary.
% - If you omit details that are vital to the proof/implementation,
%   just say so once. Everybody will be happy with that.

\section{Definitions}
\subsection{General}

\begin{frame}{The Reals}

	\begin{block}{The Completeness Axiom}
	\end{block}

\end{frame}
\begin{frame}{The Reals}
	\begin{block}{The Completeness Axiom}
		Let S be a set of real number which is nonempty and bounded above. Then the supremum of S exists and is finite.\\
		Let S be a set of real numbers which is nonempty and bounded below. The the infimum of S exists and is finite.
	\end{block}
\end{frame}

\begin{frame}{The Reals}

        \begin{block}{A Metric for the Completion of the Rationals}
        \end{block}

\end{frame}

\begin{frame}
	\begin{block}{A Metric for the Completion of the Rationals}
		let $\tilde{d}: \tilde{Q}x\tilde{Q} \rightarrow \tilde{Q} $ be defined by $\tilde{d}([(x_n)],[(y_n)]) = [(|x_n - y_n|)]$ 
	\end{block}
\end{frame}

\subsection{Complex Numbers}

\begin{frame}{Make Titles Informative.}
\end{frame}

\begin{frame}{Make Titles Informative.}
\end{frame}



\section{Theorems/Lemmas}

\subsection{Named}

\begin{frame}{Named}
	\begin{block}{Completed Reals is A Metric Space}
	\end{block}
\end{frame}

\begin{frame}{Named}
        \begin{block}{Completed Reals is A Metric Space}
		$\tilde{d}$ is a metric on $\tilde{Q}$ and $(\tilde{Q},\tilde{d})$ is a metric space
	\end{block}
\end{frame}

\begin{frame}{Named}
	\begin{block}{The Constructed Reals are Complete}
	\end{block}
\end{frame}

\begin{frame}{Named}
        \begin{block}{The Construction of Reals are Complete}
		Let $([x_n^p)])$ be a Cauchy Sequence in $(\tilde{Q},\tilde{d})$. Then there is an $[(x_n)]$ in $(\tilde{Q},\tilde{d})$ so that $\tilde{d}([(x_n^p)],[(x_n)]) \rightarrow [0]$. 
        \end{block}
\end{frame}

\begin{frame}{Named}
	\begin{block}{The Rationals are dense in the completed rationals}
	\end{block}
\end{frame}

\begin{frame}{Named}
        \begin{block}{The Rationals are dense in the completed rationals}
        	Embed $Q$ into $(\tilde{Q},\tilde{d})$ by $q \rightarrow [(\tilde{q})]$. Then this embedding of $\tilde{Q}$ is dense in $(\tilde{Q},\tilde{d})$.
	\end{block}
\end{frame}

\subsection{Unnamed}

\begin{frame}{Unnamed}
	\begin{block}{$(x_n)$ is CS sequence of rationals and $(x_{n_k})$ is one of it's subsequences. What relation is guaranteed?}
	\end{block}
\end{frame}

\begin{frame}{Unnamed}
	\begin{block}{$(x_n)$ is CS sequence of rationals and $(x_{n_k})$ is one of it's subsequences. What relation is guaranteed?}
		$(x_{n_k}) \sim (x_n)$
        \end{block}
\end{frame}

\begin{frame}{Unnamed}
	\begin{block}{$[(x_n)] \preceq [(y_n)]$ relation to $(x_n) \space \& \space (y_n)$}
	\end{block}
\end{frame}

\begin{frame}{Unnamed}
	\begin{block}{$[(x_n)] \preceq [(y_n)]$ relation to $(x_n)$ $\&$ $(y_n)$}
		$\iff (x_n) \preceq (y_n)$
	\end{block}
\end{frame}



\section*{Summary}

\begin{frame}{Summary}

  % Keep the summary *very short*.
  \begin{itemize}
  \item
    The \alert{first main message} of your talk in one or two lines.
  \item
    The \alert{second main message} of your talk in one or two lines.
  \item
    Perhaps a \alert{third message}, but not more than that.
  \end{itemize}
  
  % The following outlook is optional.
  \vskip0pt plus.5fill
  \begin{itemize}
  \item
    Outlook
    \begin{itemize}
    \item
      Something you haven't solved.
    \item
      Something else you haven't solved.
    \end{itemize}
  \end{itemize}
\end{frame}



% All of the following is optional and typically not needed. 
\appendix
\section<presentation>*{\appendixname}
\subsection<presentation>*{For Further Reading}

\begin{frame}[allowframebreaks]
  \frametitle<presentation>{For Further Reading}
    
  \begin{thebibliography}{10}
    
  \beamertemplatebookbibitems
  % Start with overview books.

  \bibitem{Author1990}
    A.~Author.
    \newblock {\em Handbook of Everything}.
    \newblock Some Press, 1990.
 
    
  \beamertemplatearticlebibitems
  % Followed by interesting articles. Keep the list short. 

  \bibitem{Someone2000}
    S.~Someone.
    \newblock On this and that.
    \newblock {\em Journal of This and That}, 2(1):50--100,
    2000.
  \end{thebibliography}
\end{frame}

\end{document}
