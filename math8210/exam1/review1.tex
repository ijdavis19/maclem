% $Header$

\documentclass{beamer}

% This file is a solution template for:

% - Talk at a conference/colloquium.
% - Talk length is about 20min.
% - Style is ornate.



% Copyright 2004 by Till Tantau <tantau@users.sourceforge.net>.
%
% In principle, this file can be redistributed and/or modified under
% the terms of the GNU Public License, version 2.
%
% However, this file is supposed to be a template to be modified
% for your own needs. For this reason, if you use this file as a
% template and not specifically distribute it as part of a another
% package/program, I grant the extra permission to freely copy and
% modify this file as you see fit and even to delete this copyright
% notice. 


\mode<presentation>
{
  \usetheme{Warsaw}
  % or ...

  \setbeamercovered{transparent}
  % or whatever (possibly just delete it)
}


\usepackage[english]{babel}
% or whatever

\usepackage[latin1]{inputenc}
% or whatever
\usepackage{listings}
\usepackage{times}
\usepackage[T1]{fontenc}
\usepackage{mathrsfs}
% Or whatever. Note that the encoding and the font should match. If T1
% does not look nice, try deleting the line with the fontenc.


\title[Math 8210: Review 1] % (optional, use only with long paper titles)
{Mapping of Inifinite Dimensional Spaces}

\subtitle
{Exam 1 Review}

\author[Davis, 2020] % (optional, use only with lots of authors)
{Ian Davis\inst{1}}
% - Give the names in the same order as the appear in the paper.
% - Use the \inst{?} command only if the authors have different
%   affiliation.

\institute[Clemson University] % (optional, but mostly needed)
{
  \inst{1}%
  John E. Walker Department of Economics\\
  Clemson University
  }
 
% - Use the \inst command only if there are several affiliations.
% - Keep it simple, no one is interested in your street address.

\date[Spring 2020] % (optional, should be abbreviation of conference name)
{Spring 2020}
\subject{Linear Analysis}
% This is only inserted into the PDF information catalog. Can be left
% out. 



% If you have a file called "university-logo-filename.xxx", where xxx
% is a graphic format that can be processed by latex or pdflatex,
% resp., then you can add a logo as follows:

% \pgfdeclareimage[height=0.5cm]{university-logo}{university-logo-filename}
% \logo{\pgfuseimage{university-logo}}



% Delete this, if you do not want the table of contents to pop up at
% the beginning of each subsection:
\AtBeginSubsection[]
{
  \begin{frame}<beamer>{Outline}
    \tableofcontents[currentsection,currentsubsection]
  \end{frame}
}


% If you wish to uncover everything in a step-wise fashion, uncomment
% the following command: 

%\beamerdefaultoverlayspecification{<+->}


\begin{document}

\begin{frame}
  \titlepage
\end{frame}

\begin{frame}{Outline}
  \tableofcontents
  % You might wish to add the option [pausesections]
\end{frame}


% Structuring a talk is a difficult task and the following structure
% may not be suitable. Here are some rules that apply for this
% solution: 

% - Exactly two or three sections (other than the summary).
% - At *most* three subsections per section.
% - Talk about 30s to 2min per frame. So there should be between about
%   15 and 30 frames, all told.

% - A conference audience is likely to know very little of what you
%   are going to talk about. So *simplify*!
% - In a 20min talk, getting the main ideas across is hard
%   enough. Leave out details, even if it means being less precise than
%   you think necessary.
% - If you omit details that are vital to the proof/implementation,
%   just say so once. Everybody will be happy with that.

\begin{frame}{The Reals}

	\begin{block}{The Completeness Axiom}
	\end{block}

\end{frame}
\begin{frame}{The Reals}
	\begin{block}{The Completeness Axiom}
		Let S be a set of real number which is nonempty and bounded above. Then the supremum of S exists and is finite.\\
		Let S be a set of real numbers which is nonempty and bounded below. The the infimum of S exists and is finite.
	\end{block}
\end{frame}

\begin{frame}{The Reals}

	\begin{block}{Describe the Process for constructing $\widetilde{Q}$}
        \end{block}

\end{frame}

\begin{frame}{The Reals}

        \begin{block}{Describe the Process for constructing $\widetilde{Q}$}
        	1) Let $s = \{(x_n)|x_n \in Q,(x_n) is a Cauchy Sequence in (Q,|.|)\}$\\
		2) Define an equivalence relation on S by $(x_n) \sim (y_n) \iff lim_{n \rightarrow \infty}|x_n - y_n| = 0$\\
		3) Let $\widetilde{Q} = \frac{S}{\sim}$, the set of all equivalence classes in S under $\sim$. Let the elements of $\widetilde{Q}$ is a field be denoted by $[(x_n)]$\\
		4) Show $\widetilde{Q}$ is a field\\
		5) Show $\widetilde{Q}$ is a totally ordered set\\
		6) Show the ordering and operators are compatible\\
		7) Show the $\widetilde{Q}$ is dense\\
		8) Show the $\widetilde{Q}$ is complete\\
	\end{block}
\end{frame}

\begin{frame}{The reals}
	\begin{block}{Describe the process by which we make $\widetilde{Q}$ a field}
	\end{block}
\end{frame}

\begin{frame}{The reals}
        \begin{block}{Describe the process by which we make $\widetilde{Q}$ a field}
		Define operators $\oplus$ and $\odot$\\
		F1: Associativity of $\oplus$ and $\odot$\\
		F2: Communativity of $\oplus$ and $\odot$\\
		F3: Distributivity of $\odot$ over $\oplus$\\
		F4: Existence of additive identity\\
		F5: Existence of multiplicative identity\\
		F6: Existence of additive inverse\\
		F7: If $x \neq 0$, then there is $y$ so that $x \odot y = 1$
	\end{block}
\end{frame}

\begin{frame}{The reals}
	\begin{block}{Definition of $\oplus$ and $\odot$ in $\widetilde{Q}$}
        \end{block}
\end{frame}

\begin{frame}{The reals}
        \begin{block}{Definition of $\oplus$ and $\odot$ in $\widetilde{Q}$}
		$[(x_n)] \oplus [(y_n)] = [(x_n + y_n)]$\\
		$[(x_n)] \odot [(y_n)] = [(x_ny_n)]$
	\end{block}
\end{frame}


\begin{frame}{The reals}
        \begin{block}{Associativity of $\oplus$ and $\odot$ in $\widetilde{Q}$}
        \end{block}
\end{frame}

\begin{frame}{The reals}
        \begin{block}{Associativity of $\oplus$ and $\odot$ in $\widetilde{Q}$}
		$([(x_n)] \oplus [(y_n)]) \oplus [(z_n)] = [(x_n)] \oplus ([(y_n)] \oplus [(z_n))]$\\
		$([(x_n)] \odot [(y_n)]) \odot [(z_n)] = [(x_n)] \odot ([(y_n)] \odot [(z_n))]$
        \end{block}
\end{frame}

\begin{frame}{The reals}
        \begin{block}{Commutativity of $\oplus$ and $\odot$ in $\widetilde{Q}$}
        \end{block}
\end{frame}

\begin{frame}{The reals}
        \begin{block}{Commutativity of $\oplus$ and $\odot$ in $\widetilde{Q}$}
		$[(x_n)] \oplus [(y_n]) = [(y_n]) \oplus [(x_n)]$
		$[(x_n)] \odot [(y_n]) = [(y_n]) \odot [(x_n)]$
	\end{block}
\end{frame}

\begin{frame}{The reals}
        \begin{block}{Distributivity of $\odot$ over $\oplus$ in $\widetilde{Q}$}
                $[(x_n)] \odot ([(y_n)] \oplus [(z_n)]) = [(x_n)] \odot [(y_n)] \oplus [(x_n)] \odot [(z_n)]$
        \end{block}
\end{frame}

\begin{frame}{The reals}
	\begin{block}{Existence of additive identity}
        \end{block}
\end{frame}

\begin{frame}{The reals}
        \begin{block}{Existence of additive identity}
		$[(x_n)] \oplus [(\bar{0})] = [(x_n)]$
	\end{block}
\end{frame}

\begin{frame}{The reals}
        \begin{block}{Existence of multiplicative identity}
        \end{block}
\end{frame}

\begin{frame}{The reals}
        \begin{block}{Existence of multiplicative identity}
                $[(x_n)] \odot [(\bar{1})] = [(x_n)]$
        \end{block}
\end{frame}

\begin{frame}{The reals}
        \begin{block}{Existence of additive inverse}
        \end{block}
\end{frame}

\begin{frame}{The reals}
        \begin{block}{Existence of additive inverse}
        	$[(x_n)] \odot [(-x_n)] = [(\bar{0})]$
	\end{block}
\end{frame}

\begin{frame}{The reals}
        \begin{block}{Existence of a multiplicative inverse for $[(x_n)] \neq [(\bar{0})]$}
        \end{block}
\end{frame}

\begin{frame}{The reals}
	\begin{block}{Existence of multiplicative inverse for $[(x_n)] \neq [(\bar{0})]$}
		$\hat{x}_n = x_n + \text{sign}(x_n)\frac{1}{n}$\\
		$\hat{y}_n = \frac{1}{\hat{x}_n}$\\
		$[(x_n)] \odot [(\hat{y}_n)] = [(\bar{1})]$
        \end{block}
\end{frame}

\begin{frame}{The reals}
        \begin{block}{Describe the process by which we add $\prec$ to $\widetilde{Q}$}
        \end{block}
\end{frame}

\begin{frame}{The reals}
        \begin{block}{Describe the process by which we add $\prec$ to $\widetilde{Q}$}
                $(x_n) \prec (y_n) \iff (x_n) \sim (y_n)$ or $\exists N \ni n > N \implies x_n \le
q y_n$
        \end{block}
\end{frame}

\begin{frame}{The reals}
	\begin{block}{Total Ordering Properties}
		TO1: $[(x_n)] \preceq [(x_n)]$ reflexivity\\
		TO2: $[(x_n)] \preceq [(y_n)]$ and $[(y_n)] \prec [(x_n)] \implies [(x_n)] = [(y_n)]$ antisymmetry\\
		TO3: $[(x_n)] \preceq [(y_n)]$ and $[(y_n)] \prec [(z_n)] \implies [(x_n)] \prec [(z_n)]$ transitivity\\
		TO2: $[(x_n)] \preceq [(y_n)]$ or $[(y_n)] \prec [(x_n)]$ totality
        \end{block}
\end{frame}

\begin{frame}{The reals}
	\begin{block}{Proof for Q to be dense in $\widetilde{Q}$}
		let $[(\hat{\epsilon})]$ for $\epsilon > 0$ be given and let $[(x_n)]$ be chosen. Then for this $\epsilon \exists N \ni |x_n - x_m| < \epsilon$ if $n,m < N$. Choose the constant rational sequence $[(x_{N+1})]$ then $d([(x_{N+1})],[(x_n)[) = [(x_{N+1} - x_n)]$\\
		But $[(x_{N+1} - x_n)] < \epsilon$ for $n > N$ which implies $[(|x_{N+1} - x_n|)] \prec [(\bar{\epsilon})]$
        \end{block}
\end{frame}

\begin{frame}{the reals}
        \begin{block}{A Metric for the Completion of the Rationals}
                let $\widetilde{d}: \widetilde{Q}x\tilde{Q} \rightarrow \widetilde{Q} $ be defined by $\tilde{d}([(x_n)],[(y_n)]) = [(|x_n - y_n|)]$
        \end{block}
\end{frame}

\begin{frame}{The reals}
	\begin{block}{Explain the process of completing $(\widetilde{Q},\widetilde{d})$}
	\end{block}
\end{frame}

\begin{frame}{The reals}
	\begin{block}{Explain the process of completing $(\widetilde{Q},\widetilde{d})$}
		1. Let $S = \{(x_n)|x_n \in Q, (x_n) \text{is a Cauchy Sequence in} (Q,d)\}$\\
		2. Define an equivalence relation on S by $(x_n) \sim (y_n) \iff lim_{n \rightarrow \infty} d(x_n,y_n) =0$\\
		3. Let $\widetilde{Q} = \frac{S}{\sim}$, the set of all equivalence classes in $S$ under $\sim$. Let the elements of $\widetilde{Q}$ be denoted by $[(x_n)]$\\
		4. Extend the metric $d$ on $Q$ to the metric $\widetilde{d}$ on $\widetilde{Q}$ by defining $\widetilde{d}(\widetilde{x},\widetilde{y}) = lim_{n \rightarrow \infty} d(x_n,y_n)$\\
		5. Define the mapping $T: Q \rightarrow \widetilde{Q}$ by $T(x) = [(\bar{x})]$\\
		6. Show $T(Q)$ is dense in $(\widetilde{Q},\widetilde{d})$\\
		7. Show $(\widetilde{Q},\widetilde{d})$ is complete
        \end{block}
\end{frame}

\begin{frame}{The reals}
        \begin{block}{Compatibility of $\preceq$ and field elements}
        \end{block}
\end{frame}

\begin{frame}{The reals}
        \begin{block}{Compatibility of $\preceq$ and field elements}
		COMP1: $x \preceq y \implies (x \oplus z) \preceq (y \oplus z)$\\
		COMP2: $0 \preceq x$ and $0 \preceq y \implies 0 \preceq x \odot y$
	\end{block}
\end{frame}

\begin{frame}{The reals}
        \begin{block}{General Metric Requirements}
	\end{block}
\end{frame}

\begin{frame}{The reals}
        \begin{block}{General Metric Requirements}
		M1: The number $d(x,y)$ is always defined and is nonnegative\\
		M2: $d(x,y) = 0 \iff x = y$\\
		M3: $d(x,y) = d(y,x)$
		M4: $d(x,y) \leq d(x,z) + x(z,y) \Delta$
        \end{block}
\end{frame}

\begin{frame}{The reals}
        \begin{block}{Explain the process of completing $(X,d)$}
        \end{block}
\end{frame}

\begin{frame}{The reals}
        \begin{block}{Explain the process of completing $(X,d)$}
                1. Let $S = \{(x_n)|x_n \in Q, (x_n) \text{is a Cauchy Sequence in} (X,d)\}$\\
                2. Define an equivalence relation on S by $(x_n) \sim (y_n) \iff lim_{n \rightarrow \infty} d(x_n,y_n) =0$\\
                3. Let $\widetilde{Q} = \frac{S}{\sim}$, the set of all equivalence classes in $S$ under $\sim$. Let the elements of $\widetilde{X}$ be denoted by $[(x_n)]$\\
                4. Extend the metric $d$ on $X$ to the metric $\widetilde{d}$ on $\widetilde{X}$ by defining $\widetilde{d}(\widetilde{x},\widetilde{y}) = lim_{n \rightarrow \infty} d(x_n,y_n)$\\
                5. Define the mapping $T: X \rightarrow \widetilde{X}$ by $T(x) = [(\bar{x})]$\\
		6. Show $T(X)$ is dense in $(\widetilde{X},\widetilde{d})$\\
                7. Show $(\widetilde{X},\widetilde{d})$ is complete
        \end{block}
\end{frame}

\begin{frame}{The reals}
	\begin{block}{Explain the process of constructing $(\widetilde{C}([a,b]),\widetilde{d}_1)$}
        \end{block}
\end{frame}

\begin{frame}{The reals}
        \begin{block}{Explain the process of constructing $(\widetilde{C}([a,b]),\widetilde{d}_1)$}
		1. Let $S = \{(x_n)|x_n \in Q, (x_n) \text{is a Cauchy Sequence in} (C([a,b]),d_1)\}$\\
                2. Define an equivalence relation on S by $(x_n) \sim (y_n) \iff lim_{n \rightarrow \infty} d_1(x_n,y_n) =0$\\
		3. Let $\widetilde{C}([a,b]) = \frac{S}{\sim}$, the set of all equivalence classes in $S$ under $\sim$. Let the elements of $C([a,b])$ be denoted by $[(x_n)]$\\
                4. Extend the metric $d_1$ on $C([a,b])$ to the metric $\widetilde{d_1}$ on $\widetilde{C}([a,b])$ by defining $\widetilde{d}_1(\widetilde{x},\widetilde{y}) = lim_{n \rightarrow \infty} d_1(x_n,y_n)$\\
		5. Define the mapping $T: C([a,b]) \rightarrow \widetilde{C}([a,b])$ by $T(x) = [(\bar{x})]$\\
		6. Show $T(C([a,b])))$ is dense in $(\widetilde{C}([a,b]),\widetilde{d_1})$\\
		7. Show $(\widetilde{C}([a,b]),\widetilde{d}_1)$ is complete
        \end{block}
\end{frame}

\begin{frame}{The reals}
	\begin{block}{Explain the process of constructing $(\mathbb{L}([a,b]),\widetilde{\mathbb{D}}_1) \equiv (\widetilde{\mathbb{RI}}([a,b]),\widetilde{\mathbb{D}}_1)$}
        \end{block}
\end{frame}

\begin{frame}{The reals}
        \begin{block}{Explain the process of constructing $(\mathbb{L}([a,b]),\widetilde{\mathbb{D}}_1) \equiv (\widetilde{\mathbb{RI}}([a,b]),\widetilde{\mathbb{D}}_1) I$}
		1. Let $S = \{([x_n])|[x_n] \in \mathbb{RI}([a,b]), (x_n) \text{is a Cauchy Sequence in} (\mathbb{RI}([a,b]),\mathbb{D}_1)\}$\\
		2. Define an equivalence relation on S by $([x_n]) \sim ([y_n]) \iff lim_{n \rightarrow \infty} \mathbb{D}_1([x_n],[y_n]) =0$\\
		3. Let $\mathbb{L}_1([a,b]) = \frac{S}{\sim}$, the set of all equivalence classes in $S$ under $\sim$. Let the elements of $\mathbb{L}_1([a,b])$ be denoted by $[[x_n]]$. Clean up the notation by using $\mathscr{x}  = [[x_n]]$\\
	\end{block}
\end{frame}

\begin{frame}{The reals}
        \begin{block}{Explain the process of constructing $(\mathbb{L}([a,b]),\widetilde{\mathbb{D}}_1) \equiv (\widetilde{\mathbb{RI}}([a,b]),\widetilde{\mathbb{D}}_1)$ II}
		4. Extend the metric $\mathbb{D}_1$ on $\mathbb{RI}([a,b])$ to the metric $\widetilde{\mathbb{D}_1}$ on $\widetilde{\mathbb{L}_1}([a,b])$ by defining $\widetilde{\mathbb{D}}_1(\mathscr{x},\mathscr{y}) = lim_{n \rightarrow \infty} \mathbb{D}_1([(x_n)],([y_n])) = lim_{n \rightarrow \infty} d_1|x_n - y_n|$\\
		5. Define the mapping $T: \mathbb{RI}([a,b]) \rightarrow \mathbb{L}([a,b])$ by $T([x]) = [\bar{x}] = \bar{\mathscr{X}}$\\
		6. Show $T(\mathbb{RI}([a,b])))$ is dense in $(\mathbb{L}([a,b]),\widetilde{\mathbb{D}_1})$\\
		7. Show $(\mathbb{L}([a,b]),\widetilde{\mathbb{D}}_1)$ is complete
        \end{block}
\end{frame}

\begin{frame}{The reals}
	\begin{block}{Holder's Inequality}
        \end{block}
\end{frame}


\begin{frame}{The reals}
        \begin{block}{Holder's Inequality}
		$\sum_{n=1}^\infty |x_ny_n| \leq (\sum_{n=1}^\infty|x_n|^p)^\frac{1}{p}(\sum_{n=1}^\infty|y_n|^q)^\frac{1}{q}$
        \end{block}
\end{frame}

\begin{frame}{The reals}
        \begin{block}{Minkowski's Inequality}
        \end{block}
\end{frame}


\begin{frame}{The reals}
        \begin{block}{Minkowski's Inequality}
                $\sum_{n=1}^\infty |x_n + y_n| \leq (\sum_{n=1}^\infty|x_n|^p)^\frac{1}{p} + (\sum_{n=1}^\infty|y_n|^q)^\frac{1}{q}$
        \end{block}
\end{frame}

\begin{frame}{General Stuff}
        \begin{block}{Open ball of radius r about x}
        \end{block}
\end{frame}

\begin{frame}{General Stuff}
        \begin{block}{Open ball of radius r about x}
		$B(x,r) = \{y \in X | d(x,y) < r\}$
        \end{block}
\end{frame}

\begin{frame}{General Stuff}
        \begin{block}{Closed ball of radius r about x}
        \end{block}
\end{frame}

\begin{frame}{General Stuff}
        \begin{block}{Open ball of radius r about x}
		$\bar{B}(x,r) = \{y \in X | d(x,y) \leq r\}$
        \end{block}
\end{frame}

\begin{frame}{General Stuff}
        \begin{block}{Punctured ball of radius r about x}
        \end{block}
\end{frame}

\begin{frame}{General Stuff}
        \begin{block}{Punctured ball of radius r about x}
		$\hat{B}(x,r) = \{y \in X |0 < d(x,y) < r\}$
        \end{block}
\end{frame}

\begin{frame}{General Stuff}
        \begin{block}{Convergence}
        \end{block}
\end{frame}

\begin{frame}{General Stuff}
        \begin{block}{Convergence}
		A sequence $(x_n)$ is said to converge to $x$ in $X$ if\\
		$\forall \epsilon > 0, \exists N \ni d(x_n,x) < \epsilon$ if $n < N$ 
        \end{block}
\end{frame}

\begin{frame}{General Stuff}
        \begin{block}{Cauchy Sequence}
        \end{block}
\end{frame}

\begin{frame}{General Stuff}
        \begin{block}{Convergence} 
                A sequence $(x_n)$ is said to a Cauchy Sequence if\\
                $\forall \epsilon > 0, \exists N \ni d(x_n,x_m) < \epsilon$ if $n, m < N$ 
        \end{block}
\end{frame}

\begin{frame}{General Stuff}
        \begin{block}{Continuity}
        \end{block}
\end{frame}

\begin{frame}{General Stuff}
        \begin{block}{Continuity}
		If $f: \rightarrow \mathcal{R}$ is a mapping from the metric space $(X,d_x)$ to the real numbers, we say $f$ is continuous at $x_0$ if\\
		$\forall \epsilon > 0 \exists \delta > 0 \ni d_Y(f(x),f(x_0)) < \epsilon if d_x(x,x_0) < \delta$
        \end{block}
\end{frame}



% All of the following is optional and typically not needed. 
\appendix
\section<presentation>*{\appendixname}
\subsection<presentation>*{For Further Reading}

\begin{frame}[allowframebreaks]
  \frametitle<presentation>{For Further Reading}
    
  \begin{thebibliography}{10}
    
  \beamertemplatebookbibitems
  % Start with overview books.

  \bibitem{Author1990}
    A.~Author.
    \newblock {\em Handbook of Everything}.
    \newblock Some Press, 1990.
 
    
  \beamertemplatearticlebibitems
  % Followed by interesting articles. Keep the list short. 

  \bibitem{Someone2000}
    S.~Someone.
    \newblock On this and that.
    \newblock {\em Journal of This and That}, 2(1):50--100,
    2000.
  \end{thebibliography}
\end{frame}

\end{document}
