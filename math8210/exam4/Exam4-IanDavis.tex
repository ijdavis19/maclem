% SampleProject.tex -- main LaTeX file for sample LaTeX article
%
%\documentclass[12pt]{article}
\documentclass[11pt]{SelfArxOneColBMN}
% add the pgf and tikz support.  This automatically loads
% xcolor so no need to load color
\usepackage{pgf}
\usepackage{tikz}
\usetikzlibrary{matrix}
\usetikzlibrary{calc}
\usepackage{xstring}
\usepackage{pbox}
\usepackage{etoolbox}
\usepackage{marginfix}
\usepackage{xparse}
\setlength{\parskip}{0pt}% fix as marginfix inserts a 1pt ghost parskip
% standard graphics support
\usepackage{graphicx,xcolor}
\usepackage{wrapfig}
%
\definecolor{color1}{RGB}{0,0,90} % Color of the article title and sections
\definecolor{color2}{RGB}{0,20,20} % Color of the boxes behind the abstract and headings
%----------------------------------------------------------------------------------------
%	HYPERLINKS
%----------------------------------------------------------------------------------------
\usepackage[pdftex]{hyperref} % Required for hyperlinks
\hypersetup{hidelinks,
colorlinks,
breaklinks=true,%
urlcolor=color2,%
citecolor=color1,%
linkcolor=color1,%
bookmarksopen=false%
,pdftitle={Exam4},%
pdfauthor={Ian Davis}}
%\usepackage[round,numbers]{natbib}
\usepackage[numbers]{natbib}
\usepackage{lmodern}
\usepackage{setspace}
\usepackage{xspace}
%
\usepackage{subfigure}
\newcommand{\goodgap}{
  \hspace{\subfigtopskip}
  \hspace{\subfigbottomskip}}
%
\usepackage{atbegshi}
%
\usepackage[hyper]{listings}
%
% use ams math packages
\usepackage{amsmath,amsthm,amssymb,amsfonts}
\usepackage{mathrsfs}
%
% use new improved Verbatim
\usepackage{fancyvrb}
%
\usepackage[titletoc,title]{appendix}
%
\usepackage{url}
%
% Create length for the baselineskip of text in footnotesize
\newdimen\footnotesizebaselineskip
\newcommand{\test}[1]{%
 \setbox0=\vbox{\footnotesize\strut Test \strut}
 \global\footnotesizebaselineskip=\ht0 \global\advance\footnotesizebaselineskip by \dp0
}
%
\usepackage{listings}

\DeclareGraphicsExtensions{.pdf, .jpg, .tif,.png}

% make sure we don't get orphaned words if at top of page
% or orphans if at bottom of page
\clubpenalty=9999
\widowpenalty=9999
\renewcommand{\textfraction}{0.15}
\renewcommand{\topfraction}{0.85}
\renewcommand{\bottomfraction}{0.85}
\renewcommand{\floatpagefraction}{0.66}
%
\DeclareMathOperator{\sech}{sech}

\newcommand{\mycite}[1]{%
(\citeauthor{#1} \citep{#1} \citeyear{#1})\xspace
}

\newcommand{\mycitetwo}[2]{%
(\citeauthor{#2} \citep[#1]{#2} \citeyear{#2})\xspace
}

\newcommand{\mycitethree}[3]{%
(\citeauthor{#3} \citep[#1][#2]{#3} \citeyear{#3})\xspace
}

\newcommand{\myincludegraphics}[3]{% file name, width, height
\includegraphics[width=#2,height=#3]{#1}
}

\newcommand{\myincludegraphicstwo}[2]{% file name, width, height
\includegraphics[scale=#1]{#2}
}

\newcommand{\mysimplegraphics}[1]{% file name, width, height
\includegraphics{#1}
}

\newcommand{\MB}[1]{
\boldsymbol{#1}
}

\newcommand{\myquotetwo}[1]{%
\small
%\singlespacing
\begin{quotation}
#1
\end{quotation}
\normalsize
%\onehalfspacing  
}

\newcommand{\jimquote}[1]{%
\small
%\singlespacing
\begin{quotation}
#1
\end{quotation}
\normalsize
%\onehalfspacing
}

\newcommand{\myquote}[1]{%
\small
%\singlespacing
\begin{quotation}
#1
\end{quotation}
\normalsize
%\onehalfspacing  
}

%A =
%
%[2 r_1 	     r_1]
%[-2r_1 + r_2  r_2 - r_1]
%
%has eigenvalues r_1 neq r_2.
% #1 = 2 r_1, #2 = r_1, #3 = -2r_1+r_2, #4 = r_2 - r_1
\newcommand{\myrealdiffA}[4]{
\left [
\begin{array}{rr}
#1  & #2\\
#3  & #4
\end{array}
\right ]
}

% args:
% 1, 2 ,3, 4, 5 = caption, label, width, height, file name
%\mysubfigure{}{}{}{}{}
\newcommand{\mysubfigure}[5]{%
\subfigure[#1]{\label{#2}\includegraphics[width=#3,height=#4]{#5}}
}

\newcommand{\mysubfiguretwo}[3]{%
\subfigure[#1]{\label{#2}\includegraphics{#3}}
}

\newcommand{\mysubfigurethree}[4]{%
\subfigure[#1]{\label{#2}\includegraphics[scale=#3]{#4}}
}

\newcommand{\myputimage}[5]{% file name, width, height
\centering
\includegraphics[width=#3,height=#4]{#5}
\caption{#1}
\label{#2}
}

\newcommand{\myputimagetwo}[4]{% caption, label, scale, file name
\centering
\includegraphics[scale=#3]{#4}
\caption{#1}
\label{#2}
}

\newcommand{\myrotateimage}[5]{% file name, width, height
\centering
\includegraphics[scale=#3,angle=#4]{#5}
\caption{#1}
\label{#2}
}

\newcommand{\myurl}[2]{%
\href{#1}{\bf #2}
}

\RecustomVerbatimEnvironment%
{Verbatim}{Verbatim}  
  {fillcolor=\color{black!20}}
  
  \DefineVerbatimEnvironment%
{MyVerbatim}{Verbatim}  
  {frame=single,
   framerule=2pt,
   fillcolor=\color{black!20},
   fontsize=\small}
   
\newcommand{\myfvset}[1]{%  
\fvset{frame=single,
       framerule=2pt,
       fontsize=\small,
       xleftmargin=#1in}}
       
\newcommand{\mylistverbatim}{%
\lstset{%
  fancyvrb, 
  basicstyle=\small,
  breaklines=true}
}  

\newcommand{\mylstinlinebf}[1]{%
{\bf #1}
}

\newcommand{\mylstinline}{%
\lstset{%
  basicstyle=\color{black!80}\bfseries\ttfamily,
  showstringspaces=false,
  showspaces=false,showtabs=false,
  breaklines=true}
\lstinline
}

\newcommand{\mylstinlinetwo}[1]{%
\lstset{%
  basicstyle=\color{black!80}\bfseries\ttfamily,
  showstringspaces=false,
  showspaces=false,showtabs=false,
  breaklines=true}
\lstinline!#1! 
}

%fontfamily=tt
%fontfamily=courier
%fontfamily=helvetica
%frame=topline,
%frame=single,
 %frame=lines,
 %framesep=10pt,
 %fontshape=it,
 %fontseries=b,
 %fontsize=\relsize{-1},
 %fillcolor=\color{black!20},
 %rulecolor=\color{yellow},
 %fillcolor=\color{red}
 %label=\fbox{\Large\emph{The code}}
\DefineVerbatimEnvironment%
{MyListVerbatim}{Verbatim}  
{
fillcolor=\color{black!10},
fontfamily=courier,
frame=single,
%formatcom=\color{white},
framesep=5mm,
labelposition=topline,
fontshape=it,
fontseries=b,
fontsize=\small,
label=\fbox{\large\emph{The code}\normalsize}
} 

%  caption={[#1] \large\bf{#1}}, 
%\centering \framebox[.6\textwidth][c]{\Large\bf{#1}}
\newcommand{\myfancyverbatim}[1]{%
\lstset{%
  fancyvrb=true, 
  %fvcmdparams= fillcolor 1,
  %morefvcmdparams = \textcolor 2,
  frame=shadowbox,framerule=2pt, 
  basicstyle=\small\bfseries,
  backgroundcolor=\color{black!08},
  showstringspaces=false,
  showspaces=false,showtabs=false,
  keywordstyle=\color{black}\bfseries,
  %numbers=left,numberstyle=\tiny,stepnumber=5,numbersep=5pt,
  stringstyle=\ttfamily,
  caption={[\quad #1] \mbox{}\\ \vspace{0.1in} \framebox{\large \bf{#1} \small} },  
  belowcaptionskip=20 pt,  
  label={},
  xleftmargin=17pt,
  framexleftmargin=17pt,
  framexrightmargin=5pt,
  framexbottommargin=4pt,
  nolol=false,
  breaklines=true}
}

\newcommand{\mylistcode}[3]{%
\lstset{%
  language=#1, 
  frame=shadowbox,framerule=2pt, 
  basicstyle=\small\bfseries,
  backgroundcolor=\color{black!16},
  showstringspaces=false,
  showspaces=false,showtabs=false,
  keywordstyle=\color{black!40}\bfseries,
  numbers=left,numberstyle=\tiny,stepnumber=5,numbersep=5pt,
  stringstyle=\ttfamily,
  caption={[\quad#2] \mbox{}\\ \vspace{0.1in} \framebox{\large \bf{#2} \small} },
  belowcaptionskip=20 pt,
  breaklines=true,
  xleftmargin=17pt,
  framexleftmargin=17pt,
  framexrightmargin=5pt,
  framexbottommargin=4pt,  
  label=#3,
  breaklines=true} 
}

  %caption={[#2] #3},
  %caption={[#2]{\mbox{}\\ \vspace{0.1in} \framebox{\large \bf{#3} \small}},
  %caption={[#2] \mbox{}\\ \bf{#3} },

% frame=single,
% caption={[Code Fragment] {\bf Code Fragment} },
% caption={[Code Fragment] \mbox{}\\ \vspace{0.1in} \framebox{\large \bf{Code Fragment} \small} },
\newcommand{\mylistcodequick}[1]{%
\lstset{%
  language=#1, 
  frame=shadowbox,framerule=2pt, 
  basicstyle=\small\bfseries,
  backgroundcolor=\color{black!16},
  showstringspaces=false,
  showspaces=false,showtabs=false,
  keywordstyle=\color{black!40}\bfseries,
  numbers=left,numberstyle=\tiny,stepnumber=5,numbersep=5pt,
  stringstyle=\ttfamily,
  caption={[\quad Code Fragment] \large \bf{Code Fragment} \small},   
  belowcaptionskip=20 pt,  
  label={},
  xleftmargin=17pt,
  framexleftmargin=17pt,
  framexrightmargin=5pt,
  framexbottommargin=4pt,
  breaklines=true} 
}

%  caption={[#2] \mbox{}\\ \vspace{0.1in} \framebox{\large \bf{#2} \small} },
\newcommand{\mylistcodequicktwo}[2]{%
\lstset{%
  language=#1, 
  frame=shadowbox,framerule=2pt, 
  basicstyle=\small\bfseries,
  extendedchars=true,
  backgroundcolor=\color{black!16},
  showstringspaces=false,
  showspaces=false,
  showtabs=false,
  keywordstyle=\color{black!40}\bfseries,
  numbers=left,numberstyle=\tiny,stepnumber=5,numbersep=5pt,
  stringstyle=\ttfamily,
  caption={[\quad#2] \large \bf{#2} \small},
  belowcaptionskip=20 pt,
  label={},
  xleftmargin=17pt,
  framexleftmargin=17pt,
  framexrightmargin=5pt,
  framexbottommargin=4pt,
  breaklines=true} 
}

%  caption={[#2] \mbox{}\\ \vspace{0.1in} \framebox{\large \bf{#2} \small} },
\newcommand{\mylistcodequickthree}[2]{%
\lstset{%
  language=#1, 
  frame=shadowbox,framerule=2pt, 
  basicstyle=\small\bfseries,
  extendedchars=true,
  backgroundcolor=\color{black!16},
  showstringspaces=false,
  showspaces=false,
  showtabs=false,
  keywordstyle=\color{black!40}\bfseries,
  numbers=left,numberstyle=\tiny,stepnumber=5,numbersep=5pt,
  stringstyle=\ttfamily,
  caption={[\quad#2] \large\bf{#2}\small},
  belowcaptionskip=20 pt,
  label={},
  xleftmargin=17pt,
  framexleftmargin=17pt,
  framexrightmargin=5pt,
  framexbottommargin=4pt,
  breaklines=true} 
}

%  frame=single,
\newcommand{\mylistset}[4]{%
\lstset{language=#1,
  basicstyle=\small,
  showstringspaces=false,
  showspaces=false,showtabs=false,
  keywordstyle=\color{black!40}\bfseries,
  numbers=left,numberstyle=\tiny,stepnumber=5,numbersep=5pt,
  stringstyle=\ttfamily,
  caption={[\quad#2]#3},
  label=#4}
}

\newcommand{\mylstinlineset}{%
\lstset{%
  basicstyle=\color{blue}\bfseries\ttfamily,
  showstringspaces=false,
  showspaces=false,showtabs=false,
  breaklines=true}
}

\newcommand{\myframedtext}[1]{%
\centering
\noindent
%\fbox{\parbox[c]{.9\textwidth}{\color{black!40} \small \singlespacing #1\onehalfspacing \normalsize \\}}
\fbox{\parbox[c]{.9\textwidth}{\color{black!40} \small  #1 \normalsize \\}}
}

\newcommand{\myemptybox}[2]{% from , to
\fbox{\begin{minipage}[t][#1in][c]{#2in}\hspace{#2in}\end{minipage}}
}

\newcommand{\myemptyboxtwo}[2]{% from , to
\centering\fbox{
\begin{minipage}{#1in}
\hfill\vspace{#2in}
\end{minipage}
}
}

\newcommand{\boldvector}[1]{
\boldsymbol{#1}
}

\newcommand{\dEdY}[2]{\frac{d E}{d Y_{#1}^{#2}}}
\newcommand{\dEdy}[2]{\frac{d E}{d y_{#1}^{#2}}}
\newcommand{\dEdT}[2]{\frac{\partial E}{\partial T_{{#1} \rightarrow {#2}}}}
\newcommand{\dEdo}[1]{\frac{\partial E}{\partial o^{#1}}}
\newcommand{\dEdg}[1]{\frac{\partial E}{\partial g^{#1}}}
\newcommand{\dYdY}[4]{\frac{\partial Y_{#1}^{#2}}{\partial Y_{#3}^{#4}}}
\newcommand{\dYdy}[4]{\frac{\partial Y_{#1}^{#2}}{\partial y_{#3}^{#4}}}
\newcommand{\dydY}[4]{\frac{\partial y_{#1}^{#2}}{\partial Y_{#3}^{#4}}}
\newcommand{\dydy}[4]{\frac{\partial y_{#1}^{#2}}{\partial y_{#3}^{#4}}}
\newcommand{\dydT}[4]{\frac{\partial y_{#1}^{#2}}{\partial T_{{#3} \rightarrow {#4}}}}
\newcommand{\dYdT}[4]{\frac{\partial Y_{#1}^{#2}}{\partial T_{{#3} \rightarrow {#4}}}}
\newcommand{\dTdT}[4]{\frac{\partial T_{{#1} \rightarrow {#2}}}{\partial T_{{#3} \rightarrow {#4}}}}
\newcommand{\ssum}[2]{\sum_{#1}^{#2}}

\newcommand{\ssty}[1]{\scriptscriptstyle #1}

\newcommand{\myparbox}[2]{%
\parbox{#1}{\color{black!20} #2}
}

\newcommand{\bs}[1]{
\boldsymbol{#1}
}

\newcommand{\parone}[2]{%
\frac{\partial #1 }{ \partial #2 }
}
\newcommand{\partwo}[2]{%
\frac{ \partial^2 {#1} }{ \partial {#2}^2 }
}

\newcommand{\twodvectorvarfun}[2]{
\left [
\begin{array}{r}
{{#1_{\ssty{1}}}(#2)} \\
{{#1_{\ssty{2}}}(#2)}
\end{array}
\right ]
}
\newcommand{\twodvectorvarprimed}[1]{
\left [
\begin{array}{r}
{{#1_{\ssty{1}}}'(t)} \\
{{#1_{\ssty{2}}}'(t)}
\end{array}
\right ]
}

\newcommand{\complex}[2]{#1 \: #2 \: \boldsymbol{i}}
\newcommand{\complexmag}[2]%
{
\sqrt{(#1)^2 \: + \: (#2)^2}
}
\newcommand{\threenorm}[3]%
{
\sqrt{(#1)^2 \: + \: (#2)^2 \: + \: (#3)^2}
}
\newcommand{\norm}[1]{\mid \mid #1 \mid \mid}

\newcommand{\myderiv}[2]{\frac{d #1}{d #2}}
\newcommand{\myderivb}[2]{\frac{d}{d #2} \left ( #1 \right )}
\newcommand{\myrate}[3]%
{#1^\prime(#2) &=& #3 \: #1(#2)
}
\newcommand{\myrateexter}[4]%
{#1^\prime(#2) &=& #3 \: #1(#2) \: + \: #4
}
\newcommand{\myrateic}[3]%
{#1( \: #2 \:) &=& #3 
}

\newcommand{\mytwodsystemeqn}[6]{
#1 \: x    #2 \: y &=& #3\\
#4 \: x    #5 \: y &=& #6\\
}

\newcommand{\mytwodsystem}[8]{
#3 \: #1 \: + \: #4 \: #2 &=& #5\\
#6 \: #1 \: + \: #7 \: #2 &=& #8\\
}  

\newcommand{\mytwodarray}[4]{
\left [
\begin{array}{rr}
#1 & #2\\
#3 & #4
\end{array}
\right ]
}

\newcommand{\mytwoid}{
\left [
\begin{array}{rr}
1 & 0\\
0 & 1
\end{array}
\right ]
}

\newcommand{\myxprime}[2]{
\left [
\begin{array}{r}
#1^\prime(t)\\
#2^\prime(t)
\end{array}
\right ]
}

\newcommand{\myxprimepacked}[2]{
\left [
\begin{array}{r}
#1^\prime\\
#2^\prime
\end{array}
\right ]
}

\newcommand{\myx}[2]{
\left [
\begin{array}{r}
#1(t)\\
#2(t)
\end{array}
\right ]
}

\newcommand{\myxonly}[2]{
\left [
\begin{array}{r}
#1\\
#2
\end{array}
\right ]
}

\newcommand{\myv}[2]{
\left [
\begin{array}{r}
#1\\
#2
\end{array}
\right ]
}

\newcommand{\myxinitial}[2]{
\left [
\begin{array}{r}
#1(0)\\
#2(0)
\end{array}
\right ]
}

\newcommand{\twodboldv}[1]{
\boldsymbol{#1}
}

\newcommand{\mytwodvector}[2]{
\left [
\begin{array}{r}
#1\\
#2
\end{array}
\right ]
}

\newcommand{\mythreedvector}[3]{
\left [
\begin{array}{r}
#1\\
#2\\
#3
\end{array}
\right ]
}

\newcommand{\mytwodsystemvector}[6]{
\left [
\begin{array}{rr}
#1 & #2\\
#4 & #5
\end{array}
\right ]
\:
\left [
\begin{array}{r}
x \\
y 
\end{array}
\right ]
&=&
\left [
\begin{array}{r}
#3\\
#6
\end{array}
\right ]
}

\newcommand{\mythreedarray}[9]{
\left [
\begin{array}{rrr}
#1 & #2 & #3\\
#4 & #5 & #6\\
#7 & #8 & #9
\end{array}
\right ]
}

\newcommand{\myodetwo}[6]{
#1 \: #6^{\prime \prime}(t) \: #2 \: #6^{\prime}(t) \: #3 \: #6(t) &=& 0\\
#6(0)                                           &=& #4\\
#6^{\prime}(0)                                  &=& #5
}

\newcommand{\myodetwoNoIC}[4]{
#1 \: #4^{\prime \prime}(t) \: #2 \: #4^{\prime}(t) \: #3 \: #4(t) &=& 0
}

\newcommand{\myodetwopacked}[5]{
\hspace{-0.3in}& & #1 u^{\prime \prime} #2 u^{\prime} #3 u \: = \: 0\\
\hspace{-0.3in}& & u(0) \: = \: #4, \: \: u^{\prime}(0)    \: = \:  #5
}

\newcommand{\myodetwoforced}[6]{
#1\: u^{\prime \prime}(t) \: #2 \: u^{\prime}(t) \: #3 \: u(t) &=& #6\\
u(0)                                           &=& #4\\
u^{\prime}(0)                                  &=& #5\\
}

\newcommand{\myodesystemtwo}[8]{
#1 \: x^\prime(t) \: #2 \: y^\prime(t) \: #3 \: x(t) \: #4 \: y(t) &=& 0\\
#5 \: x^\prime(t) \: #6 \: y^\prime(t) \: #7 \: x(t) \: #8 \: y(t) &=& 0\\
}

\newcommand{\myodesystemtwoic}[2]{
x(0)                                       &=& #1\\ 
y(0)                                       &=& #2
}

\newcommand{\mypredprey}[4]{
x^\prime(t) &=& #1 \: x(t) \: - \: #2 \: x(t) \: y(t)\\
y^\prime(t) &=& -#3 \: y(t) \: + \: #4 \: x(t) \: y(t)
}

\newcommand{\mypredpreypacked}[4]{
x^\prime &=& #1 \: x - #2 \: x \: y\\
y^\prime &=& -#3 \: y + #4 \: x \: y
}

\newcommand{\mypredpreyself}[6]{
x^\prime(t) &=&  #1 \: x(t) \: - \: #2 \: x(t) \: y(t) \: - \: #3 \: x(t)^2\\
y^\prime(t) &=& -#4 \: y(t) \: + \: #5 \: x(t) \: y(t) \: - \: #6 \: y(t)^2
}

\newcommand{\mypredpreyfish}[5]{
x^\prime(t) &=&  #1 \: x(t) \: - \: #2 \: x(t) \: y(t) \: - \: #5 \: x(t)\\
y^\prime(t) &=& -#3 \: y(t) \: + \: #4 \: x(t) \: y(t) \: - \: #5 \: y(t)
}

\newcommand{\myepidemic}[4]{
S^\prime(t) &=& - #1 \: S(t) \: I(t)\\
I^\prime(t) &=&   #1 \: S(t) \: I(t) \: - \: #2 \: I(t)\\
S(0)        &=&   #3\\
I(0)        &=&   #4\\
}

\newcommand{\bsred}[1]{%
\textcolor{red}{\boldsymbol{#1}}
}

\newcommand{\bsblue}[1]{%
\textcolor{blue}{\boldsymbol{#1}}
}


\newcommand{\myfloor}[1]{%
\lfloor{#1}\rfloor
}

\newcommand{\cubeface}[7]{%
\begin{bmatrix}
\bs{#3}          & \longrightarrow & \bs{#4}\\
\uparrow          &                         &  \uparrow  \\
\bs{#1} & \longrightarrow & \bs{#2}\\
              & \text{ \bfseries #5:} \: \bs{#6} \: \text{\bfseries  #7 } & 
\end{bmatrix}
}

\newcommand{\cubefacetwo}[5]{%
\begin{bmatrix}
\bs{#3}          & \longrightarrow & \bs{#4}\\
\uparrow          &                         &  \uparrow  \\
\bs{#1} & \longrightarrow & \bs{#2}\\
              & \text{ \bfseries #5} & 
\end{bmatrix}
}

\newcommand{\cubefacethree}[9]{%
\begin{bmatrix}
\bs{#3}                  & \overset{#9}{\longrightarrow} & \bs{#4}\\
\uparrow \: #7         &                                             &  \uparrow  \: #8 \\
\bs{#1}                  & \overset{#6}{\longrightarrow} & \bs{#2}\\
                               & \text{ \bfseries #5} & 
\end{bmatrix}
}

\renewcommand{\qedsymbol}{\hfill \blacksquare}
\newcommand{\subqedsymbol}{\hfill \Box}
%\theoremstyle{plain}

\newtheoremstyle{mystyle}% name
  {6pt}%      Space above
  {6pt}%      Space below
  {\itshape}%         Body font
  {}%         Indent amount (empty = no indent, \parindent = para indent)
  {\bfseries}% Thm head font
  {}%        Punctuation after thm head
  { }%     Space after thm head: " " = normal interword space; \newline = linebreak
  {}%         Thm head spec (can be left empty, meaning `normal')
\theoremstyle{mystyle}
 
\newtheorem{axiom}{Axiom}
%\newtheorem{solution}{Solution}[section]
\newtheorem*{solution}{Solution}
\newtheorem{exercise}{Exercise}[section]
\newtheorem{theorem}{Theorem}[section]
\newtheorem{proposition}[theorem]{Proposition}
\newtheorem{prop}[theorem]{Proposition}
\newtheorem{assumption}{Assumption}[section]
\newtheorem{definition}{Definition}[section]
\newtheorem{comment}{Comment}[section]
\newtheorem*{question}{Question}
\newtheorem{program}{Program}[section]
%\newtheorem{myproof}{Proof}
%\newtheorem*{myproof}{Proof}[section]
\newtheorem{myproof}{Proof}[section]
\newtheorem{hint}{Hint}[section]
\newtheorem*{phint}{Hint}
\newtheorem{lemma}[theorem]{Lemma}
\newtheorem{example}{Example}[section]
      
\newenvironment{myassumption}[4]
{
\centering
\begin{assumption}[{\textbf{#1}\nopunct}]%
\index{#2}
\mbox{}\\  \vskip6pt \colorbox{black!15}{\fbox{\parbox{.9\textwidth}{#3}}}
\label{#4}
\end{assumption}
%\renewcommand{\theproposition}{\arabic{chapter}.\arabic{section}.\arabic{assumption}} 
}%
{}

\newenvironment{myproposition}[4]
{
\centering
\begin{proposition}[{\textbf{#1}\nopunct}]%
\index{#2} 
\mbox{}\\  \vskip6pt \colorbox{black!15}{\fbox{\parbox{.9\textwidth}{#3}}}
\label{#4}
\end{proposition}
%\renewcommand{\theproposition}{\arabic{chapter}.\arabic{section}.\arabic{proposition}} 
}%
{}

\newenvironment{mytheorem}[4]
{
\centering
\begin{theorem}[{\textbf{#1}\nopunct}]%
\index{#2} 
\mbox{}\\ \vskip6pt \colorbox{black!15}{\fbox{\parbox{.9\textwidth}{#3}}}
\label{#4}
\end{theorem}
%\renewcommand{\thetheorem}{\arabic{chapter}.\arabic{section}.\arabic{theorem}} 
}%
{}

\newenvironment{mydefinition}[4]
{
\centering
\begin{definition}[{\textbf{#1}\nopunct}]%
\index{#2} 
\mbox{}\\  \vskip6pt \colorbox{black!15}{\fbox{\parbox{.9\textwidth}{#3}}}
\label{#4}
\end{definition}
%\renewcommand{\thedefinitio{n}{\arabic{chapter}.\arabic{section}.\arabic{definition}} 
}%
{}

\newenvironment{myaxiom}[4]
{
\centering
\begin{axiom}[{\textbf{#1}\nopunct}]%
\index{#2} 
\mbox{}\\  \vskip6pt \colorbox{black!15}{\fbox{\parbox{.9\textwidth}{#3}}}
\label{#4}
\end{axiom}
%\renewcommand{\theaxiom}{\arabic{chapter}.\arabic{section}.\arabic{axiom}} 
}%
{}

\newenvironment{mylemma}[4]
{
\centering
\begin{lemma}[{\textbf{#1}\nopunct}]%
\index{#2} 
\mbox{}\\  \vskip6pt \colorbox{black!15}{\fbox{\parbox{.9\textwidth}{#3}}}
\label{#4}
\end{lemma}
%\renewcommand{\thelemma}{\arabic{chapter}.\arabic{section}.\arabic{lemma}} 
}%
{}
   
\newenvironment{reason}[1]
{
 \vskip0.05in
 \begin{myproof}
 \mbox{}\\#1
 $\qedsymbol$
 \end{myproof}  
 \vskip0.05in
}%
{}

\newenvironment{reasontwo}[1]
{
 \vskip+.05in
 \begin{myproof}
 \mbox{}\\#1
 \end{myproof}  
 \vskip+0.05in
}%
{}

\newenvironment{subreason}[1]
{
 \vskip0.05in
 \renewcommand{\themyproof}{}
 \begin{myproof}
 #1
 $\subqedsymbol$
 \end{myproof}
 \vskip0.05in
 \renewcommand{\themyproof}{\thetheorem}
 %\renewcommand{\themyproof}{\arabic{chapter}.\arabic{section}.\arabic{myproof}}   
 %
}%
{}  

\newenvironment{myhint}[1]
{
 \vskip0.05in
 \begin{hint}
 #1
 $\subqedsymbol$ 
 \end{hint}  
 \vskip0.05in
}%
{} 

\newenvironment{myeqn}[3]
{
 \renewcommand{\theequation}{$\boldsymbol{#1}$}
 \begin{eqnarray}
 \label{equation:#2}
 #3 
 \end{eqnarray}
 \renewcommand{\theequation}{\arabic{chapter}.\arabic{eqnarray}}   
}%
{} 


\JournalInfo{MATH 8210: Exam 4, 1-\pageref{LastPage}, 2020} % Journal information
\Archive{Draft Version \today} % Additional notes (e.g. copyright, DOI, review/research article)

\PaperTitle{MATH 8210 Exam 4}
\Authors{Ian Davis\textsuperscript{1}}
\affiliation{\textsuperscript{1}\textit{John E. Walker Department of Economics,
Clemson University,Clemson, SC: email ijdavis@g.clemson.edu}}
%\affiliation{*\textbf{Corresponding author}: yournamehere@clemson.edu} % Corresponding author

\date{\small{Version ~\today}}
\Abstract{Exam 4}
\Keywords{}
\newcommand{\keywordname}{Keywords}
%
\onehalfspacing
\begin{document}
\flushbottom

\addcontentsline{toc}{section}{Title}
\maketitle

\renewcommand{\theexercise}{\arabic{exercise}}
\textbf{Problem 1: }Let $A$ and $B$ be self adjoint operators on the Hilbert space $H$
\begin{itemize}
  \item Prove if $A$ and $B$ commute, then $AB$ is self adjoint.
  \begin{solution}
    Because $A$ and $B$ are self-adjoint in the Hilbert space $H$ and commute we know
    \begin{enumerate}
      \item $<A(\mathbf{x}),\mathbf{y}> = <\mathbf{x},A(\mathbf{y})>$
      \item $<B(\mathbf{x}),\mathbf{y}> = <\mathbf{x},B(\mathbf{y})>$
      \item $A(B(\mathbf{x})) = B(A(\mathbf{x}))$
    \end{enumerate}
    From these conditions, we can deduce the following
    \begin{eqnarray*}
      <A(B(\mathbf{x})),\mathbf{y}> &=& <B(\mathbf{x}),A(\mathbf{y})>\\
      &=& <\mathbf{x},B(A(\mathbf{y}))>\\
      &=& <\mathbf{x},A(B(\mathbf{y}))>
    \end{eqnarray*}
    proving AB is self adjoint.
  \end{solution}
  \item If $E$ is an eigenvector of $A$ for eigenvalue $\lambda$, and $A$ and $B$ commute, then $BE$ is an eigenvector of $A$ with eigenvalue $\lambda$
  \begin{solution}
    We know from above that
    \begin{eqnarray*}
      A(E) = \lambda E
    \end{eqnarray*}
    and then, because $A$ and $B$ commute, we can get
    \begin{eqnarray*}
      A(BE) &=& B(AE)\\
      &=& B(\lambda E)\\
      &=& \lambda B(E)
    \end{eqnarray*}
    which tells us that $B(E)$ is an eigenvector of $A$ with eigenvalue $\lambda$
  \end{solution}
  \item If $F$ is an eigenvector of $B$ for eigenvalue $\mu$, and $A$ and $B$ commute, then $AF$ is an eigenvector of $B$ with eigenvalue $\mu$
  \begin{solution}
    Similar to the proof above, we know
    \begin{eqnarray*}
      B(F) = \mu F
    \end{eqnarray*}
    and then, because $A$ and $B$ commute, we can get
    \begin{eqnarray*}
      B(AF) &=& A(BF)\\
      &=& A(\mu F)\\
      &=& \mu A(F)
    \end{eqnarray*}
    which tells us that $A(F)$ is an eigenvector of $B$ with eigenvalue $\mu$
  \end{solution}
\end{itemize}

\textbf{Problem 2: }Let $X = \Re^n$. Define $\mathbf{F_i} \in X^\prime$ by $F_i(\mathbf{E_j}) = \delta_i^j$ where $\mathbf{E = \{E_1,...,E_n\}}$ is a given o.n. basis of X. Then any $\mathbf{x} \in X$ has a unique expansion $\mathbf{x} = \sum_{i=1}^n\:x_i\mathbf{E_i}$. Prove $\mathbf{F = \{F_1,...,F_n\}}$ is an o.n. basis for $X^\prime$ and if 
\begin{eqnarray*}
\mathbf{z_F} = 
  \begin{bmatrix}
    F(\mathbf{E_1})\\
    \vdots\\
    F(\mathbf{E_n})
  \end{bmatrix}
\end{eqnarray*}
then $F(\mathbf{x}) = <\mathbf{x},\mathbf{z_F}>$. Prove this representation and explain how you know $\mathbf{z_F}$ is the unique choice since $X \equiv X^\prime$
\begin{solution}
  First, we want to show that $\mathbf{F}$ is an orthonormal basis for $X^\prime$. To start, note that $X = \Re^N$ is a Hilbert Space with the usual inner product space. Because of this, we know that we can extend said inner product to $X^\prime$.\\
  Also, we know that we can identify $\mathbf{F_i}$ as the following $n\times1$ matricies
  \begin{eqnarray*}
    \mathbf{F_1} &=&
    \begin{bmatrix}
      1,0,0,\hdots,0
    \end{bmatrix}
    \\
    \mathbf{F_2} &=&
    \begin{bmatrix}
      0,1,0,\hdots,0
    \end{bmatrix}
    \\
    \mathbf{F_3} &=&
    \begin{bmatrix}
      0,0,1,\hdots,0
    \end{bmatrix}
    \\
    &\vdots&\\
    \mathbf{F_n} &=&
    \begin{bmatrix}
      0,0,0,\hdots,1
    \end{bmatrix}
  \end{eqnarray*}
  which gives us 
  \begin{eqnarray*}
    \mathbf{F_j}(\mathbf{x}) &=& \mathbf{E_j^T}\mathbf{x}\\
    &=& <\mathbf{x},\mathbf{E_j}>\\
    &=& x_j\\
  \end{eqnarray*}
  and $\mathbf{F}$ is an o.n. basis of $X^\prime$. Extending this idea futher, we get
  \begin{eqnarray*}
    F(\mathbf{x}) &=& \sum_{j=1}^n\:<\mathbf{x},\mathbf{E_j}>\\
    &=& <\mathbf{x},\sum_{j=1}^n\:\mathbf{E_j}>\\
    &=& <\mathbf{x},\mathbf{z_F}>
  \end{eqnarray*}
  To show that $z_F$ is unique, we simply look at
  \begin{eqnarray*}
    f(\mathbf{x}) &=& <\mathbf{x},\mathbf{z}>\\
    &=& <\mathbf{x},\mathbf{z_F}>\\
    \implies <\mathbf{x},\mathbf{z} - \mathbf{z_F}> &=& 0 \; \forall \; \mathbf{x}
  \end{eqnarray*}.
  In the specific case of $\mathbf{x} = \mathbf{z} - \mathbf{z_F}$, we get $\|\mathbf{z} - \mathbf{z_F}\|^2 \implies \mathbf{z} = \mathbf{z_F}$. Hence, $\mathbf{z_F}$ is unique. Finally, to show $X \equiv X^\prime$, need to show that $T(\mathbf{F}) = \mathbf{z_F}$ is a linear bijective norm isometry. First, we know from above that $T$ is well defined but we still need to show it is 1 - 1. We can do this but showing it is a norm isometry i.e. $\|T(F)\| = \|\mathbf{z_F}\| = \|F\|_{op}$. First, $\mathbf{F}(\mathbf{x}) = <\mathbf{z_F}> \; \forall \mathbf{x}>$. This means that $\mathbf{F}(\mathbf{z_F}) \implies \frac{\mathbf{F}(\mathbf{z_F})}{{z_F}\|} = \|\mathbf{z_F}\| \implies \|\mathbf{F}\|_{op} = \|\mathbf{z_F}\|$. For the reverse equality, $|\mathbf{F}(\mathbf{x})| = |<\mathbf{x},\mathbf{z_F}>| \leq \|\mathbf{x}\|\|\mathbf{z_F}\| \implies \|\mathbf{F}\|_{op} = \sup_{\mathbf{x \neq 0}}\frac{\|\mathbf{F(x)}}{\|\mathbf{x}} \ leq \|\mathbf{z_F}\|$. When we combine the two inequalities we get $\|\mathbf{z_F}\|= \|\mathbf{F}\|_{op}$ which confirms that $T(\mathbf{F})$ is 1 -1. Next,
  \begin{eqnarray*}
    \mathbf{F(x)} &=& <\mathbf{x,z_F}>\\
    \implies (\alpha \mathbf{F})(\mathbf{x}) &=& \alpha<\mathbf{x,z_F}>\\
    &=& <\mathbf{x},\bar{\alpha}\mathbf{z_F}>
  \end{eqnarray*}
  which says that $T(\alpha\mathbf{F}) = <\mathbf{x},\bar{\alpha}\mathbf{z_F}$ so $\mathbf{z}_{\alpha\mathbf{F}} = \bar{\alpha}\mathbf{z_F}$. Then $T$ is additive because $(T(\mathbf{F} + \mathbf{G}))(\mathbf{x}) = <\mathbf{x},\mathbf{z}_{\alpha\mathbf{F}} + \mathbf{z}_{\beta\mathbf{G}}>$. To show $T$ is linear, we just consider 
  \begin{eqnarray*}
    (T(\mathbf{F} + \mathbf{G}))(\mathbf{x}) &=& <\mathbf{x},\mathbf{z}_{\alpha\mathbf{F}} + \mathbf{z}_{\beta\mathbf{G}}>\\
    &=& <\mathbf{x},\bar{\alpha}\mathbf{z_F} + \bar{\beta}\mathbf{z_G}>\\
    &=& \alpha<\mathbf{x},\mathbf{z_F}> + \beta<\mathbf{x},\mathbf{z_G}>\\
    &=& (\alpha T(\mathbf{F}) + \beta T(\mathbf{G}))(\mathbf{x})
  \end{eqnarray*}
  Finally we can show that $T$ is because any $\mathbf{z} \in X$ defines the bounded linear functional $F(\mathbf{x}) = <\mathbf{x},\mathbf{z}$. The choice of element assigned to $\mathbf{F}$ must be unique so $T(\mathbf{F}) = \mathbf{z}$. Hence, $T(\mathbf{F}) = \mathbf{z_F}$ is a linear bijective norm isometry and $X \equiv X^\prime$
\end{solution}

\textbf{Problem 3: }Prove the following statements
\begin{itemize}
  \item From our approximation theorem, we also know
  \begin{eqnarray*}
    T_N^0 = \sum_{j=0}^n\:\frac{1}{\Theta - \frac{n^2\pi^2}{L^2}}<\hat{u}_j,f>\hat{u}_j
  \end{eqnarray*}
  is the best approximation to the solution to $u'' + \Theta u = f;\;u'(0) = u'(L) = 0$ for $f \in \L([0,L])$ data to the subspace spanned by $\{\hat{u}\}_{n=0}^N$\\
  \begin{solution}
    First we assume $\Theta \neq \frac{n^2\pi^2}{L^2} \; \forall \; n > 1$ and $\Theta \neq 1$. Then, for $n = 0$, our eigenvalue and eigenfunction is $\beta_0 = 0$ and $u_0(x) = 1$ respectively. For the case of $n > 1$, the eigenvalue and eigenfunction is $\beta_n = \Theta - \frac{n^2\pi^2}{L^2}$ and $u_n(x) = \cos(\frac{n\pi x}{L})$. Normalized, these eigenfunctions are $\hat{u}_0(x) = \sqrt{\frac{1}{L}}$ and $\hat{u}_n(x) = \sqrt{\frac{2}{L}}\cos(\frac{n\pi x}{L})$. The sequence $\hat{u}_n(x)$ is a complete orthonormal sequence because it is a St\"urm - Liouville solution. From here, we get that any $f \in \L_2([0,L])$ has a unique expansion following $f = \sum_{n=0}^\infty<f,\hat{u}_n>\hat{u}_n$ and the patial sums converge in the $\|\cdot\|_2$ norm. Also, we interpret $f$ as a representative of the $[f]$ equivalence class. Now, from the Convergence of Approximations to the St\"urm Liouville Model, we know that partial sums, $\sum_{j=0}^n\:\mu_j\mathbf{u_j}<\mathbf{u_j},\mathbf{f}>$, converge to the continuous function $\mathbf{u} = J_{\lambda_0}(\mathbf{f})$ both uniformly and in the $\L_2([0,L])$ norm. We know that $\mathbf{u} = \frac{1}{\lambda_n - \lambda_0} \rightarrow 0$ where the lambdas represent the eigenvalues discussed above. This gives us
    an approximation for $f$ as
    \begin{eqnarray*}
    T_N^0 = \sum_{j=0}^n\:\frac{1}{\Theta - \frac{n^2\pi^2}{L^2}}<\hat{u}_j,f>\hat{u}_j
  \end{eqnarray*}
  \end{solution}
  \item
  \begin{eqnarray*}
    T_N^1 = \sum_{j=1}^n\:\sqrt{\frac{1}{\Theta - \frac{n^2\pi^2}{L^2}}}<\hat{v}_j,f>\hat{v}_j
  \end{eqnarray*}
  is the best approximation to the solution to $u'' + \Theta u = f;\;u'(0) = u(L) = 0$ for $f \in \L([0,L])$ data to the subspace spanned by $\{\hat{v}\}_{n=1}^N$
  \begin{solution}
    The following arguement is similar to that which was used above. Again, we assume $\Theta \neq \frac{n^2\pi^2}{L^2} \; \forall \; n > 1$ and $\Theta \neq 1$. Then, for $n > 1$, the eigenvalue and eigenfunction is $\beta_n = \Theta - \frac{n^2\pi^2}{L^2}$ and $v_n(x) = \sin(\frac{n\pi x}{L})$. Normalized, this eigenfunction is $\hat{v}_n(x) = \sqrt{\frac{2}{L}}$. The sequence $\hat{v}_n(x)$ is a complete orthonormal sequence because it is a St\"urm - Liouville solution. From here, we get that any $f \in \L_2([0,L])$ has a unique expansion following $f = \sum_{n=0}^\infty<f,\hat{v}_n>\hat{v}_n$ and the patial sums converge in the $\|\cdot\|_2$ norm. Also, we interpret $f$ as a representative of the $[f]$ equivalence class. Now, from the Convergence of Approximations to the St\"urm Liouville Model, we know that partial sums, $\sum_{j=0}^n\:\mu_j\mathbf{v_j}<\mathbf{v_j},\mathbf{f}>$, converge to the continuous function $\mathbf{v} = J_{\lambda_0}(\mathbf{f})$ both uniformly and in the $\L_2([0,L])$ norm. We know that $\mathbf{v} = \frac{1}{\lambda_n - \lambda_0} \rightarrow 0$ where the lambdas represent the eigenvalues discussed above. This gives us
    an approximation for $f$ as
    \begin{eqnarray*}
    T_N^1 = \sum_{j=1}^n\:\frac{1}{\Theta - \frac{n^2\pi^2}{L^2}}<\hat{v}_j,f>\hat{v}_j
  \end{eqnarray*}
  \end{solution}
\end{itemize}

\end{document}