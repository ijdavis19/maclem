% SampleProject.tex -- main LaTeX file for sample LaTeX article
%
%\documentclass[12pt]{article}
\documentclass[11pt]{SelfArxOneColBMN}
% add the pgf and tikz support.  This automatically loads
% xcolor so no need to load color
\usepackage{pgf}
\usepackage{tikz}
\usetikzlibrary{matrix}
\usetikzlibrary{calc}
\usepackage{xstring}
\usepackage{pbox}
\usepackage{etoolbox}
\usepackage{marginfix}
\usepackage{xparse}
\setlength{\parskip}{0pt}% fix as marginfix inserts a 1pt ghost parskip
% standard graphics support
\usepackage{graphicx,xcolor}
\usepackage{wrapfig}
%
\definecolor{color1}{RGB}{0,0,90} % Color of the article title and sections
\definecolor{color2}{RGB}{0,20,20} % Color of the boxes behind the abstract and headings
%----------------------------------------------------------------------------------------
%	HYPERLINKS
%----------------------------------------------------------------------------------------
\usepackage[pdftex]{hyperref} % Required for hyperlinks
\hypersetup{hidelinks,
colorlinks,
breaklinks=true,%
urlcolor=color2,%
citecolor=color1,%
linkcolor=color1,%
bookmarksopen=false%
,pdftitle={Lecture27HW},%
pdfauthor={Ian Davis}}
%\usepackage[round,numbers]{natbib}
\usepackage[numbers]{natbib}
\usepackage{lmodern}
\usepackage{setspace}
\usepackage{xspace}
%
\usepackage{subfigure}
\newcommand{\goodgap}{
  \hspace{\subfigtopskip}
  \hspace{\subfigbottomskip}}
%
\usepackage{atbegshi}
%
\usepackage[hyper]{listings}
%
% use ams math packages
\usepackage{amsmath,amsthm,amssymb,amsfonts}
\usepackage{mathrsfs}
%
% use new improved Verbatim
\usepackage{fancyvrb}
%
\usepackage[titletoc,title]{appendix}
%
\usepackage{url}
%
% Create length for the baselineskip of text in footnotesize
\newdimen\footnotesizebaselineskip
\newcommand{\test}[1]{%
 \setbox0=\vbox{\footnotesize\strut Test \strut}
 \global\footnotesizebaselineskip=\ht0 \global\advance\footnotesizebaselineskip by \dp0
}
%
\usepackage{listings}

\DeclareGraphicsExtensions{.pdf, .jpg, .tif,.png}

% make sure we don't get orphaned words if at top of page
% or orphans if at bottom of page
\clubpenalty=9999
\widowpenalty=9999
\renewcommand{\textfraction}{0.15}
\renewcommand{\topfraction}{0.85}
\renewcommand{\bottomfraction}{0.85}
\renewcommand{\floatpagefraction}{0.66}
%
\DeclareMathOperator{\sech}{sech}

\newcommand{\mycite}[1]{%
(\citeauthor{#1} \citep{#1} \citeyear{#1})\xspace
}

\newcommand{\mycitetwo}[2]{%
(\citeauthor{#2} \citep[#1]{#2} \citeyear{#2})\xspace
}

\newcommand{\mycitethree}[3]{%
(\citeauthor{#3} \citep[#1][#2]{#3} \citeyear{#3})\xspace
}

\newcommand{\myincludegraphics}[3]{% file name, width, height
\includegraphics[width=#2,height=#3]{#1}
}

\newcommand{\myincludegraphicstwo}[2]{% file name, width, height
\includegraphics[scale=#1]{#2}
}

\newcommand{\mysimplegraphics}[1]{% file name, width, height
\includegraphics{#1}
}

\newcommand{\MB}[1]{
\boldsymbol{#1}
}

\newcommand{\myquotetwo}[1]{%
\small
%\singlespacing
\begin{quotation}
#1
\end{quotation}
\normalsize
%\onehalfspacing  
}

\newcommand{\jimquote}[1]{%
\small
%\singlespacing
\begin{quotation}
#1
\end{quotation}
\normalsize
%\onehalfspacing
}

\newcommand{\myquote}[1]{%
\small
%\singlespacing
\begin{quotation}
#1
\end{quotation}
\normalsize
%\onehalfspacing  
}

%A =
%
%[2 r_1 	     r_1]
%[-2r_1 + r_2  r_2 - r_1]
%
%has eigenvalues r_1 neq r_2.
% #1 = 2 r_1, #2 = r_1, #3 = -2r_1+r_2, #4 = r_2 - r_1
\newcommand{\myrealdiffA}[4]{
\left [
\begin{array}{rr}
#1  & #2\\
#3  & #4
\end{array}
\right ]
}

% args:
% 1, 2 ,3, 4, 5 = caption, label, width, height, file name
%\mysubfigure{}{}{}{}{}
\newcommand{\mysubfigure}[5]{%
\subfigure[#1]{\label{#2}\includegraphics[width=#3,height=#4]{#5}}
}

\newcommand{\mysubfiguretwo}[3]{%
\subfigure[#1]{\label{#2}\includegraphics{#3}}
}

\newcommand{\mysubfigurethree}[4]{%
\subfigure[#1]{\label{#2}\includegraphics[scale=#3]{#4}}
}

\newcommand{\myputimage}[5]{% file name, width, height
\centering
\includegraphics[width=#3,height=#4]{#5}
\caption{#1}
\label{#2}
}

\newcommand{\myputimagetwo}[4]{% caption, label, scale, file name
\centering
\includegraphics[scale=#3]{#4}
\caption{#1}
\label{#2}
}

\newcommand{\myrotateimage}[5]{% file name, width, height
\centering
\includegraphics[scale=#3,angle=#4]{#5}
\caption{#1}
\label{#2}
}

\newcommand{\myurl}[2]{%
\href{#1}{\bf #2}
}

\RecustomVerbatimEnvironment%
{Verbatim}{Verbatim}  
  {fillcolor=\color{black!20}}
  
  \DefineVerbatimEnvironment%
{MyVerbatim}{Verbatim}  
  {frame=single,
   framerule=2pt,
   fillcolor=\color{black!20},
   fontsize=\small}
   
\newcommand{\myfvset}[1]{%  
\fvset{frame=single,
       framerule=2pt,
       fontsize=\small,
       xleftmargin=#1in}}
       
\newcommand{\mylistverbatim}{%
\lstset{%
  fancyvrb, 
  basicstyle=\small,
  breaklines=true}
}  

\newcommand{\mylstinlinebf}[1]{%
{\bf #1}
}

\newcommand{\mylstinline}{%
\lstset{%
  basicstyle=\color{black!80}\bfseries\ttfamily,
  showstringspaces=false,
  showspaces=false,showtabs=false,
  breaklines=true}
\lstinline
}

\newcommand{\mylstinlinetwo}[1]{%
\lstset{%
  basicstyle=\color{black!80}\bfseries\ttfamily,
  showstringspaces=false,
  showspaces=false,showtabs=false,
  breaklines=true}
\lstinline!#1! 
}

%fontfamily=tt
%fontfamily=courier
%fontfamily=helvetica
%frame=topline,
%frame=single,
 %frame=lines,
 %framesep=10pt,
 %fontshape=it,
 %fontseries=b,
 %fontsize=\relsize{-1},
 %fillcolor=\color{black!20},
 %rulecolor=\color{yellow},
 %fillcolor=\color{red}
 %label=\fbox{\Large\emph{The code}}
\DefineVerbatimEnvironment%
{MyListVerbatim}{Verbatim}  
{
fillcolor=\color{black!10},
fontfamily=courier,
frame=single,
%formatcom=\color{white},
framesep=5mm,
labelposition=topline,
fontshape=it,
fontseries=b,
fontsize=\small,
label=\fbox{\large\emph{The code}\normalsize}
} 

%  caption={[#1] \large\bf{#1}}, 
%\centering \framebox[.6\textwidth][c]{\Large\bf{#1}}
\newcommand{\myfancyverbatim}[1]{%
\lstset{%
  fancyvrb=true, 
  %fvcmdparams= fillcolor 1,
  %morefvcmdparams = \textcolor 2,
  frame=shadowbox,framerule=2pt, 
  basicstyle=\small\bfseries,
  backgroundcolor=\color{black!08},
  showstringspaces=false,
  showspaces=false,showtabs=false,
  keywordstyle=\color{black}\bfseries,
  %numbers=left,numberstyle=\tiny,stepnumber=5,numbersep=5pt,
  stringstyle=\ttfamily,
  caption={[\quad #1] \mbox{}\\ \vspace{0.1in} \framebox{\large \bf{#1} \small} },  
  belowcaptionskip=20 pt,  
  label={},
  xleftmargin=17pt,
  framexleftmargin=17pt,
  framexrightmargin=5pt,
  framexbottommargin=4pt,
  nolol=false,
  breaklines=true}
}

\newcommand{\mylistcode}[3]{%
\lstset{%
  language=#1, 
  frame=shadowbox,framerule=2pt, 
  basicstyle=\small\bfseries,
  backgroundcolor=\color{black!16},
  showstringspaces=false,
  showspaces=false,showtabs=false,
  keywordstyle=\color{black!40}\bfseries,
  numbers=left,numberstyle=\tiny,stepnumber=5,numbersep=5pt,
  stringstyle=\ttfamily,
  caption={[\quad#2] \mbox{}\\ \vspace{0.1in} \framebox{\large \bf{#2} \small} },
  belowcaptionskip=20 pt,
  breaklines=true,
  xleftmargin=17pt,
  framexleftmargin=17pt,
  framexrightmargin=5pt,
  framexbottommargin=4pt,  
  label=#3,
  breaklines=true} 
}

  %caption={[#2] #3},
  %caption={[#2]{\mbox{}\\ \vspace{0.1in} \framebox{\large \bf{#3} \small}},
  %caption={[#2] \mbox{}\\ \bf{#3} },

% frame=single,
% caption={[Code Fragment] {\bf Code Fragment} },
% caption={[Code Fragment] \mbox{}\\ \vspace{0.1in} \framebox{\large \bf{Code Fragment} \small} },
\newcommand{\mylistcodequick}[1]{%
\lstset{%
  language=#1, 
  frame=shadowbox,framerule=2pt, 
  basicstyle=\small\bfseries,
  backgroundcolor=\color{black!16},
  showstringspaces=false,
  showspaces=false,showtabs=false,
  keywordstyle=\color{black!40}\bfseries,
  numbers=left,numberstyle=\tiny,stepnumber=5,numbersep=5pt,
  stringstyle=\ttfamily,
  caption={[\quad Code Fragment] \large \bf{Code Fragment} \small},   
  belowcaptionskip=20 pt,  
  label={},
  xleftmargin=17pt,
  framexleftmargin=17pt,
  framexrightmargin=5pt,
  framexbottommargin=4pt,
  breaklines=true} 
}

%  caption={[#2] \mbox{}\\ \vspace{0.1in} \framebox{\large \bf{#2} \small} },
\newcommand{\mylistcodequicktwo}[2]{%
\lstset{%
  language=#1, 
  frame=shadowbox,framerule=2pt, 
  basicstyle=\small\bfseries,
  extendedchars=true,
  backgroundcolor=\color{black!16},
  showstringspaces=false,
  showspaces=false,
  showtabs=false,
  keywordstyle=\color{black!40}\bfseries,
  numbers=left,numberstyle=\tiny,stepnumber=5,numbersep=5pt,
  stringstyle=\ttfamily,
  caption={[\quad#2] \large \bf{#2} \small},
  belowcaptionskip=20 pt,
  label={},
  xleftmargin=17pt,
  framexleftmargin=17pt,
  framexrightmargin=5pt,
  framexbottommargin=4pt,
  breaklines=true} 
}

%  caption={[#2] \mbox{}\\ \vspace{0.1in} \framebox{\large \bf{#2} \small} },
\newcommand{\mylistcodequickthree}[2]{%
\lstset{%
  language=#1, 
  frame=shadowbox,framerule=2pt, 
  basicstyle=\small\bfseries,
  extendedchars=true,
  backgroundcolor=\color{black!16},
  showstringspaces=false,
  showspaces=false,
  showtabs=false,
  keywordstyle=\color{black!40}\bfseries,
  numbers=left,numberstyle=\tiny,stepnumber=5,numbersep=5pt,
  stringstyle=\ttfamily,
  caption={[\quad#2] \large\bf{#2}\small},
  belowcaptionskip=20 pt,
  label={},
  xleftmargin=17pt,
  framexleftmargin=17pt,
  framexrightmargin=5pt,
  framexbottommargin=4pt,
  breaklines=true} 
}

%  frame=single,
\newcommand{\mylistset}[4]{%
\lstset{language=#1,
  basicstyle=\small,
  showstringspaces=false,
  showspaces=false,showtabs=false,
  keywordstyle=\color{black!40}\bfseries,
  numbers=left,numberstyle=\tiny,stepnumber=5,numbersep=5pt,
  stringstyle=\ttfamily,
  caption={[\quad#2]#3},
  label=#4}
}

\newcommand{\mylstinlineset}{%
\lstset{%
  basicstyle=\color{blue}\bfseries\ttfamily,
  showstringspaces=false,
  showspaces=false,showtabs=false,
  breaklines=true}
}

\newcommand{\myframedtext}[1]{%
\centering
\noindent
%\fbox{\parbox[c]{.9\textwidth}{\color{black!40} \small \singlespacing #1\onehalfspacing \normalsize \\}}
\fbox{\parbox[c]{.9\textwidth}{\color{black!40} \small  #1 \normalsize \\}}
}

\newcommand{\myemptybox}[2]{% from , to
\fbox{\begin{minipage}[t][#1in][c]{#2in}\hspace{#2in}\end{minipage}}
}

\newcommand{\myemptyboxtwo}[2]{% from , to
\centering\fbox{
\begin{minipage}{#1in}
\hfill\vspace{#2in}
\end{minipage}
}
}

\newcommand{\boldvector}[1]{
\boldsymbol{#1}
}

\newcommand{\dEdY}[2]{\frac{d E}{d Y_{#1}^{#2}}}
\newcommand{\dEdy}[2]{\frac{d E}{d y_{#1}^{#2}}}
\newcommand{\dEdT}[2]{\frac{\partial E}{\partial T_{{#1} \rightarrow {#2}}}}
\newcommand{\dEdo}[1]{\frac{\partial E}{\partial o^{#1}}}
\newcommand{\dEdg}[1]{\frac{\partial E}{\partial g^{#1}}}
\newcommand{\dYdY}[4]{\frac{\partial Y_{#1}^{#2}}{\partial Y_{#3}^{#4}}}
\newcommand{\dYdy}[4]{\frac{\partial Y_{#1}^{#2}}{\partial y_{#3}^{#4}}}
\newcommand{\dydY}[4]{\frac{\partial y_{#1}^{#2}}{\partial Y_{#3}^{#4}}}
\newcommand{\dydy}[4]{\frac{\partial y_{#1}^{#2}}{\partial y_{#3}^{#4}}}
\newcommand{\dydT}[4]{\frac{\partial y_{#1}^{#2}}{\partial T_{{#3} \rightarrow {#4}}}}
\newcommand{\dYdT}[4]{\frac{\partial Y_{#1}^{#2}}{\partial T_{{#3} \rightarrow {#4}}}}
\newcommand{\dTdT}[4]{\frac{\partial T_{{#1} \rightarrow {#2}}}{\partial T_{{#3} \rightarrow {#4}}}}
\newcommand{\ssum}[2]{\sum_{#1}^{#2}}

\newcommand{\ssty}[1]{\scriptscriptstyle #1}

\newcommand{\myparbox}[2]{%
\parbox{#1}{\color{black!20} #2}
}

\newcommand{\bs}[1]{
\boldsymbol{#1}
}

\newcommand{\parone}[2]{%
\frac{\partial #1 }{ \partial #2 }
}
\newcommand{\partwo}[2]{%
\frac{ \partial^2 {#1} }{ \partial {#2}^2 }
}

\newcommand{\twodvectorvarfun}[2]{
\left [
\begin{array}{r}
{{#1_{\ssty{1}}}(#2)} \\
{{#1_{\ssty{2}}}(#2)}
\end{array}
\right ]
}
\newcommand{\twodvectorvarprimed}[1]{
\left [
\begin{array}{r}
{{#1_{\ssty{1}}}'(t)} \\
{{#1_{\ssty{2}}}'(t)}
\end{array}
\right ]
}

\newcommand{\complex}[2]{#1 \: #2 \: \boldsymbol{i}}
\newcommand{\complexmag}[2]%
{
\sqrt{(#1)^2 \: + \: (#2)^2}
}
\newcommand{\threenorm}[3]%
{
\sqrt{(#1)^2 \: + \: (#2)^2 \: + \: (#3)^2}
}
\newcommand{\norm}[1]{\mid \mid #1 \mid \mid}

\newcommand{\myderiv}[2]{\frac{d #1}{d #2}}
\newcommand{\myderivb}[2]{\frac{d}{d #2} \left ( #1 \right )}
\newcommand{\myrate}[3]%
{#1^\prime(#2) &=& #3 \: #1(#2)
}
\newcommand{\myrateexter}[4]%
{#1^\prime(#2) &=& #3 \: #1(#2) \: + \: #4
}
\newcommand{\myrateic}[3]%
{#1( \: #2 \:) &=& #3 
}

\newcommand{\mytwodsystemeqn}[6]{
#1 \: x    #2 \: y &=& #3\\
#4 \: x    #5 \: y &=& #6\\
}

\newcommand{\mytwodsystem}[8]{
#3 \: #1 \: + \: #4 \: #2 &=& #5\\
#6 \: #1 \: + \: #7 \: #2 &=& #8\\
}  

\newcommand{\mytwodarray}[4]{
\left [
\begin{array}{rr}
#1 & #2\\
#3 & #4
\end{array}
\right ]
}

\newcommand{\mytwoid}{
\left [
\begin{array}{rr}
1 & 0\\
0 & 1
\end{array}
\right ]
}

\newcommand{\myxprime}[2]{
\left [
\begin{array}{r}
#1^\prime(t)\\
#2^\prime(t)
\end{array}
\right ]
}

\newcommand{\myxprimepacked}[2]{
\left [
\begin{array}{r}
#1^\prime\\
#2^\prime
\end{array}
\right ]
}

\newcommand{\myx}[2]{
\left [
\begin{array}{r}
#1(t)\\
#2(t)
\end{array}
\right ]
}

\newcommand{\myxonly}[2]{
\left [
\begin{array}{r}
#1\\
#2
\end{array}
\right ]
}

\newcommand{\myv}[2]{
\left [
\begin{array}{r}
#1\\
#2
\end{array}
\right ]
}

\newcommand{\myxinitial}[2]{
\left [
\begin{array}{r}
#1(0)\\
#2(0)
\end{array}
\right ]
}

\newcommand{\twodboldv}[1]{
\boldsymbol{#1}
}

\newcommand{\mytwodvector}[2]{
\left [
\begin{array}{r}
#1\\
#2
\end{array}
\right ]
}

\newcommand{\mythreedvector}[3]{
\left [
\begin{array}{r}
#1\\
#2\\
#3
\end{array}
\right ]
}

\newcommand{\mytwodsystemvector}[6]{
\left [
\begin{array}{rr}
#1 & #2\\
#4 & #5
\end{array}
\right ]
\:
\left [
\begin{array}{r}
x \\
y 
\end{array}
\right ]
&=&
\left [
\begin{array}{r}
#3\\
#6
\end{array}
\right ]
}

\newcommand{\mythreedarray}[9]{
\left [
\begin{array}{rrr}
#1 & #2 & #3\\
#4 & #5 & #6\\
#7 & #8 & #9
\end{array}
\right ]
}

\newcommand{\myodetwo}[6]{
#1 \: #6^{\prime \prime}(t) \: #2 \: #6^{\prime}(t) \: #3 \: #6(t) &=& 0\\
#6(0)                                           &=& #4\\
#6^{\prime}(0)                                  &=& #5
}

\newcommand{\myodetwoNoIC}[4]{
#1 \: #4^{\prime \prime}(t) \: #2 \: #4^{\prime}(t) \: #3 \: #4(t) &=& 0
}

\newcommand{\myodetwopacked}[5]{
\hspace{-0.3in}& & #1 u^{\prime \prime} #2 u^{\prime} #3 u \: = \: 0\\
\hspace{-0.3in}& & u(0) \: = \: #4, \: \: u^{\prime}(0)    \: = \:  #5
}

\newcommand{\myodetwoforced}[6]{
#1\: u^{\prime \prime}(t) \: #2 \: u^{\prime}(t) \: #3 \: u(t) &=& #6\\
u(0)                                           &=& #4\\
u^{\prime}(0)                                  &=& #5\\
}

\newcommand{\myodesystemtwo}[8]{
#1 \: x^\prime(t) \: #2 \: y^\prime(t) \: #3 \: x(t) \: #4 \: y(t) &=& 0\\
#5 \: x^\prime(t) \: #6 \: y^\prime(t) \: #7 \: x(t) \: #8 \: y(t) &=& 0\\
}

\newcommand{\myodesystemtwoic}[2]{
x(0)                                       &=& #1\\ 
y(0)                                       &=& #2
}

\newcommand{\mypredprey}[4]{
x^\prime(t) &=& #1 \: x(t) \: - \: #2 \: x(t) \: y(t)\\
y^\prime(t) &=& -#3 \: y(t) \: + \: #4 \: x(t) \: y(t)
}

\newcommand{\mypredpreypacked}[4]{
x^\prime &=& #1 \: x - #2 \: x \: y\\
y^\prime &=& -#3 \: y + #4 \: x \: y
}

\newcommand{\mypredpreyself}[6]{
x^\prime(t) &=&  #1 \: x(t) \: - \: #2 \: x(t) \: y(t) \: - \: #3 \: x(t)^2\\
y^\prime(t) &=& -#4 \: y(t) \: + \: #5 \: x(t) \: y(t) \: - \: #6 \: y(t)^2
}

\newcommand{\mypredpreyfish}[5]{
x^\prime(t) &=&  #1 \: x(t) \: - \: #2 \: x(t) \: y(t) \: - \: #5 \: x(t)\\
y^\prime(t) &=& -#3 \: y(t) \: + \: #4 \: x(t) \: y(t) \: - \: #5 \: y(t)
}

\newcommand{\myepidemic}[4]{
S^\prime(t) &=& - #1 \: S(t) \: I(t)\\
I^\prime(t) &=&   #1 \: S(t) \: I(t) \: - \: #2 \: I(t)\\
S(0)        &=&   #3\\
I(0)        &=&   #4\\
}

\newcommand{\bsred}[1]{%
\textcolor{red}{\boldsymbol{#1}}
}

\newcommand{\bsblue}[1]{%
\textcolor{blue}{\boldsymbol{#1}}
}


\newcommand{\myfloor}[1]{%
\lfloor{#1}\rfloor
}

\newcommand{\cubeface}[7]{%
\begin{bmatrix}
\bs{#3}          & \longrightarrow & \bs{#4}\\
\uparrow          &                         &  \uparrow  \\
\bs{#1} & \longrightarrow & \bs{#2}\\
              & \text{ \bfseries #5:} \: \bs{#6} \: \text{\bfseries  #7 } & 
\end{bmatrix}
}

\newcommand{\cubefacetwo}[5]{%
\begin{bmatrix}
\bs{#3}          & \longrightarrow & \bs{#4}\\
\uparrow          &                         &  \uparrow  \\
\bs{#1} & \longrightarrow & \bs{#2}\\
              & \text{ \bfseries #5} & 
\end{bmatrix}
}

\newcommand{\cubefacethree}[9]{%
\begin{bmatrix}
\bs{#3}                  & \overset{#9}{\longrightarrow} & \bs{#4}\\
\uparrow \: #7         &                                             &  \uparrow  \: #8 \\
\bs{#1}                  & \overset{#6}{\longrightarrow} & \bs{#2}\\
                               & \text{ \bfseries #5} & 
\end{bmatrix}
}

\renewcommand{\qedsymbol}{\hfill \blacksquare}
\newcommand{\subqedsymbol}{\hfill \Box}
%\theoremstyle{plain}

\newtheoremstyle{mystyle}% name
  {6pt}%      Space above
  {6pt}%      Space below
  {\itshape}%         Body font
  {}%         Indent amount (empty = no indent, \parindent = para indent)
  {\bfseries}% Thm head font
  {}%        Punctuation after thm head
  { }%     Space after thm head: " " = normal interword space; \newline = linebreak
  {}%         Thm head spec (can be left empty, meaning `normal')
\theoremstyle{mystyle}
 
\newtheorem{axiom}{Axiom}
%\newtheorem{solution}{Solution}[section]
\newtheorem*{solution}{Solution}
\newtheorem{exercise}{Exercise}[section]
\newtheorem{theorem}{Theorem}[section]
\newtheorem{proposition}[theorem]{Proposition}
\newtheorem{prop}[theorem]{Proposition}
\newtheorem{assumption}{Assumption}[section]
\newtheorem{definition}{Definition}[section]
\newtheorem{comment}{Comment}[section]
\newtheorem*{question}{Question}
\newtheorem{program}{Program}[section]
%\newtheorem{myproof}{Proof}
%\newtheorem*{myproof}{Proof}[section]
\newtheorem{myproof}{Proof}[section]
\newtheorem{hint}{Hint}[section]
\newtheorem*{phint}{Hint}
\newtheorem{lemma}[theorem]{Lemma}
\newtheorem{example}{Example}[section]
      
\newenvironment{myassumption}[4]
{
\centering
\begin{assumption}[{\textbf{#1}\nopunct}]%
\index{#2}
\mbox{}\\  \vskip6pt \colorbox{black!15}{\fbox{\parbox{.9\textwidth}{#3}}}
\label{#4}
\end{assumption}
%\renewcommand{\theproposition}{\arabic{chapter}.\arabic{section}.\arabic{assumption}} 
}%
{}

\newenvironment{myproposition}[4]
{
\centering
\begin{proposition}[{\textbf{#1}\nopunct}]%
\index{#2} 
\mbox{}\\  \vskip6pt \colorbox{black!15}{\fbox{\parbox{.9\textwidth}{#3}}}
\label{#4}
\end{proposition}
%\renewcommand{\theproposition}{\arabic{chapter}.\arabic{section}.\arabic{proposition}} 
}%
{}

\newenvironment{mytheorem}[4]
{
\centering
\begin{theorem}[{\textbf{#1}\nopunct}]%
\index{#2} 
\mbox{}\\ \vskip6pt \colorbox{black!15}{\fbox{\parbox{.9\textwidth}{#3}}}
\label{#4}
\end{theorem}
%\renewcommand{\thetheorem}{\arabic{chapter}.\arabic{section}.\arabic{theorem}} 
}%
{}

\newenvironment{mydefinition}[4]
{
\centering
\begin{definition}[{\textbf{#1}\nopunct}]%
\index{#2} 
\mbox{}\\  \vskip6pt \colorbox{black!15}{\fbox{\parbox{.9\textwidth}{#3}}}
\label{#4}
\end{definition}
%\renewcommand{\thedefinitio{n}{\arabic{chapter}.\arabic{section}.\arabic{definition}} 
}%
{}

\newenvironment{myaxiom}[4]
{
\centering
\begin{axiom}[{\textbf{#1}\nopunct}]%
\index{#2} 
\mbox{}\\  \vskip6pt \colorbox{black!15}{\fbox{\parbox{.9\textwidth}{#3}}}
\label{#4}
\end{axiom}
%\renewcommand{\theaxiom}{\arabic{chapter}.\arabic{section}.\arabic{axiom}} 
}%
{}

\newenvironment{mylemma}[4]
{
\centering
\begin{lemma}[{\textbf{#1}\nopunct}]%
\index{#2} 
\mbox{}\\  \vskip6pt \colorbox{black!15}{\fbox{\parbox{.9\textwidth}{#3}}}
\label{#4}
\end{lemma}
%\renewcommand{\thelemma}{\arabic{chapter}.\arabic{section}.\arabic{lemma}} 
}%
{}
   
\newenvironment{reason}[1]
{
 \vskip0.05in
 \begin{myproof}
 \mbox{}\\#1
 $\qedsymbol$
 \end{myproof}  
 \vskip0.05in
}%
{}

\newenvironment{reasontwo}[1]
{
 \vskip+.05in
 \begin{myproof}
 \mbox{}\\#1
 \end{myproof}  
 \vskip+0.05in
}%
{}

\newenvironment{subreason}[1]
{
 \vskip0.05in
 \renewcommand{\themyproof}{}
 \begin{myproof}
 #1
 $\subqedsymbol$
 \end{myproof}
 \vskip0.05in
 \renewcommand{\themyproof}{\thetheorem}
 %\renewcommand{\themyproof}{\arabic{chapter}.\arabic{section}.\arabic{myproof}}   
 %
}%
{}  

\newenvironment{myhint}[1]
{
 \vskip0.05in
 \begin{hint}
 #1
 $\subqedsymbol$ 
 \end{hint}  
 \vskip0.05in
}%
{} 

\newenvironment{myeqn}[3]
{
 \renewcommand{\theequation}{$\boldsymbol{#1}$}
 \begin{eqnarray}
 \label{equation:#2}
 #3 
 \end{eqnarray}
 \renewcommand{\theequation}{\arabic{chapter}.\arabic{eqnarray}}   
}%
{} 


\JournalInfo{MATH 8210:  Lecture 27 Homework, 1-\pageref{LastPage}, 2020} % Journal information
\Archive{Draft Version \today} % Additional notes (e.g. copyright, DOI, review/research article)

\PaperTitle{MATH 8210 Lecture 27 Homework}
\Authors{Ian Davis\textsuperscript{1}}
\affiliation{\textsuperscript{1}\textit{John E. Walker Department of Economics,
Clemson University,Clemson, SC: email ijdavis@g.clemson.edu}}
%\affiliation{*\textbf{Corresponding author}: yournamehere@clemson.edu} % Corresponding author

\date{\small{Version ~\today}}
\Abstract{Homework from Lecture 27}
\Keywords{}
\newcommand{\keywordname}{Keywords}
%
\onehalfspacing
\begin{document}

\flushbottom

\addcontentsline{toc}{section}{Title}
\maketitle

\renewcommand{\theexercise}{\arabic{exercise}}

\begin{exercise}
    Let $M$ be the space of all 2 x 1 vectors whos components are in $l^1$.
    \begin{itemize}
        \item Endow M with the norm $\|A\| = \sum_{i = 1}^2\|A_i\|_1$. Prove this is a norm.
        \begin{solution}
            First, lets write out $\|A\|$ so we can show that it satifies the four conditions for a norm.
            \begin{eqnarray*}
                \|A\| &=& \sum_{i=1}^2\|A_i\|_1\\
                &=& \|A_1\|_1 + \|A_2\|_1\\
            \end{eqnarray*}
            Where both $\|A_1\|_1$ and $\|A_2\|_1$ are both the $\ell^1$ norm. So 
            \begin{enumerate}
                \item $\|A_1\|_1 \geq 0$ and $\|A_2\|_1 \geq 0$ so $\|A_1\|_1 + \|A_2\|_1 \geq 0$
                \item $\|A\| = 0 \iff \|A_1\|_1 + \|A_2\|_1 = 0 \iff \|A_1\|_1 = -\|A_2\|_1$ but becuase $\|A_1\|_1 \geq 0$ and $\|A_2\|_1 \geq 0$, then $\|A_1\|_1 = -\|A_2\|_1 \iff \|A_1\|_1 = \|A_2\|_1 = 0$
                \item $\|\alpha A\| = \|\alpha A_1\|_1 + \|\alpha A_2\|_1 = |\alpha|(\|A_1\|_1 + \|A_2\|_1)$
                \item Finally, know that $\|A + B\| < \|A\| + \|B\|$ by Minkowski's Inequality
            \end{enumerate}
        \end{solution}
        \item define $\mathbf{(F_n)}$ by
        \begin{eqnarray*}
            \mathbf{F_n^1} \; = \;
            \begin{bmatrix}
                \mathbf{E_n}\\
                \mathbf{0}
            \end{bmatrix}
            \; & \;
            \mathbf{F_n^2} \; = \;
            \begin{bmatrix}
                \mathbf{0}\\
                \mathbf{E_n}
            \end{bmatrix}
        \end{eqnarray*}
        where $(E_n)$ is the usual Schauder basis for $l^1$. Prove this is a Schauder Basis for M.
        \begin{solution}
            Because we know both $A_1$ and $A_2$ are in $\ell^1$ and $(E_n)$ is the $\ell^1$ Schauder Basis, then $\exists$ scalars $(\alpha_n)_{n=1}^\infty$ and $(\beta_n)_{n=1}^\infty$ such that, for $m^i \in \mathbf{E_n}$
            \begin{eqnarray*}
                A_1 = lim_{n\rightarrow \infty}\sum_{j=1}^n\:\alpha_j m^1_j\\
                A_2 = lim_{n\rightarrow \infty}\sum_{j=1}^n\:\beta_j m^2_j
            \end{eqnarray*}
            So now, we consider the following element,
            $
            \begin{bmatrix}
            x\\
            y
            \end{bmatrix}
            $
            , of M
            \begin{eqnarray*}
                \begin{bmatrix}
                    x\\
                    y
                \end{bmatrix}
                &=&
                \begin{bmatrix}
                    \sum_{i=1}^\infty\;\alpha_i\mathbf{(E_i)}\\
                    \sum_{i=1}^\infty\;\beta_i\mathbf{(E_i)}
                \end{bmatrix}
                \\
                &=& \sum_{i=1}^\infty\;\alpha_i
                \begin{bmatrix}
                    \mathbf{E_i}\\
                    \mathbf{0}
                \end{bmatrix}
                + \sum_{i=1}^\infty\;\beta_i
                \begin{bmatrix}
                    \mathbf{0}\\
                    \mathbf{E_i}
                \end{bmatrix}
                \\
                &=& \sum_{i=1}^\infty\;\alpha_i\mathbf{(F_i)}^1 + \sum_{i=1}^\infty\;\beta_i\mathbf{(F_i)}^2
            \end{eqnarray*}
            Hence, we can conclude that $\mathbf{(F_n)}$ is a Schauder Basis for M.
        \end{solution}
        \item Go through all the steps of the proof the characterize $M^\prime$.
        \begin{solution}
        First, we define $K$ to be the space of all $2 \times 1$ vectors whose components are in $\ell^\infty$, We want to prove that, with $f \in M^\prime, \; \exists \; !\mathbf{z_f} \in K$ so that for any $\mathbf{x} \in M$
        \begin{eqnarray*}
            f(\mathbf{x}) = \sum_{n=1}^\infty\;x_n(z_f)_i
        \end{eqnarray*}
        where $\mathbf{x} = (x_n)$ and $\mathbf{z_f} = ((z_f)_n)$. Further $\|z_f\|_\infty = \|f\|_{op}$. To do so, let $f \in M^\prime$. We have $A =
        \begin{bmatrix}
        x\\
        y
        \end{bmatrix}
        $
        which has representation $A = \sum_{i=1}^\infty\;\alpha_i\mathbf{(F_i)}^1 + \sum_{i=1}^\infty\;\beta_i\mathbf{(F_i)}^2$ and $\|A\| < \infty$ and $\|\mathbf{F_n}\|_1 = 1 \; \forall \; n$. Let $\mathbf{S_n} = \sum_{i=1}^\infty\;\alpha_i\mathbf{(F_i)}^1 + \sum_{i=1}^\infty\;\beta_i\mathbf{(F_i)}^2$. Then by linearity of f, $f\mathbf{(S_n)} = \sum_{i=1}^\infty\;\alpha_i f(\mathbf{(F_i)}^1) + \sum_{i=1}^\infty\;\beta_i  f(\mathbf{(F_i)}^2)$. We know that $lim_{n\rightarrow\infty}\mathbf{S_n} = A$ because $\mathbf{(F_n)}$ is a Schauder Basis and we also know f is continuous. So $lim_{n\rightarrow\infty}f(\mathbf{S_n}) = f(lim_{n\rightarrow\infty}\mathbf{S_n}) = f(A)$. Hence, $f(A) = lim_{n\rightarrow\infty} \sum_{i=1}^n\;\alpha_if(\mathbf{(F_i)}^1) + \sum_{i=1}^n\;\beta_if(\mathbf{(F_i)}^2)$ and we have shown that the series $\sum_{i=1}^\infty\;\alpha_if(\mathbf{(F_i)}^1) + \sum_{i=1}^\infty\;\beta_if(\mathbf{(F_i)}^2)$ converges.\\
        \\
        Let the sequence $\mathbf{z_f}$ be defined by $(z_f)_n = f(\mathbf{F_n})$, Next, we need to show the sequence $\mathbf{z_f} \in K$ and its norm is $\|\mathbf{z_f}\|_\infty = \|f\|_{op}$.\\
        \\
        To show $\mathbf{z_f} \in K$, note 
        \begin{eqnarray*}
            |(z_f)_n| = |f_(\mathbf{F_n})| \leq \|f\|_{op}\|\mathbf{F_n}\|_1 = \|f\|_{op}
        \end{eqnarray*}
        which implies that all the entries of $\mathbf{z_f}$ are bounded and $\mathbf{z_f} \in K$. To show $\|\mathbf{z_f}\|_K = \|f\|_{op}$, we only have to show the reverse inequality of the already proven $\|\mathbf{z_f}\|_K \leq \|f\|_{op}$. To do this, consider
        \begin{eqnarray*}
            |f(A)| &=& |f(\sum_{i=1}^\infty\;\alpha_i\mathbf{(F_i)}^1 + \sum_{i=1}^\infty\;\beta_i\mathbf{(F_i)}^2)|\\
            &=& |f(lim_{n\rightarrow\infty}\mathbf{S_n})|\\
            &=& lim_{n\rightarrow\infty}|f(\mathbf{S_n})| \; \text{by the continuity of f and} \; |\cdot|\\
            &\leq& lim_{n\rightarrow\infty}\sum_{i=1}^n\;\alpha_i\mathbf{(F_i)}^1 + lim_{n\rightarrow\infty}\sum_{i=1}^n\;\beta_i\mathbf{(F_i)}^2\\
            &=& lim_{n\rightarrow\infty}\sum_{i=1}^n\;\alpha_i(z_f)_i^1 + lim_{n\rightarrow\infty}\sum_{i=1}^n\;\beta_i(z_f)_i^2\\
            &\leq& \|\mathbf{z_f}\|_K \sum_{i=1}^n|\alpha_i| \|\mathbf{z_f}\|_K \sum_{i=1}^n|\beta_i|\\
            &\leq& \|\mathbf{z_f}\|_K\|A\|_1\\
            &\implies& \|f\|_{op} = \sup_{A\neq\mathbf{0}}\frac{|f(A)|}{\|x\|_1} \leq \|\mathbf{z_f}\|_k
        \end{eqnarray*}
        Combining, we get the elements of $\mathbf{z_f} \in \ell^\infty$. Lastly we want to know if $\mathbf{z_f}$ is unique. If we could find $z \in K$ with $f(A) = \sum_{n=1}^\infty\:A_n^1z_n^1 + \sum_{n=1}^\infty\:A_n^2z_n^2$ then $\sum_{n=1}^\infty\:A_n^1z_n^1 + \sum_{n=1}^\infty\:A_n^2z_n^2 = \sum_{n=1}^\infty\:A_n^1(z_f)_n^1 + \sum_{n=1}^\infty\:A_n^2(z_f)_n^2$. Specifically we would find $f(1\mathbf{F_i}) = 1 z_i$ and so $\mathbf{z_f} = z$
        \end{solution}
        \item Prove this duel space is equivalent to the space $N$ of two dimensional vectors whose components are in $l^\infty$ using the obvious norm.
        \begin{solution}
            We need to show $M^\prime \equiv K$ where K is the same K as defined above. For $f \in M^\prime$, there is unique $\mathbf{z_f} = ((z_f)_n) =
            \begin{bmatrix}
                (z_f)_n^1\\
                (z_f)_n^2
            \end{bmatrix}
             \in K$
        so $f(A) = \sum_{n=1}^\infty A_n^1(z_f)_n^1 + \sum_{n=1}^\infty A_n^2(z_f)_n^2$ with $A \in K$ and $\|z_f\|_K = \|f\|_{op}$ Define $T: M^\prime \rightarrow K$ by $T(f) = \mathbf{z_f}$. Then there is a linear bijective isometry and $M^\prime \equiv K$. To prove this, we want to show that T is a linear bijective isometry. We already know $\|T(f)\|_K = \|\mathbf{z_f}\|_K = \|f\|_{op}$ so $T$ preserves norms. We also know $\mathbf{z_f}$ is unique so T is 1 - 1. Then we have $T(\alpha f + \beta g) = \mathbf{z}_{\alpha f + \beta g}$ and $(\alpha f + \beta g)(\mathbf{A}) = \sum_{n = 1}^\infty A_n^1(z_{\alpha f + \beta g}^1)_n + \sum_{n = 1}^\infty A_n^2(z_{\alpha f + \beta g}^2)_n$. But
        \begin{eqnarray*}
            (\alpha f)(A) &=& \sum_{n=1}^\infty A_n^1(z_{\alpha f}^1)_n + \sum_{n=1}^\infty A_n^2(z_{\alpha f}^2)_n\\
            (\beta g)(A) &=& \sum_{n=1}^\infty A_n^1(z_{\beta g}^1)_n + \sum_{n=1}^\infty A_n^2(z_{\beta g}^2)_n\\
            f(A) &=& \sum_{n=1}^\infty A_n^1(z_{f}^1)_n + \sum_{n=1}^\infty A_n^2(z_{f}^2)_n\\
            g(A) &=& \sum_{n=1}^\infty A_n^1(z_{g}^1)_n + \sum_{n=1}^\infty A_n^2(z_{g}^2)_n
        \end{eqnarray*}
        and
        \begin{eqnarray*}
            \sum_{n=1}^\infty\:A_n^1(z^1_{\alpha f}) + \sum_{n=1}^\infty\:A_n^2(z^2_{\alpha f}) &=& (\alpha f)(A)\\
            &=& \alpha(\sum_{n=1}^\infty\:A_n^1(z^1_{f}) + \sum_{n=1}^\infty\:A_n^2(z^2_{f}))\\
            &=& \sum_{n=1}^\infty\:A_n^1\alpha(z^1_{f}) + \sum_{n=1}^\infty\:A_n^2\alpha(z^2_{f})
        \end{eqnarray*}
        So we have $(z_{\alpha f})_n = \alpha(z_f)_n$ for all n. A similar argument shows $(z_{\beta g})_n = \beta(z_g)_n$. Thus
        \begin{eqnarray*}
            \sum_{n=1}^\infty A_n^1(z^1_{\alpha f + \beta g})_n + \sum_{n=1}^\infty A_n^2(z^2_{\alpha f + \beta g})_n &=& (\alpha f + \beta g)(A_n^1) + (\alpha f + \beta g)A_n^2\\
            &=& \alpha f(A_n^1) + \alpha f(A_n^2) + \beta f(A_n^1) + \beta f(A_n^2)\\
            &=& \sum_{n=1}^\infty A_n^1\alpha(z_f^1)_n \sum_{n=1}^\infty A_n^2\alpha(z_f^2)_n + \sum_{n=1}^\infty A_n^1\beta(z_g^1)_n + \sum_{n=1}^\infty A_n^2\beta(z_g^2)_n\\
            &=& \sum_{n=1}^\infty x_n^1(\alpha (z_f^1)_n + \beta(z_g^1)_n) + \sum_{n=1}^\infty x_n^2(\alpha (z_f^2)_n + \beta(z_g^2)_n)
        \end{eqnarray*}
        By uniqueness, we must have $(z^i_{\alpha f + \beta g})_n = \alpha (z^i_f)_n + \beta(z^i_g)_n$ which tells us that $T(\alpha f + \beta g) = \alpha T(f) + \beta T(g)$. We conclude T is linear. To see T is onto note if $\mathbf{w} \in K$, we can define f at any $A \in M$ by $f(A) = \sum_{n=1}^\infty \sum_{i=1}^2 A_n^i w_n^i$ which convergers by Holders as $|f(A)| \leq \|w\|_K \|A\|$. This tells us that $\|f\|_{op} \leq \|w\|_K$ and so f is bounded. It is also clealy linear. Thus $f \in M^\prime$. Finally, we have $f(\mathbf{F_n}) = w_n$ and so $(z_f)_n = f(\mathbf{F_n}) = w_n$. This shows $T(f) = \mathbf{z}_f = \mathbf{w}$ and therefore T is onto. Hence, $M^\prime \equiv K$
        \end{solution}
    \end{itemize}
\end{exercise}

\begin{exercise}
    Let $M$ be the space of all $2 \times 1$ vectors whose components are in $\ell^p$.
    \begin{itemize}
        \item Endow M with the norm $\|A\| = \sum_{i = 1}^2\|A_i\|_p$. Prove this is a norm.
        \begin{solution}
            First, lets write out $\|A\|$ so we can show that it satifies the four conditions for a norm.
            \begin{eqnarray*}
                \|A\| &=& \sum_{i=1}^2\|A_i\|_p\\
                &=& \|A_1\|_p + \|A_2\|_p\\
            \end{eqnarray*}
            Where both $\|A_1\|_1$ and $\|A_2\|_1$ are both the $\ell^p$ norm. So 
            \begin{enumerate}
                \item $\|A_1\|_p \geq 0$ and $\|A_2\|_p \geq 0$ so $\|A_1\|_p + \|A_2\|_p \geq 0$
                \item $\|A\| = 0 \iff \|A_1\|_p + \|A_2\|_p = 0 \iff \|A_1\|_p = -\|A_2\|_p$ but becuase $\|A_1\|_p \geq 0$ and $\|A_2\|_p \geq 0$, then $\|A_1\|_p = -\|A_2\|_p \iff \|A_1\|_p = \|A_2\|_p = 0$
                \item $\|\alpha A\| = \|\alpha A_1\|_p + \|\alpha A_2\|_p = |\alpha|(\|A_1\|_p + \|A_2\|_p)$
                \item Finally, know that $\|A + B\| < \|A\| + \|B\|$ by Minkowski's Inequality
            \end{enumerate}
        \end{solution}
        \item define $\mathbf{(F_n)}$ by
        \begin{eqnarray*}
            \mathbf{F_n^1} \; = \;
            \begin{bmatrix}
                \mathbf{E_n}\\
                \mathbf{0}
            \end{bmatrix}
            \; & \;
            \mathbf{F_n^2} \; = \;
            \begin{bmatrix}
                \mathbf{0}\\
                \mathbf{E_n}
            \end{bmatrix}
        \end{eqnarray*}
        where $(E_n)$ is the usual Schauder basis for $l^p$. Prove this is a Schauder Basis for M.
        \begin{solution}
            Because we know both $A_1$ and $A_2$ are in $\ell^p$ and $(E_n)$ is the $\ell^p$ Schauder Basis, then $\exists$ scalars $(\alpha_n)_{n=1}^\infty$ and $(\beta_n)_{n=1}^\infty$ such that, for $m^i \in \mathbf{E_n}$
            \begin{eqnarray*}
                A_1 = lim_{n\rightarrow \infty}\sum_{j=1}^n\:\alpha_j m^1_j\\
                A_2 = lim_{n\rightarrow \infty}\sum_{j=1}^n\:\beta_j m^2_j
            \end{eqnarray*}
            So now, we consider the following element,
            $
            \begin{bmatrix}
            x\\
            y
            \end{bmatrix}
            $
            , of M
            \begin{eqnarray*}
                \begin{bmatrix}
                    x\\
                    y
                \end{bmatrix}
                &=&
                \begin{bmatrix}
                    \sum_{i=1}^\infty\;\alpha_i\mathbf{(E_i)}\\
                    \sum_{i=1}^\infty\;\beta_i\mathbf{(E_i)}
                \end{bmatrix}
                \\
                &=& \sum_{i=1}^\infty\;\alpha_i
                \begin{bmatrix}
                    \mathbf{E_i}\\
                    \mathbf{0}
                \end{bmatrix}
                + \sum_{i=1}^\infty\;\beta_i
                \begin{bmatrix}
                    \mathbf{0}\\
                    \mathbf{E_i}
                \end{bmatrix}
                \\
                &=& \sum_{i=1}^\infty\;\alpha_i\mathbf{(F_i)}^1 + \sum_{i=1}^\infty\;\beta_i\mathbf{(F_i)}^2
            \end{eqnarray*}
            Hence, we can conclude that $\mathbf{(F_n)}$ is a Schauder Basis for M.
        \end{solution}
        \item Go through all the steps of the proof the characterize M'. 
        \begin{solution}
            First, we define $K$ to be the space of all $2 \times 1$ vectors whose components are in $\ell^q$ with p and q being conjugate indicies. We want to prove that, with $f \in M^\prime, \; \exists \; !\mathbf{z_f} \in K$ so that for any $\mathbf{x} \in M$
            \begin{eqnarray*}
                f(A) = \sum_{n=1}^\infty\sum_{i=1}^2\;A_n^i(z^i_f)_n
            \end{eqnarray*}
            where $A = (A_n^i)$ for $i = \{1,2\}$ and $\mathbf{z_f} = ((z_f^i)_n)$ for $i = \{1,2\}$. Further $\|z_f\|_\infty = \|f\|_{op}$. To do so, let $f \in M^\prime$. We have $A =
            \begin{bmatrix}
            A^1\\
            A^2
            \end{bmatrix}
            $
            which has representation $A = \sum_{i=1}^\infty\;\alpha_i\mathbf{(F_i)}^1 + \sum_{i=1}^\infty\;\beta_i\mathbf{(F_i)}^2$ and $\|A\| < \infty$ and $\|\mathbf{F_n}\|_1 = 1 \; \forall \; n$. Let $\mathbf{S_n} = \sum_{i=1}^\infty\;\alpha_i\mathbf{(F_i)}^1 + \sum_{i=1}^\infty\;\beta_i\mathbf{(F_i)}^2$. We use the continuity and linearity of f as discussed earlier to give us $f(A_n) = \sum_{i=1}^\infty\;\alpha_i f(\mathbf{(F_i)}^1) + \sum_{i=1}^\infty\;\beta_i  f(\mathbf{(F_i)}^2)$. Define the sequence $\mathbf{z_f}$ be defined by $(z_f)_n = f(\mathbf{F_n})$, Next, we need to show the sequence $\mathbf{z_f} \in K$ and its norm is $\|\mathbf{z_f}\|_\infty = \|f\|_{op}$.\\
            \\
            To show $\mathbf{z_f} \in K$, note 
            \begin{eqnarray*}
                |(z_f)_n| = |f_(\mathbf{F_n})| \leq \|f\|_{op}\|\mathbf{F_n}\|_1 = \|f\|_{op}
            \end{eqnarray*}
            which implies that all the entries of $\mathbf{z_f}$ are bounded and $\mathbf{z_f} \in K$. To show $\|\mathbf{z_f}\|_K = \|f\|_{op}$, we only have to show the reverse inequality of the already proven $\|\mathbf{z_f}\|_K \leq \|f\|_{op}$. To do this, we consider the following. For each n define the sequence $y_n$ by
                \begin{eqnarray*}
                f(x) &=&
                \left \{
                    \begin{array}{ll}
                        \frac{|(z_f)_i|^q}{(z_f)_i}, & 1 \leq i < n \; \text{and} (z_f)_i \neq 0\\
                        0, & i > n \; \text{or} \; (z_f)_i = b
                    \end{array}
                \right .
                \end{eqnarray*}
                and $y =
                \begin{bmatrix}
                    y_{ni}^1\\
                    y_{ni}^2
                \end{bmatrix}
                $
                . Then, we get
                \begin{eqnarray*}
                    f(y) = \sum_{i=1}^n\sum_{j=1}^2y^j_{ni}f(\mathbf{F_i}^j) = \sum_{i=1}^n\sum_{j=1}^2\frac{|(\gamma_f)_i|^q}{(\gamma_f)_i}f(\mathbf{F_i}^j)
                \end{eqnarray*}
                But since $\gamma_i = f(\mathbf{F_i})$
                \begin{eqnarray*}
                    f(y) = \sum_{i=1}^n\sum_{j=1}^2y^j_{ni}f(\mathbf{F_i}^j) = \sum_{i=1}^n\sum_{j=1}^2|\gamma_i^j|^q
                \end{eqnarray*}
                It is clear then that $f(y) \geq 0$ and
                \begin{eqnarray*}
                    |f(y)| = \sum_{i=1}^n|\gamma_i^j|^q \leq \|f\|_{op}\|y\|_M
                \end{eqnarray*}
                Now,
                \begin{eqnarray*}
                    \|y\|_M &=& (\sum_{i=1}^n\sum_{j=1}^2|\frac{|\gamma_i^j|^q}{\gamma_i^j}|^p)^\frac{1}{p} = (\sum_{i=1}^n\sum_{j=1}^2(|\gamma_i^j|^{q-1})^p)^\frac{1}{p}\\
                    &=& (\sum_{i=1}^n\sum_{j=1}^2|\gamma_i^j|)^\frac{1}{p}
                \end{eqnarray*}
                So we can now say
                \begin{eqnarray*}
                    |f(y)| &=& \sum_{i=1}^n\sum_{j=1}^2|\gamma_i^j| \leq \|f\|_{op}\|y^j\| = \|f\|_{op}(\sum_{i=1}^n\sum_{j=1}^2|\gamma_i^j|^q)^\frac{1}{p}\\
                    &\implies& (\sum_{i=1}^n\sum_{j=1}^2|\gamma_i^j|^q)^{1 - \frac{1}{p} = \frac{1}{q}} \leq \|f\|_{op}
                \end{eqnarray*}
                But because $\gamma_i = (z_f)_i$, so we can conclude $\|z_f\|_q \leq \|f\|_{op}$ telling us $z_f \in K$. We know want to show that $\|z_f\|_q = \|f\|_{op}$. We have already $\|z_f\|_q \leq \|f\|_{op}$ so we only need to show the reverse inequality. So
                \begin{eqnarray*}
                    |f(A)| = |\sum_{i=1}^\infty\sum_{j=1}^2A^jf(\mathbf{F_i})| \leq \|A\| \|z_f\|_q
                \end{eqnarray*}
                By Holders. Taking the sup over all nonzero A, we find
                \begin{eqnarray*}
                    \|f\|_{op} =sup_{A \neq \mathbf{0}}\frac{|f(A)|}{\|A\|} \leq \|z_f\|_q
                \end{eqnarray*}
                which is the reverse inequality we are looking for. Finally, we need to show $z_f$ is unique. To do so, consider if there were to be another $z \in M$ with 
                \begin{eqnarray*}
                    f(A) = \sum_{i=1}^\infty\sum_{j=1}^2A_i^jz_i^j = \sum_{i=1}^\infty\sum_{j=1}^2A_i^j(z_f)^j_i
                \end{eqnarray*}
        \end{solution}
        In particular, we would have $f(1\mathbf{F_i}) = 1z_i$ and so $z_f = z$
        \item Prove this duel space is equivalent to the space $N$ of two dimensional vectors whose components are in $l^q$ using the obvious norm.
        \begin{solution}
            We need to show $M^\prime \equiv K$ where K is the same K as defined above. For $f \in M^\prime$, there is unique $\mathbf{z_f} = ((z_f)_n) =
            \begin{bmatrix}
                (z_f)_n^1\\
                (z_f)_n^2
            \end{bmatrix}
             \in K$
        so $f(A) = \sum_{n=1}^\infty A_n^1(z_f)_n^1 + \sum_{n=1}^\infty A_n^2(z_f)_n^2$ with $A \in K$ and $\|z_f\|_K = \|f\|_{op}$ Define $T: M^\prime \rightarrow K$ by $T(f) = \mathbf{z_f}$. Then there is a linear bijective isometry and $M^\prime \equiv K$. To prove this, we want to show that T is a linear bijective norm isometry. We already know $\|T(f)\|_K = \|\mathbf{z_f}\|_K = \|f\|_{op}$ so $T$ preserves norms. We also know $\mathbf{z_f}$ is unique so T is 1 - 1. Then we have $T(\alpha f + \beta g) = \mathbf{z}_{\alpha f + \beta g}$ and $(\alpha f + \beta g)(\mathbf{A}) = \sum_{n = 1}^\infty A_n^1(z_{\alpha f + \beta g}^1)_n + \sum_{n = 1}^\infty A_n^2(z_{\alpha f + \beta g}^2)_n$. But
        \begin{eqnarray*}
            (\alpha f)(A) &=& \sum_{n=1}^\infty A_n^1(z_{\alpha f}^1)_n + \sum_{n=1}^\infty A_n^2(z_{\alpha f}^2)_n\\
            (\beta g)(A) &=& \sum_{n=1}^\infty A_n^1(z_{\beta g}^1)_n + \sum_{n=1}^\infty A_n^2(z_{\beta g}^2)_n\\
            f(A) &=& \sum_{n=1}^\infty A_n^1(z_{f}^1)_n + \sum_{n=1}^\infty A_n^2(z_{f}^2)_n\\
            g(A) &=& \sum_{n=1}^\infty A_n^1(z_{g}^1)_n + \sum_{n=1}^\infty A_n^2(z_{g}^2)_n
        \end{eqnarray*}
        and
        \begin{eqnarray*}
            \sum_{n=1}^\infty\:A_n^1(z^1_{\alpha f}) + \sum_{n=1}^\infty\:A_n^2(z^2_{\alpha f}) &=& (\alpha f)(A)\\
            &=& \alpha(\sum_{n=1}^\infty\:A_n^1(z^1_{f}) + \sum_{n=1}^\infty\:A_n^2(z^2_{f}))\\
            &=& \sum_{n=1}^\infty\:A_n^1\alpha(z^1_{f}) + \sum_{n=1}^\infty\:A_n^2\alpha(z^2_{f})
        \end{eqnarray*}
        So we have $(z_{\alpha f})_n = \alpha(z_f)_n$ for all n. A similar argument shows $(z_{\beta g})_n = \beta(z_g)_n$. Thus
        \begin{eqnarray*}
            \sum_{n=1}^\infty A_n^1(z^1_{\alpha f + \beta g})_n + \sum_{n=1}^\infty A_n^2(z^2_{\alpha f + \beta g})_n &=& (\alpha f + \beta g)(A_n^1) + (\alpha f + \beta g)A_n^2\\
            &=& \alpha f(A_n^1) + \alpha f(A_n^2) + \beta f(A_n^1) + \beta f(A_n^2)\\
            &=& \sum_{n=1}^\infty A_n^1\alpha(z_f^1)_n \sum_{n=1}^\infty A_n^2\alpha(z_f^2)_n + \sum_{n=1}^\infty A_n^1\beta(z_g^1)_n + \sum_{n=1}^\infty A_n^2\beta(z_g^2)_n\\
            &=& \sum_{n=1}^\infty x_n^1(\alpha (z_f^1)_n + \beta(z_g^1)_n) + \sum_{n=1}^\infty x_n^2(\alpha (z_f^2)_n + \beta(z_g^2)_n)
        \end{eqnarray*}
        By uniqueness, we must have $(z^i_{\alpha f + \beta g})_n = \alpha (z^i_f)_n + \beta(z^i_g)_n$ which tells us that $T(\alpha f + \beta g) = \alpha T(f) + \beta T(g)$. We conclude T is linear. To see T is onto note if $\mathbf{w} \in K$, we can define f at any $A \in M$ by $f(A) = \sum_{n=1}^\infty \sum_{i=1}^2 A_n^i w_n^i$ which convergers by Holders as $|f(A)| \leq \|w\|_K \|A\|$. This tells us that $\|f\|_{op} \leq \|w\|_K$ and so f is bounded. It is also clealy linear. Thus $f \in M^\prime$. Finally, we have $f(\mathbf{F_n}) = w_n$ and so $(z_f)_n = f(\mathbf{F_n}) = w_n$. This shows $T(f) = \mathbf{z}_f = \mathbf{w}$ and therefore T is onto. Hence, $M^\prime \equiv K$
        \end{solution}
    \end{itemize}
\end{exercise}

\begin{exercise}
    If $(X,\|\cdot\|)$ is a normed space, prove $\|\cdot\|$ is a sublinear functional which is positive homogeneous.
    \begin{solution}
        Because $(X,\|\cdot\|)$ is a normed space, it has to satisfy the 4 norm conditions. Specifcially N4: $\rho(u + v) \leq \rho(u) + \rho(v) \; \forall \; u,v \in X$ and N2: $\rho(\alpha v) = |\alpha|\rho(v) \implies \rho(\alpha v) = \alpha\rho(v) \; \forall \; v \in X, \alpha \geq 0$. These two condition being satisfied tells us that $\|\cdot\|$ is a sublinear, positive homogeneous functional on X.
    \end{solution}
\end{exercise}

\end{document}