% SampleProject.tex -- main LaTeX file for sample LaTeX article
%
%\documentclass[12pt]{article}
\documentclass[11pt]{SelfArxOneColBMN}
% add the pgf and tikz support.  This automatically loads
% xcolor so no need to load color
\usepackage{pgf}
\usepackage{tikz}
\usetikzlibrary{matrix}
\usetikzlibrary{calc}
\usepackage{xstring}
\usepackage{pbox}
\usepackage{etoolbox}
\usepackage{marginfix}
\usepackage{xparse}
\setlength{\parskip}{0pt}% fix as marginfix inserts a 1pt ghost parskip
% standard graphics support
\usepackage{graphicx,xcolor}
\usepackage{wrapfig}
%
\definecolor{color1}{RGB}{0,0,90} % Color of the article title and sections
\definecolor{color2}{RGB}{0,20,20} % Color of the boxes behind the abstract and headings
%----------------------------------------------------------------------------------------
%	HYPERLINKS
%----------------------------------------------------------------------------------------
\usepackage[pdftex]{hyperref} % Required for hyperlinks
\hypersetup{hidelinks,
colorlinks,
breaklinks=true,%
urlcolor=color2,%
citecolor=color1,%
linkcolor=color1,%
bookmarksopen=false%
,pdftitle={SampleProject},%
pdfauthor={Peterson}}
%\usepackage[round,numbers]{natbib}
\usepackage[numbers]{natbib}
\usepackage{lmodern}
\usepackage{setspace}
\usepackage{xspace}
%
\usepackage{subfigure}
\newcommand{\goodgap}{
  \hspace{\subfigtopskip}
  \hspace{\subfigbottomskip}}
%
\usepackage{atbegshi}
%
\usepackage[hyper]{listings}
%
% use ams math packages
\usepackage{amsmath,amsthm,amssymb,amsfonts}
\usepackage{mathrsfs}
%
% use new improved Verbatim
\usepackage{fancyvrb}
%
\usepackage[titletoc,title]{appendix}
%
\usepackage{url}
%
% Create length for the baselineskip of text in footnotesize
\newdimen\footnotesizebaselineskip
\newcommand{\test}[1]{%
 \setbox0=\vbox{\footnotesize\strut Test \strut}
 \global\footnotesizebaselineskip=\ht0 \global\advance\footnotesizebaselineskip by \dp0
}
%
\usepackage{listings}

\DeclareGraphicsExtensions{.pdf, .jpg, .tif,.png}

% make sure we don't get orphaned words if at top of page
% or orphans if at bottom of page
\clubpenalty=9999
\widowpenalty=9999
\renewcommand{\textfraction}{0.15}
\renewcommand{\topfraction}{0.85}
\renewcommand{\bottomfraction}{0.85}
\renewcommand{\floatpagefraction}{0.66}
%
\DeclareMathOperator{\sech}{sech}

\newcommand{\mycite}[1]{%
(\citeauthor{#1} \citep{#1} \citeyear{#1})\xspace
}

\newcommand{\mycitetwo}[2]{%
(\citeauthor{#2} \citep[#1]{#2} \citeyear{#2})\xspace
}

\newcommand{\mycitethree}[3]{%
(\citeauthor{#3} \citep[#1][#2]{#3} \citeyear{#3})\xspace
}

\newcommand{\myincludegraphics}[3]{% file name, width, height
\includegraphics[width=#2,height=#3]{#1}
}

\newcommand{\myincludegraphicstwo}[2]{% file name, width, height
\includegraphics[scale=#1]{#2}
}

\newcommand{\mysimplegraphics}[1]{% file name, width, height
\includegraphics{#1}
}

\newcommand{\MB}[1]{
\boldsymbol{#1}
}

\newcommand{\myquotetwo}[1]{%
\small
%\singlespacing
\begin{quotation}
#1
\end{quotation}
\normalsize
%\onehalfspacing  
}

\newcommand{\jimquote}[1]{%
\small
%\singlespacing
\begin{quotation}
#1
\end{quotation}
\normalsize
%\onehalfspacing
}

\newcommand{\myquote}[1]{%
\small
%\singlespacing
\begin{quotation}
#1
\end{quotation}
\normalsize
%\onehalfspacing  
}

%A =
%
%[2 r_1 	     r_1]
%[-2r_1 + r_2  r_2 - r_1]
%
%has eigenvalues r_1 neq r_2.
% #1 = 2 r_1, #2 = r_1, #3 = -2r_1+r_2, #4 = r_2 - r_1
\newcommand{\myrealdiffA}[4]{
\left [
\begin{array}{rr}
#1  & #2\\
#3  & #4
\end{array}
\right ]
}

% args:
% 1, 2 ,3, 4, 5 = caption, label, width, height, file name
%\mysubfigure{}{}{}{}{}
\newcommand{\mysubfigure}[5]{%
\subfigure[#1]{\label{#2}\includegraphics[width=#3,height=#4]{#5}}
}

\newcommand{\mysubfiguretwo}[3]{%
\subfigure[#1]{\label{#2}\includegraphics{#3}}
}

\newcommand{\mysubfigurethree}[4]{%
\subfigure[#1]{\label{#2}\includegraphics[scale=#3]{#4}}
}

\newcommand{\myputimage}[5]{% file name, width, height
\centering
\includegraphics[width=#3,height=#4]{#5}
\caption{#1}
\label{#2}
}

\newcommand{\myputimagetwo}[4]{% caption, label, scale, file name
\centering
\includegraphics[scale=#3]{#4}
\caption{#1}
\label{#2}
}

\newcommand{\myrotateimage}[5]{% file name, width, height
\centering
\includegraphics[scale=#3,angle=#4]{#5}
\caption{#1}
\label{#2}
}

\newcommand{\myurl}[2]{%
\href{#1}{\bf #2}
}

\RecustomVerbatimEnvironment%
{Verbatim}{Verbatim}  
  {fillcolor=\color{black!20}}
  
  \DefineVerbatimEnvironment%
{MyVerbatim}{Verbatim}  
  {frame=single,
   framerule=2pt,
   fillcolor=\color{black!20},
   fontsize=\small}
   
\newcommand{\myfvset}[1]{%  
\fvset{frame=single,
       framerule=2pt,
       fontsize=\small,
       xleftmargin=#1in}}
       
\newcommand{\mylistverbatim}{%
\lstset{%
  fancyvrb, 
  basicstyle=\small,
  breaklines=true}
}  

\newcommand{\mylstinlinebf}[1]{%
{\bf #1}
}

\newcommand{\mylstinline}{%
\lstset{%
  basicstyle=\color{black!80}\bfseries\ttfamily,
  showstringspaces=false,
  showspaces=false,showtabs=false,
  breaklines=true}
\lstinline
}

\newcommand{\mylstinlinetwo}[1]{%
\lstset{%
  basicstyle=\color{black!80}\bfseries\ttfamily,
  showstringspaces=false,
  showspaces=false,showtabs=false,
  breaklines=true}
\lstinline!#1! 
}

%fontfamily=tt
%fontfamily=courier
%fontfamily=helvetica
%frame=topline,
%frame=single,
 %frame=lines,
 %framesep=10pt,
 %fontshape=it,
 %fontseries=b,
 %fontsize=\relsize{-1},
 %fillcolor=\color{black!20},
 %rulecolor=\color{yellow},
 %fillcolor=\color{red}
 %label=\fbox{\Large\emph{The code}}
\DefineVerbatimEnvironment%
{MyListVerbatim}{Verbatim}  
{
fillcolor=\color{black!10},
fontfamily=courier,
frame=single,
%formatcom=\color{white},
framesep=5mm,
labelposition=topline,
fontshape=it,
fontseries=b,
fontsize=\small,
label=\fbox{\large\emph{The code}\normalsize}
} 

%  caption={[#1] \large\bf{#1}}, 
%\centering \framebox[.6\textwidth][c]{\Large\bf{#1}}
\newcommand{\myfancyverbatim}[1]{%
\lstset{%
  fancyvrb=true, 
  %fvcmdparams= fillcolor 1,
  %morefvcmdparams = \textcolor 2,
  frame=shadowbox,framerule=2pt, 
  basicstyle=\small\bfseries,
  backgroundcolor=\color{black!08},
  showstringspaces=false,
  showspaces=false,showtabs=false,
  keywordstyle=\color{black}\bfseries,
  %numbers=left,numberstyle=\tiny,stepnumber=5,numbersep=5pt,
  stringstyle=\ttfamily,
  caption={[\quad #1] \mbox{}\\ \vspace{0.1in} \framebox{\large \bf{#1} \small} },  
  belowcaptionskip=20 pt,  
  label={},
  xleftmargin=17pt,
  framexleftmargin=17pt,
  framexrightmargin=5pt,
  framexbottommargin=4pt,
  nolol=false,
  breaklines=true}
}

\newcommand{\mylistcode}[3]{%
\lstset{%
  language=#1, 
  frame=shadowbox,framerule=2pt, 
  basicstyle=\small\bfseries,
  backgroundcolor=\color{black!16},
  showstringspaces=false,
  showspaces=false,showtabs=false,
  keywordstyle=\color{black!40}\bfseries,
  numbers=left,numberstyle=\tiny,stepnumber=5,numbersep=5pt,
  stringstyle=\ttfamily,
  caption={[\quad#2] \mbox{}\\ \vspace{0.1in} \framebox{\large \bf{#2} \small} },
  belowcaptionskip=20 pt,
  breaklines=true,
  xleftmargin=17pt,
  framexleftmargin=17pt,
  framexrightmargin=5pt,
  framexbottommargin=4pt,  
  label=#3,
  breaklines=true} 
}

  %caption={[#2] #3},
  %caption={[#2]{\mbox{}\\ \vspace{0.1in} \framebox{\large \bf{#3} \small}},
  %caption={[#2] \mbox{}\\ \bf{#3} },

% frame=single,
% caption={[Code Fragment] {\bf Code Fragment} },
% caption={[Code Fragment] \mbox{}\\ \vspace{0.1in} \framebox{\large \bf{Code Fragment} \small} },
\newcommand{\mylistcodequick}[1]{%
\lstset{%
  language=#1, 
  frame=shadowbox,framerule=2pt, 
  basicstyle=\small\bfseries,
  backgroundcolor=\color{black!16},
  showstringspaces=false,
  showspaces=false,showtabs=false,
  keywordstyle=\color{black!40}\bfseries,
  numbers=left,numberstyle=\tiny,stepnumber=5,numbersep=5pt,
  stringstyle=\ttfamily,
  caption={[\quad Code Fragment] \large \bf{Code Fragment} \small},   
  belowcaptionskip=20 pt,  
  label={},
  xleftmargin=17pt,
  framexleftmargin=17pt,
  framexrightmargin=5pt,
  framexbottommargin=4pt,
  breaklines=true} 
}

%  caption={[#2] \mbox{}\\ \vspace{0.1in} \framebox{\large \bf{#2} \small} },
\newcommand{\mylistcodequicktwo}[2]{%
\lstset{%
  language=#1, 
  frame=shadowbox,framerule=2pt, 
  basicstyle=\small\bfseries,
  extendedchars=true,
  backgroundcolor=\color{black!16},
  showstringspaces=false,
  showspaces=false,
  showtabs=false,
  keywordstyle=\color{black!40}\bfseries,
  numbers=left,numberstyle=\tiny,stepnumber=5,numbersep=5pt,
  stringstyle=\ttfamily,
  caption={[\quad#2] \large \bf{#2} \small},
  belowcaptionskip=20 pt,
  label={},
  xleftmargin=17pt,
  framexleftmargin=17pt,
  framexrightmargin=5pt,
  framexbottommargin=4pt,
  breaklines=true} 
}

%  caption={[#2] \mbox{}\\ \vspace{0.1in} \framebox{\large \bf{#2} \small} },
\newcommand{\mylistcodequickthree}[2]{%
\lstset{%
  language=#1, 
  frame=shadowbox,framerule=2pt, 
  basicstyle=\small\bfseries,
  extendedchars=true,
  backgroundcolor=\color{black!16},
  showstringspaces=false,
  showspaces=false,
  showtabs=false,
  keywordstyle=\color{black!40}\bfseries,
  numbers=left,numberstyle=\tiny,stepnumber=5,numbersep=5pt,
  stringstyle=\ttfamily,
  caption={[\quad#2] \large\bf{#2}\small},
  belowcaptionskip=20 pt,
  label={},
  xleftmargin=17pt,
  framexleftmargin=17pt,
  framexrightmargin=5pt,
  framexbottommargin=4pt,
  breaklines=true} 
}

%  frame=single,
\newcommand{\mylistset}[4]{%
\lstset{language=#1,
  basicstyle=\small,
  showstringspaces=false,
  showspaces=false,showtabs=false,
  keywordstyle=\color{black!40}\bfseries,
  numbers=left,numberstyle=\tiny,stepnumber=5,numbersep=5pt,
  stringstyle=\ttfamily,
  caption={[\quad#2]#3},
  label=#4}
}

\newcommand{\mylstinlineset}{%
\lstset{%
  basicstyle=\color{blue}\bfseries\ttfamily,
  showstringspaces=false,
  showspaces=false,showtabs=false,
  breaklines=true}
}

\newcommand{\myframedtext}[1]{%
\centering
\noindent
%\fbox{\parbox[c]{.9\textwidth}{\color{black!40} \small \singlespacing #1\onehalfspacing \normalsize \\}}
\fbox{\parbox[c]{.9\textwidth}{\color{black!40} \small  #1 \normalsize \\}}
}

\newcommand{\myemptybox}[2]{% from , to
\fbox{\begin{minipage}[t][#1in][c]{#2in}\hspace{#2in}\end{minipage}}
}

\newcommand{\myemptyboxtwo}[2]{% from , to
\centering\fbox{
\begin{minipage}{#1in}
\hfill\vspace{#2in}
\end{minipage}
}
}

\newcommand{\boldvector}[1]{
\boldsymbol{#1}
}

\newcommand{\dEdY}[2]{\frac{d E}{d Y_{#1}^{#2}}}
\newcommand{\dEdy}[2]{\frac{d E}{d y_{#1}^{#2}}}
\newcommand{\dEdT}[2]{\frac{\partial E}{\partial T_{{#1} \rightarrow {#2}}}}
\newcommand{\dEdo}[1]{\frac{\partial E}{\partial o^{#1}}}
\newcommand{\dEdg}[1]{\frac{\partial E}{\partial g^{#1}}}
\newcommand{\dYdY}[4]{\frac{\partial Y_{#1}^{#2}}{\partial Y_{#3}^{#4}}}
\newcommand{\dYdy}[4]{\frac{\partial Y_{#1}^{#2}}{\partial y_{#3}^{#4}}}
\newcommand{\dydY}[4]{\frac{\partial y_{#1}^{#2}}{\partial Y_{#3}^{#4}}}
\newcommand{\dydy}[4]{\frac{\partial y_{#1}^{#2}}{\partial y_{#3}^{#4}}}
\newcommand{\dydT}[4]{\frac{\partial y_{#1}^{#2}}{\partial T_{{#3} \rightarrow {#4}}}}
\newcommand{\dYdT}[4]{\frac{\partial Y_{#1}^{#2}}{\partial T_{{#3} \rightarrow {#4}}}}
\newcommand{\dTdT}[4]{\frac{\partial T_{{#1} \rightarrow {#2}}}{\partial T_{{#3} \rightarrow {#4}}}}
\newcommand{\ssum}[2]{\sum_{#1}^{#2}}

\newcommand{\ssty}[1]{\scriptscriptstyle #1}

\newcommand{\myparbox}[2]{%
\parbox{#1}{\color{black!20} #2}
}

\newcommand{\bs}[1]{
\boldsymbol{#1}
}

\newcommand{\parone}[2]{%
\frac{\partial #1 }{ \partial #2 }
}
\newcommand{\partwo}[2]{%
\frac{ \partial^2 {#1} }{ \partial {#2}^2 }
}

\newcommand{\twodvectorvarfun}[2]{
\left [
\begin{array}{r}
{{#1_{\ssty{1}}}(#2)} \\
{{#1_{\ssty{2}}}(#2)}
\end{array}
\right ]
}
\newcommand{\twodvectorvarprimed}[1]{
\left [
\begin{array}{r}
{{#1_{\ssty{1}}}'(t)} \\
{{#1_{\ssty{2}}}'(t)}
\end{array}
\right ]
}

\newcommand{\complex}[2]{#1 \: #2 \: \boldsymbol{i}}
\newcommand{\complexmag}[2]%
{
\sqrt{(#1)^2 \: + \: (#2)^2}
}
\newcommand{\threenorm}[3]%
{
\sqrt{(#1)^2 \: + \: (#2)^2 \: + \: (#3)^2}
}
\newcommand{\norm}[1]{\mid \mid #1 \mid \mid}

\newcommand{\myderiv}[2]{\frac{d #1}{d #2}}
\newcommand{\myderivb}[2]{\frac{d}{d #2} \left ( #1 \right )}
\newcommand{\myrate}[3]%
{#1^\prime(#2) &=& #3 \: #1(#2)
}
\newcommand{\myrateexter}[4]%
{#1^\prime(#2) &=& #3 \: #1(#2) \: + \: #4
}
\newcommand{\myrateic}[3]%
{#1( \: #2 \:) &=& #3 
}

\newcommand{\mytwodsystemeqn}[6]{
#1 \: x    #2 \: y &=& #3\\
#4 \: x    #5 \: y &=& #6\\
}

\newcommand{\mytwodsystem}[8]{
#3 \: #1 \: + \: #4 \: #2 &=& #5\\
#6 \: #1 \: + \: #7 \: #2 &=& #8\\
}  

\newcommand{\mytwodarray}[4]{
\left [
\begin{array}{rr}
#1 & #2\\
#3 & #4
\end{array}
\right ]
}

\newcommand{\mytwoid}{
\left [
\begin{array}{rr}
1 & 0\\
0 & 1
\end{array}
\right ]
}

\newcommand{\myxprime}[2]{
\left [
\begin{array}{r}
#1^\prime(t)\\
#2^\prime(t)
\end{array}
\right ]
}

\newcommand{\myxprimepacked}[2]{
\left [
\begin{array}{r}
#1^\prime\\
#2^\prime
\end{array}
\right ]
}

\newcommand{\myx}[2]{
\left [
\begin{array}{r}
#1(t)\\
#2(t)
\end{array}
\right ]
}

\newcommand{\myxonly}[2]{
\left [
\begin{array}{r}
#1\\
#2
\end{array}
\right ]
}

\newcommand{\myv}[2]{
\left [
\begin{array}{r}
#1\\
#2
\end{array}
\right ]
}

\newcommand{\myxinitial}[2]{
\left [
\begin{array}{r}
#1(0)\\
#2(0)
\end{array}
\right ]
}

\newcommand{\twodboldv}[1]{
\boldsymbol{#1}
}

\newcommand{\mytwodvector}[2]{
\left [
\begin{array}{r}
#1\\
#2
\end{array}
\right ]
}

\newcommand{\mythreedvector}[3]{
\left [
\begin{array}{r}
#1\\
#2\\
#3
\end{array}
\right ]
}

\newcommand{\mytwodsystemvector}[6]{
\left [
\begin{array}{rr}
#1 & #2\\
#4 & #5
\end{array}
\right ]
\:
\left [
\begin{array}{r}
x \\
y 
\end{array}
\right ]
&=&
\left [
\begin{array}{r}
#3\\
#6
\end{array}
\right ]
}

\newcommand{\mythreedarray}[9]{
\left [
\begin{array}{rrr}
#1 & #2 & #3\\
#4 & #5 & #6\\
#7 & #8 & #9
\end{array}
\right ]
}

\newcommand{\myodetwo}[6]{
#1 \: #6^{\prime \prime}(t) \: #2 \: #6^{\prime}(t) \: #3 \: #6(t) &=& 0\\
#6(0)                                           &=& #4\\
#6^{\prime}(0)                                  &=& #5
}

\newcommand{\myodetwoNoIC}[4]{
#1 \: #4^{\prime \prime}(t) \: #2 \: #4^{\prime}(t) \: #3 \: #4(t) &=& 0
}

\newcommand{\myodetwopacked}[5]{
\hspace{-0.3in}& & #1 u^{\prime \prime} #2 u^{\prime} #3 u \: = \: 0\\
\hspace{-0.3in}& & u(0) \: = \: #4, \: \: u^{\prime}(0)    \: = \:  #5
}

\newcommand{\myodetwoforced}[6]{
#1\: u^{\prime \prime}(t) \: #2 \: u^{\prime}(t) \: #3 \: u(t) &=& #6\\
u(0)                                           &=& #4\\
u^{\prime}(0)                                  &=& #5\\
}

\newcommand{\myodesystemtwo}[8]{
#1 \: x^\prime(t) \: #2 \: y^\prime(t) \: #3 \: x(t) \: #4 \: y(t) &=& 0\\
#5 \: x^\prime(t) \: #6 \: y^\prime(t) \: #7 \: x(t) \: #8 \: y(t) &=& 0\\
}

\newcommand{\myodesystemtwoic}[2]{
x(0)                                       &=& #1\\ 
y(0)                                       &=& #2
}

\newcommand{\mypredprey}[4]{
x^\prime(t) &=& #1 \: x(t) \: - \: #2 \: x(t) \: y(t)\\
y^\prime(t) &=& -#3 \: y(t) \: + \: #4 \: x(t) \: y(t)
}

\newcommand{\mypredpreypacked}[4]{
x^\prime &=& #1 \: x - #2 \: x \: y\\
y^\prime &=& -#3 \: y + #4 \: x \: y
}

\newcommand{\mypredpreyself}[6]{
x^\prime(t) &=&  #1 \: x(t) \: - \: #2 \: x(t) \: y(t) \: - \: #3 \: x(t)^2\\
y^\prime(t) &=& -#4 \: y(t) \: + \: #5 \: x(t) \: y(t) \: - \: #6 \: y(t)^2
}

\newcommand{\mypredpreyfish}[5]{
x^\prime(t) &=&  #1 \: x(t) \: - \: #2 \: x(t) \: y(t) \: - \: #5 \: x(t)\\
y^\prime(t) &=& -#3 \: y(t) \: + \: #4 \: x(t) \: y(t) \: - \: #5 \: y(t)
}

\newcommand{\myepidemic}[4]{
S^\prime(t) &=& - #1 \: S(t) \: I(t)\\
I^\prime(t) &=&   #1 \: S(t) \: I(t) \: - \: #2 \: I(t)\\
S(0)        &=&   #3\\
I(0)        &=&   #4\\
}

\newcommand{\bsred}[1]{%
\textcolor{red}{\boldsymbol{#1}}
}

\newcommand{\bsblue}[1]{%
\textcolor{blue}{\boldsymbol{#1}}
}


\newcommand{\myfloor}[1]{%
\lfloor{#1}\rfloor
}

\newcommand{\cubeface}[7]{%
\begin{bmatrix}
\bs{#3}          & \longrightarrow & \bs{#4}\\
\uparrow          &                         &  \uparrow  \\
\bs{#1} & \longrightarrow & \bs{#2}\\
              & \text{ \bfseries #5:} \: \bs{#6} \: \text{\bfseries  #7 } & 
\end{bmatrix}
}

\newcommand{\cubefacetwo}[5]{%
\begin{bmatrix}
\bs{#3}          & \longrightarrow & \bs{#4}\\
\uparrow          &                         &  \uparrow  \\
\bs{#1} & \longrightarrow & \bs{#2}\\
              & \text{ \bfseries #5} & 
\end{bmatrix}
}

\newcommand{\cubefacethree}[9]{%
\begin{bmatrix}
\bs{#3}                  & \overset{#9}{\longrightarrow} & \bs{#4}\\
\uparrow \: #7         &                                             &  \uparrow  \: #8 \\
\bs{#1}                  & \overset{#6}{\longrightarrow} & \bs{#2}\\
                               & \text{ \bfseries #5} & 
\end{bmatrix}
}

\renewcommand{\qedsymbol}{\hfill \blacksquare}
\newcommand{\subqedsymbol}{\hfill \Box}
%\theoremstyle{plain}

\newtheoremstyle{mystyle}% name
  {6pt}%      Space above
  {6pt}%      Space below
  {\itshape}%         Body font
  {}%         Indent amount (empty = no indent, \parindent = para indent)
  {\bfseries}% Thm head font
  {}%        Punctuation after thm head
  { }%     Space after thm head: " " = normal interword space; \newline = linebreak
  {}%         Thm head spec (can be left empty, meaning `normal')
\theoremstyle{mystyle}
 
\newtheorem{axiom}{Axiom}
%\newtheorem{solution}{Solution}[section]
\newtheorem*{solution}{Solution}
\newtheorem{exercise}{Exercise}[section]
\newtheorem{theorem}{Theorem}[section]
\newtheorem{proposition}[theorem]{Proposition}
\newtheorem{prop}[theorem]{Proposition}
\newtheorem{assumption}{Assumption}[section]
\newtheorem{definition}{Definition}[section]
\newtheorem{comment}{Comment}[section]
\newtheorem*{question}{Question}
\newtheorem{program}{Program}[section]
%\newtheorem{myproof}{Proof}
%\newtheorem*{myproof}{Proof}[section]
\newtheorem{myproof}{Proof}[section]
\newtheorem{hint}{Hint}[section]
\newtheorem*{phint}{Hint}
\newtheorem{lemma}[theorem]{Lemma}
\newtheorem{example}{Example}[section]
      
\newenvironment{myassumption}[4]
{
\centering
\begin{assumption}[{\textbf{#1}\nopunct}]%
\index{#2}
\mbox{}\\  \vskip6pt \colorbox{black!15}{\fbox{\parbox{.9\textwidth}{#3}}}
\label{#4}
\end{assumption}
%\renewcommand{\theproposition}{\arabic{chapter}.\arabic{section}.\arabic{assumption}} 
}%
{}

\newenvironment{myproposition}[4]
{
\centering
\begin{proposition}[{\textbf{#1}\nopunct}]%
\index{#2} 
\mbox{}\\  \vskip6pt \colorbox{black!15}{\fbox{\parbox{.9\textwidth}{#3}}}
\label{#4}
\end{proposition}
%\renewcommand{\theproposition}{\arabic{chapter}.\arabic{section}.\arabic{proposition}} 
}%
{}

\newenvironment{mytheorem}[4]
{
\centering
\begin{theorem}[{\textbf{#1}\nopunct}]%
\index{#2} 
\mbox{}\\ \vskip6pt \colorbox{black!15}{\fbox{\parbox{.9\textwidth}{#3}}}
\label{#4}
\end{theorem}
%\renewcommand{\thetheorem}{\arabic{chapter}.\arabic{section}.\arabic{theorem}} 
}%
{}

\newenvironment{mydefinition}[4]
{
\centering
\begin{definition}[{\textbf{#1}\nopunct}]%
\index{#2} 
\mbox{}\\  \vskip6pt \colorbox{black!15}{\fbox{\parbox{.9\textwidth}{#3}}}
\label{#4}
\end{definition}
%\renewcommand{\thedefinitio{n}{\arabic{chapter}.\arabic{section}.\arabic{definition}} 
}%
{}

\newenvironment{myaxiom}[4]
{
\centering
\begin{axiom}[{\textbf{#1}\nopunct}]%
\index{#2} 
\mbox{}\\  \vskip6pt \colorbox{black!15}{\fbox{\parbox{.9\textwidth}{#3}}}
\label{#4}
\end{axiom}
%\renewcommand{\theaxiom}{\arabic{chapter}.\arabic{section}.\arabic{axiom}} 
}%
{}

\newenvironment{mylemma}[4]
{
\centering
\begin{lemma}[{\textbf{#1}\nopunct}]%
\index{#2} 
\mbox{}\\  \vskip6pt \colorbox{black!15}{\fbox{\parbox{.9\textwidth}{#3}}}
\label{#4}
\end{lemma}
%\renewcommand{\thelemma}{\arabic{chapter}.\arabic{section}.\arabic{lemma}} 
}%
{}
   
\newenvironment{reason}[1]
{
 \vskip0.05in
 \begin{myproof}
 \mbox{}\\#1
 $\qedsymbol$
 \end{myproof}  
 \vskip0.05in
}%
{}

\newenvironment{reasontwo}[1]
{
 \vskip+.05in
 \begin{myproof}
 \mbox{}\\#1
 \end{myproof}  
 \vskip+0.05in
}%
{}

\newenvironment{subreason}[1]
{
 \vskip0.05in
 \renewcommand{\themyproof}{}
 \begin{myproof}
 #1
 $\subqedsymbol$
 \end{myproof}
 \vskip0.05in
 \renewcommand{\themyproof}{\thetheorem}
 %\renewcommand{\themyproof}{\arabic{chapter}.\arabic{section}.\arabic{myproof}}   
 %
}%
{}  

\newenvironment{myhint}[1]
{
 \vskip0.05in
 \begin{hint}
 #1
 $\subqedsymbol$ 
 \end{hint}  
 \vskip0.05in
}%
{} 

\newenvironment{myeqn}[3]
{
 \renewcommand{\theequation}{$\boldsymbol{#1}$}
 \begin{eqnarray}
 \label{equation:#2}
 #3 
 \end{eqnarray}
 \renewcommand{\theequation}{\arabic{chapter}.\arabic{eqnarray}}   
}%
{} 


\JournalInfo{MATH 8210: Boop Boop Beep, 1-\pageref{LastPage}, 2020} % Journal information
\Archive{Draft Version \today} % Additional notes (e.g. copyright, DOI, review/research article)

\PaperTitle{MATH 8210: Sample Project}
\Authors{Ian Davis\textsuperscript{1}}
\affiliation{\textsuperscript{1}\textit{School of Mathematical and Statistical Sciences,
Clemson University,Clemson, SC: email ijdavis@g.clemson.edu}}
\affiliation{*\textbf{Corresponding author}: yournamehere@clemson.edu} % Corresponding author

\date{\small{Version 01142020 : Compiled ~\today}}

\Abstract{
Here you write a paragraph or so that is your abstract.
Below that you supply some keywords.
}
\Keywords{
{\bf Key Phrase One} --
{\bf Key Phrase Two} --
{\bf KeyPhrase Three} -- 
{\bf Key Phrase Four} }
\newcommand{\keywordname}{Keywords}
%
\onehalfspacing
\begin{document}

\flushbottom

\addcontentsline{toc}{section}{Title}
\maketitle
\addcontentsline{toc}{section}{Table Of Contents}
\tableofcontents
\thispagestyle{empty}
\newpage
	
\section{Introduction}
\label{sec:introduction}

\noindent
{\bf Here you introduce the problem.  What you are doing for
    the project. }
    
\section{Background}
\label{sec:background}

\noindent
{\bf Here you talk about your problem, introduce whatever
     science and math are necessary to understand what is going on.
     Sort of like a tutorial or a lecture to bright, yet
     untrained in this area, colleagues.  This can be split over several
     sections if required.}
     
\section{YOUR SPECIFIC TASK }
\label{sec:task}

For the title of this section, use something better
than {\bf YOUR SPECIFIC TASK}.  Pick a title relevant to the project.

\noindent
{\bf What your task was and how you accomplished it.
    This can include theoretical development, algorithm development
    and software issues.  This can be spread over several sections
    if necessary.}
    
\section {Results}
\label{sec:results}

\noindent
{\bf Discuss your accomplishments.}

\section{Acknowledgements}

This is where you acknowledge people, grants and so forth
that helped you do the work.  For example,
my recent work is funded by an Army grant and so I would write
this line.\\

\noindent
This research has been supported by support of the Army grant {\bf W911NF-17-1-0455}
from the Army Research Office.

\noindent
Here you put the bibliography.  The \mylstinline{\phantomsection}
is there to make sure it appears in the table of contents in the pdf
we generate.

\phantomsection
\section*{Bibliography}
\bibliographystyle{unsrtnat}
\addcontentsline{toc}{section}{Bibliography}
%\renewcommand{\refname}{References}
\bibliography{HomologyInBio} 

\noindent
We can also add a list of figures and code listings as follows:

\singlespacing 
\myfancyverbatim{List of Figures and Code}
\begin{lstlisting}
\listoffigures                                                                   % sets up a list of figures
\addcontentsline{toc}{part}{List Of Figures}              % adds figures list to pdf hyperrefs
\lstlistoflistings                                                               % sets of a list of code listings
\addcontentsline{toc}{part}{List Of Code Examples} % add code list to pdf hyperrefs
\end{lstlisting}
\onehalfspacing
\lstset{fancyvrb=false}
%
\listoffigures                                 % sets up a list of figures
\addcontentsline{toc}{part}{List Of Figures}   % adds figures list to pdf hyperrefs
%
\lstlistoflistings                                 % sets of a list of code listings
\addcontentsline{toc}{part}{List Of Code Examples} % add code list to pdf hyperrefs

\section{Some ~\LaTeX Pointers}
\label{sec:'latexstuff}

Now here are some pointers on ~\LaTeX.

\subsection{Setting up a subsection}

\noindent
Here is how you do a subsection: note we can move heading around using the
\verb+\hspace+ command.

\singlespacing 
\myfancyverbatim{Setting Up a subsection}
\begin{lstlisting}
\subsection{Your Title Goes Here}
\label{basic}
\end{lstlisting}
\onehalfspacing
\lstset{fancyvrb=false}

\subsection{Referencing A Paper:}

Here is ~\LaTeX~ code for referencing a paper:

\singlespacing 
\myfancyverbatim{Referencing a work you want to cite}
\begin{lstlisting}
A good way to learn immunology is to read \mycite{abbas2010}!
\end{lstlisting}
\onehalfspacing
\lstset{fancyvrb=false}

\noindent
A good way to learn immunology is to read \mycite{abbas2010}!\\

\noindent
In the \mylstinline{HomologyInBio.bib} file you'll see the lines

\singlespacing 
\myfancyverbatim{A book .bib entry}
\begin{lstlisting}
@book{abbas2010,
author= {A. K. Abbas and A. H. Lichtman and S. Pillai},
title={Cellular and Molecular Immunology.},
publisher={Saunders Elsevier}, 
year={2010}}
\end{lstlisting}
\onehalfspacing
\lstset{fancyvrb=false}

\noindent
and to get the references right you do this

\singlespacing 
\myfancyverbatim{Generating the Bibliography}
\begin{lstlisting}
pdflatex SampleProject
bibtex SampleProject
pdflatex SampleProject
\end{lstlisting}
\onehalfspacing
\lstset{fancyvrb=false}

\noindent
either manually in a terminal like I do in Linux or using an integrated development
for ~\LaTeX~ in Windows or MacOS.

\subsection{Some Math Equations:}

Here are some mathematical equations.  Here is the ~\LaTeX~ 
code:

\singlespacing 
\myfancyverbatim{Math Equations}
\begin{lstlisting}
\begin{eqnarray}
\begin{array}{cc}
\min         & \hat{J}(x,u,t)\\
u \in \cal U & \\
             & \\
\end{array}
\label{con11}
\end{eqnarray}

\noindent
where

\begin{eqnarray}
\hat{J}(x,u,t)     &=& dist(y(t_f),\Gamma) 
                  +  \int_t^{t_f} \: {\cal L}( y(s),u(s),s) \: ds 
\label{con11b}
\end{eqnarray}

\noindent
Subject to:
\begin{eqnarray}
\label{con12}
y^{\prime}(s) &=& f(y(s),u(s),s), \: \:  t \leq \: s \: \leq t_f \\ 
\label{con13}
y(t)          &=& x\\
\label{con14}
y(s)        &\in& {\cal Y}(s) \subseteq R^N,  \: \:  t \leq \: s \: \leq t_f\\
\label{con15}
u(s)   &\in & U(s) \subseteq R^M, \: \:  t \leq \: s \: \leq t_f 
\end{eqnarray}
\end{lstlisting}
\onehalfspacing
\lstset{fancyvrb=false}

\noindent
Here is the ~\LaTeX~ output:

\begin{eqnarray}
\begin{array}{cc}
\min         & \hat{J}(x,u,t)\\
u \in \cal U & \\
             & \\
\end{array}
\label{con11}
\end{eqnarray}

\noindent
where

\begin{eqnarray}
\hat{J}(x,u,t)     &=& dist(y(t_f),\Gamma) 
                  +  \int_t^{t_f} \: {\cal L}( y(s),u(s),s) \: ds 
\label{con11b}
\end{eqnarray}

\noindent
Subject to:
\begin{eqnarray}
\label{con12}
y^{\prime}(s) &=& f(y(s),u(s),s), \: \:  t \leq \: s \: \leq t_f \\ 
\label{con13}
y(t)          &=& x\\
\label{con14}
y(s)        &\in& {\cal Y}(s) \subseteq R^N,  \: \:  t \leq \: s \: \leq t_f\\
\label{con15}
u(s)   &\in & U(s) \subseteq R^M, \: \:  t \leq \: s \: \leq t_f 
\end{eqnarray}

subsection{Embedding Mathematics and Referencing Equations:}

Here is ~\LaTeX~ code to handle embedded mathematics
and referencing equations:

\singlespacing 
\myfancyverbatim{Embedding Mathematics}
\begin{lstlisting}
Here $y$ and $u$ are the {\em state vector} and
{\em control vector} of the system, respectively;
$\cal U$ is the space of functions that the control must be chosen from
during the minimization process and (~\ref{con13}) --
(~\ref{con15}) give the initialization and constraint conditions that
the state and control must satisfy. 
The set $\Gamma$ represents a target constraint set
and $dist(y(t_f),\Gamma)$ indicates the distance from
the final state $y(t_f)$ to the constraint set $\Gamma$.
\end{lstlisting}
\onehalfspacing
\lstset{fancyvrb=false}

\noindent
Here is the ~\LaTeX~ output:

\noindent
Here $y$ and $u$ are the {\em state vector} and
{\em control vector} of the system, respectively;
$\cal U$ is the space of functions that the control must be chosen from
during the minimization process and (~\ref{con13}) --
(~\ref{con15}) give the initialization and constraint conditions that
the state and control must satisfy. 
The set $\Gamma$ represents a target constraint set
and $dist(y(t_f),\Gamma)$ indicates the distance from
the final state $y(t_f)$ to the constraint set $\Gamma$.

\subsection{Setting Up a Table:}

Here is a ~\LaTeX~ code for a table and how it is referenced in the text:

\singlespacing 
\myfancyverbatim{Setting up a table}
\begin{lstlisting}
A schematic of the learning algorithm for the network is given 
in Table ~\ref{criticlearn}; note that this training method is
essentially a Gauss-Seidel procedure. 

\begin{table}[hhh]
\caption{Learning Algorithm}
\begin{center}
\begin{tabular}{cl} \\
1 & $W_0 = 0$\\
2 & $j=0$\\
3 & $i=1$\\
4 &  Increment $j$\\
5 &  Set $W_j = W{j-1}$ \\
6 & Input $y_i$, $\theta_i$ \\
7 & Using the previous weights, $W_{j-1}$\\
  & Calculate the desired target\\ 
  & $D_i = \zeta {\phi}(W_{j-1},y_{i+1},\theta_{i+1})
          +{\cal T}_{i}$\\
8 &  Update the CMAC weights using gradient descent\\
  &  to set ${\phi}(W_{j},y_i,\theta_i) = D_i$\\
9 & Increment $i$\\
10 & If $i< \eta$ Go To (6)\\
   &   Else Continue\\
11 & If $W_j \neq W_{j-1}$ Go To (3)\\
   &  Else Continue\\
12 & Convergence\\
\end{tabular}
\end{center}
\label{criticlearn}
\end{table}
\end{lstlisting}
\onehalfspacing
\lstset{fancyvrb=false}

\noindent
The ~\LaTeX~ output:

\noindent
A schematic of the learning algorithm for the network is given 
in Table ~\ref{criticlearn}; note that this training method is
essentially a Gauss-Seidel procedure. 

\begin{table}[hhh]
\caption{Learning Algorithm}
\begin{center}
\begin{tabular}{cl} \\
1 & $W_0 = 0$\\
2 & $j=0$\\
3 & $i=1$\\
4 &  Increment $j$\\
5 &  Set $W_j = W{j-1}$ \\
6 & Input $y_i$, $\theta_i$ \\
7 & Using the previous weights, $W_{j-1}$\\
  & Calculate the desired target\\ 
  & $D_i = \zeta {\phi}(W_{j-1},y_{i+1},\theta_{i+1})
          +{\cal T}_{i}$\\
8 &  Update the CMAC weights using gradient descent\\
  &  to set ${\phi}(W_{j},y_i,\theta_i) = D_i$\\
9 & Increment $i$\\
10 & If $i< \eta$ Go To (6)\\
   &   Else Continue\\
11 & If $W_j \neq W_{j-1}$ Go To (3)\\
   &  Else Continue\\
12 & Convergence\\
\end{tabular}
\end{center}
\label{criticlearn}
\end{table}

\subsection{Mathematics in a Enumerated Environment:}

Here, ~\LaTeX~ code to put math stuff in an enumerated envoronment:

\singlespacing 
\myfancyverbatim{Math in an enumerate list}
\begin{lstlisting}
The CMAC architectures are used to 
construct predictions for each state variable using the temporal difference
approach already outlined. 
We used the following $6$ temporal differences:

\begin{enumerate}
\item $\sin(\alpha_c(t_i))-\sin(\alpha_c(t_{i-1}))$
\item $\cos(\alpha_c(t_i))-\cos(\alpha_c(t_{i-1}))$
\item $\sin(\alpha_t(t_i))-\sin(\alpha_t(t_{i-1}))$
\item $\sin(\alpha_t(t_i))-\sin(\alpha_t(t_{i-1}))$
\item $x_c(t_i)-x_c(t_{i-1})/XSCALE$
\item $y_c(t_i)-y_c(t_{i-1})/YSCALE$
\end{enumerate}
\end{lstlisting}
\onehalfspacing
\lstset{fancyvrb=false}

\noindent
~\LaTeX~ output:

\noindent
The CMAC architectures are used to 
construct predictions for each state variable using the temporal difference
approach already outlined in earlier sections. 
We used the following $6$ temporal differences:

\begin{enumerate}
\item $\sin(\alpha_c(t_i))-\sin(\alpha_c(t_{i-1}))$
\item $\cos(\alpha_c(t_i))-\cos(\alpha_c(t_{i-1}))$
\item $\sin(\alpha_t(t_i))-\sin(\alpha_t(t_{i-1}))$
\item $\sin(\alpha_t(t_i))-\sin(\alpha_t(t_{i-1}))$
\item $x_c(t_i)-x_c(t_{i-1})/XSCALE$
\item $y_c(t_i)-y_c(t_{i-1})/YSCALE$
\end{enumerate}

\subsection{Another Table and How to Reference It:}

Here is ~\LaTeX~ code for another table with how to reference it:

\singlespacing 
\myfancyverbatim{Table Referencing}
\begin{lstlisting}
We will attempt to discuss generic software design issues, but
we will use the truck backer-upper control problem as our
framework so we can refer to concrete details when we need to 
do so.  First, we look at a basic sketch of program flow as presented in
Table ~\ref{neurocontrolflow}.  
 
\begin{table}[hhh]
\caption{Neurocontrol Flow}
\begin{center}
\begin{tabular}{ll} \\
1 & Set simulation parameters  \\
2 & Set up CMAC architectures  \\
3 & Choose random truck start and control as state \\
4 & i = 0 \\
5 & Save state as sample i \\
6 & Wait DATA\_FREQ time steps and store state\\
7 &    \hspace{.2in}If state lies in a new CMAC sensor, \\
  &   \hspace{.4in}  increment i\\
  &    \hspace{.4in} save state as sample i of training set\\
  &   \hspace{.4in}  Compute next predicted optimal\\
  &   \hspace{.6in} control using CMAC model\\
  &   \hspace{.4in} If i $<$ num\_samples GoTo 6 \\
  &    \hspace{.2in}Else, If i $<$ num\_samples GoTo 6 \\
  & Now have eta = num\_samples training samples  \\
8 & For each output component d  \\
  &    \hspace{.2in}For each input i, compute desired \\
  &   \hspace{.4in} D = FUTURE\_WEIGHT*CMAC OUTPUT for input i \\
  &   \hspace{.4in} + T for input i\\
9& Train CMAC to match I/O set \\
10& Compute next predicted optimal control \\
  & \hspace{.2in} using CMAC model  \\
11& Save sample eta as state and GoTo 4
\end{tabular}
\end{center}
\label{neurocontrolflow}
\end{table}
\end{lstlisting}
\onehalfspacing
\lstset{fancyvrb=false}

\noindent
\LaTeX~ output:

\noindent
We will attempt to discuss generic software design issues, but
we will use the truck backer-upper control problem as our
framework so we can refer to concrete details when we need to 
do so.  First, we look at a basic sketch of program flow as presented in
Table ~\ref{neurocontrolflow}.  
 
\begin{table}[hhh]
\caption{Neurocontrol Flow}
\begin{center}
\begin{tabular}{ll} \\
1 & Set simulation parameters  \\
2 & Set up CMAC architectures  \\
3 & Choose random truck start and control as state \\
4 & i = 0 \\
5 & Save state as sample i \\
6 & Wait DATA\_FREQ time steps and store state\\
7 &    \hspace{.2in}If state lies in a new CMAC sensor, \\
  &   \hspace{.4in}  increment i\\
  &    \hspace{.4in} save state as sample i of training set\\
  &   \hspace{.4in}  Compute next predicted optimal\\
  &   \hspace{.6in} control using CMAC model\\
  &   \hspace{.4in} If i $<$ num\_samples GoTo 6 \\
  &    \hspace{.2in}Else, If i $<$ num\_samples GoTo 6 \\
  & Now have eta = num\_samples training samples  \\
8 & For each output component d  \\
  &    \hspace{.2in}For each input i, compute desired \\
  &   \hspace{.4in} D = FUTURE\_WEIGHT*CMAC OUTPUT for input i \\
  &   \hspace{.4in} + T for input i\\
9& Train CMAC to match I/O set \\
10& Compute next predicted optimal control \\
  & \hspace{.2in} using CMAC model  \\
11& Save sample eta as state and GoTo 4
\end{tabular}
\end{center}
\label{neurocontrolflow}
\end{table}

\subsection{Inserting C Source Into The Document:}

Here is simple ~\LaTeX~ code I have used to
insert C code into a paper:

\singlespacing 
\begin{Verbatim}
Each output component of the CMAC structure has the identical input
architecture specified by the {\em "C" structures} {\bf cmacinput}
and {\bf cmac}.  For expositional cleanness, these structures have been
stripped down to bare essentials: in a working simulation, there will
of course be additional diagnostic and useful items included, such as
the name and location of configuration files where all the
important setup information is stored.  We allocate space for
one cmac structure and $M$ blocks of working memory.  We will
use a fixed offset strategy of offset $O = \frac{1}{L}$, where
$L$ is the number of levels.

\mylistcode{C}{The cmacinput structure}{code:cmacone}
\begin{lstlisting}
struct cmacinput{ 
  int training_set_size;   //the number of training samples 
  int in_dimensions;       //the dimension of the input space 
  int out_dimensions;      //the dimension of the output space 
  float learning_rate;     //the learning rate used in training 
  float *min_x;            //lower bound for input components  
  float *max_x;            //upper bound for input components 
  }; 
struct cmac{ 
  int levels;              //the number of levels  
  int hash_value;          //hash value 
  float offset_value;      //offset = 1/levels 
  float *width_value;      //sensor width 
  float **offset;          //store size of all offsets 
  float **receptive_field_width; 
                           //store size of all widths 
  fhash hash;              //hash function 
  }; 
struct working_memory{      
  float **mem;             //working memory levels by hash_size 
  };   
\end{lstlisting}
\end{Verbatim}
\onehalfspacing

\noindent
\LaTeX~ output:

Each output component of the CMAC structure has the identical input
architecture specified by the {\em "C" structures} {\bf cmacinput}
and {\bf cmac}.  For expositional cleanness, these structures have been
stripped down to bare essentials: in a working simulation, there will
of course be additional diagnostic and useful items included, such as
the name and location of configuration files where all the
important setup information is stored.  We allocate space for
one cmac structure and $M$ blocks of working memory.  We will
use a fixed offset strategy of offset $O = \frac{1}{L}$, where
$L$ is the number of levels.

\mylistcode{C}{The cmacinput structure}{codeL:cmacone}
\singlespacing
\begin{lstlisting}
struct cmacinput{ 
  int training_set_size;   //the number of training samples 
  int in_dimensions;       //the dimension of the input space 
  int out_dimensions;      //the dimension of the output space 
  float learning_rate;     //the learning rate used in training 
  float *min_x;            //lower bound for input components  
  float *max_x;            //upper bound for input components 
  }; 
struct cmac{ 
  int levels;              //the number of levels  
  int hash_value;          //hash value 
  float offset_value;      //offset = 1/levels 
  float *width_value;      //sensor width 
  float **offset;          //store size of all offsets 
  float **receptive_field_width; 
                           //store size of all widths 
  fhash hash;              //hash function 
  }; 
struct working_memory{      
  float **mem;             //working memory levels by hash_size 
  };  
\end{lstlisting}
\onehalfspacing

\subsection{Using subfigures}

\noindent
Here's how you use the \mylstinline{subfigure} environment.

\singlespacing
\myfancyverbatim{Subfigures}
\begin{lstlisting}
Consider the average sodium and potassium time constant values given in Figure \ref{figure:averagenat} and
Figure \ref{figure:averagek1t} respectively.

\begin{figure}[h]
\begin{center}
\mysubfigurethree{Average Sodium Time constant}{fig:averagenat}%
                            {.4}{BookFigures/average_t_m_NA.png}
\goodgap
\mysubfigurethree{Average Potassium Time constant}{fig:averagek1t}%
                            {.4}{BookFigures/average_t_m_K1.png}
\caption{Time Constants}
\label{fig:iontimeconstants}
\end{center}
\end{figure}
\end{lstlisting}
\onehalfspacing
\lstset{fancyvrb=false}

\noindent
\LaTeX~ output:

Consider the average sodium and potassium time constant values given in Figure \ref{fig:averagenat} and
Figure \ref{fig:averagek1t} respectively.

\begin{figure}[h]
\begin{center}
\mysubfigurethree{Average Sodium Time constant}{fig:averagenat}%
                            {.4}{BookFigures/average_t_m_NA.png}
\goodgap
\mysubfigurethree{Average Potassium Time constant}{fig:averagek1t}%
                            {.4}{BookFigures/average_t_m_K1.png}
\caption{Time Constants}
\label{fig:iontimeconstants}
\end{center}
\end{figure}

\subsection{Including a .png file}

\noindent
Here's how you embed a .png file: note the full path name must be entered as one line--do not 
carriage return because the line is too long.  That will screw things
up!

\singlespacing
\myfancyverbatim{Including a .png file}
\begin{lstlisting}
Consider the outline for cognitive processing given in Figure \ref{fig:metaone}.

\begin{figure}
\myputimagetwo{A Model of Cognition}{fig:metaone}{.5}{BookFigures/meta.png}
\end{figure}
\end{lstlisting}
\onehalfspacing
\lstset{fancyvrb=false}

\noindent
\LaTeX~ output:

Consider the outline for cognitive processing given in Figure \ref{figure:metaone}.

\begin{figure}
\myputimagetwo{A Model of Cognition}{figure:metaone}{.5}{BookFigures/meta.png}
\end{figure}

\subsection{A Sample with citations, quotes etc}

Here is another nice sample.

\myfancyverbatim{Sample with references}
\begin{lstlisting}
Here we use \verb+\mycite+ as defined at the top of this document
to call \verb+natbib+ to setup the references.

In \mycite{peterson2015a} and \mycite{peterson2015b}, we explore a micro
level simulation model of a single host's response to varying levels of
West Nile Virus (WNV) infection.  In that infection, there is a substantial
self damage component and in those papers, we show that
this is probably due to the way that the virus infects two
cell populations differently.  This difference, which involves
an larger upregulation of MHC-1 sites on the surface
of non-dividing infected cells over dividing infected cells,
is critical in establishing a self damage or collateral damage response.
In \mycite{peterson2015c}, we develop a macro level model
of the nonlinear interactions between two critical \emph{signaling}
agents that mediate the interaction between these two sets of cell populations.
In the case of WNV infection, the two signals are the MHC-1 upregulation level
of the cell and the free WNV antigen level.  This macro model allowed
us to predict a host's health response to varying levels of initial virus dose.
Hence, we could begin to understand the oscillations in collateral damage
and host health that lead to the survival data we see in WNV infections.
The more general musings of \mycite{peterson2015c} were then 
extended to the setting of auto immune interactions in general
in \mycite{peterson2015d}.  In this work, we explore how viral infections
in the Central Nervous System (CNS) can generate an auto immune
response.

For background, we will follow the presentation of \mycite{kasahara2010}
in what follows.
The adaptive immune system (AIS) arose about $450$ million years ago
from a common ancestor of jawed vertebrates.  The emergence of this system
is probably two rounds of whole-genome duplication.  This event gave the jawed
vertebrate ancestor many paralogous genes which play essential roles
in T cell and B cell immunity.  These genes are known as \emph{ohnologs}
and are very important as they encode key components of the antigen
presentation machinery and also the signal transduction pathway for
lymphocyte activation.  When and how T cell immunity arose is important
because it sheds light on how T cell problems with self versus non self may have arisen
and continue to be an issue in modern vertebrates.  We know that
T cell receptors (TCRs) and major histocompatibility complex (MHC) molecules
are present in all classes of jawed vertebrates but not there in jawless vertebrates.

Now we quote from the source \mycite{kasahara2010} using \verb+\jimquote+

\jimquote{
Jawless vertebrates are equipped with rearranging antigen receptors that are
clonally expressed on lymphocyte-like-cells.  However, their receptors,...,
generate diversity through somatic recombination of leucine-rich repeat
molecules, and, ... are hence structurally unrelated to TCRs and B cell receptors
(BCRs)}
\end{lstlisting}
\onehalfspacing
\lstset{fancyvrb=false}

\noindent
Here we use \verb+\mycite+ as defined at the top of this document
to call \verb+natbib+ to setup the references.

In \mycite{peterson2015a} and \mycite{peterson2016c}, we explore a micro
level simulation model of a single host's response to varying levels of
West Nile Virus (WNV) infection.  In that infection, there is a substantial
self damage component and in those papers, we show that
this is probably due to the way that the virus infects two
cell populations differently.  This difference, which involves
an larger upregulation of MHC-1 sites on the surface
of non-dividing infected cells over dividing infected cells,
is critical in establishing a self damage or collateral damage response.
In \mycite{peterson2017a}, we develop a macro level model
of the nonlinear interactions between two critical \emph{signaling}
agents that mediate the interaction between these two sets of cell populations.
In the case of WNV infection, the two signals are the MHC-1 upregulation level
of the cell and the free WNV antigen level.  This macro model allowed
us to predict a host's health response to varying levels of initial virus dose.
Hence, we could begin to understand the oscillations in collateral damage
and host health that lead to the survival data we see in WNV infections.
The more general musings of \mycite{peterson2017a} were then 
extended to the setting of auto immune interactions in general
in \mycite{peterson2017b}.  In this work, we explore how viral infections
in the Central Nervous System (CNS) can generate an auto immune
response.

For background, we will follow the presentation of \mycite{kasahara2010}
in what follows.
The adaptive immune system (AIS) arose about $450$ million years ago
from a common ancestor of jawed vertebrates.  The emergence of this system
is probably two rounds of whole-genome duplication.  This event gave the jawed
vertebrate ancestor many paralogous genes which play essential roles
in T cell and B cell immunity.  These genes are known as \emph{ohnologs}
and are very important as they encode key components of the antigen
presentation machinery and also the signal transduction pathway for
lymphocyte activation.  When and how T cell immunity arose is important
because it sheds light on how T cell problems with self versus non self may have arisen
and continue to be an issue in modern vertebrates.  We know that
T cell receptors (TCRs) and major histocompatibility complex (MHC) molecules
are present in all classes of jawed vertebrates but not there in jawless vertebrates.

Now we quote from the source \mycite{kasahara2010} using \verb+\jimquote+

\jimquote{
Jawless vertebrates are equipped with rearranging antigen receptors that are
clonally expressed on lymphocyte-like-cells.  However, their receptors,...,
generate diversity through somatic recombination of leucine-rich repeat
molecules, and, ... are hence structurally unrelated to TCRs and B cell receptors
(BCRs)}

		
\end{document}
