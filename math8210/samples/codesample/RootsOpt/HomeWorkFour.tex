% SampleProject.tex -- main LaTeX file for sample LaTeX article
%
%\documentclass[12pt]{article}
\documentclass[11pt]{SelfArxOneColBMN}
% add the pgf and tikz support.  This automatically loads
% xcolor so no need to load color
\usepackage{pgf}
\usepackage{tikz}
\usetikzlibrary{matrix}
\usetikzlibrary{calc}
\usepackage{xstring}
\usepackage{pbox}
\usepackage{etoolbox}
\usepackage{marginfix}
\usepackage{xparse}
\setlength{\parskip}{0pt}% fix as marginfix inserts a 1pt ghost parskip
% standard graphics support
\usepackage{graphicx,xcolor}
\usepackage{wrapfig}
%
\definecolor{color1}{RGB}{0,0,90} % Color of the article title and sections
\definecolor{color2}{RGB}{0,20,20} % Color of the boxes behind the abstract and headings
%----------------------------------------------------------------------------------------
%	HYPERLINKS
%----------------------------------------------------------------------------------------
\usepackage[pdftex]{hyperref} % Required for hyperlinks
\hypersetup{hidelinks,
colorlinks,
breaklinks=true,%
urlcolor=color2,%
citecolor=color1,%
linkcolor=color1,%
bookmarksopen=false%
,pdftitle={SampleProject},%
pdfauthor={Peterson}}
%\usepackage[round,numbers]{natbib}
\usepackage[numbers]{natbib}
\usepackage{lmodern}
\usepackage{setspace}
\usepackage{xspace}
%
\usepackage{subfigure}
\newcommand{\goodgap}{
  \hspace{\subfigtopskip}
  \hspace{\subfigbottomskip}}
%
\usepackage{atbegshi}
%
\usepackage[hyper]{listings}
%
% use ams math packages
\usepackage{amsmath,amsthm,amssymb,amsfonts}
\usepackage{mathrsfs}
%
% use new improved Verbatim
\usepackage{fancyvrb}
%
\usepackage[titletoc,title]{appendix}
%
\usepackage{url}
%
% Create length for the baselineskip of text in footnotesize
\newdimen\footnotesizebaselineskip
\newcommand{\test}[1]{%
 \setbox0=\vbox{\footnotesize\strut Test \strut}
 \global\footnotesizebaselineskip=\ht0 \global\advance\footnotesizebaselineskip by \dp0
}
%
\usepackage{listings}

\DeclareGraphicsExtensions{.pdf, .jpg, .tif,.png}

% make sure we don't get orphaned words if at top of page
% or orphans if at bottom of page
\clubpenalty=9999
\widowpenalty=9999
\renewcommand{\textfraction}{0.15}
\renewcommand{\topfraction}{0.85}
\renewcommand{\bottomfraction}{0.85}
\renewcommand{\floatpagefraction}{0.66}
%
\DeclareMathOperator{\sech}{sech}

\newcommand{\mycite}[1]{%
(\citeauthor{#1} \citep{#1} \citeyear{#1})\xspace
}

\newcommand{\mycitetwo}[2]{%
(\citeauthor{#2} \citep[#1]{#2} \citeyear{#2})\xspace
}

\newcommand{\mycitethree}[3]{%
(\citeauthor{#3} \citep[#1][#2]{#3} \citeyear{#3})\xspace
}

\newcommand{\myincludegraphics}[3]{% file name, width, height
\includegraphics[width=#2,height=#3]{#1}
}

\newcommand{\myincludegraphicstwo}[2]{% file name, width, height
\includegraphics[scale=#1]{#2}
}

\newcommand{\mysimplegraphics}[1]{% file name, width, height
\includegraphics{#1}
}

\newcommand{\MB}[1]{
\boldsymbol{#1}
}

\newcommand{\myquotetwo}[1]{%
\small
%\singlespacing
\begin{quotation}
#1
\end{quotation}
\normalsize
%\onehalfspacing  
}

\newcommand{\jimquote}[1]{%
\small
%\singlespacing
\begin{quotation}
#1
\end{quotation}
\normalsize
%\onehalfspacing
}

\newcommand{\myquote}[1]{%
\small
%\singlespacing
\begin{quotation}
#1
\end{quotation}
\normalsize
%\onehalfspacing  
}

%A =
%
%[2 r_1 	     r_1]
%[-2r_1 + r_2  r_2 - r_1]
%
%has eigenvalues r_1 neq r_2.
% #1 = 2 r_1, #2 = r_1, #3 = -2r_1+r_2, #4 = r_2 - r_1
\newcommand{\myrealdiffA}[4]{
\left [
\begin{array}{rr}
#1  & #2\\
#3  & #4
\end{array}
\right ]
}

% args:
% 1, 2 ,3, 4, 5 = caption, label, width, height, file name
%\mysubfigure{}{}{}{}{}
\newcommand{\mysubfigure}[5]{%
\subfigure[#1]{\label{#2}\includegraphics[width=#3,height=#4]{#5}}
}

\newcommand{\mysubfiguretwo}[3]{%
\subfigure[#1]{\label{#2}\includegraphics{#3}}
}

\newcommand{\mysubfigurethree}[4]{%
\subfigure[#1]{\label{#2}\includegraphics[scale=#3]{#4}}
}

\newcommand{\myputimage}[5]{% file name, width, height
\centering
\includegraphics[width=#3,height=#4]{#5}
\caption{#1}
\label{#2}
}

\newcommand{\myputimagetwo}[4]{% caption, label, scale, file name
\centering
\includegraphics[scale=#3]{#4}
\caption{#1}
\label{#2}
}

\newcommand{\myrotateimage}[5]{% file name, width, height
\centering
\includegraphics[scale=#3,angle=#4]{#5}
\caption{#1}
\label{#2}
}

\newcommand{\myurl}[2]{%
\href{#1}{\bf #2}
}

\RecustomVerbatimEnvironment%
{Verbatim}{Verbatim}  
  {fillcolor=\color{black!20}}
  
  \DefineVerbatimEnvironment%
{MyVerbatim}{Verbatim}  
  {frame=single,
   framerule=2pt,
   fillcolor=\color{black!20},
   fontsize=\small}
   
\newcommand{\myfvset}[1]{%  
\fvset{frame=single,
       framerule=2pt,
       fontsize=\small,
       xleftmargin=#1in}}
       
\newcommand{\mylistverbatim}{%
\lstset{%
  fancyvrb, 
  basicstyle=\small,
  breaklines=true}
}  

\newcommand{\mylstinlinebf}[1]{%
{\bf #1}
}

\newcommand{\mylstinline}{%
\lstset{%
  basicstyle=\color{black!80}\bfseries\ttfamily,
  showstringspaces=false,
  showspaces=false,showtabs=false,
  breaklines=true}
\lstinline
}

\newcommand{\mylstinlinetwo}[1]{%
\lstset{%
  basicstyle=\color{black!80}\bfseries\ttfamily,
  showstringspaces=false,
  showspaces=false,showtabs=false,
  breaklines=true}
\lstinline!#1! 
}

%fontfamily=tt
%fontfamily=courier
%fontfamily=helvetica
%frame=topline,
%frame=single,
 %frame=lines,
 %framesep=10pt,
 %fontshape=it,
 %fontseries=b,
 %fontsize=\relsize{-1},
 %fillcolor=\color{black!20},
 %rulecolor=\color{yellow},
 %fillcolor=\color{red}
 %label=\fbox{\Large\emph{The code}}
\DefineVerbatimEnvironment%
{MyListVerbatim}{Verbatim}  
{
fillcolor=\color{black!10},
fontfamily=courier,
frame=single,
%formatcom=\color{white},
framesep=5mm,
labelposition=topline,
fontshape=it,
fontseries=b,
fontsize=\small,
label=\fbox{\large\emph{The code}\normalsize}
} 

%  caption={[#1] \large\bf{#1}}, 
%\centering \framebox[.6\textwidth][c]{\Large\bf{#1}}
\newcommand{\myfancyverbatim}[1]{%
\lstset{%
  fancyvrb=true, 
  %fvcmdparams= fillcolor 1,
  %morefvcmdparams = \textcolor 2,
  frame=shadowbox,framerule=2pt, 
  basicstyle=\small\bfseries,
  backgroundcolor=\color{black!08},
  showstringspaces=false,
  showspaces=false,showtabs=false,
  keywordstyle=\color{black}\bfseries,
  %numbers=left,numberstyle=\tiny,stepnumber=5,numbersep=5pt,
  stringstyle=\ttfamily,
  caption={[\quad #1] \mbox{}\\ \vspace{0.1in} \framebox{\large \bf{#1} \small} },  
  belowcaptionskip=20 pt,  
  label={},
  xleftmargin=17pt,
  framexleftmargin=17pt,
  framexrightmargin=5pt,
  framexbottommargin=4pt,
  nolol=false,
  breaklines=true}
}

\newcommand{\mylistcode}[3]{%
\lstset{%
  language=#1, 
  frame=shadowbox,framerule=2pt, 
  basicstyle=\small\bfseries,
  backgroundcolor=\color{black!16},
  showstringspaces=false,
  showspaces=false,showtabs=false,
  keywordstyle=\color{black!40}\bfseries,
  numbers=left,numberstyle=\tiny,stepnumber=5,numbersep=5pt,
  stringstyle=\ttfamily,
  caption={[\quad#2] \mbox{}\\ \vspace{0.1in} \framebox{\large \bf{#2} \small} },
  belowcaptionskip=20 pt,
  breaklines=true,
  xleftmargin=17pt,
  framexleftmargin=17pt,
  framexrightmargin=5pt,
  framexbottommargin=4pt,  
  label=#3,
  breaklines=true} 
}

  %caption={[#2] #3},
  %caption={[#2]{\mbox{}\\ \vspace{0.1in} \framebox{\large \bf{#3} \small}},
  %caption={[#2] \mbox{}\\ \bf{#3} },

% frame=single,
% caption={[Code Fragment] {\bf Code Fragment} },
% caption={[Code Fragment] \mbox{}\\ \vspace{0.1in} \framebox{\large \bf{Code Fragment} \small} },
\newcommand{\mylistcodequick}[1]{%
\lstset{%
  language=#1, 
  frame=shadowbox,framerule=2pt, 
  basicstyle=\small\bfseries,
  backgroundcolor=\color{black!16},
  showstringspaces=false,
  showspaces=false,showtabs=false,
  keywordstyle=\color{black!40}\bfseries,
  numbers=left,numberstyle=\tiny,stepnumber=5,numbersep=5pt,
  stringstyle=\ttfamily,
  caption={[\quad Code Fragment] \large \bf{Code Fragment} \small},   
  belowcaptionskip=20 pt,  
  label={},
  xleftmargin=17pt,
  framexleftmargin=17pt,
  framexrightmargin=5pt,
  framexbottommargin=4pt,
  breaklines=true} 
}

%  caption={[#2] \mbox{}\\ \vspace{0.1in} \framebox{\large \bf{#2} \small} },
\newcommand{\mylistcodequicktwo}[2]{%
\lstset{%
  language=#1, 
  frame=shadowbox,framerule=2pt, 
  basicstyle=\small\bfseries,
  extendedchars=true,
  backgroundcolor=\color{black!16},
  showstringspaces=false,
  showspaces=false,
  showtabs=false,
  keywordstyle=\color{black!40}\bfseries,
  numbers=left,numberstyle=\tiny,stepnumber=5,numbersep=5pt,
  stringstyle=\ttfamily,
  caption={[\quad#2] \large \bf{#2} \small},
  belowcaptionskip=20 pt,
  label={},
  xleftmargin=17pt,
  framexleftmargin=17pt,
  framexrightmargin=5pt,
  framexbottommargin=4pt,
  breaklines=true} 
}

%  caption={[#2] \mbox{}\\ \vspace{0.1in} \framebox{\large \bf{#2} \small} },
\newcommand{\mylistcodequickthree}[2]{%
\lstset{%
  language=#1, 
  frame=shadowbox,framerule=2pt, 
  basicstyle=\small\bfseries,
  extendedchars=true,
  backgroundcolor=\color{black!16},
  showstringspaces=false,
  showspaces=false,
  showtabs=false,
  keywordstyle=\color{black!40}\bfseries,
  numbers=left,numberstyle=\tiny,stepnumber=5,numbersep=5pt,
  stringstyle=\ttfamily,
  caption={[\quad#2] \large\bf{#2}\small},
  belowcaptionskip=20 pt,
  label={},
  xleftmargin=17pt,
  framexleftmargin=17pt,
  framexrightmargin=5pt,
  framexbottommargin=4pt,
  breaklines=true} 
}

%  frame=single,
\newcommand{\mylistset}[4]{%
\lstset{language=#1,
  basicstyle=\small,
  showstringspaces=false,
  showspaces=false,showtabs=false,
  keywordstyle=\color{black!40}\bfseries,
  numbers=left,numberstyle=\tiny,stepnumber=5,numbersep=5pt,
  stringstyle=\ttfamily,
  caption={[\quad#2]#3},
  label=#4}
}

\newcommand{\mylstinlineset}{%
\lstset{%
  basicstyle=\color{blue}\bfseries\ttfamily,
  showstringspaces=false,
  showspaces=false,showtabs=false,
  breaklines=true}
}

\newcommand{\myframedtext}[1]{%
\centering
\noindent
%\fbox{\parbox[c]{.9\textwidth}{\color{black!40} \small \singlespacing #1\onehalfspacing \normalsize \\}}
\fbox{\parbox[c]{.9\textwidth}{\color{black!40} \small  #1 \normalsize \\}}
}

\newcommand{\myemptybox}[2]{% from , to
\fbox{\begin{minipage}[t][#1in][c]{#2in}\hspace{#2in}\end{minipage}}
}

\newcommand{\myemptyboxtwo}[2]{% from , to
\centering\fbox{
\begin{minipage}{#1in}
\hfill\vspace{#2in}
\end{minipage}
}
}

\newcommand{\boldvector}[1]{
\boldsymbol{#1}
}

\newcommand{\dEdY}[2]{\frac{d E}{d Y_{#1}^{#2}}}
\newcommand{\dEdy}[2]{\frac{d E}{d y_{#1}^{#2}}}
\newcommand{\dEdT}[2]{\frac{\partial E}{\partial T_{{#1} \rightarrow {#2}}}}
\newcommand{\dEdo}[1]{\frac{\partial E}{\partial o^{#1}}}
\newcommand{\dEdg}[1]{\frac{\partial E}{\partial g^{#1}}}
\newcommand{\dYdY}[4]{\frac{\partial Y_{#1}^{#2}}{\partial Y_{#3}^{#4}}}
\newcommand{\dYdy}[4]{\frac{\partial Y_{#1}^{#2}}{\partial y_{#3}^{#4}}}
\newcommand{\dydY}[4]{\frac{\partial y_{#1}^{#2}}{\partial Y_{#3}^{#4}}}
\newcommand{\dydy}[4]{\frac{\partial y_{#1}^{#2}}{\partial y_{#3}^{#4}}}
\newcommand{\dydT}[4]{\frac{\partial y_{#1}^{#2}}{\partial T_{{#3} \rightarrow {#4}}}}
\newcommand{\dYdT}[4]{\frac{\partial Y_{#1}^{#2}}{\partial T_{{#3} \rightarrow {#4}}}}
\newcommand{\dTdT}[4]{\frac{\partial T_{{#1} \rightarrow {#2}}}{\partial T_{{#3} \rightarrow {#4}}}}
\newcommand{\ssum}[2]{\sum_{#1}^{#2}}

\newcommand{\ssty}[1]{\scriptscriptstyle #1}

\newcommand{\myparbox}[2]{%
\parbox{#1}{\color{black!20} #2}
}

\newcommand{\bs}[1]{
\boldsymbol{#1}
}

\newcommand{\parone}[2]{%
\frac{\partial #1 }{ \partial #2 }
}
\newcommand{\partwo}[2]{%
\frac{ \partial^2 {#1} }{ \partial {#2}^2 }
}

\newcommand{\twodvectorvarfun}[2]{
\left [
\begin{array}{r}
{{#1_{\ssty{1}}}(#2)} \\
{{#1_{\ssty{2}}}(#2)}
\end{array}
\right ]
}
\newcommand{\twodvectorvarprimed}[1]{
\left [
\begin{array}{r}
{{#1_{\ssty{1}}}'(t)} \\
{{#1_{\ssty{2}}}'(t)}
\end{array}
\right ]
}

\newcommand{\complex}[2]{#1 \: #2 \: \boldsymbol{i}}
\newcommand{\complexmag}[2]%
{
\sqrt{(#1)^2 \: + \: (#2)^2}
}
\newcommand{\threenorm}[3]%
{
\sqrt{(#1)^2 \: + \: (#2)^2 \: + \: (#3)^2}
}
\newcommand{\norm}[1]{\mid \mid #1 \mid \mid}

\newcommand{\myderiv}[2]{\frac{d #1}{d #2}}
\newcommand{\myderivb}[2]{\frac{d}{d #2} \left ( #1 \right )}
\newcommand{\myrate}[3]%
{#1^\prime(#2) &=& #3 \: #1(#2)
}
\newcommand{\myrateexter}[4]%
{#1^\prime(#2) &=& #3 \: #1(#2) \: + \: #4
}
\newcommand{\myrateic}[3]%
{#1( \: #2 \:) &=& #3 
}

\newcommand{\mytwodsystemeqn}[6]{
#1 \: x    #2 \: y &=& #3\\
#4 \: x    #5 \: y &=& #6\\
}

\newcommand{\mytwodsystem}[8]{
#3 \: #1 \: + \: #4 \: #2 &=& #5\\
#6 \: #1 \: + \: #7 \: #2 &=& #8\\
}  

\newcommand{\mytwodarray}[4]{
\left [
\begin{array}{rr}
#1 & #2\\
#3 & #4
\end{array}
\right ]
}

\newcommand{\mytwoid}{
\left [
\begin{array}{rr}
1 & 0\\
0 & 1
\end{array}
\right ]
}

\newcommand{\myxprime}[2]{
\left [
\begin{array}{r}
#1^\prime(t)\\
#2^\prime(t)
\end{array}
\right ]
}

\newcommand{\myxprimepacked}[2]{
\left [
\begin{array}{r}
#1^\prime\\
#2^\prime
\end{array}
\right ]
}

\newcommand{\myx}[2]{
\left [
\begin{array}{r}
#1(t)\\
#2(t)
\end{array}
\right ]
}

\newcommand{\myxonly}[2]{
\left [
\begin{array}{r}
#1\\
#2
\end{array}
\right ]
}

\newcommand{\myv}[2]{
\left [
\begin{array}{r}
#1\\
#2
\end{array}
\right ]
}

\newcommand{\myxinitial}[2]{
\left [
\begin{array}{r}
#1(0)\\
#2(0)
\end{array}
\right ]
}

\newcommand{\twodboldv}[1]{
\boldsymbol{#1}
}

\newcommand{\mytwodvector}[2]{
\left [
\begin{array}{r}
#1\\
#2
\end{array}
\right ]
}

\newcommand{\mythreedvector}[3]{
\left [
\begin{array}{r}
#1\\
#2\\
#3
\end{array}
\right ]
}

\newcommand{\mytwodsystemvector}[6]{
\left [
\begin{array}{rr}
#1 & #2\\
#4 & #5
\end{array}
\right ]
\:
\left [
\begin{array}{r}
x \\
y 
\end{array}
\right ]
&=&
\left [
\begin{array}{r}
#3\\
#6
\end{array}
\right ]
}

\newcommand{\mythreedarray}[9]{
\left [
\begin{array}{rrr}
#1 & #2 & #3\\
#4 & #5 & #6\\
#7 & #8 & #9
\end{array}
\right ]
}

\newcommand{\myodetwo}[6]{
#1 \: #6^{\prime \prime}(t) \: #2 \: #6^{\prime}(t) \: #3 \: #6(t) &=& 0\\
#6(0)                                           &=& #4\\
#6^{\prime}(0)                                  &=& #5
}

\newcommand{\myodetwoNoIC}[4]{
#1 \: #4^{\prime \prime}(t) \: #2 \: #4^{\prime}(t) \: #3 \: #4(t) &=& 0
}

\newcommand{\myodetwopacked}[5]{
\hspace{-0.3in}& & #1 u^{\prime \prime} #2 u^{\prime} #3 u \: = \: 0\\
\hspace{-0.3in}& & u(0) \: = \: #4, \: \: u^{\prime}(0)    \: = \:  #5
}

\newcommand{\myodetwoforced}[6]{
#1\: u^{\prime \prime}(t) \: #2 \: u^{\prime}(t) \: #3 \: u(t) &=& #6\\
u(0)                                           &=& #4\\
u^{\prime}(0)                                  &=& #5\\
}

\newcommand{\myodesystemtwo}[8]{
#1 \: x^\prime(t) \: #2 \: y^\prime(t) \: #3 \: x(t) \: #4 \: y(t) &=& 0\\
#5 \: x^\prime(t) \: #6 \: y^\prime(t) \: #7 \: x(t) \: #8 \: y(t) &=& 0\\
}

\newcommand{\myodesystemtwoic}[2]{
x(0)                                       &=& #1\\ 
y(0)                                       &=& #2
}

\newcommand{\mypredprey}[4]{
x^\prime(t) &=& #1 \: x(t) \: - \: #2 \: x(t) \: y(t)\\
y^\prime(t) &=& -#3 \: y(t) \: + \: #4 \: x(t) \: y(t)
}

\newcommand{\mypredpreypacked}[4]{
x^\prime &=& #1 \: x - #2 \: x \: y\\
y^\prime &=& -#3 \: y + #4 \: x \: y
}

\newcommand{\mypredpreyself}[6]{
x^\prime(t) &=&  #1 \: x(t) \: - \: #2 \: x(t) \: y(t) \: - \: #3 \: x(t)^2\\
y^\prime(t) &=& -#4 \: y(t) \: + \: #5 \: x(t) \: y(t) \: - \: #6 \: y(t)^2
}

\newcommand{\mypredpreyfish}[5]{
x^\prime(t) &=&  #1 \: x(t) \: - \: #2 \: x(t) \: y(t) \: - \: #5 \: x(t)\\
y^\prime(t) &=& -#3 \: y(t) \: + \: #4 \: x(t) \: y(t) \: - \: #5 \: y(t)
}

\newcommand{\myepidemic}[4]{
S^\prime(t) &=& - #1 \: S(t) \: I(t)\\
I^\prime(t) &=&   #1 \: S(t) \: I(t) \: - \: #2 \: I(t)\\
S(0)        &=&   #3\\
I(0)        &=&   #4\\
}

\newcommand{\bsred}[1]{%
\textcolor{red}{\boldsymbol{#1}}
}

\newcommand{\bsblue}[1]{%
\textcolor{blue}{\boldsymbol{#1}}
}


\newcommand{\myfloor}[1]{%
\lfloor{#1}\rfloor
}

\newcommand{\cubeface}[7]{%
\begin{bmatrix}
\bs{#3}          & \longrightarrow & \bs{#4}\\
\uparrow          &                         &  \uparrow  \\
\bs{#1} & \longrightarrow & \bs{#2}\\
              & \text{ \bfseries #5:} \: \bs{#6} \: \text{\bfseries  #7 } & 
\end{bmatrix}
}

\newcommand{\cubefacetwo}[5]{%
\begin{bmatrix}
\bs{#3}          & \longrightarrow & \bs{#4}\\
\uparrow          &                         &  \uparrow  \\
\bs{#1} & \longrightarrow & \bs{#2}\\
              & \text{ \bfseries #5} & 
\end{bmatrix}
}

\newcommand{\cubefacethree}[9]{%
\begin{bmatrix}
\bs{#3}                  & \overset{#9}{\longrightarrow} & \bs{#4}\\
\uparrow \: #7         &                                             &  \uparrow  \: #8 \\
\bs{#1}                  & \overset{#6}{\longrightarrow} & \bs{#2}\\
                               & \text{ \bfseries #5} & 
\end{bmatrix}
}

\renewcommand{\qedsymbol}{\hfill \blacksquare}
\newcommand{\subqedsymbol}{\hfill \Box}
%\theoremstyle{plain}

\newtheoremstyle{mystyle}% name
  {6pt}%      Space above
  {6pt}%      Space below
  {\itshape}%         Body font
  {}%         Indent amount (empty = no indent, \parindent = para indent)
  {\bfseries}% Thm head font
  {}%        Punctuation after thm head
  { }%     Space after thm head: " " = normal interword space; \newline = linebreak
  {}%         Thm head spec (can be left empty, meaning `normal')
\theoremstyle{mystyle}
 
\newtheorem{axiom}{Axiom}
%\newtheorem{solution}{Solution}[section]
\newtheorem*{solution}{Solution}
\newtheorem{exercise}{Exercise}[section]
\newtheorem{theorem}{Theorem}[section]
\newtheorem{proposition}[theorem]{Proposition}
\newtheorem{prop}[theorem]{Proposition}
\newtheorem{assumption}{Assumption}[section]
\newtheorem{definition}{Definition}[section]
\newtheorem{comment}{Comment}[section]
\newtheorem*{question}{Question}
\newtheorem{program}{Program}[section]
%\newtheorem{myproof}{Proof}
%\newtheorem*{myproof}{Proof}[section]
\newtheorem{myproof}{Proof}[section]
\newtheorem{hint}{Hint}[section]
\newtheorem*{phint}{Hint}
\newtheorem{lemma}[theorem]{Lemma}
\newtheorem{example}{Example}[section]
      
\newenvironment{myassumption}[4]
{
\centering
\begin{assumption}[{\textbf{#1}\nopunct}]%
\index{#2}
\mbox{}\\  \vskip6pt \colorbox{black!15}{\fbox{\parbox{.9\textwidth}{#3}}}
\label{#4}
\end{assumption}
%\renewcommand{\theproposition}{\arabic{chapter}.\arabic{section}.\arabic{assumption}} 
}%
{}

\newenvironment{myproposition}[4]
{
\centering
\begin{proposition}[{\textbf{#1}\nopunct}]%
\index{#2} 
\mbox{}\\  \vskip6pt \colorbox{black!15}{\fbox{\parbox{.9\textwidth}{#3}}}
\label{#4}
\end{proposition}
%\renewcommand{\theproposition}{\arabic{chapter}.\arabic{section}.\arabic{proposition}} 
}%
{}

\newenvironment{mytheorem}[4]
{
\centering
\begin{theorem}[{\textbf{#1}\nopunct}]%
\index{#2} 
\mbox{}\\ \vskip6pt \colorbox{black!15}{\fbox{\parbox{.9\textwidth}{#3}}}
\label{#4}
\end{theorem}
%\renewcommand{\thetheorem}{\arabic{chapter}.\arabic{section}.\arabic{theorem}} 
}%
{}

\newenvironment{mydefinition}[4]
{
\centering
\begin{definition}[{\textbf{#1}\nopunct}]%
\index{#2} 
\mbox{}\\  \vskip6pt \colorbox{black!15}{\fbox{\parbox{.9\textwidth}{#3}}}
\label{#4}
\end{definition}
%\renewcommand{\thedefinitio{n}{\arabic{chapter}.\arabic{section}.\arabic{definition}} 
}%
{}

\newenvironment{myaxiom}[4]
{
\centering
\begin{axiom}[{\textbf{#1}\nopunct}]%
\index{#2} 
\mbox{}\\  \vskip6pt \colorbox{black!15}{\fbox{\parbox{.9\textwidth}{#3}}}
\label{#4}
\end{axiom}
%\renewcommand{\theaxiom}{\arabic{chapter}.\arabic{section}.\arabic{axiom}} 
}%
{}

\newenvironment{mylemma}[4]
{
\centering
\begin{lemma}[{\textbf{#1}\nopunct}]%
\index{#2} 
\mbox{}\\  \vskip6pt \colorbox{black!15}{\fbox{\parbox{.9\textwidth}{#3}}}
\label{#4}
\end{lemma}
%\renewcommand{\thelemma}{\arabic{chapter}.\arabic{section}.\arabic{lemma}} 
}%
{}
   
\newenvironment{reason}[1]
{
 \vskip0.05in
 \begin{myproof}
 \mbox{}\\#1
 $\qedsymbol$
 \end{myproof}  
 \vskip0.05in
}%
{}

\newenvironment{reasontwo}[1]
{
 \vskip+.05in
 \begin{myproof}
 \mbox{}\\#1
 \end{myproof}  
 \vskip+0.05in
}%
{}

\newenvironment{subreason}[1]
{
 \vskip0.05in
 \renewcommand{\themyproof}{}
 \begin{myproof}
 #1
 $\subqedsymbol$
 \end{myproof}
 \vskip0.05in
 \renewcommand{\themyproof}{\thetheorem}
 %\renewcommand{\themyproof}{\arabic{chapter}.\arabic{section}.\arabic{myproof}}   
 %
}%
{}  

\newenvironment{myhint}[1]
{
 \vskip0.05in
 \begin{hint}
 #1
 $\subqedsymbol$ 
 \end{hint}  
 \vskip0.05in
}%
{} 

\newenvironment{myeqn}[3]
{
 \renewcommand{\theequation}{$\boldsymbol{#1}$}
 \begin{eqnarray}
 \label{equation:#2}
 #3 
 \end{eqnarray}
 \renewcommand{\theequation}{\arabic{chapter}.\arabic{eqnarray}}   
}%
{} 


\JournalInfo{MATH 8210:  Homework Four, 1-\pageref{LastPage}, 2020} % Journal information
\Archive{Draft Version \today} % Additional notes (e.g. copyright, DOI, review/research article)

\PaperTitle{MATH 8210 Homework Four}
\Authors{Your Name Here\textsuperscript{1}}
\affiliation{\textsuperscript{1}\textit{School of Mathematical and Statistical Sciences,
Clemson University,Clemson, SC: email yournamehere@clemson.edu}}
\affiliation{*\textbf{Corresponding author}: yournamehere@clemson.edu} % Corresponding author

\date{\small{Version ~\today}}
\Abstract{Some problems metric spaces}
\Keywords{}
\newcommand{\keywordname}{Keywords}
%
\onehalfspacing
\begin{document}

\flushbottom

\addcontentsline{toc}{section}{Title}
\maketitle

\renewcommand{\theexercise}{\arabic{exercise}}%

\noindent
Here are some questions about the LCT.

\begin{exercise}
Let $f(t) = t$ and $g(t) = t^2$ on $[0,1]$.
We want to find a constant $c>0$ so that for all $\alpha$ and
$\beta$ the following inequality hold.
\begin{eqnarray*}
\int_0^1 \: | \alpha t + \beta t^2| dt &\geq c( |\alpha| + |\beta|)
\end{eqnarray*}
\noindent
This is a lot harder than it looks. It is enough to look at
the cases where both $\alpha$ and $\beta$ are non zero.
\begin{itemize}
\item If both $\alpha$ and $\beta$ are positive or the case
where they are both negative are easy.  Show we have
\begin{eqnarray*}
\int_0^1 \: | |\alpha| t + |\beta| t^2| dt 
&=&
\frac{|\alpha|}{2} + \frac{|\beta|}{3} \geq \frac{1}{6} ( |\alpha| + |\beta|)
\end{eqnarray*}
\item Show the only other case is when $\alpha$ and $\beta$ have opposite
sign in which case it is enough to study
$\int_0^1 | |\alpha|t - |\beta| t^2 | dt$.    Show there are two cases to consider:
\begin{enumerate}
\item $\frac{|\alpha|}{|\beta|} \geq 1$ which implies the integrand is
always positive and we find
\begin{eqnarray*}
\int_0^1 \: | |\alpha| t - |\beta| t^2| dt &=& \frac{|\alpha|}{2} - \frac{|\beta|}{3}
\geq \frac{1}{12}( |\alpha| + |\beta|)
\end{eqnarray*}
\noindent
because $|\alpha| > |\beta|$ here.
\item $\frac{|\alpha|}{|\beta|} < 1$ so that we write
\begin{eqnarray*}
\int_0^1 \: | |\alpha| t - |\beta| t^2| dt &=& 
\int_0^{\frac{|\alpha|}{|\beta|}} (|\alpha| t - |\beta| t^2) dt
+ \int_{\frac{|\alpha|}{|\beta|}}^1 (|\beta| t^2-|\alpha| t)  dt\\
&=&
\frac{|\alpha|}{3} \: \left( \frac{|\alpha|}{|\beta|} \right)^2
+ \frac{1}{3}|\beta| - \frac{1}{2} |\alpha|
\end{eqnarray*}
\item Show if $u = \frac{|\alpha|}{|\beta|}$ we have
\begin{eqnarray*}
\frac{\frac{|\alpha|}{3} \: \left( \frac{|\alpha|}{|\beta|} \right)^2
+ \frac{1}{3}|\beta| - \frac{1}{2} |\alpha|}{|\alpha| + |\beta|}
&=& f(u) =
\frac{\frac{1}{3} - \frac{1}{2} u + \frac{1}{3}u^3}{1+u}
\end{eqnarray*}
\noindent
We only need to consider this function on $(0,1)$.  Show using
standard calculus that $f$ has a positive absolute minimum on $(0,1)$
which furnishes another value of $c$.
\end{enumerate}
\item The smallest of all these $c$ values is the one we seek.
\end{itemize}
\end{exercise}

\begin{solution}
Your solution here.
\end{solution}
	
\begin{exercise}
Let $f(t) = t$ and $g(t) = t^2$ on $[0,1]$.
We want to find a constant $c>0$ so that for all $\alpha$ and
$\beta$ the following inequality hold.
\begin{eqnarray*}
\sqrt{\int_0^1 \: | \alpha t + \beta t^2|^2 dt} &\geq c( |\alpha| + |\beta|)
\end{eqnarray*}
\noindent
This is a again a lot harder than it looks. It is enough to look at
the cases where both $\alpha$ and $\beta$ are non zero.
\begin{itemize}
\item If both $\alpha$ and $\beta$ are positive or the case
where they are both negative is not so bad.  Show we have
\begin{eqnarray*}
\int_0^1 \: ( |\alpha| t + |\beta| t^2|)^2 dt 
&=&
\int_0^1 ( |\alpha|^2 t^2 + 2 |\alpha| |\beta| t^3 + |\beta|^2 t^4 ) dt\\
&=&
\frac{|\alpha|^2}{3} + 2\frac{|\alpha| |\beta|}{4} + \frac{|\beta|^2}{5}
\geq
\frac{|\alpha|^2}{5} + 2\frac{|\alpha| |\beta|}{5} + \frac{|\beta|^2}{5}\\
&=&
\frac{\left( |\alpha| + |\beta| \right)^2}{5}
\Longrightarrow
\sqrt{\int_0^1 \: ( |\alpha| t + |\beta| t^2|)^2 dt}
\geq \frac{1}{\sqrt{5}} ( |\alpha| + |\beta|)
\end{eqnarray*}
\noindent
\item 
If $\alpha$ and $\beta$ are different signs, then the work is more difficult.
We only need to consider
\begin{eqnarray*}
\int_0^1 \: ( |\alpha| t - |\beta| t^2|)^2 dt 
&=&
\int_0^1 ( |\alpha|^2 t^2 - 2 |\alpha| |\beta| t^3 + |\beta|^2 t^4 ) dt\\
&=&
\frac{|\alpha|^2}{3} - \frac{|\alpha| |\beta|}{2 } + \frac{|\beta|^2}{5}
\end{eqnarray*}
\noindent
We want
\begin{eqnarray*}
\sqrt{\frac{|\alpha|^2}{3} - \frac{|\alpha| |\beta|}{2 } + \frac{|\beta|^2}{5}}
&\geq& c (|\alpha| + |\beta|)
\Longrightarrow
\frac{\frac{|\alpha|^2}{3} - \frac{|\alpha| |\beta|}{2 } + \frac{|\beta|^2}{5}}{(|\alpha| + |\beta|)^2}
\geq c^2
\end{eqnarray*}
\noindent
Let $u = \frac{|\alpha|}{|\beta}$ and show
\begin{eqnarray*}
\frac{\frac{|\alpha|^2}{3} - \frac{|\alpha| |\beta|}{2 } + \frac{|\beta|^2}{5}}{(|\alpha| + |\beta|)^2}
&=& f(u)
= \frac{\frac{1}{3}u^2 - \frac{1}{2}u + \frac{1}{5}}{(1+u)^2}
\end{eqnarray*}
\noindent
We need to consider this function on $(0,\infty)$.  Show using
standard calculus that $f$ has a positive absolute minimum on $(0,1)$
which furnishes another value of $c$.
\item The smallest of all these $c$ values is the one we seek.
\end{itemize}
\end{exercise}

\begin{solution}
Your solution here.
\end{solution}

\begin{exercise}
Let $f(t) = t$ and $g(t) = t^3$ on $[0,1]$.
We want to find a constant $c>0$ so that for all $\alpha$ and
$\beta$ the following inequality hold.
\begin{eqnarray*}
\int_0^1 \: | \alpha t + \beta t^3| dt &\geq c( |\alpha| + |\beta|)
\end{eqnarray*}
\noindent
Oh boy, this is ugly!
\end{exercise}

\begin{solution}
Your solution here.
\end{solution}
	
\begin{exercise}
Let $f(t) = t$ and $g(t) = t^3$ on $[0,1]$.
We want to find a constant $c>0$ so that for all $\alpha$ and
$\beta$ the following inequality hold.
\begin{eqnarray*}
\sqrt{\int_0^1 \: | \alpha t + \beta t^3|^2 dt} &\geq c( |\alpha| + |\beta|)
\end{eqnarray*}
\noindent
This is even uglier but look at it this way: you always wanted to
show the world you know your calculus right?
\end{exercise}

\begin{solution}
Your solution here.
\end{solution}

\section{Finding Roots}

You probably noticed that the problems above lead to root finding
problems.  You might find these notes on root finding in MatLab helpful.
There are plenty of exercises here which you don't have to do
to turn in but you might find them useful.  Now if you use these methods
in your exercises make sure you document how you do the code solutions.\\

\noindent
Let's discuss root finding using both the {\bf bisection} and
{Newton} method.

\subsection{The Bisection Method:}

We need a simple function to find the root of a 
{\em nice} function $f$ of the real variable
$x$ using what is called bisection.  The method is actually
quite simple.  We know that if $f$ is a continuous function on
the finite interval $[a,b]$ then $f$ must have a zero inside the
interval $[a,b]$ if $f$ has a different algebraic sign at the endpoints
$a$ and $b$.  This means the product $f(a) \: f(b)$ is not zero.
So we assume we can find an interval $[a,b]$ on which this change in
sign satisfies $f(a) \: f(b) \: \leq \: 0$ (which we can do
by switching to $-f$ if we have to!) and then if we
divide the interval $[a,b]$ into two equal pieces
$[a,m]$ and $[m,b]$, $f(m)$ can't have the same sign as
both $f(a)$ and $f(b)$ because of the assumed sign difference.
So at least one of the two halves has a sign change.\\

\noindent
Note that if $f(a)$ and $f(b)$ was zero then we still have
$f(a) \: f(b) \: \leq \: 0$ and either $a$ or 
$b$ could be our chosen root and either half interval
works fine.  If only one of the endpoint
function values is zero, then the bisection of $[a,b]$ into the
two halves still finds the one half interval that has the root.\\

\noindent
So our prototyping Matlab code should use tests like
$f(x) \: f(y) \: \leq \: 0$ rather than
$f(x) \: f(y) \: < \: 0$ to make sure we catch the root.

\subsubsection{The Bisection Matlab Code:}

Here is a simple Matlab function to perform the Bisection routine.

\mylistcode{matlab}{Bisection Code}{rootoptcode1}
\singlespacing
\lstinputlisting{Source/RootsOpt/Bisection.m}
\onehalfspacing

\noindent
We should look at some of these lines more closely.
First, to use this routine, we need to write a function definition
for the function we want to apply bisection to.  We will do this
in a file called {\tt func.m} (Inspired Name, eh?)
An example would be the one we wrote for the function
\begin{eqnarray*}
f(x) &=& \tan(\frac{x}{4}) \: - \: 1;
\end{eqnarray*}
\noindent
which is coded in Matlab by

\mylistcode{matlab}{Function Definition In MatLab}{rootoptcode2}
\singlespacing
\lstinputlisting{Source/RootsOpt/func.m}
\onehalfspacing

\noindent
So to apply bisection to this function on the
interval $[2,4]$ with a stopping tolerance of say
$10^{-4}$, in Matlab, we would type the command

\singlespacing
\myfancyverbatim{Using the code}
\begin{lstlisting}
 root = Bisection('func',2,4,10^-4)}
\end{lstlisting}
\lstset{fancyvrb=false}
\onehalfspacing

\noindent
Note that the name of our supplied function, the uninspired
choice {\em func} is passed in as the first argument in single
quotes as it is a string.\\

\noindent
Also, in the Bisection routine, we have added the code to
print out what is happening at each iteration of the while loop.
Matlab handles prints to the screen a little funny, so do set up
a table of printed values we use this syntax:

\singlespacing
\myfancyverbatim{Printing out diagnostics}
\begin{lstlisting}
  % this prints a blank line and then a table heading.
  % note disp prints a string only
  disp(' ')
  disp('      k  |       a(k)    |      b(k)     |       b(k) - a(k) ')
  % now to print the k, a, b and b-a, we must first
  % put their values into a string using the c like function
  % sprintf and then use disp to disply that string.
  % so we do this 
  % disp( sprintf(' output specifications here ',variables here))
  % so inside the while loop we use
  disp(sprintf(' %6d  | %12.7f  | %12.7f  | %12.7f',k,a,b,b-a));
\end{lstlisting}
\lstset{fancyvrb=false}
\onehalfspacing

\subsubsection{Running the Code:}

As mentioned above, we will test this code on the function
\begin{eqnarray*}
f(x) &=& \tan(\frac{x}{4}) \: - \: 1;
\end{eqnarray*}
\noindent
on the interval $[2,4]$ with a stopping tolerance of
$\delta \: = \: 10^{-6}$.  Our function has been
written as the Matlab function {\tt func} supplied in
the file {\tt func.m}.\\

\noindent
The Matlab run time looks like this:

\mylistcode{matlab}{Bisection MatLab Session}{rootoptcode3}
\singlespacing
\lstinputlisting{Source/RootsOpt/Bisection.prompts}
\onehalfspacing

\subsubsection{Exercises:}

Well, you have to practice this stuff to see what is going on.
So here are two problems to sink your teeth into!

\begin{enumerate}
\item Use bisection to find the first five positive solutions of the
      equation $x \: = \: \tan(x)$.  You can see where this is
      roughly by graphing $\tan(x)$ and $x$ simultaneously.
      Do this for tolerances 
      $$\{10^{-1}, 10^{-2}, 10^{-3}, 10^{-4}, 10^{-5}, 10^{-6}, 10^{-7}\}$$
      For each root, choose a reasonable bracketing interval $[a,b]$,
      explain why you chose it,
      provide a table of the number of iterations to achieve the
      accuracy and a graph of this number vs. accuracy.
\item Use the Bisection Method to find the largest real root of the function
      $f(x) \: = \: x^6 \: - \: x \: - \: 1$. 
      Do this for tolerances \\
      $\{10^{-1}, 10^{-2}, 10^{-3}, 10^{-4}, 10^{-5}, 10^{-6}, 10^{-7}\}$.
      Choose a reasonable bracketing interval $[a,b]$,
      explain why you chose it,
      provide a table of the number of iterations to achieve the
      accuracy and a graph of this number vs. accuracy.     
\end{enumerate}

\subsection{Newton's Method:}

\subsubsection{When Do We Do A Newton Step?}

The following code uses a simple test to see
if we should use a bisection or a newton method
step in our zero finding routine.

\mylistcode{matlab}{Should We Do A Newton Step?}{rootoptcode4}
\singlespacing
\lstinputlisting{Source/RootsOpt/StepIsIn.m}
\onehalfspacing

\subsubsection{A Global Newton Method:}

\mylistcode{matlab}{Global Newton Method}{rootoptcode5}
\singlespacing
\lstinputlisting{Source/RootsOpt/GlobalNewton.m}
\onehalfspacing

\subsubsection{A Run Time Example:}

We will apply our global newton method root finding code to
a simple example: find a root for $f(x) \: = \: \sin(x)$ in the
interval $[\frac{-7 \pi}{2}, 15 \pi + 0.1]$.  We code the
function and its derivative in two simple Matlab files;
{\tt f1.m} and {\tt f1p.m}.  These are

\mylistcode{matlab}{Global Newton Function}{rootoptcode6}
\singlespacing
\lstinputlisting{Source/RootsOpt/f1.m}
\onehalfspacing

\noindent
and

\mylistcode{matlab}{Global Newton Function Derivatived}{rootoptcode7}
\singlespacing
\lstinputlisting{Source/RootsOpt/f1p.m}
\onehalfspacing

\noindent
To run this code on this example, we would then type
a phrase like the one below:

\singlespacing
\myfancyverbatim{Using the Global Newton Code}
\begin{lstlisting}
[x,fx,nEvals,aLast,bLast] = GlobalNewton('f1','f1p',-7*pi/2,15*pi+.1,...
                                         10^-6,10^-8,200)
\end{lstlisting}
\lstset{fancyvrb=false}
\onehalfspacing	

\noindent
Here is the runtime output:				 

\mylistcode{matlab}{Global Newton MatLab Session}{rootoptcode8}
\singlespacing
\lstinputlisting{Source/RootsOpt/GlobalMethod.prompts}
\onehalfspacing

\subsubsection{Some Exercises:}

\begin{enumerate}
\item Use the Global Newton Method to find the first five 
      positive solutions of the
      equation $x \: = \: \tan(x)$.  You can see where this is
      roughly by graphing $\tan(x)$ and $x$ simultaneously.
      Do this for tolerances \\
      $$\{10^{-1}, 10^{-2}, 10^{-3}, 10^{-4}, 10^{-5}, 10^{-6}, 10^{-7}\}$$
      For each root, choose a reasonable bracketing interval $[a,b]$,
      explain why you chose it,
      provide a table of the number of iterations to achieve the
      accuracy and a graph of this number vs. accuracy.
\item Use the Global Newton Method to find the largest real root of the function
      $f(x) \: = \: x^6 \: - \: x \: - \: 1$. 
      Do this for tolerances \\
      $$\{10^{-1}, 10^{-2}, 10^{-3}, 10^{-4}, 10^{-5}, 10^{-6}, 10^{-7}\}$$
      Choose a reasonable bracketing interval $[a,b]$,
      explain why you chose it,
      provide a table of the number of iterations to achieve the
      accuracy and a graph of this number vs. accuracy.     
\end{enumerate}

\subsection{Adding Finite Difference Approximations to the Derivative:}

We can also choose to replace the derivative function for
$f$ with a finite difference approximation.  We will use
\begin{eqnarray*}
f^\prime(x) &\approx& \frac{f(x_c \: + \: \delta_c) \: - \: f(x_c)}{\delta_c}
\end{eqnarray*}
\noindent
to approximate the value of the derivative at the point $x_c$.
As we have discussed earlier, some care is required to pick a 
size for $\delta_c$ so that round-off errors do not destroy the accuracy of
our finite difference approximation to $f^\prime$.\\

\noindent
The simple Matlab code to implement this is given below:

\singlespacing
\myfancyverbatim{Simple Finite Difference Approximation}
\begin{lstlisting}
fval = feval(fname,x);
fpval = (feval(fname,x+delta) - fval)/delta;
\end{lstlisting}
\lstset{fancyvrb=false}
\onehalfspacing

\noindent
We can also use a secant approximation as follows:
\begin{eqnarray*}
f^\prime(x) &\approx& \frac{f(x_c) \: - \: f(x_{-}}{x_c \: - \: x_{-}}
\end{eqnarray*}
\noindent
where $x_{-}$ is the previous iterate from our routine.
The Matlab fragment we need is then:

\singlespacing
\myfancyverbatim{Secant Approximation}
\begin{lstlisting}
fpc = (fc - f_)/(xc - x_);
\end{lstlisting}
\lstset{fancyvrb=false}
\onehalfspacing

\subsubsection{A Finite Difference Global Newton Method:}

We add the finite difference routines into our Global Newton's
Method as follows:

\mylistcode{matlab}{Finite Difference Global Newton Method}{rootoptcode9}
\singlespacing
\lstinputlisting{Source/RootsOpt/GlobalNewtonFD.m}
\onehalfspacing

\noindent
Note, we use for our finite difference stepsize 
$\sqrt{\epsilon_{machine}} |x|$.

\subsubsection{A Run Time Example:}

We will apply our finite difference global newton method root finding code to
the same simple example: find a root for $f(x) \: = \: \sin(x)$ in the
interval $[\frac{-7 \pi}{2}, 15 \pi + 0.1]$.  We only need the code 
for the function now which is as usual in the file
{\tt f1.m}.

\noindent
To run this code on this example, we would then type
a phrase like the one below:

\singlespacing
\myfancyverbatim{Using the Finite Difference Code}
\begin{lstlisting}
[x,fx,nEvals,aLast,bLast] = GlobalNewtonFD('f1',-7*pi/2,15*pi+.1,...
                                         10^-6,10^-8,200)
\end{lstlisting}
\lstset{fancyvrb=false}
\onehalfspacing	

\noindent
Here is the runtime output:				 

\mylistcode{matlab}{Finite Difference Netwon Method MatLab Session}%
{rootoptcode10}
\singlespacing
\lstinputlisting{Source/RootsOpt/GlobalNewtonFD.prompts}
\onehalfspacing

\subsubsection{Some Exercises:}

\begin{enumerate}
\item Use the Finite Difference Global Newton Method to find the second
      positive solution of the equation $x \: = \: \tan(x)$.  
      Do this for tolerances 
      $10^{-8}$.  This time alter the GlobalNewtonFD code to
      allow the finite difference step size {\tt delta}
      to be a parameter and do a parametric study on the
      effects of {\tt delta}.  Note that the code now uses
      the reasonable choice of $\sqrt{\epsilon_{machine}} |x|$ but you
      need to use the additional $\delta$ choices
      $\{10^{-4}, 10^{-6}, 10^{-8}, 10^{-10}\}$.  This will give you
      five $\delta$ choices. 
      Provide a table
      and a graph of $\delta$ vs. accuracy of the root approximation.
\item Use the Finite Difference Global Newton Method to 
      find the largest real root of the function
      $f(x) \: = \: x^6 \: - \: x \: - \: 1$.
      Do this for tolerances 
      $10^{-8}$.  Again use altered GlobalNewtonFD code
      with the finite difference step size {\tt delta}
      as a parameter and do a parametric study on the
      effects of {\tt delta}.  Note that the code now uses
      the reasonable choice of $\sqrt{\epsilon_{machine}} |x|$ but you
      need to use the additional $\delta$ choices
      $\{10^{-4}, 10^{-6}, 10^{-8}, 10^{-10}\}$.  This will give you
      five $\delta$ choices. 
      Provide a table
      and a graph of $\delta$ vs. accuracy of the root approximation. 
\item Do the same thing for the problems above, but replace the
      Finite Difference Global Newton Code with a Secant Global
      Newton Code.  This will only require a few lines of code to
      change really, so don't freak out!          
\end{enumerate}

\end{document}