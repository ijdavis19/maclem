% SampleProject.tex -- main LaTeX file for sample LaTeX article
%
%\documentclass[12pt]{article}
\documentclass[11pt]{SelfArxOneColBMN}
% add the pgf and tikz support.  This automatically loads
% xcolor so no need to load color
\usepackage{pgf}
\usepackage{tikz}
\usetikzlibrary{matrix}
\usetikzlibrary{calc}
\usepackage{xstring}
\usepackage{pbox}
\usepackage{etoolbox}
\usepackage{marginfix}
\usepackage{xparse}
\setlength{\parskip}{0pt}% fix as marginfix inserts a 1pt ghost parskip
% standard graphics support
\usepackage{graphicx,xcolor}
\usepackage{wrapfig}
%
\definecolor{color1}{RGB}{0,0,90} % Color of the article title and sections
\definecolor{color2}{RGB}{0,20,20} % Color of the boxes behind the abstract and headings
%----------------------------------------------------------------------------------------
%	HYPERLINKS
%----------------------------------------------------------------------------------------
\usepackage[pdftex]{hyperref} % Required for hyperlinks
\hypersetup{hidelinks,
colorlinks,
breaklinks=true,%
urlcolor=color2,%
citecolor=color1,%
linkcolor=color1,%
bookmarksopen=false%
,pdftitle={SampleProject},%
pdfauthor={Peterson}}
%\usepackage[round,numbers]{natbib}
\usepackage[numbers]{natbib}
\usepackage{lmodern}
\usepackage{setspace}
\usepackage{xspace}
%
\usepackage{subfigure}
\newcommand{\goodgap}{
  \hspace{\subfigtopskip}
  \hspace{\subfigbottomskip}}
%
\usepackage{atbegshi}
%
\usepackage[hyper]{listings}
%
% use ams math packages
\usepackage{amsmath,amsthm,amssymb,amsfonts}
\usepackage{mathrsfs}
%
% use new improved Verbatim
\usepackage{fancyvrb}
%
\usepackage[titletoc,title]{appendix}
%
\usepackage{url}
%
% Create length for the baselineskip of text in footnotesize
\newdimen\footnotesizebaselineskip
\newcommand{\test}[1]{%
 \setbox0=\vbox{\footnotesize\strut Test \strut}
 \global\footnotesizebaselineskip=\ht0 \global\advance\footnotesizebaselineskip by \dp0
}
%
\usepackage{listings}

\DeclareGraphicsExtensions{.pdf, .jpg, .tif,.png}

% make sure we don't get orphaned words if at top of page
% or orphans if at bottom of page
\clubpenalty=9999
\widowpenalty=9999
\renewcommand{\textfraction}{0.15}
\renewcommand{\topfraction}{0.85}
\renewcommand{\bottomfraction}{0.85}
\renewcommand{\floatpagefraction}{0.66}
%
\DeclareMathOperator{\sech}{sech}

\newcommand{\mycite}[1]{%
(\citeauthor{#1} \citep{#1} \citeyear{#1})\xspace
}

\newcommand{\mycitetwo}[2]{%
(\citeauthor{#2} \citep[#1]{#2} \citeyear{#2})\xspace
}

\newcommand{\mycitethree}[3]{%
(\citeauthor{#3} \citep[#1][#2]{#3} \citeyear{#3})\xspace
}

\newcommand{\myincludegraphics}[3]{% file name, width, height
\includegraphics[width=#2,height=#3]{#1}
}

\newcommand{\myincludegraphicstwo}[2]{% file name, width, height
\includegraphics[scale=#1]{#2}
}

\newcommand{\mysimplegraphics}[1]{% file name, width, height
\includegraphics{#1}
}

\newcommand{\MB}[1]{
\boldsymbol{#1}
}

\newcommand{\myquotetwo}[1]{%
\small
%\singlespacing
\begin{quotation}
#1
\end{quotation}
\normalsize
%\onehalfspacing  
}

\newcommand{\jimquote}[1]{%
\small
%\singlespacing
\begin{quotation}
#1
\end{quotation}
\normalsize
%\onehalfspacing
}

\newcommand{\myquote}[1]{%
\small
%\singlespacing
\begin{quotation}
#1
\end{quotation}
\normalsize
%\onehalfspacing  
}

%A =
%
%[2 r_1 	     r_1]
%[-2r_1 + r_2  r_2 - r_1]
%
%has eigenvalues r_1 neq r_2.
% #1 = 2 r_1, #2 = r_1, #3 = -2r_1+r_2, #4 = r_2 - r_1
\newcommand{\myrealdiffA}[4]{
\left [
\begin{array}{rr}
#1  & #2\\
#3  & #4
\end{array}
\right ]
}

% args:
% 1, 2 ,3, 4, 5 = caption, label, width, height, file name
%\mysubfigure{}{}{}{}{}
\newcommand{\mysubfigure}[5]{%
\subfigure[#1]{\label{#2}\includegraphics[width=#3,height=#4]{#5}}
}

\newcommand{\mysubfiguretwo}[3]{%
\subfigure[#1]{\label{#2}\includegraphics{#3}}
}

\newcommand{\mysubfigurethree}[4]{%
\subfigure[#1]{\label{#2}\includegraphics[scale=#3]{#4}}
}

\newcommand{\myputimage}[5]{% file name, width, height
\centering
\includegraphics[width=#3,height=#4]{#5}
\caption{#1}
\label{#2}
}

\newcommand{\myputimagetwo}[4]{% caption, label, scale, file name
\centering
\includegraphics[scale=#3]{#4}
\caption{#1}
\label{#2}
}

\newcommand{\myrotateimage}[5]{% file name, width, height
\centering
\includegraphics[scale=#3,angle=#4]{#5}
\caption{#1}
\label{#2}
}

\newcommand{\myurl}[2]{%
\href{#1}{\bf #2}
}

\RecustomVerbatimEnvironment%
{Verbatim}{Verbatim}  
  {fillcolor=\color{black!20}}
  
  \DefineVerbatimEnvironment%
{MyVerbatim}{Verbatim}  
  {frame=single,
   framerule=2pt,
   fillcolor=\color{black!20},
   fontsize=\small}
   
\newcommand{\myfvset}[1]{%  
\fvset{frame=single,
       framerule=2pt,
       fontsize=\small,
       xleftmargin=#1in}}
       
\newcommand{\mylistverbatim}{%
\lstset{%
  fancyvrb, 
  basicstyle=\small,
  breaklines=true}
}  

\newcommand{\mylstinlinebf}[1]{%
{\bf #1}
}

\newcommand{\mylstinline}{%
\lstset{%
  basicstyle=\color{black!80}\bfseries\ttfamily,
  showstringspaces=false,
  showspaces=false,showtabs=false,
  breaklines=true}
\lstinline
}

\newcommand{\mylstinlinetwo}[1]{%
\lstset{%
  basicstyle=\color{black!80}\bfseries\ttfamily,
  showstringspaces=false,
  showspaces=false,showtabs=false,
  breaklines=true}
\lstinline!#1! 
}

%fontfamily=tt
%fontfamily=courier
%fontfamily=helvetica
%frame=topline,
%frame=single,
 %frame=lines,
 %framesep=10pt,
 %fontshape=it,
 %fontseries=b,
 %fontsize=\relsize{-1},
 %fillcolor=\color{black!20},
 %rulecolor=\color{yellow},
 %fillcolor=\color{red}
 %label=\fbox{\Large\emph{The code}}
\DefineVerbatimEnvironment%
{MyListVerbatim}{Verbatim}  
{
fillcolor=\color{black!10},
fontfamily=courier,
frame=single,
%formatcom=\color{white},
framesep=5mm,
labelposition=topline,
fontshape=it,
fontseries=b,
fontsize=\small,
label=\fbox{\large\emph{The code}\normalsize}
} 

%  caption={[#1] \large\bf{#1}}, 
%\centering \framebox[.6\textwidth][c]{\Large\bf{#1}}
\newcommand{\myfancyverbatim}[1]{%
\lstset{%
  fancyvrb=true, 
  %fvcmdparams= fillcolor 1,
  %morefvcmdparams = \textcolor 2,
  frame=shadowbox,framerule=2pt, 
  basicstyle=\small\bfseries,
  backgroundcolor=\color{black!08},
  showstringspaces=false,
  showspaces=false,showtabs=false,
  keywordstyle=\color{black}\bfseries,
  %numbers=left,numberstyle=\tiny,stepnumber=5,numbersep=5pt,
  stringstyle=\ttfamily,
  caption={[\quad #1] \mbox{}\\ \vspace{0.1in} \framebox{\large \bf{#1} \small} },  
  belowcaptionskip=20 pt,  
  label={},
  xleftmargin=17pt,
  framexleftmargin=17pt,
  framexrightmargin=5pt,
  framexbottommargin=4pt,
  nolol=false,
  breaklines=true}
}

\newcommand{\mylistcode}[3]{%
\lstset{%
  language=#1, 
  frame=shadowbox,framerule=2pt, 
  basicstyle=\small\bfseries,
  backgroundcolor=\color{black!16},
  showstringspaces=false,
  showspaces=false,showtabs=false,
  keywordstyle=\color{black!40}\bfseries,
  numbers=left,numberstyle=\tiny,stepnumber=5,numbersep=5pt,
  stringstyle=\ttfamily,
  caption={[\quad#2] \mbox{}\\ \vspace{0.1in} \framebox{\large \bf{#2} \small} },
  belowcaptionskip=20 pt,
  breaklines=true,
  xleftmargin=17pt,
  framexleftmargin=17pt,
  framexrightmargin=5pt,
  framexbottommargin=4pt,  
  label=#3,
  breaklines=true} 
}

  %caption={[#2] #3},
  %caption={[#2]{\mbox{}\\ \vspace{0.1in} \framebox{\large \bf{#3} \small}},
  %caption={[#2] \mbox{}\\ \bf{#3} },

% frame=single,
% caption={[Code Fragment] {\bf Code Fragment} },
% caption={[Code Fragment] \mbox{}\\ \vspace{0.1in} \framebox{\large \bf{Code Fragment} \small} },
\newcommand{\mylistcodequick}[1]{%
\lstset{%
  language=#1, 
  frame=shadowbox,framerule=2pt, 
  basicstyle=\small\bfseries,
  backgroundcolor=\color{black!16},
  showstringspaces=false,
  showspaces=false,showtabs=false,
  keywordstyle=\color{black!40}\bfseries,
  numbers=left,numberstyle=\tiny,stepnumber=5,numbersep=5pt,
  stringstyle=\ttfamily,
  caption={[\quad Code Fragment] \large \bf{Code Fragment} \small},   
  belowcaptionskip=20 pt,  
  label={},
  xleftmargin=17pt,
  framexleftmargin=17pt,
  framexrightmargin=5pt,
  framexbottommargin=4pt,
  breaklines=true} 
}

%  caption={[#2] \mbox{}\\ \vspace{0.1in} \framebox{\large \bf{#2} \small} },
\newcommand{\mylistcodequicktwo}[2]{%
\lstset{%
  language=#1, 
  frame=shadowbox,framerule=2pt, 
  basicstyle=\small\bfseries,
  extendedchars=true,
  backgroundcolor=\color{black!16},
  showstringspaces=false,
  showspaces=false,
  showtabs=false,
  keywordstyle=\color{black!40}\bfseries,
  numbers=left,numberstyle=\tiny,stepnumber=5,numbersep=5pt,
  stringstyle=\ttfamily,
  caption={[\quad#2] \large \bf{#2} \small},
  belowcaptionskip=20 pt,
  label={},
  xleftmargin=17pt,
  framexleftmargin=17pt,
  framexrightmargin=5pt,
  framexbottommargin=4pt,
  breaklines=true} 
}

%  caption={[#2] \mbox{}\\ \vspace{0.1in} \framebox{\large \bf{#2} \small} },
\newcommand{\mylistcodequickthree}[2]{%
\lstset{%
  language=#1, 
  frame=shadowbox,framerule=2pt, 
  basicstyle=\small\bfseries,
  extendedchars=true,
  backgroundcolor=\color{black!16},
  showstringspaces=false,
  showspaces=false,
  showtabs=false,
  keywordstyle=\color{black!40}\bfseries,
  numbers=left,numberstyle=\tiny,stepnumber=5,numbersep=5pt,
  stringstyle=\ttfamily,
  caption={[\quad#2] \large\bf{#2}\small},
  belowcaptionskip=20 pt,
  label={},
  xleftmargin=17pt,
  framexleftmargin=17pt,
  framexrightmargin=5pt,
  framexbottommargin=4pt,
  breaklines=true} 
}

%  frame=single,
\newcommand{\mylistset}[4]{%
\lstset{language=#1,
  basicstyle=\small,
  showstringspaces=false,
  showspaces=false,showtabs=false,
  keywordstyle=\color{black!40}\bfseries,
  numbers=left,numberstyle=\tiny,stepnumber=5,numbersep=5pt,
  stringstyle=\ttfamily,
  caption={[\quad#2]#3},
  label=#4}
}

\newcommand{\mylstinlineset}{%
\lstset{%
  basicstyle=\color{blue}\bfseries\ttfamily,
  showstringspaces=false,
  showspaces=false,showtabs=false,
  breaklines=true}
}

\newcommand{\myframedtext}[1]{%
\centering
\noindent
%\fbox{\parbox[c]{.9\textwidth}{\color{black!40} \small \singlespacing #1\onehalfspacing \normalsize \\}}
\fbox{\parbox[c]{.9\textwidth}{\color{black!40} \small  #1 \normalsize \\}}
}

\newcommand{\myemptybox}[2]{% from , to
\fbox{\begin{minipage}[t][#1in][c]{#2in}\hspace{#2in}\end{minipage}}
}

\newcommand{\myemptyboxtwo}[2]{% from , to
\centering\fbox{
\begin{minipage}{#1in}
\hfill\vspace{#2in}
\end{minipage}
}
}

\newcommand{\boldvector}[1]{
\boldsymbol{#1}
}

\newcommand{\dEdY}[2]{\frac{d E}{d Y_{#1}^{#2}}}
\newcommand{\dEdy}[2]{\frac{d E}{d y_{#1}^{#2}}}
\newcommand{\dEdT}[2]{\frac{\partial E}{\partial T_{{#1} \rightarrow {#2}}}}
\newcommand{\dEdo}[1]{\frac{\partial E}{\partial o^{#1}}}
\newcommand{\dEdg}[1]{\frac{\partial E}{\partial g^{#1}}}
\newcommand{\dYdY}[4]{\frac{\partial Y_{#1}^{#2}}{\partial Y_{#3}^{#4}}}
\newcommand{\dYdy}[4]{\frac{\partial Y_{#1}^{#2}}{\partial y_{#3}^{#4}}}
\newcommand{\dydY}[4]{\frac{\partial y_{#1}^{#2}}{\partial Y_{#3}^{#4}}}
\newcommand{\dydy}[4]{\frac{\partial y_{#1}^{#2}}{\partial y_{#3}^{#4}}}
\newcommand{\dydT}[4]{\frac{\partial y_{#1}^{#2}}{\partial T_{{#3} \rightarrow {#4}}}}
\newcommand{\dYdT}[4]{\frac{\partial Y_{#1}^{#2}}{\partial T_{{#3} \rightarrow {#4}}}}
\newcommand{\dTdT}[4]{\frac{\partial T_{{#1} \rightarrow {#2}}}{\partial T_{{#3} \rightarrow {#4}}}}
\newcommand{\ssum}[2]{\sum_{#1}^{#2}}

\newcommand{\ssty}[1]{\scriptscriptstyle #1}

\newcommand{\myparbox}[2]{%
\parbox{#1}{\color{black!20} #2}
}

\newcommand{\bs}[1]{
\boldsymbol{#1}
}

\newcommand{\parone}[2]{%
\frac{\partial #1 }{ \partial #2 }
}
\newcommand{\partwo}[2]{%
\frac{ \partial^2 {#1} }{ \partial {#2}^2 }
}

\newcommand{\twodvectorvarfun}[2]{
\left [
\begin{array}{r}
{{#1_{\ssty{1}}}(#2)} \\
{{#1_{\ssty{2}}}(#2)}
\end{array}
\right ]
}
\newcommand{\twodvectorvarprimed}[1]{
\left [
\begin{array}{r}
{{#1_{\ssty{1}}}'(t)} \\
{{#1_{\ssty{2}}}'(t)}
\end{array}
\right ]
}

\newcommand{\complex}[2]{#1 \: #2 \: \boldsymbol{i}}
\newcommand{\complexmag}[2]%
{
\sqrt{(#1)^2 \: + \: (#2)^2}
}
\newcommand{\threenorm}[3]%
{
\sqrt{(#1)^2 \: + \: (#2)^2 \: + \: (#3)^2}
}
\newcommand{\norm}[1]{\mid \mid #1 \mid \mid}

\newcommand{\myderiv}[2]{\frac{d #1}{d #2}}
\newcommand{\myderivb}[2]{\frac{d}{d #2} \left ( #1 \right )}
\newcommand{\myrate}[3]%
{#1^\prime(#2) &=& #3 \: #1(#2)
}
\newcommand{\myrateexter}[4]%
{#1^\prime(#2) &=& #3 \: #1(#2) \: + \: #4
}
\newcommand{\myrateic}[3]%
{#1( \: #2 \:) &=& #3 
}

\newcommand{\mytwodsystemeqn}[6]{
#1 \: x    #2 \: y &=& #3\\
#4 \: x    #5 \: y &=& #6\\
}

\newcommand{\mytwodsystem}[8]{
#3 \: #1 \: + \: #4 \: #2 &=& #5\\
#6 \: #1 \: + \: #7 \: #2 &=& #8\\
}  

\newcommand{\mytwodarray}[4]{
\left [
\begin{array}{rr}
#1 & #2\\
#3 & #4
\end{array}
\right ]
}

\newcommand{\mytwoid}{
\left [
\begin{array}{rr}
1 & 0\\
0 & 1
\end{array}
\right ]
}

\newcommand{\myxprime}[2]{
\left [
\begin{array}{r}
#1^\prime(t)\\
#2^\prime(t)
\end{array}
\right ]
}

\newcommand{\myxprimepacked}[2]{
\left [
\begin{array}{r}
#1^\prime\\
#2^\prime
\end{array}
\right ]
}

\newcommand{\myx}[2]{
\left [
\begin{array}{r}
#1(t)\\
#2(t)
\end{array}
\right ]
}

\newcommand{\myxonly}[2]{
\left [
\begin{array}{r}
#1\\
#2
\end{array}
\right ]
}

\newcommand{\myv}[2]{
\left [
\begin{array}{r}
#1\\
#2
\end{array}
\right ]
}

\newcommand{\myxinitial}[2]{
\left [
\begin{array}{r}
#1(0)\\
#2(0)
\end{array}
\right ]
}

\newcommand{\twodboldv}[1]{
\boldsymbol{#1}
}

\newcommand{\mytwodvector}[2]{
\left [
\begin{array}{r}
#1\\
#2
\end{array}
\right ]
}

\newcommand{\mythreedvector}[3]{
\left [
\begin{array}{r}
#1\\
#2\\
#3
\end{array}
\right ]
}

\newcommand{\mytwodsystemvector}[6]{
\left [
\begin{array}{rr}
#1 & #2\\
#4 & #5
\end{array}
\right ]
\:
\left [
\begin{array}{r}
x \\
y 
\end{array}
\right ]
&=&
\left [
\begin{array}{r}
#3\\
#6
\end{array}
\right ]
}

\newcommand{\mythreedarray}[9]{
\left [
\begin{array}{rrr}
#1 & #2 & #3\\
#4 & #5 & #6\\
#7 & #8 & #9
\end{array}
\right ]
}

\newcommand{\myodetwo}[6]{
#1 \: #6^{\prime \prime}(t) \: #2 \: #6^{\prime}(t) \: #3 \: #6(t) &=& 0\\
#6(0)                                           &=& #4\\
#6^{\prime}(0)                                  &=& #5
}

\newcommand{\myodetwoNoIC}[4]{
#1 \: #4^{\prime \prime}(t) \: #2 \: #4^{\prime}(t) \: #3 \: #4(t) &=& 0
}

\newcommand{\myodetwopacked}[5]{
\hspace{-0.3in}& & #1 u^{\prime \prime} #2 u^{\prime} #3 u \: = \: 0\\
\hspace{-0.3in}& & u(0) \: = \: #4, \: \: u^{\prime}(0)    \: = \:  #5
}

\newcommand{\myodetwoforced}[6]{
#1\: u^{\prime \prime}(t) \: #2 \: u^{\prime}(t) \: #3 \: u(t) &=& #6\\
u(0)                                           &=& #4\\
u^{\prime}(0)                                  &=& #5\\
}

\newcommand{\myodesystemtwo}[8]{
#1 \: x^\prime(t) \: #2 \: y^\prime(t) \: #3 \: x(t) \: #4 \: y(t) &=& 0\\
#5 \: x^\prime(t) \: #6 \: y^\prime(t) \: #7 \: x(t) \: #8 \: y(t) &=& 0\\
}

\newcommand{\myodesystemtwoic}[2]{
x(0)                                       &=& #1\\ 
y(0)                                       &=& #2
}

\newcommand{\mypredprey}[4]{
x^\prime(t) &=& #1 \: x(t) \: - \: #2 \: x(t) \: y(t)\\
y^\prime(t) &=& -#3 \: y(t) \: + \: #4 \: x(t) \: y(t)
}

\newcommand{\mypredpreypacked}[4]{
x^\prime &=& #1 \: x - #2 \: x \: y\\
y^\prime &=& -#3 \: y + #4 \: x \: y
}

\newcommand{\mypredpreyself}[6]{
x^\prime(t) &=&  #1 \: x(t) \: - \: #2 \: x(t) \: y(t) \: - \: #3 \: x(t)^2\\
y^\prime(t) &=& -#4 \: y(t) \: + \: #5 \: x(t) \: y(t) \: - \: #6 \: y(t)^2
}

\newcommand{\mypredpreyfish}[5]{
x^\prime(t) &=&  #1 \: x(t) \: - \: #2 \: x(t) \: y(t) \: - \: #5 \: x(t)\\
y^\prime(t) &=& -#3 \: y(t) \: + \: #4 \: x(t) \: y(t) \: - \: #5 \: y(t)
}

\newcommand{\myepidemic}[4]{
S^\prime(t) &=& - #1 \: S(t) \: I(t)\\
I^\prime(t) &=&   #1 \: S(t) \: I(t) \: - \: #2 \: I(t)\\
S(0)        &=&   #3\\
I(0)        &=&   #4\\
}

\newcommand{\bsred}[1]{%
\textcolor{red}{\boldsymbol{#1}}
}

\newcommand{\bsblue}[1]{%
\textcolor{blue}{\boldsymbol{#1}}
}


\newcommand{\myfloor}[1]{%
\lfloor{#1}\rfloor
}

\newcommand{\cubeface}[7]{%
\begin{bmatrix}
\bs{#3}          & \longrightarrow & \bs{#4}\\
\uparrow          &                         &  \uparrow  \\
\bs{#1} & \longrightarrow & \bs{#2}\\
              & \text{ \bfseries #5:} \: \bs{#6} \: \text{\bfseries  #7 } & 
\end{bmatrix}
}

\newcommand{\cubefacetwo}[5]{%
\begin{bmatrix}
\bs{#3}          & \longrightarrow & \bs{#4}\\
\uparrow          &                         &  \uparrow  \\
\bs{#1} & \longrightarrow & \bs{#2}\\
              & \text{ \bfseries #5} & 
\end{bmatrix}
}

\newcommand{\cubefacethree}[9]{%
\begin{bmatrix}
\bs{#3}                  & \overset{#9}{\longrightarrow} & \bs{#4}\\
\uparrow \: #7         &                                             &  \uparrow  \: #8 \\
\bs{#1}                  & \overset{#6}{\longrightarrow} & \bs{#2}\\
                               & \text{ \bfseries #5} & 
\end{bmatrix}
}

\renewcommand{\qedsymbol}{\hfill \blacksquare}
\newcommand{\subqedsymbol}{\hfill \Box}
%\theoremstyle{plain}

\newtheoremstyle{mystyle}% name
  {6pt}%      Space above
  {6pt}%      Space below
  {\itshape}%         Body font
  {}%         Indent amount (empty = no indent, \parindent = para indent)
  {\bfseries}% Thm head font
  {}%        Punctuation after thm head
  { }%     Space after thm head: " " = normal interword space; \newline = linebreak
  {}%         Thm head spec (can be left empty, meaning `normal')
\theoremstyle{mystyle}
 
\newtheorem{axiom}{Axiom}
%\newtheorem{solution}{Solution}[section]
\newtheorem*{solution}{Solution}
\newtheorem{exercise}{Exercise}[section]
\newtheorem{theorem}{Theorem}[section]
\newtheorem{proposition}[theorem]{Proposition}
\newtheorem{prop}[theorem]{Proposition}
\newtheorem{assumption}{Assumption}[section]
\newtheorem{definition}{Definition}[section]
\newtheorem{comment}{Comment}[section]
\newtheorem*{question}{Question}
\newtheorem{program}{Program}[section]
%\newtheorem{myproof}{Proof}
%\newtheorem*{myproof}{Proof}[section]
\newtheorem{myproof}{Proof}[section]
\newtheorem{hint}{Hint}[section]
\newtheorem*{phint}{Hint}
\newtheorem{lemma}[theorem]{Lemma}
\newtheorem{example}{Example}[section]
      
\newenvironment{myassumption}[4]
{
\centering
\begin{assumption}[{\textbf{#1}\nopunct}]%
\index{#2}
\mbox{}\\  \vskip6pt \colorbox{black!15}{\fbox{\parbox{.9\textwidth}{#3}}}
\label{#4}
\end{assumption}
%\renewcommand{\theproposition}{\arabic{chapter}.\arabic{section}.\arabic{assumption}} 
}%
{}

\newenvironment{myproposition}[4]
{
\centering
\begin{proposition}[{\textbf{#1}\nopunct}]%
\index{#2} 
\mbox{}\\  \vskip6pt \colorbox{black!15}{\fbox{\parbox{.9\textwidth}{#3}}}
\label{#4}
\end{proposition}
%\renewcommand{\theproposition}{\arabic{chapter}.\arabic{section}.\arabic{proposition}} 
}%
{}

\newenvironment{mytheorem}[4]
{
\centering
\begin{theorem}[{\textbf{#1}\nopunct}]%
\index{#2} 
\mbox{}\\ \vskip6pt \colorbox{black!15}{\fbox{\parbox{.9\textwidth}{#3}}}
\label{#4}
\end{theorem}
%\renewcommand{\thetheorem}{\arabic{chapter}.\arabic{section}.\arabic{theorem}} 
}%
{}

\newenvironment{mydefinition}[4]
{
\centering
\begin{definition}[{\textbf{#1}\nopunct}]%
\index{#2} 
\mbox{}\\  \vskip6pt \colorbox{black!15}{\fbox{\parbox{.9\textwidth}{#3}}}
\label{#4}
\end{definition}
%\renewcommand{\thedefinitio{n}{\arabic{chapter}.\arabic{section}.\arabic{definition}} 
}%
{}

\newenvironment{myaxiom}[4]
{
\centering
\begin{axiom}[{\textbf{#1}\nopunct}]%
\index{#2} 
\mbox{}\\  \vskip6pt \colorbox{black!15}{\fbox{\parbox{.9\textwidth}{#3}}}
\label{#4}
\end{axiom}
%\renewcommand{\theaxiom}{\arabic{chapter}.\arabic{section}.\arabic{axiom}} 
}%
{}

\newenvironment{mylemma}[4]
{
\centering
\begin{lemma}[{\textbf{#1}\nopunct}]%
\index{#2} 
\mbox{}\\  \vskip6pt \colorbox{black!15}{\fbox{\parbox{.9\textwidth}{#3}}}
\label{#4}
\end{lemma}
%\renewcommand{\thelemma}{\arabic{chapter}.\arabic{section}.\arabic{lemma}} 
}%
{}
   
\newenvironment{reason}[1]
{
 \vskip0.05in
 \begin{myproof}
 \mbox{}\\#1
 $\qedsymbol$
 \end{myproof}  
 \vskip0.05in
}%
{}

\newenvironment{reasontwo}[1]
{
 \vskip+.05in
 \begin{myproof}
 \mbox{}\\#1
 \end{myproof}  
 \vskip+0.05in
}%
{}

\newenvironment{subreason}[1]
{
 \vskip0.05in
 \renewcommand{\themyproof}{}
 \begin{myproof}
 #1
 $\subqedsymbol$
 \end{myproof}
 \vskip0.05in
 \renewcommand{\themyproof}{\thetheorem}
 %\renewcommand{\themyproof}{\arabic{chapter}.\arabic{section}.\arabic{myproof}}   
 %
}%
{}  

\newenvironment{myhint}[1]
{
 \vskip0.05in
 \begin{hint}
 #1
 $\subqedsymbol$ 
 \end{hint}  
 \vskip0.05in
}%
{} 

\newenvironment{myeqn}[3]
{
 \renewcommand{\theequation}{$\boldsymbol{#1}$}
 \begin{eqnarray}
 \label{equation:#2}
 #3 
 \end{eqnarray}
 \renewcommand{\theequation}{\arabic{chapter}.\arabic{eqnarray}}   
}%
{} 


\JournalInfo{MATH 8210:  Homework Five, 1-\pageref{LastPage}, 2020} % Journal information
\Archive{Draft Version \today} % Additional notes (e.g. copyright, DOI, review/research article)

\PaperTitle{MATH 8210 Homework Five}
\Authors{Ian Davis\textsuperscript{1}}
\affiliation{\textsuperscript{1}\textit{John E. Walker Department of Economics,
Clemson University,Clemson, SC: email ijdavis@g.clemson.edu}}
\affiliation{*\textbf{Corresponding author}: yournamehere@clemson.edu} % Corresponding author

\date{\small{Version ~\today}}
\Abstract{Compactness and Bounded Operators}
\Keywords{}
\newcommand{\keywordname}{Keywords}
%
\onehalfspacing
\begin{document}

\flushbottom

\addcontentsline{toc}{section}{Title}
\maketitle

\renewcommand{\theexercise}{\arabic{exercise}}%

\section{Compactness}

Here are some problems about compactness.

\begin{exercise}
Let $X = C([0,1])$ with the supremum norm.
Define the set $S$ by 
\begin{eqnarray*}
S &=& \{ \bs{y} \in X \: | \: y(t) = \int_0^t \: f(x(s)) \: ds, \: \text{ for some } \bs{x} \in X \text{ with } \|\bs{x}\|_\infty \leq 1 \}
\end{eqnarray*}
\noindent
where $f(u) = u^2$.
\begin{itemize}
\item Note if $\bs{y} \in S$, then $\|\bs{y}\|_\infty \leq 2$ so that $S$ is a uniformly bounded subset of $X$.
\item Note if $\bs{y} \in S$, then for any $\bs{y_1}$ and $\bs{y_2}$ in $S$
\begin{eqnarray*}
|y_1(t)-y_2(t)| &\leq& 2 \|\bs{x_1} - \bs{x_2}\|_\infty 
\end{eqnarray*}
\item Note if $\bs{y} \in S$, then
\begin{eqnarray*}
|y(t_1)-y_(t_2)| &\leq&   \|\bs{x}\|^2_\infty \: |t_1 - t_2| < |t_1 - t_2|
\end{eqnarray*}
\noindent
Show this says $S$ is a uniformly equicontinuous family in $C([0,1])$.
\item Use the Arzela - Ascoli Theorem to prove $S$ is compact.
\item Prove $S$ is therefore nowhere dense in $C([0,1])$.
\end{itemize}
\end{exercise}

\begin{solution}
  First, we want to show why
  \begin{eqnarray*}
    |y(t_1)-y_(t_2)| &\leq&   \|\bs{x}\|^2_\infty \: |t_1 - t_2| < |t_1 - t_2|
  \end{eqnarray*}
  says S is a uniformly equicontinuous family in C([0,1]). To do this, pick some $\varepsilon > 0$ and set $\delta = \frac{\varepsilon}{\|\bs{x}\|^2_\infty }$, then we get $|t_1-t_2| < \delta \implies$
  \begin{eqnarray*}
    |y(t_1)-y_(t_2)| &\leq&   \|\bs{x}\|^2_\infty \: |t_1 - t_2| < \|\bs{x}\|^2_\infty \: \delta = \|\bs{x}\|^2_\infty \: \frac{\varepsilon}{\|\bs{x}\|^2_\infty } = \varepsilon
  \end{eqnarray*}
  and, because $\varepsilon$ was arbitrarily chosen, we know it holds for all $\varepsilon > 0$. Hence, S is a uniformly continuous family on C([0,1])\\
  \\
  Next, we want to use Arzela - Ascoli to prove S is compact. To do so, recall that Arzela - Ascoli tells us that there exists a sequence $(y_n) \subset S$ that converges uniformily to a continuous function y on [0,1] and we know that the theorem holds because S is uniformly bounded and equicontinuous. We identify Q with the enumeration $(r_n)$. Since S is UB by 2, the set of points $\{(y(r_1))|y \in S\}$ is bounded by 2 also. By Bolzano - Weierstrass, there is a sequence of distinct functions $(y_n^1)$ from S so that $(y_n^1(r_1))$ converges to a real number $\alpha_1$. Continuing this process for $\{(y(r_2))|y \in S\}$, there is a sequence of disctinct functions $(y_n^2) \subset (y_n^1)$ from S so that $(y_n^2(r_2)$ converges to a real number $\alpha_2$. Note, since $(y_n^2) \subset (y_n^1)$, we know $(y_n^2(r_2))$ also converges to $\alpha_1$. We continue this process on to get the chain of sequences
  \begin{eqnarray*}
    ... \subset (y_n^p) ... \subset (y_n^3) \subset (y_n^2) \subset (y_n^1)
  \end{eqnarray*}
  so that for each k, $(y_n^k)$ converges at $r_1,...,r_k$ to $\alpha_1,...,\alpha_k$. The sequence looks like
  \begin{eqnarray*}
    (y_n^k) = ((y_n^k)_1),(y_n^k)_2),(y_n^k)_3),...,(y_n^k)_k),...)
  \end{eqnarray*}
  where the $(y_n^k)_j)$ is the $j^{th}$ element of the sequence. If we fix the rational number $r_p$ and consider the sequence $((y_n^k)_k)(r_p)$. This is a subsequence of $(y_n^p)(r_p)$ when $k > p$ and so converges to $\alpha_p$. So this sequence converges on Q. For a given $\epsilon > 0$, there is a $N(r_k)$ so that $p,q > N(r_k)$ implies
  \begin{eqnarray*}
    |(y_n^p)_p(r_k) - (y_n^q)q(r_k)| < \frac{\epsilon}{3}
  \end{eqnarray*}
  Since the familiy is equicontinuous, for each $x \in [0,1]$, there is a $\delta_x > 0$ so that $|y(x) - y(t)| < \frac{\epsilon}{3}$ for all y in S and $|x - t| < \delta_x$ and we are also able to get $p,q > N(r_l)$ implies
  \begin{eqnarray*}
    |(y_n^p)_p(r_l) - (y_n^q)_q(r_l)| < \frac{\epsilon}{3}
  \end{eqnarray*}
  Thus, if $p,q > N(r_l), |(y_n^p)_p(x) - (y_n^q)_q(x)| < \epsilon$. We can do this for any x in [0,1] with only picking a new $r_l$. So if we let $N = max\{N_{r_{l_1}},....,N_{r_{l_P}}\}$we see $p,q > N$ implies $|(y_n^p)_p(x) - (y_n^q)_q(x)| < \epsilon$ which satisfies the Cauchy Criterion for uniform convergence on [0,1]. Hence, there is a continuous function x on [0,1] so that $(f_n^k)_k \rightarrow x$ uniformly on [0,1]. Hence, S is compact.\\
  \\
  Lastly, we want to prove S is nowhere dense in C([0,1]). To do so, start with the fact that we have proven S is compact in C([0,1]) which tells us that S is closed and bounded. We then assume that S has an interior point \textbf{p} and there is an $r > 0$ so that $B(\bs{p};r) \subset S$ which also implies $\overline{B(\bs{p};\frac{r}{2})} \subset S$. This implies the translation $-\bs{p} + \overline{B(\bs{p};\frac{r}{2})} = \overline{B(0;\frac{r}{2})}$ is compact and the scaled set $\frac{2}{r}(-\bs{p} + \overline{B(\bs{p};\frac{r}{2})}) = \overline{B(0;1)}$ is compact. But C([a,b]) is infinitely dimensionalso our assumption of S having an interior point must not hold. Hence, we can conlude S is nowhere dense. 
\end{solution}

\begin{exercise}
Let $X = C([0,1])$ with the supremum norm.
Define the set $S$ by 
\begin{eqnarray*}
S &=& \{ \bs{y} \in X \: | \: y(t) = \int_0^t \: f(x(s)) \: ds, \: \text{ for some } \bs{x} \in X \text{ with } \|\bs{x}\|_\infty \leq 1 \}
\end{eqnarray*}
\noindent
where $f$ is continuous and globally Lipschitz on $\Re$; i.e. there is a $K > 0$ so that
$|f(u) - f(v)| < K |u - v|$ for all $u$ and $v$ is $\Re$.  Note this says
$|f(x_1(t)) - f(x_2(t))| < K|x_1(t) - x_2(t)|$.
\begin{itemize}
\item Note if $\bs{y} \in S$, then there is a $B > 0$ so that $\|\bs{y}\|_\infty \leq B $. Hence, $S$ is a uniformly bounded subset of $X$.
\item Note if $\bs{y} \in S$, then for any $\bs{y_1}$ and $\bs{y_2}$ in $S$
\begin{eqnarray*}
|y_1(t)-y_2(t)| &\leq& K \|\bs{x_1} - \bs{x_2}\|_\infty 
\end{eqnarray*}
\item Note if $\bs{y} \in S$, then
\begin{eqnarray*}
|y(t_1)-y_(t_2)| &\leq&   \max_{0 \leq u \leq 1} |f(u)| \: |t_1 - t_2|
\end{eqnarray*}
\noindent
Show this says $S$ is a uniformly equicontinuous family in $C([0,1])$.
\item Use the Arzela - Ascoli Theorem to prove $S$ is compact.
\item Prove $S$ is therefore nowhere dense in $C([0,1])$.
\end{itemize}
\end{exercise}

\begin{solution}
  First, we want to show why
  \begin{eqnarray*}
    |y(t_1)-y_(t_2)| &\leq&   \max_{0 \leq u \leq 1} |f(u)| \: |t_1 - t_2|
  \end{eqnarray*}
  tells us that S is a uniformly equicontinuous family on in C([0,1]). To do so, we pick $\varepsilon > 0$ and set $\delta = \frac{\varepsilon}{\max_{0 \leq u \leq 1} |f(u)|}$. This gives us $|t_1-t_2| < \delta \implies$
  \begin{eqnarray*}
    |y(t_1)-y_(t_2)| &\leq&   \max_{0 \leq u \leq 1} |f(u)| \: |t_1 - t_2| < \max_{0 \leq u \leq 1} |f(u)| \: \delta = \max_{0 \leq u \leq 1} |f(u)| \: \frac{\varepsilon}{\max_{0 \leq u \leq 1} |f(u)|} = \varepsilon
  \end{eqnarray*}
  \begin{center}
    \textbf Next, we want to use Azrela Ascoli to prove S is compact. Because S is uniformly bounded by K, the exact argument from 1 holds to show S is compact. 
  \end{center}
  Becauase we have now shown that S is continuous by Azrela - Ascoli, we know by theorem 5.4.5 (the proof of which is above in the final part of exercise 1), that compact subsets of an indfinite dimensional normed linear space are nowhere dense.
\end{solution}

\section{Bounded Operators}

\begin{exercise}
Let $T : \ell^2 \rightarrow \ell^2$ be the right shift operator; i.e. if $\bs{x} = \{x_1, x_2, \ldots \}$
then $T(\bs{x}) = \{x_2, x_3, \ldots \}$.  Show $\sup_{\|\bs{x}\|_2 = 1} \|T(\bs{x})\|_2 \leq 1$.
\end{exercise}

\begin{solution}
  Note that $||x||_2 = 1 \implies \sum_{n+1}^\infty(x_n^2)^{1/2} = 1$. This gives us two important cases to consider.\\
  1. $x_1 \neq 0$\\
  \begin{eqnarray*}
    ||T(x)||_2 = ||x||_2 - x_1 = 1 - x < 1\\
    \implies sup||T(x)||_2 < 1
  \end{eqnarray*} 
  2. $x_1 = 0$
  \begin{eqnarray*}
    ||T(x)||_2 = ||x||_2 - x_1 = 1 - 0 = 1
    \implies \implies sup||T(x)||_2 = 1
  \end{eqnarray*}
  Hence, $\implies \implies sup||T(x)||_2 \leq 1$
\end{solution}

\begin{exercise}
Let $A$ be an $n \times n$ matrix with $n$ linearly independent
eigenvectors $\bs{E_i}$ with real eigenvalues $\lambda_i$.  Let
$\| \cdot \|$ denote any norm used on $\Re^n$ for both the domain and
the range. Prove

\begin{eqnarray*} \sup_{\|\bs{x}\| = 1} \|A(\bs{x})\|
&\leq& \frac{EL}{c} \frac{\sum_{i=1}^n |x_i|}{\sum_{i=1}^n |x_i|} =
\frac{EL}{c}
\end{eqnarray*}

\noindent
where the components of any $\bs{x}$ are denoted by $x_i$ as usual, $L
= \max_{1 \leq i \;eq n} |\lambda_i|$, $E = \max_{1 \leq i \leq n}
\|\bs{E_i}\|$ and $c$ is the constant we obtain by invoking the Linear
Combination Theorem.
\end{exercise}

\begin{solution}
Your solution here.
\end{solution}

\section{Rotating Surfaces}

We are going to do some problems with Hessians in the context
of graduate analysis, but let's review a bit to set the stage.
We go through  these ideas both by
hand and using MatLab so you can see how both approaches work.
\index{Symmetric Matrices!Rotating Surfaces}
Consider the surface $z = 3x^2 + 2xy + 4y^2$ which is a rotated
paraboloid with elliptical cross sections.  Let's fine the angle $\theta$ so that
the change of variable $\bs{R_{\theta}} \begin{bmatrix} x \\y \end{bmatrix}$
to the new coordinates $\overline{x}$ and $\overline{y}$ leads to a surface
equation with no cross terms $\overline{x} \overline{y}$.  First note
\begin{eqnarray*}
3x^2 + 2xy + 4y^2 &=&
\begin{bmatrix} x\\y\end{bmatrix}^T 
\: \begin{bmatrix} 3 & 1 \\ 1 & 4 \end{bmatrix}
\: \begin{bmatrix} x\\y\end{bmatrix}
\end{eqnarray*}
\noindent
where we let
\begin{eqnarray*}
\bs{H} &=& \begin{bmatrix} 3 & 1 \\ 1 & 4 \end{bmatrix}
\end{eqnarray*}
\noindent
The matrix $\bs{H}$ here is symmetric so using the eigenvectors and eigenvalues 
of $\bs{H}$,
we can write
\begin{eqnarray*}
\begin{bmatrix} \bs{E_1} & \bs{E_2}\end{bmatrix}^T
\: \begin{bmatrix} 3 & 1 \\ 1 & 4 \end{bmatrix}
\: \begin{bmatrix} \bs{E_1} & \bs{E_2}\end{bmatrix}
&=&
\begin{bmatrix} \lambda_1 & 0 \\ 0 & \lambda_2 \end{bmatrix}
\end{eqnarray*}
\noindent
The orthonormal basis $\{ \bs{E_1}, \bs{E_2}\}$ can be identified
with either a rotation $\bs{R_\theta}$ or a reflection $\bs{R_\theta^r}$.
We will choose the rotation.  Then using
the change of variables
\begin{eqnarray*}
\begin{bmatrix} \overline{x}\\\overline{y}\end{bmatrix}
&=& \bs{R_\theta} \: \begin{bmatrix} x\\y\end{bmatrix}
\end{eqnarray*}
\noindent
we find
\begin{eqnarray*}
z &=&
\begin{bmatrix} x\\y\end{bmatrix}^T  \bs{R_\theta}^T
\: 
\begin{bmatrix} 3 & 1 \\ 1 & 4 \end{bmatrix}
\: \bs{R_\theta} \begin{bmatrix} x\\y\end{bmatrix}
\: = \:
\lambda_1 \overline{x}^2 + \lambda_2 \overline{y}^2.
\end{eqnarray*}
\noindent
The characteristic equation for $\bs{H}$ is
\begin{eqnarray*}
det(\lambda \bs{I} - \bs{H}) &=& \lambda^2 -7 \lambda +11 = 0
\end{eqnarray*}
\noindent
with roots
\begin{eqnarray*}
\lambda_1 &=& \frac{7 + \sqrt{5}}{2} = 4.618, \quad
\lambda_2 = \frac{7 - \sqrt{5}}{2} = 2.382
\end{eqnarray*}
\noindent
It is straightforward to find the corresponding eigenvectors
\begin{eqnarray*}
\bs{W_1} = \begin{bmatrix} 1 \\ \frac{1+\sqrt{5}}{2} \end{bmatrix}, \quad
\bs{W_1} = \begin{bmatrix} 1 \\ \frac{1-\sqrt{5}}{2} \end{bmatrix},
\end{eqnarray*}
\noindent
Then 
\begin{eqnarray*}
\| \bs{W_1}\|_2^2 &=& \sqrt{ 1 + \left(\frac{1+\sqrt{5}}{2}\right)^2}
\Longrightarrow 
\| \bs{W_1}\|_2 = \sqrt{3.618} = 1.902\\
\| \bs{W_2}\|_2^2 &=& \sqrt{ 1 + \left(\frac{1-\sqrt{5}}{2}\right)^2}
\Longrightarrow 
\| \bs{W_2}\|_2 = \sqrt{1.382} = 1.176
\end{eqnarray*}
\noindent
Then the orthonormal basis we need is
\begin{eqnarray*}
\bs{E_1} &=& \frac{\bs{W_1}}{\| \bs{W_1}\|_2} 
= \begin{bmatrix} .526\\ .8505 \end{bmatrix}\\
\bs{E_2} &=& \pm \frac{\bs{W_2}}{\| \bs{W_2}\|_2} 
= \pm \begin{bmatrix} .8505 \\ -.526 \end{bmatrix}
\end{eqnarray*}
\noindent
If we choose the minus choice, we get
\begin{eqnarray*}
\bs{E_2} &=& \begin{bmatrix} -.8505 \\ .526 \end{bmatrix}
\end{eqnarray*}
\noindent
and $\begin{bmatrix} \bs{E_1}, \bs{E_2} \end{bmatrix}$
is the rotation matrix $\bs{R_\theta}$.  If we choose the plus choice, we get
\begin{eqnarray*}
\bs{E_2} &=& \begin{bmatrix} .8505 \\ -.526 \end{bmatrix}
\end{eqnarray*}
\noindent
and $\begin{bmatrix} \bs{E_1}, \bs{E_2} \end{bmatrix}$
is the reflected rotation matrix $\bs{R_\theta^r}$.  Our rotation matrix is then
\begin{eqnarray*}
\bs{R_\theta} &=& \begin{bmatrix} .526 & -.8505\\ .8505 & .526 \end{bmatrix}
 \: = \:
  \begin{bmatrix} \cos(1.017) & -\sin(1.017)\\ \sin(1.017)& \cos(1.017) \end{bmatrix}
\end{eqnarray*}
\noindent
where $\theta = 1.017$ radians.
As a check, 
\begin{eqnarray*}
\begin{bmatrix} .526 & .8505\\ -.8505 & .526 \end{bmatrix}
\: \begin{bmatrix} 3 & 1 \\ 1 & 4 \end{bmatrix}
\: \begin{bmatrix} .526 & -.8505\\ .8505 & .526 \end{bmatrix}
&=&
\begin{bmatrix} .526 & .8505\\ -.8505 & .526 \end{bmatrix}
\: \begin{bmatrix} 2.4285 & -2.0255 \\ 3.928 & 1.2535 \end{bmatrix}\\
&= &
\begin{bmatrix} 4.618& 0\\ 0 & 2.382 \end{bmatrix}
\: = \:
\begin{bmatrix} \lambda_1& 0\\ 0 & \lambda_2 \end{bmatrix}
\end{eqnarray*}
\noindent
The change of variables is
\begin{eqnarray*}
\begin{bmatrix} \overline{x}\\\overline{y}\end{bmatrix}
&=& \bs{R_\theta} \: \begin{bmatrix} x\\y\end{bmatrix}
\: = \:
\begin{bmatrix} .526 & -.8505\\ .8505 & .526 \end{bmatrix}
\: \begin{bmatrix} x\\y\end{bmatrix}
\: = \:
\begin{bmatrix} .526 \:x  - .8505 \:y\\.8505 \:x + .526\: y\end{bmatrix}
\end{eqnarray*}
\noindent
Now let's do the same computations in MatLab.

\myfancyverbatim{Surface Rotation: Eigenvalues and Eigenvectors}
\begin{lstlisting}
>> H = [3,1;1,4]
H =

   3   1
   1   4

>> [W,D] = eig(H)
W =

  -0.85065   0.52573
   0.52573   0.85065

D =

Diagonal Matrix

   2.3820        0
        0   4.6180
\end{lstlisting}
\lstset{fancyvrb=false}

\noindent
The command \mylstinline{eig} returns the unit eigenvectors as columns of
$\bs{V}$ and the corresponding eigenvalues as the diagonal entries of $\bs{D}$.
You can see by looking at the matrix $\bs{V}$ the columns are not set up
correctly to be a rotation matrix.
We pick our orthonormal basis from $\bs{V}$ and hence
our rotation matrix $\bs{R}$ as follows:

\myfancyverbatim{Surface Rotation: Setting the Rotation Matrix}
\begin{lstlisting}
>> E1 = W(:,2)
E1 =

   0.52573
   0.85065

>> E2 = W(:,1)
E2 =

  -0.85065
   0.52573

>> R = [E1,E2]
R =

   0.52573  -0.85065
   0.85065   0.52573
\end{lstlisting}
\lstset{fancyvrb=false}

\noindent
We then check the decomposition.

\myfancyverbatim{Surface Rotation: Checking the decomposition}
\begin{lstlisting}
>> R'*H*R
ans =

   4.61803   0.00000
  -0.00000   2.38197
\end{lstlisting}
\lstset{fancyvrb=false}

\noindent
You can practice on these problems if you want:\\

\begin{itemize}
\item Analyze $4x^2 + 6xy + 7y^2$.
\item Analyze $-7x^2 + 2xy + 8y^2$.
\item Analyze $2x^2 -8xy + 6y^2$.
\item Analyze $5x^2 + 12xy + 2y^2$.
\end{itemize}

\noindent
Now on to the real problems.

\begin{exercise}
Let's study the Hessian of a nice two dimensional map.
Let $f(x,y) = 2x^2 + 5xy + 6y^2 + 7x + 9y + 10$.
\begin{itemize}
\item Find the Hessian for this map and its eigenvalues and eigenvectors.
Note we can choose the eigenvectors to be an orthonormal basis for $\Re^2$.
\item Use the $\| \cdot\|_2$ norm on $\Re^2$ here and prove
\begin{eqnarray*}
\|H\| = \sup_{\|\bs{x} = 1} \|H(\bs{x})\| & \leq & \sqrt{|\lambda_1|^2 |x_1|^2 + |\lambda_2|^2 |x_2|^2}
\leq \max\{ |\lambda_1|,|\lambda_2|\}
\end{eqnarray*}
\item Prove $\|H\|  \geq \max\{ |\lambda_1|,|\lambda_2|\}$
\item Combining, this proves $\|H\|  = \max\{ |\lambda_1|,|\lambda_2|\}$.
\end{itemize}
\noindent
Note make sure you fill in the numerical values here also.
\end{exercise}

\begin{solution}
  The hession for this map is
  \begin{eqnarray*}
    \begin{bmatrix}
      4 & 5\\
      5 & 12
    \end{bmatrix}
  \end{eqnarray*}
  and, using the coding provided, we get
  \begin{eqnarray*}
    \lambda_1 &=& 1.5969\\
    \lambda_2 &=& 14.14031\\
    E_1 &=&
    \begin{vmatrix}
      -0.9013\\
      0.43319
    \end{vmatrix}
    \\
    E_2 &=&
    \begin{vmatrix}
      0.43319\\
      -0.9013
    \end{vmatrix} 
  \end{eqnarray*}
\end{solution}

\begin{exercise}
Using the approach of the last exercise, now consider
Let $f(x,y) = 5x^2 + 15xy + 8y^2 + 70x + 90y + 30$.
\begin{itemize}
\item Find the Hessian for this map and its eigenvalues and eigenvectors.
Note we can choose the eigenvectors to be an orthonormal basis for $\Re^2$.
\item Use the $\| \cdot\|_2$ norm on $\Re^2$ here and prove
\begin{eqnarray*}
\|H\| = \sup_{\|\bs{x} = 1} \|H(\bs{x})\| & \leq & \sqrt{|\lambda_1|^2 |x_1|^2 + |\lambda_2|^2 |x_2|^2}
\leq \max\{ |\lambda_1|,|\lambda_2|\}
\end{eqnarray*}
\item Prove $\|H\|  \geq \max\{ |\lambda_1|,|\lambda_2|\}$
\item Combining, this proves $\|H\|  = \max\{ |\lambda_1|,|\lambda_2|\}$.
\end{itemize}
\noindent
Note make sure you fill in the numerical values here also.
\end{exercise}

\begin{solution}
  The hession for this map is
  \begin{eqnarray*}
    \begin{bmatrix}
      10 & 15\\
      15 & 16
    \end{bmatrix}
  \end{eqnarray*}
  and, using the coding provided, we get
  \begin{eqnarray*}
    \lambda_1 &=& -2.2971\\
    \lambda_2 &=& 28.2971\\
    E_1 &=&
    \begin{vmatrix}
      -0.77334\\
      0.63399
    \end{vmatrix}
    \\
    E_2 &=&
    \begin{vmatrix}
      0.63399\\
      -0.7733
    \end{vmatrix} 
  \end{eqnarray*}
\end{solution}

\end{document}