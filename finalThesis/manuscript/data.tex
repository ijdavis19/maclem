\chapter{Data and Variables}
\section{Sulfamethoxazole-Trimethoprim}
\indent I focus my analysis on prescriptions of sulfamethoxazole-trimethoprim. This decision was made because of the antibiotic's generic counterpart entering the market in the middle of the observational period and a sufficient number of prescriptions occurring in the data. Sulfamethoxazole-trimethoprim is a combination antibiotic from the class antimetabolite/sulfonamide and was first introduced in 1968. The brand versions of the drug include Bactrim, Bactrim DS, Septra, and Septra DS and the generic form entered the American market in July of 2012. The antibiotic was popular even before entry of the generic due its high familiarity among physicians and low cost \cite{noauthor_sulfamethoxazole_nodate,ho_considerations_2011}.\\
\indent The drug has FDA approval to fight urinary tract infections, ear infections (acute otitis media), acute exacerbations of chronic bronchitis in adults, Shigellosis, treatment and prophylaxis of \textit{Pneumocystis jirovecii} pneumonia, and Traveler's diarrhea in adults. The antibiotic is also approved for use against infections due to \textit{Listeria, Nocardia, Salmonella, Brucella, Paracoccidioides,} melioidosis, \textit{Burkholderia, Stenotrophomonas,} cyclospora, isospora, Whipple's disease, and alternative therapy for toxoplasmosis and community-acquired MRSA skin infections \cite{schlossberg_antibiotics_2017}.

\section{National Ambulatory Medical Care Survey}
Data used are from the National Ambulatory Medical Care Survey (NAMCS) which is a nationally representative survey of outpatient medical visits. Included in the scope of the survey are freestanding clinics/urgicenters, community health centers, mental health centers, health maintenance organizations,  non-federal government clinics, family practice plans, and private solo or group practices. Not included are hospital emergency or outpatient departments, ambulatory surgicenters, institutional settings such as schools or prisons, industrial outpatient facilities, clinics operated by the federal government, and laser vision surgery centers \cite{hing_basic_nodate}. The surveys include information about the patient, the visit, and the provider seen. Weights are provided in order to create national estimates.\\
\indent I pool observations from the years 2006 to 2016 and drop variables which are not consistently tracked across this time or have more than 30\% missing values as instructed in the survey documentation \cite{myrick_understanding_nodate}. For the specific cases of diagnoses and prescriptions, the maximum amount of available entries increased during the study period. The 2006 NAMCS survey provided three slots to record diagnoses and eight slots to record prescriptions. This set up means that even if more than three diagnoses were made during a medical visit, only three of them would be recorded as there was no option in the survey to add additional diagnoses. The same restriction applies in the case where more than eight medications were prescribed. In 2012, the maximum number of medications recorded was raised to twelve and rose again in 2014 to thirty. The maximum number of diagnoses recorded was raised from 3 to 5 in 2014 as well. In order to accurately measure trends across the study period, I only use the first three diagnoses and the first eight prescriptions as instructed in the survey documentation \cite{schappert_analyzing_nodate}.\\
\indent In order to restrict observations to only those which may have led to a prescription of sulfamethoxazole-trimethoprim, I track all diagnoses which occurred during visits where the antibiotic was prescribed. It must be noted that, while sulfamethoxazole and trimethoprim can be each prescribed on their own, the combination sulfamethoxazole-trimethoprim is distinct enough to be considered its own entity. Hence, when considering visits where sulfamethoxazole-trimethoprim was prescribed, I only consider visits where the distinct combination is prescribed and not when each element of the combination is prescribed seperately. For the sake of analysis, these diagnoses which occured during visits where sulfamethoxazole-trimethoprim was prescribed are considered relevant diagnoses. Then, all visits where one of these relevant diagnoses are made are then marked as relevant visits. This characterization indicates that at least one of the diagnoses made during this visit could have led to the prescription of sulfamethoxazole-trimethoprim based on the behavior of other prescribing physicians. Hence, this visit is considered relevant because it could have led to a prescription of sulfamethoxazole-trimethoprim.\\
\indent Visits where no diagnosis made ever leads to a prescription of the antibiotic are dropped from the sample. For years 2006-2015 diagnoses are labeled using ICD-9-CM codes and ICD-10-CM codes are used for the year 2016. To allow for comparability across all years of the study, each ICD-10-CM code was recoded as its exact or closest ICD-9-CM counterpart. Because all relevant diagnoses are given equal importance regardless of whether it was the specific one which led to the prescription, it is possible that this strategy does not fully rid the sample of all non relevant visits which would negatively bias estimates. An illustration of this process as well as an example of a possible shortcoming are included in the appendix.\\
\indent To further control for nonrelevant visits, I create an indicator for visits where at least one diagnosis is associated with an FDA approved use of sulfamethoxazole-trimethoprim. The reasons for prescribing antibiotics can be categorized as on-label and off-label. On-label uses of the antibiotic are the FDA approved reasons for prescribing sulfamethoxazole-trimethoprim mentioned previously while off-label indications are non FDA approved uses. Each on-label indication is mapped to one or more ICD-9-CM codes shown in the appendix.figure 3.1 shows the 12-month moving average probability of prescription for on-label and off-label uses of sulfamethoxazole-trimethoprim conditional on the medical visit being considered relevant.
\begin{figure}
\centering
\includegraphics{twelveMonthMAonoff.png}
\caption{12-Month Moving-Average Probabilities of Prescription of Sulfamethoxazole-Trimethoprim by Type of Diagnosis}
\end{figure}
\newpage
\section{Variables}
The independent variables considered in the analysis are as follows. \textbf{TimeSinceGeneric} is a continuous variable from -79 to 52 and indicates the number months since the first full month in which the generic version of sulfamethoxazole-trimethoprim was in the market the market, August of 2012. Defining the timeline this way means that the months before August of 2012 take negative values. For example, June of 2012 (2 months before generic entry) would be coded as \textbf{TimeSinceGeneric}$=-2$. I include \textbf{Age} and \textbf{AgeSQ} which are the ages and squared ages of the patient at the time of the visit. \textbf{OffLabel} indicates that none of the diagnoses which resulted from the visit were associated with an FDA approved usage of sulfamethoxazole-trimethoprim. \textbf{GovInsurance} and \textbf{NonWhite} indicate whether a patient was on Medicare or Medicaid and if the patient was of an ethnicity other than white. Controlling for Medicare and Medicaid help to control for differences in the choice set faced by consumers due to insurance and \textbf{NonWhite} is used as a proxy for lower income patients.\\
\indent \autoref{tab:Table3.1} is a statistical summary of the continuous variables \textbf{TimeSinceGeneric} and \textbf{age}. I present the variables in the context of the entire study followed by summaries for before and after entry of the generic. The study goes across 131 months from January of 2006 (\textbf{TimeSinceGeneric}$=-82$) to December of 2016 (\textbf{TimeSinceGeneric}$=49$) with \textbf{TimeSinceGeneric}$=0$ indicating August of 2012 which was the first full month the generic sulfamethoxazole-trimethoprim had been in the market. It is important to note that \textbf{TimeSinceGeneric}$=0$ is included in "After Entry of Generic". The final item of note from the table is the average age of the patient during the study period increased in the time after entry of the generic from 45.2 years old to 47.1 years old.\\
%\begin{landscape}
\begin{tabular}{l*{6}{c}}
\hline\hline
            Variable&\multicolumn{1}{c}{Time frame}&\multicolumn{1}{c}{Weighted Mean}&\multicolumn{1}{c}{Weighted Median}&\multicolumn{1}{c}{Standard Deviation}&\multicolumn{1}{c}{Minimum}&\multicolumn{1}{c}{Maximum}\\
\hline
\textbf{TimeSinceGeneric}                    &     2006-2026&             -15.316&    -17&   37.167&     -79&  52\\
(Time in months since entry &     Before Entry of Generic&     -39.703&    -40&    22.497 &     -98&  -1\\
 of generic)   &     After Entry of Generic&       25.013 &    25&      14.792&     0&  52\\
[1em]
\textbf{Age}                                 &     2006-2026&             45.917&    50&    25.09 &     0&  100\\
(Age of patient in years)           &     Before Entry of Generic&     45.221&    49&    25.207&     0&  100\\
                                    &     After Entry of Generic&      47.069&    51&    24.853&     0&  92\\
[1em]
\textbf{AgeSQ}                               &     2006-2026&             2737.892&    2500&  2218.939&     0&  10000\\
(Age of patient squared)   &     Before Entry of Generic&    2680.279&    2401&  2216.963 &     0&  10000\\
                                    &     After Entry of Generic&      2833.167&    2601&  2218.929 &     0&  8464\\
\hline
$\text{Sample Size for Years 2006-2016} = 399245$\\
$\text{Before Entry of Generic} = 230182$\\
$\text{After Entry of Generic} = 169063$\\
\hline\hline
\multicolumn{4}{l}{\footnotesize All observations after July 2012 are considered to be after entry of generic.}\\
\end{tabular}

\newpage
%\end{landscape} 
\indent \autoref{tab:Table3.2} is a statistical summary of the categorical variables \textbf{offLabel}, \textbf{govInsurance}, and \textbf{NonWhite}. The vast majority of visits in the sample did not have a diagnosis associated with an on-label use of sulfamethoxazole-trimethoprim. These visits accounted for over 96\% of the weighted sample across each time period. Patients on government insurance (Medicare or Medicaid) made up over a quarter of visits over the entire time period and their weighted share of the sample increased from 24.8\% before entry of the generic to 28\% during the time after the generic was introduced. Nonwhite patients make up less of the sample with a weighted average of 16.4\% over the entire study. Similar to the government insurance group, this category saw an increase in their weighted proportion of the sample after the generic came on from 16\% to 17.1\%.
%\begin{landscape}
\begin{tabular}{l*{4}{c}}
\hline\hline
            Variable&\multicolumn{1}{c}{Time frame}&\multicolumn{1}{c}{Total}&\multicolumn{1}{c}{Weighted Share of Sample}\\
\hline
\textbf{OffLabel}                                                &     Entire Study&             387264&      .967\\
(=1 if no diagnoses made were FDA approved          &     Before Entry of Generic&    223267&      .966\\
indications of sulfamethoxazole-trimethoprim)  &     After Entry of Generic&      163997&      .968\\
[1em]
\textbf{GovInsurance}                                            &     Entire Study&             105273&      .26 \\
(=1 if patient is on either Medicare or Medicaid)       &     Before Entry of Generic&     30480 &      .248\\
                                                        &     After Entry of Generic&      44793 &      .28\\
[1em]
\textbf{NonWhite}                                                &     Entire Study&             61442&      .164\\
(=1 if patient is a race other than white)                &     Before Entry of Generic&     37733&      .16\\
                                                        &     After Entry of Generic&      23709&      .171\\
\hline
$\text{Sample Size for Years 2006-2016} = 399245$\\
$\text{Before Entry of Generic} = 230182$\\
$\text{After Entry of Generic} = 169063$\\
\hline\hline
\multicolumn{4}{l}{\footnotesize All observations after August 2012 are considered to be after entry of generic.}\\
\end{tabular}

%multicolumn{5}{l}{"Share of Sample" and "Proportion Prescribed ST" are both weighted proportions}\\
%\multicolumn{5}{l}{offLabel(=1) indicates no diagnoses made were on label indicators, govInsurance(=1) indicates patient is on Medicare or Medicaid,}\\
%\multicolumn{5}{l}{nonWhite(=1) indicates patient is race other than white}\\
%\end{tabular}
%\label{tab:Table4.2}
%\end{table}

%\end{landscape}
\autoref{tab:Table4.4} provides the weighted proportion of patients of each type who were prescribed sulfamethoxazole-trimethoprim. Overall, .861\% of visits lead to a prescription of sulfamethoxazole-trimethoprim over the study period with an increase of .014 percentage points post entry of the generic. Looking only at visits that did not have a diagnosis associated with an on-label use of sulfamethoxazole-trimethoprim, the numbers do not change a great deal. The weighted proportion prescribed the drug was .718\% across the study and increased from .7\% to .749\% between the time before the generic entered and after. The story does change for the compliment of the off-label category, however. Visits that resulted in at least one diagnosis associated with an on-label use of sulfamethoxazole-trimethoprim led to a prescription of the antibiotic 5.07\% of the time. However, this proportion decreased from 5.35\% to 4.78\% between the two time periods.\\
\indent The other categories share a similar trend with the entire sample. Patients on government insurance were prescribed sulfamethoxazole-trimethoprim for .871\% of their total visits and saw an increase from .803\% to .97\%. The proportion of patients not on any form of government insurance who were prescribed the antibiotic fell slightly from .873\% to .831\%. For nonwhite patients, the weighted proportion of visits leading to a prescription of the drug was .933\% with an increase from .857\% to 1.05\% between between before entry of the generic and after. The proportion of white patients prescribed the antibiotic decreased from .856\% to .832\%. 
%\begin{landscape}
\begin{table}[htbp]\centering
\def\sym#1{\ifmmode^{#1}\else\(^{#1}\)\fi}
\caption{Proportions of Patients Prescribed Sulfamethoxazole-Trimethoprim (SXT) by Group\label{tab1}}
\begin{tabular}{l*{3}{c}}
\hline\hline
            Variable&\multicolumn{1}{c}{Timeframe}&\multicolumn{1}{c}{Total Prescriptions of SXT}&\multicolumn{1}{c}{Weighted Proportion Prescribed SXT}\\
\hline
\textbf{Total Sample}                                   &     Entire Study&             3340&     .00861\\
                                                        &     Before Generic Entry&    2072&     .00851\\
                                                        &     After Generic Entry&      1268&     .0088\\
[1em]
\textbf{offLabel}                                       &     Entire Study&             2736&     .00718\\
(=1 if no diagnoses made were FDA approved         &     Before Generic Entry&    1663&     .00696\\
indications of Sulfamethoxazole-Trimethoprim)  &     After Generic Entry&      1073&     .0078\\
[1em]
(=0 if at least one diagnosis made during               &     Entire Study&             604&     .0507\\
visit is an FDA approved indication of                  &     Before Generic Entry&    409&     .0535\\
Sulfamethoxazole-Trimethoprim)                          &     After Generic Entry&      195&     .0454\\
[1em]
\textbf{govInsurance}                                   &     Entire Study&             869&     .00871\\
(=1 if patient is on either Medicare of Medicaid)       &     Before Generic Entry&     533 &     .00806\\
                                                        &     After Generic Entry&      336 &     .00975\\
[1em]
(=0 if patient is on neither Medicare nor Medicaid)     &     Entire Study&             2471&     .00858\\
                                                        &     Before Generic Entry&     1539 &     .00865\\
                                                        &     After Generic Entry&      932 &     .00843\\
[1em]
\textbf{nonWhite}                                       &     Entire Study&             534&      .00933\\
(=1 if patient is race other than white)                &     Before Generic Entry&     354&      .00846\\
                                                        &     After Generic Entry&      180&      .0108\\
[1em]
(=0 if patient is white)                                &     Entire Study&             2806&      .00847\\
                                                        &     Before Generic Entry&     1718&      .00852\\
                                                        &     After Generic Entry&      1088&      .00838\\
[1em]
\textbf{UnspecCellAbscess}                              &     Entire Study&             349 &      .175\\
(=1 if patient was diagnosed with an unspecified        &     Before Generic Entry&     224 &      .170\\
 skin abscess or cellulitis)                            &     After Generic Entry&      125  &      .184\\
[1em]
(=0 if patient was not diagnosed with an unspecified    &     Entire Study&             2291 &      .00761\\
skin abscess or cellulitis)                             &     Before Generic Entry&     1848 &      .0075\\
                                                        &     After Generic Entry&      1143  &      .0078\\
\hline
$n(\text{Entire Study}) = 399245$\\
$n(\text{Before Generic Entry}) = 249345$\\
$n(\text{After Generic Entry}) = 149900$\\
\hline\hline
%\multicolumn{5}{l}{"Share of Sample" and "Proportion Prescribed ST" are both weighted proportions}\\
%\multicolumn{5}{l}{offLabel(=1) indicates no diagnoses made were on label indicators, govInsurance(=1) indicates patient is on Medicare or Medicaid, nonWhite(=1) indicates}\\
%\multicolumn{5}{l}{patient is race other than white, UnspecCellAbscess(=1) indicates patient was diagnosed with and unspecified skin abscess or cellulitis}\\
\end{tabular}
\label{tab:Table4.4}
\end{table}

%multicolumn{5}{l}{"Share of Sample" and "Proportion Prescribed ST" are both weighted proportions}\\
%\multicolumn{5}{l}{offLabel(=1) indicates no diagnoses made were on label indicators, govInsurance(=1) indicates patient is on Medicare or Medicaid,}\\
%\multicolumn{5}{l}{nonWhite(=1) indicates patient is race other than white}\\
%\end{tabular}
%\label{tab:Table4.2}
%\end{table}

%\end{landscape}