\chapter{Introduction}
An antibiotic's effectiveness is an exhaustible resource. Efficient use of antibiotics would eventually render them ineffective due to bacterial resistance. Additionally, privately optimal levels of antibiotic usage increases the rate in which these drugs lose their effectiveness. In the United States alone, over 2.8 million Americans per year are sickened by antibiotic resistant bacteria leading to at least 35,000 deaths \cite{centers_for_disease_control_and_prevention_us_antibiotic_2019}. Over usage and over medication of antibiotics amplify these adverse effects \cite{gerber_outpatient_2019}.\\
\indent The entry of manufacturers of generic versions of an antibiotic increases the amount of treatment options available to a patient and their doctor. Furthermore, the entry of generic manufacturers shifts the supply curve to the right which lowers the price of consumption of an antibiotic. This price decrease would lead to an increase in the equilibrium quantity of prescriptions demanded. As consumption increases along with prescriptions, an acceleration in the evolution of resistance may develop.\\
\indent This paper aims to test the first part this hypothesis that entry of generic antibiotic manufacturers leads to an increase in the demand of prescriptions for said antibiotic. To do so, prescription trends of sulfamethoxazole-trimethoprim before and after entry of its generic counterpart in July of 2012 are compared. Data from the National Ambulatory Medical Care Survey, a nationally representative survey of medical visits, are used track prescriptions of sulfamethoxazole-trimethoprim from January of 2006 to December of 2016. Differences in sets of linear probability models are then used to determine the effect of entry of generics on the probability a given patient will be prescribed the antibiotic assuming that an increase in this probability indicates an increase in aggregate consumption. Specific attention is given to patients with Food and Drug Administration (FDA) approved reasons for prescription, also known as on-label indications, of sulfamethoxazole-trimethoprim.\\
\indent I find that, despite negative trends in probability of prescription over time, a small but significant (90\% CI) increase is present in the probability of prescription of sulfamethoxazole-trimethoprim for individuals diagnosed with FDA approved indications. This trend becomes larger and more significant for patients on Medicare or Medicaid and patients who are a race other than white. These changes were not present in individuals diagnosed solely with non FDA approved indications of sulfamethoxazole-trimethoprim although these visits made up a majority of the drug's prescriptions.\\
\indent The market for generic medications as it is known in the United States now did not exist until 1984. Prior to then, FDA rules required generic drug companies to repeat clinical trials which had already been undertaken by their on brand predecessors. Hence, even after a formula had proven to be safe and effective, generic companies would have to subject a similar formula to the same level of scrutiny as it had already overcome. This costly redundancy served as a barrier to entry preventing generic drug companies from competing with larger brands \cite{eban_bottle_2019}.\\
\indent Senators Orrin Hatch and Henry Waxman authored the Drug Price Competition and Patent Term Restoration Act (known now as the Hatch-Waxman Act) which served to lower the cost and expedite the process for FDA approval of generic drugs. The Hatch-Waxman Act, passed in 1984, removed the requirements that a generic medication needed to undergo all of the clinical trials and safety procedures first undertaken by the original product. Instead, generics simply need to prove bioequivalence to the brand name and demonstrate the drugs exhibit a similar behavior inside of the body. Proving bioequivalence requires determining that, in addition to having the same active ingredient, the rate and extent at which the active ingredient becomes available to the body is not significantly different between the branded drug and the generic version being tested \cite{fda_primer}. This new standard greatly lowered the fixed cost of bringing a generic to market and built the foundation for the modern generic drug industry in the United States.\\
\indent In the case of antibiotics, evidence shows that markets behave consistently with economic theory regarding entrance of a close or perfect substitute. Demand for an antibiotic's active ingredient persists beyond patent expiration and and entry of generics. In the 15 to 30 years after initial patent expiration, between 64\% and 99\% demand remains \cite{mansley_utilization_2008}. The price of the antibiotic is negatively correlated, with the number of suppliers \cite{alpern_trends_2017}. Also, the average price decreases significantly upon entry of generic medications \cite{frank_generic_1997, grabowski_brand_1992}. One study found this price decrease to be between 6.6\% and 66\% of the original market price \cite{vondeling_impact_2018}.\\
\indent Empirical evidence has found treatment outcomes using generic medications to be comparable or equivalent to therapies with brand name medications \cite{lin_comparative_2017,desai_comparative_2019} In spite of the evidence, patients still do hold some negative views about generics. A 2015 meta analysis of generic medicine usage found patients have strong opinions that cheaper drugs are of lower quality although doctors do not share such views\cite{dunne_what_2015}. The meta analysis goes on to show, however, that more educated patients are significantly more likely to accept generic treatment and overall trust of generics has improved over time. Finally, a patient's trust in their doctor's judgment also tends to overrule biases that a patient has about cheaper generics.\\

%\indent Extensive research has been conducted on both patient and doctor perceptions of generic medicines, as well as the consequences of usage or non usage of generic drugs both in America and abroad. One study of Turkish healthcare costs estimated the country's total healthcare expenditure could be diminished by 31\% by adopting more widespread use of generic antibiotics \cite{mercanoglu_evaluation_2018}. In the United States, further adoption of generic antiretrovial therapies would save consumers an estimated \$920 Million \cite{walensky_economic_2013}. Additionally, a case study of Ireland found generic medicines to be 20-90\% cheaper than their on brand competition \cite{dunne_review_2013}. Studies have found treatment outcomes using generics and brand name medicines to be comparable or equivalent both abroad \cite{lin_comparative_2017} and in the United States \cite{desai_comparative_2019}.\\ 

