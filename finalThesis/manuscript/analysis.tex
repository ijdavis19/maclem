\chapter{Empirical Analysis and Results}
\indent \autoref{tab:Table4.1} shows the results of the two binary linear probability models. In both cases, the probability of a white patient with an ailment whose treatment is associated with an on-label use of sulfamethozaxole-trimethoprim who is not on government insurance has a statistically nonzero probability of demanding treatment with the antibiotic. The change in this probability of prescription after entry of the generic is 2.03 percentage points and significant at the 90\% confidence level. The probability of being prescribed sulfamethozaxole-trimethoprim was decreasing over time before and after the generic entered. The positive coefficients for the interactions between \textbf{TimeSinceGeneric} and \textbf{OffLabel} are greater than the negative coefficient attached to \textbf{TimeSinceGeneric} in absolute value indiacting the probability of prescription for off-label visits was increasing with time. Significant positive coefficents on \textbf{Age} and significant negative coefficients on \textbf{AgeSQ} indicate that a patient's probability of prescription is increasing with age but this effect decreases as the patient gets older. The interactions between \textbf{Age} and \textbf{TimeSinceGeneric} as well as \textbf{AgeSQ} and \textbf{TimeSinceGeneric} show how this age effect changes over the course of time. During the period of time before the generic had entered, there was a small but significant decrease in the probability of prescription for older patients over time. This effect decreased further as a patient increased in age. This effect loses significance once the generic enters the market.\\
\indent Both patients on Medicare or Medicaid and patients of a race other than white did not have a probability of being prescribed the antibiotic which was significantly different from a white patient not on government insurance before entry of the generic. After entry of the generic, however, patients on government insurance became .185 percentage points more likely to be prescribed sulfamethozaxole-trimethoprim than those not on Medicare or Medicaid. Similarly, patients of a race other than white became .219 percentage points more likely to be prescribed sulfamethozaxole-trimethoprim than their white counterparts. Both of these estimates are significant at the 99\% confidence level. 
\def\sym#1{\ifmmode^{#1}\else\(^{#1}\)\fi}
\begin{tabular}{l*{3}{c}}
\hline\hline
Variable            &\multicolumn{1}{c}{Before Generic Entry}&\multicolumn{1}{c}{After Generic Entry}&\multicolumn{1}{c}{Difference}\\
\hline
\textbf{timeSinceGeneric}&                           -0.00032\sym{***}&   -0.000592\sym{***}&   -.000272\\
(Time in months since generic entry            &     (-6.51)         &     (-6.72)         &     [0.368]         \\
of Sulfamethoxazole-Trimethoprim)\\
[.5em]
\textbf{offLabel}    &                                     -0.0326\sym{***}&     -0.0513\sym{***}&   -.0188\\
(=1 if no diagnoses made were FDA approved            &    (-14.65)         &    (-20.05)         &    [0.0676]         \\
indications of Sulfamethoxazole-Trimethoprim)\\
[.5em]
\textbf{offLabel}$\times$\textbf{timeSinceGeneric} &    0.000372\sym{***}&    0.00047\sym{***}&   .0000984\\
            &                                           (7.7)         &      (5.64)         &    [0.757]         \\
[.5em]
\textbf{age}         &                      0.000172\sym{*}  &    0.0000696\sym{*}  &   -.000103\\
(Age of patient in years)            &      (3.08)         &      (1.06)         &    [0.378]         \\
[.5em]
\textbf{ageSQ}       &                             -0.00000215\sym{**} & -0.00000128\sym{*}  &   0.000000863\\
(Age of patient in years squared)            &     (-3.36)         &     (-1.74)         &    [0.495]         \\
[.5em]
\textbf{age}$\times$\textbf{timeSinceGeneric}      & -0.00000229\sym{**} &  0.00000419         &   0.000000648\\
            &                                        (-1.88)         &      (1.88)         &    [0.107]         \\
[.5em]
\textbf{ageSQ}$\times$\textbf{timeSinceGeneric}    &    .0000000274\sym{**} &   -0.000000031         &   -0.000000059\\
            &                                           (1.98)         &     (-1.23)         &    [0.184]         \\
[.5em]
\textbf{govInsurance}&                                    0.000176         &     0.00185\sym{**} &   0.00168\\
(=1 if patient is on either Medicare or            &      (0.38)         &      (3.54)         &    [0.161]         \\
Medicaid)\\
[.5em]
\textbf{nonwhite}    &                                    -0.00000675         &     0.00219\sym{***}&   0.00222\\
(=1 if patient is race other than white)            &     (-0.01)         &      (3.65)         &    [0.164]         \\
[.5em]
\textbf{\_cons}      &      0.0383\sym{***}&      0.0586\sym{***}&   0.0203\sym{*}\\
            &               (16.83)         &     (21.83)         &    [0.0403]         \\
\hline
r2          &     0.009         &     0.0058\\
N           &      230182         &      169063\\
\hline\hline
\multicolumn{3}{l}{\footnotesize \textit{t} statistics in parentheses, \scalebox{1.25}{$\text{Pr}(\frac{\hat{\beta}^\text{before}_i - \hat{\beta}^\text{after}_i}{[\hat{\sigma}^2\{\hat{\beta}^\text{before}_i\} + \hat{\sigma}^2\{\hat{\beta}^\text{after}_i\}]^\frac{1}{2}} > X^2)$} in brackets}\\
\multicolumn{3}{l}{\footnotesize \sym{*} \(p<0.05\), \sym{**} \(p<0.01\), \sym{***} \(p<0.001\)}\\
\multicolumn{4}{l}{\footnotesize "Probability of prescription" refers to probability that medical visit will have a prescription of Sulfamethoxazole-Trimethoprim}
\end{tabular}
\\
\indent To find the expected changes in probability for patient's before and after entry of the generic, I use the expected value of the categorical variables and their coefficients. I calculate the mean effects of the variables to hold them constant while fixing \textbf{TimeSinceGeneric} = 0. The resulting estimates can be interpreted as the expected probability an individual is prescribed sulfamethozaxole-trimethoprim immediately before and after the antibiotic entered the market. The results are shown in \autoref{tab:Table4.2} along with the results of the same hypothesis test used previously.\\
\indent Patient's with at least one diagnosis associated with an on-label use of sulfamethozaxole-trimethoprim are estimated to be 1.87 percentage points more likely to demand a prescription of the antibiotic immediately after entry of the generic. This increase is significant at the 90\% confidence level. Patients without a diagnosis associated with an on-label use of the antibiotic are estimated to have a much smaller but insignificant increase in the probability of being prescribed the antibiotic. Visits associated with on-label uses with patients who were on Medicare or Medicaid are estimated to be 1.99 percentage points more likely to result in a prescription of the antibiotic after entry of the generic. The same visits but with non-white patients are estimated to be 2.05 percentage points more likely to result in a prescription of sulfamethozaxole-trimethoprim.\\
\begin{table}[htbp]\centering
\def\sym#1{\ifmmode^{#1}\else\(^{#1}\)\fi}
\caption{Estimated Probability of Prescription of Sulfamethoxazole-Trimethoprim Immediately Before and After Generic Entry \label{tab1}}
\begin{tabular}{l*{3}{c}}
\hline\hline
Patient Group  &\multicolumn{1}{c}{(Before Generic Entry)}&\multicolumn{1}{c}{(After Generic Entry)}&\multicolumn{1}{c}{Difference}\\
\hline
\textbf{onLabel}                                               &   0.0405\sym{***}   &   0.0594\sym{***}   &   .019\\
(=1 if at least one diagnosis made during visit is an          &     [0.000]         &     [0.000]         &     [0.0736]       \\
FDA approved indication of Sulfamethoxazole-Trimethoprim)\\
[1em]
\textbf{offLabel}                                     &     0.00756\sym{***}&     .00824\sym{***}  &   .000679\\
(=1 if no diagnoses made were FDA approved            &    [0.000]          &    [.00195]         &    [0.389]         \\
indications of Sulfamethoxazole-Trimethoprim)\\
[1em]
\textbf{govInsurance}$\times$\textbf{onLabel}&    0.0405\sym{***}         &     0.0607\sym{***} &   0.0202\\
(Patient on Medicare or Medicaid        &      [0.000]         &      [0.000]        &    [0.0600]         \\
and \textbf{onLabel}=1)\\
[1em]
\textbf{nonwhite}$\times$\textbf{onLabel}    &  0.0405\sym{***}         &     0.0614\sym{***}&   0.0209\sym{*}\\
(Patient race other than white            &     [0.000]        &      [0.000]         &    [0.0484]         \\
and \textbf{onLabel}=1)\\
[1em]
\textbf{UnspecCellAbscess}                   &  0.228\sym{***}         &     0.435&   0.207\\
(=1 if patient was diagnosed with an unspecified            &     [0.004]        &      [0.000]         &    [0.162]         \\
 skin abscess or cellulitis)\\
\hline\hline
\multicolumn{3}{l}{\footnotesize \scalebox{1.25}{$\text{Pr}(\frac{\hat{\beta}^\text{before}_i - \hat{\beta}^\text{after}_i}{[\hat{\sigma}^2\{\hat{\beta}^\text{before}_i\} + \hat{\sigma}^2\{\hat{\beta}^\text{after}_i\}]^\frac{1}{2}} > X^2)$} in brackets}\\
\multicolumn{3}{l}{\footnotesize \sym{*} \(p<0.05\), \sym{**} \(p<0.01\), \sym{***} \(p<0.001\)}\\
\end{tabular}
\end{table}

