\chapter*{Abstract} %still need to fix first sentence
The nonrenewable nature of antibiotic therapy has led many medical professionals, policy makers, and academics to questions regarding the optimal use of antibiotics. To build out a more robust understanding of antibiotic usage and its contributing factors, careful attention must be paid to the study of antibiotic demand and its characteristics. The purpose of this paper is to estimate the effect of generic entry on antibiotic demand through changes in the probability of prescription using sulfamethoxazole-trimethoprim as a case study. The data used are pooled from outpatient ambulatory care visits from the National Ambulatory Medical Care Surveys over the years 2006-2016. The probabilities of prescription of sulfamethoxazole-trimethoprim for various patient groups who could have been prescribed the drug both before and after entry of an off brand version are estimated. If a patient has at least one diagnosis of a medical condition for which the Food and Drug Administration (FDA) of the United States has approved the use of sulfamethoxazole-trimethoprim, the probability the patient is prescribed the antibiotic initially increases, on average, an estimated 1.87 percentage points in the first months the generic is on sale. Visits where patients were not diagnosed with an FDA approved reason for prescription of the antibiotic saw no significant change in the probability of being prescribed the antibiotic after entry of the generic antibiotic.