\chapter{Conclusions}
\indent While not uniform across all patient groups, I find a small but significant increase in the probability of prescription of sulfamethoxazole-trimethoprim due to generic entry for patients with diagnoses associated with FDA approved indications of the antibiotic. The expected probability of demanding treatment which used the antibiotic increased by 1.87 percentage points (90\% CI) for patients with diagnoses associated with on-label uses of the antibiotic immediately after generic entry. Members of this patient group who were on Medicare or Medicaid or were non white saw increases in expected probability of prescription of 1.99 percentage points (90\% CI) and 2.05 percentage points (95\% CI) respectively. I attribute these changes to a decrease in price caused by generic entry. Patients with no diagnoses associated with FDA approved indications of sulfamethoxazole-trimethoprim did not see significant changes their probability of prescription. This finding is important because these cases made up more the majority of all prescriptions of sulfamethoxazole-trimethoprim.

\section{Limitations of the Study}
These results may be limited by the specific setting from which they come. Because emergency departments are not included in the analysis, populations more likely to leverage those services as well as diagnoses which are more likely to occur in that setting are underrepresented. Second, sulfamethoxazole-trimethoprim is only one antibiotic which had already seen resistance forming before entry of the generic. Because of this, consumers may have been more reluctant to demand the drug as time progressed. Generic entry of antibiotics with less reported resistance may have a higher proportion of patients adopting generic treatment. This study is additionally limited by its selection of variables. Considerations of a patient's gender, region of visit, and characteristics of the physician may reveal some omitted variable bias albeit at the cost of increased complexity within the model. Finally, a more refined process could be implemented in order to determine what visits can be considered relevant in the study of a given antibiotic. Including all diagnoses which lead to the prescription of sulfamethoxazole-trimethoprim may lead to inclusion of irrelevant visits. One possible example would be and individual who is diagnosed with a diagnosis associated with an on-label use of the drug as well one not associated such as hypertension. The method employed in this study would go on to count all visits where a diagnosis of hypertension was made to be relevant visits even though sulfamethoxazole-trimethoprim would not be prescribed for that condition. This occurrence could could negatively bias empirical results. Another possible shortcoming of the method used to select observations is only prescriptions of the distinct entity sulfamethoxazole-trimethoprim were counted as prescriptions. There is the possiblity that some prescriptions could have been recorded as seperate prescriptions of the components sulfamethoxazole and trimethoprim.

\section{Recommendations for Future Research}
The limitations mentioned above serve as directions for future research. First, expanding the scope of the study to include emergency department and non ambulatory care would help provide a more complete understanding of sulfamethoxazole-trimethoprim before and after generic entry. Second, subjecting different drugs to a similar methodology is needed to determine if results from this study are products of specific characteristics of sulfamethoxazole-trimethoprim. Third, inclusion of additional variables about the patient and physician may help lead to additional findings not seen in this study. Fourth, development of a more sophisticated method to control for relevant diagnoses could further ensure unbiased results. Finally, controlling for prescription trends of close substitutes would provide the researcher with an idea of what the opportunity cost of adoption of generics may specifically be. 