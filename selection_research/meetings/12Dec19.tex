%Input preamble
\input{preamble}
\let\counterwithout\relax
\let\counterwithin\relax
\definecolor{maroon}{HTML}{4B0082}

\begin{document}
\noindent \textbf{12/16/2019}\\
\noindent \textbf{4:00 PM Eastern}\\
\bigskip
Updates:\\
\begin{itemize}
\item Collected do files and data from Casey Mulligan's data
\item Read Mulligan and Rubinstein's paper
\item Did not understand most of the paper or the code
\item Want to split time this week continuing to move through their code but also building out a "needs" list to do the analysis
\item Figure out what they were bringing to the table and determine a path forward to replicate it within the DHS data
\begin{itemize}
\item Starts with identifying needs
\item Compare with Krishna
\end{itemize} 
\end{itemize}

More generally
\begin{itemize}
\item Want to be able to replicate the first figure: inequality across gender
\item end goal is figure 3: 'X\% of the closing of the gap is due to selection into the workforce
\item 1: understand the method 2: Build the road map 3: Estimation(not for later)
\item check out DHS datasets for one option
\item IPUMS international is another possible source for standardize census with 
\item DHS v. IPUMS International in terms of which data set do we want to use.
\item If you really wanted to get excited, try and extend this problem to black/white pay gaps in the United States
\item Build out a broad story that explains wage differential beyond just discrimination
\item DHS has quick country data
\end{itemize}

Next Time
\begin{itemize}
\item Have to have a good understanding of the paper
\item Build out that road map of what we need data-wise
\item Maybe make a few slides about what regressions we need to run and what variables we need at least for the bivariate. Identification at infinity should be easier
\item "What is the ideal data set for each country?"
\i
\end{itemize}
\end{document}