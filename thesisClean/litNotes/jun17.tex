% pdflatex litNotes.tex
% bibtex litNotes.aux
% pdflatex litNotes.tex
% pdflatex litNotes.tex



\input{preamble}
\let\counterwithout\relax
\let\counterwithin\relax
\definecolor{maroon}{HTML}{4B0082}

\begin{document}

\noindent\textbf{A System Dynamics Model of Infection Risk, Expectations, and Perceptions on Antibiotic Prescribing in the United States} \cite{kianmehr_system_2020}\\
\begin{itemize}
    \item Simulation which uses parameters taken from NAMCS 1993-2015
    \item Specifically considering Acute Resperiatory Tract Infections
    \item "Simulation results reveal that physician diagnosis for prescribing antibiotics is based on physician's experience from their prior prescribing behaviour, their perception of patient's infection risk, and patient's expectation to receive antibiotics. Also, there are some variations depending on patient's age and residential region. The simulation analysis also depicts the decreasing trend in patient's expectation over the past two decades for most age groups and regions."
\end{itemize}

\noindent\textbf{Characteristics of Demand for Antibiotics in Primary Care: An Almost Ideal Demand System Approach} \cite{filippini_characteristics_nodate}\\
\begin{itemize}
    \item Model of demand for different classes of antibiotics
    \item Specifically looking at demand for resperiatory infection in outpation care settings
    \item Data are from Switzerland
    \item Cites a couple earlier studies about antibiotic demand (Ellison et al. (1997)) and (Chaudhuri (2003)).
    \item In the study, highest elasticity of demand is for 3rd Generation Cephalosporins and Quinolones which are the newest and more expensive categories.
    \item Complementary effects found with relatively narrow spectrum antibiotics and relatively large spectrum
\end{itemize}

\noindent\textbf{National Trends in Incidence of Purulent Skin and Soft Tissue Infections in Patients Presenting to Ambulatory and Emergency Department Settings, 2000-2015} \cite{fritz_national_2020}\\
\begin{itemize}
    \item Uses NAMCS 2000-2015
    \item 69\% of patients received antibiotics
    \item Talks about some of the trappings involved with using NAMCS data.
\end{itemize}

\noindent\textbf{Outpatient Antibiotic Prescribing in the United States: Are Pediatricians Leading the Way?} \cite{gerber_outpatient_2019}\\
\begin{itemize}
    \item Editorial Commentary that does not do much to provide new information but makes important claims about the state of antiboitic perscribing in America and its ramifications.
\end{itemize}

\noindent\textbf{Patients, Doctors, and Contracts: An Application of Principal-Agent Theory to the Doctor-Patient Relationship} \cite{scott_patients_1999}\\
\begin{itemize}
    \item Discrete Choice Experiment to determine patient preferences of the patient-doctor relationship
    \item Clearly describes the history of applying this framework to the doctor-patient relationship
    \item Finds that the utility of younger, female patients is affected more by the level of involement in decision making.
\end{itemize}

\noindent\textbf{The Economics of Direct-to-Consumer Advertising of Prescription Only Drugs} \cite{morgan_economics_2003}\\
\begin{itemize}
    \item Most drugs that are adverstised directly to consumers are patented and brand name drugs
    \item Paper determines that advertising creates biases and incomplete information in prescription drug market
\end{itemize}

\noindent\textbf{The Impact of Pricing and Patent Expiration on Demand for Pharmaceuticals: An Examination of the Use of Broad-Spectrum Antimicrobials} \cite{kaier_impact_2013}\\
\begin{itemize}
    \item German study
    \item Still used ambulatory care settings
    \item In the outpatient setting, all antibiotics studied had significant negative own price elasticities of demand
    \item Only some antibiotics showed this behavior in the hospital setting.
    \item Concludes that price dependency of demand is present in both the ambulatory and hospital settings.
\end{itemize}

\noindent\textbf{Uncertainty and the Welfare Economics of Medical Care} \cite{arrow_uncertainty_2004}\\
\begin{itemize}
    \item citation is not the original publication of the work
    \item Kind of the first paper to deal with the specific trapping of healthcare Economics
    \item Entire section devoted to the role of the physician in the medical care markets
\end{itemize}

\noindent\textbf{Using Stated Preference and Revealed Preference Modeling to Evaluate Prescribing Decisions} \cite{mark_using_2004}\\
\begin{itemize}
    \item Experiment with physicians prescrbing a hypothetical alcoholism medcation
    \item May be too far removed from task at hand to use
    \item Negative correlation between price and liklihood of prescription
\end{itemize}


\newpage

\bibliography{jun17} 
\bibliographystyle{unsrt}

\end{document}