% pdflatex litNotes.tex
% bibtex litNotes.aux
% pdflatex litNotes.tex
% pdflatex litNotes.tex



%Style
\documentclass[12pt]{article}
\usepackage[top=1in, bottom=1in, left=1in, right=1in]{geometry}
\parindent 22pt
\usepackage{fancyhdr}

%Packages
\usepackage{adjustbox}
\usepackage{amsmath}
\usepackage{amsfonts}
\usepackage{amssymb}
\usepackage{bm}
\usepackage[table]{xcolor}
\usepackage{tabu}
\usepackage{color,soul}
\usepackage{makecell}
\usepackage{longtable}
\usepackage{multirow}
\usepackage[normalem]{ulem}
\usepackage{etoolbox}
\usepackage{graphicx}
\usepackage{tabularx}
\usepackage{ragged2e}
\usepackage{booktabs}
\usepackage{caption}
\usepackage{fixltx2e}
\usepackage[para, flushleft]{threeparttablex}
\usepackage[capposition=top,objectset=centering]{floatrow}
\usepackage{subcaption}
\usepackage{pdfpages}
\usepackage{pdflscape}
\usepackage{natbib}
\usepackage{bibunits}
\definecolor{maroon}{HTML}{990012}
\usepackage[bottom]{footmisc}
\usepackage[colorlinks=true,linkcolor=maroon,citecolor=maroon,urlcolor=maroon,anchorcolor=maroon]{hyperref}
\usepackage{marvosym}
\usepackage{makeidx}
\usepackage{tikz}
\usetikzlibrary{shapes}
\usepackage{setspace}
\usepackage{enumerate}
\usepackage{rotating}
\usepackage{tocloft}
\usepackage{epstopdf}
\usepackage[titletoc]{appendix}
\usepackage{framed}
\usepackage{comment}
\usepackage{xr}
\usepackage{titlesec}
\usepackage{footnote}
\usepackage{longtable}
\newlength{\tablewidth}
\setlength{\tablewidth}{9.3in}
\setcounter{secnumdepth}{4}
\usepackage{textgreek}

\titleformat{\paragraph}
{\normalfont\normalsize\bfseries}{\theparagraph}{1em}{}
\titlespacing*{\paragraph}
{0pt}{3.25ex plus 1ex minus .2ex}{1.5ex plus .2ex}
\makeatletter
\pretocmd\start@align
{%
  \let\everycr\CT@everycr
  \CT@start
}{}{}
\apptocmd{\endalign}{\CT@end}{}{}
\makeatother
%Watermark
\usepackage[printwatermark]{xwatermark}
\usepackage{lipsum}
\definecolor{lightgray}{RGB}{220,220,220}
%\newwatermark[allpages,color=lightgray,angle=45,scale=3,xpos=0,ypos=0]{Preliminary Draft}

%Further subsection level
\usepackage{titlesec}
\setcounter{secnumdepth}{4}
\titleformat{\paragraph}
{\normalfont\normalsize\bfseries}{\theparagraph}{1em}{}
\titlespacing*{\paragraph}
{0pt}{3.25ex plus 1ex minus .2ex}{1.5ex plus .2ex}

\setcounter{secnumdepth}{5}
\titleformat{\subparagraph}
{\normalfont\normalsize\bfseries}{\thesubparagraph}{1em}{}
\titlespacing*{\subparagraph}
{0pt}{3.25ex plus 1ex minus .2ex}{1.5ex plus .2ex}

%Functions
\DeclareMathOperator{\cov}{Cov}
\DeclareMathOperator{\corr}{Corr}
\DeclareMathOperator{\var}{Var}
\DeclareMathOperator{\plim}{plim}
\DeclareMathOperator*{\argmin}{arg\,min}
\DeclareMathOperator*{\argmax}{arg\,max}
\DeclareMathOperator{\supp}{supp}

%Math Environments
\newtheorem{theorem}{Theorem}
\newtheorem{claim}{Claim}
\newtheorem{condition}{Condition}
\renewcommand\thecondition{C--\arabic{condition}}
\newtheorem{algorithm}{Algorithm}
\newtheorem{assumption}{Assumption}
\renewcommand\theassumption{A--\arabic{assumption}}
\newtheorem{remark}{Remark}
\renewcommand\theremark{R--\arabic{remark}}
\newtheorem{definition}[theorem]{Definition}
\newtheorem{hypothesis}[theorem]{Hypothesis}
\newtheorem{property}[theorem]{Property}
\newtheorem{example}[theorem]{Example}
\newtheorem{result}[theorem]{Result}
\newenvironment{proof}{\textbf{Proof:}}{$\bullet$}

%Commands
\newcommand\independent{\protect\mathpalette{\protect\independenT}{\perp}}
\def\independenT#1#2{\mathrel{\rlap{$#1#2$}\mkern2mu{#1#2}}}
\newcommand{\overbar}[1]{\mkern 1.5mu\overline{\mkern-1.5mu#1\mkern-1.5mu}\mkern 1.5mu}
\newcommand{\equald}{\ensuremath{\overset{d}{=}}}
\captionsetup[table]{skip=10pt}
%\makeindex

\setlength\parindent{20pt}
\setlength{\parskip}{0pt}

\newcolumntype{L}[1]{>{\raggedright\let\newline\\\arraybackslash\hspace{0pt}}m{#1}}
\newcolumntype{C}[1]{>{\centering\let\newline\\\arraybackslash\hspace{0pt}}m{#1}}
\newcolumntype{R}[1]{>{\raggedleft\let\newline\\\arraybackslash\hspace{0pt}}m{#1}}



%Logo
%\AddToShipoutPictureBG{%
%  \AtPageUpperLeft{\raisebox{-\height}{\includegraphics[width=1.5cm]{uchicago.png}}}
%}

\newcolumntype{L}[1]{>{\raggedright\let\newline\\\arraybackslash\hspace{0pt}}m{#1}}
\newcolumntype{C}[1]{>{\centering\let\newline\\\arraybackslash\hspace{0pt}}m{#1}}
\newcolumntype{R}[1]{>{\raggedleft\let\newline\\\arraybackslash\hspace{0pt}}m{#1}}

\newcommand{\mr}{\multirow}
\newcommand{\mc}{\multicolumn}

%\newcommand{\comment}[1]{}

\let\counterwithout\relax
\let\counterwithin\relax
\definecolor{maroon}{HTML}{4B0082}

\begin{document}

\noindent\textbf{Evaluation of Generic Antibiotic Usage Under the Perspective of Drug Cost} \cite{mercanoglu_evaluation_2018}\\
\begin{itemize}
    \item Specific study about Turkey but quantifies the savings of using generics
    \item Finds that switching to generic antibiotics would reduce total healthcare expenditure in Turkey by 31\%.
\end{itemize}

\noindent\textbf{Off-label use of oral fluoroquinolone antibiotics in outpatient settings in the United States, 2006 to 2012} \cite{almalki_off-label_2016}.\\
\begin{itemize}
    \item \textbf{Paper warrents a careful read!}
    \item Uses NAMCS 2006-2012
    \item Study of how fluoroquinolone antibiotics are used in ambulatory settings
    \item Finds that many times (over 50\%) the drugs are prescribed for indicators "in which the safety and the efficacy are not yet established."
\end{itemize}

\noindent\textbf{Trends in Antibiotic Prescribing for Adults in the United States—1995 to 2002} \cite{roumie_trends_2005}.\\
\begin{itemize}
    \item Study used NAMCS data to trace trends of antibiotic prescriptions from 95-02.
    \item "During the study period, outpatient antibiotic prescribing for respiratory infections where antibiotics are rarely indicated has declined, while the proportion of broad-spectrum antibiotics prescribed for these diagnoses has increased significantly. This trend resulted in a 15\% decline in the total proportion of outpatient visits in which antibiotics were prescribed. However, because outpatient visits increased 17\% over this time period, the population burden of outpatient antibiotic prescriptions changed little.
\end{itemize}

\noindent\textbf{Outpatient Antibiotic Prescribing for Older Adults in the United States: 2011 to 2014} \cite{kabbani_outpatient_2018}.\\
\begin{itemize}
    \item Used IQVIA Xponent database to study trends in prescription of antibiotics across the united states
    \item Prescription rates were stable across study period
    \item Presciption rates were highest in the south and lowest in the west
    \item Most commonly prescribed class was quinolones followed by pennicillins and macrolides
    \item Azithromycin, amoxicillin, and ciprofloxacin were the most commonly prescribed drugs respectively.
\end{itemize}

\noindent\textbf{Comparative analysis of the cost and effectiveness of generic and brand-name antibiotics: the case of uncomplicated urinary tract infection} \cite{lin_comparative_2017}.\\
\begin{itemize}
    \item Uses the Longitudinal Health Insurace Database of Taiwan to look at healthcare costs of individuals with UTIs
    \item Found individuals treated with generics had equivalent outcomes to those treated with brand names but at a much lower cost.
\end{itemize}

\noindent\textbf{Past, Present, and Future of Antibacterial Economics: Increasing Bacterial Resistance, Limited Antibiotic Pipeline, and Societal Implications} \cite{luepke_past_2017}.\\
\begin{itemize}
    \item Big picture look at ramifications of antibiotic usage and current innovation efforts
    \item Discusses need for improved antibiotic stewardship in light of slow down in innovation. 
\end{itemize}

\noindent\textbf{A review of the differences and similarities between generic drugs and their originator counterparts, including economic benefits associated with usage of generic medicines, using Ireland as a case study} \cite{dunne_review_2013}.\\
\begin{itemize}
    \item Primarily focuses on Ireland but does have a section comparing US policy with Europe.
    \item Very good and specific definitions of generics in the United States, what is required by the FDA for approval, and meaning of biosimilarity.
    \item Cites source claiming generics are typically between 20-90\% cheaper than their originators. Source is from the European Union so a number for the United States needs to be found. 
\end{itemize}

\noindent\textbf{What do people really think of generic medicines? A systematic review and critical appraisal of literature on stakeholder perceptions of generic drugs} \cite{dunne_what_2015}.\\
\begin{itemize}
    \item Meta-analysis of literature on PubMed regarding generic medicine usage
    \item  "...as patient trust in their physician often overrules their personal mistrust of generic medicines, enhancing the opinions of physicians regarding generics may have particular importance in strategies to promote usage and acceptance of generic medicines in the future."
    \item Patients hold strong opinion that cheaper drugs are of lower quality
    \item Trust in generics has imporved over time. 
    \item Acceptance of generics is higher in consumers with higher levels of education.
\end{itemize}

\noindent\textbf{Economic Savings Versus Health Losses: The Cost-Effectiveness of Generic Antiretroviral Therapy in the United States} \cite{walensky_economic_2013}.\\
\begin{itemize}
    \item Not very much to say on this paper. Uses a mathematical simulation to show that if all HIV positive individuals switched to a generic treatment savings are estimated to be \$920 Million.
    \item interesting but may not be very useful
\end{itemize}

\noindent\textbf{Comparative effectiveness of generic and brand-name medication use: A database study of US health insurance claims} \cite{desai_comparative_2019}.\\
\begin{itemize}
    \item Recent US study that found "comparable" results between patients using brand name and generic medications.
    \item Should be noted that the study used "Authorized Generics" which are identical in composition and appearance to brand names. Need to check if the Sulfamethoxazole-Trimethroprim fits this category.
\end{itemize}

\newpage

\bibliography{litNotes} 
\bibliographystyle{unsrt}

\end{document}