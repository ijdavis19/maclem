% $Header$

\documentclass{beamer}

% This file is a solution template for:

% - Giving a talk on some subject.
% - The talk is between 15min and 45min long.
% - Style is ornate.



% Copyright 2004 by Till Tantau <tantau@users.sourceforge.net>.
%
% In principle, this file can be redistributed and/or modified under
% the terms of the GNU Public License, version 2.
%
% However, this file is supposed to be a template to be modified
% for your own needs. For this reason, if you use this file as a
% template and not specifically distribute it as part of a another
% package/program, I grant the extra permission to freely copy and
% modify this file as you see fit and even to delete this copyright
% notice. 


\mode<presentation>
{
  \usetheme{Warsaw}
  % or ...

  \setbeamercovered{transparent}
  % or whatever (possibly just delete it)
}


\usepackage[english]{babel}
% or whatever

\usepackage[latin1]{inputenc}
% or whatever

\usepackage{times}
\usepackage[T1]{fontenc}
\usepackage{graphicx}
\graphicspath{{./figures/}}
% Or whatever. Note that the encoding and the font should match. If T1
% does not look nice, try deleting the line with the fontenc.


\title[Economics 8160] % (optional, use only with long paper titles)
{Paid Parental Leave Laws in the United States}

\subtitle
{Does Short-Duration Leave Affect Women's Labor Force Attachment?} % (optional)

\author[Ian Davis] % (optional, use only with lots of authors)
{Tanya S. Byker}
% - Use the \inst{?} command only if the authors have different
%   affiliation.

\institute[American Economic Review] % (optional, but mostly needed)
{American Economic Review: Papers \& Proceedings 2016,\\
106(5): 242-246}
% - Use the \inst command only if there are several affiliations.
% - Keep it simple, no one is interested in your street address.

\date[Short Occasion] % (optional)
{22 October 19}
\begin{document}

\begin{frame}
  \titlepage
\end{frame}

\begin{frame}{Outline}
	\begin{itemize}
	\item
	 Purpose of the Presentation
	\item
	 Policy Specifics and Context
	\item
	 Empirical Methodology
	\item
	 Results
	\item
	 Replication
	\item
	 Concluding Remarks
	\end{itemize}
\end{frame}

\begin{frame}{Purpose of the Presentation}
 
  \begin{itemize}
  \item
    Analyze the effect of paid family leave on female labor force participation
  \item
    Analyze the specific event-study difference-in-differences methodology utilized
  \end{itemize}
\end{frame}

\begin{frame}{Policy Specifics and Context}
	\begin{itemize}		
	\item
	 Family \& Medical Leave Act (1993)
		\begin{itemize}
		\item
		 National Mandate guaranteeing up to 12 weeks of un-paid, job protected medical leave for qualified medical and family reasons
		\item
		 Qualified workers
		\item
		 Eligible firms
		\item
		 Birth of child covered by policy
		\item
		 Studies have shown little impact on woman's labor force attachment
		\end{itemize}
	\item
	 No national mandate for paid family leave has been passed
	\end{itemize}
\end{frame}
	
\begin{frame}{Policy Specifics and Context}
	\begin{itemize}
	\item
	 In 1980, less than 20\% of women reported having used paid parental leave
	\item
	 Substantial gains were made by the late 2000's but not equally distributed
	\begin{itemize}
	\item 44\% of women with bachelor's reported using paid parental leave
	\item 26\% of women with less than a bachelor's degree reported using paid parental leave
	\end{itemize}
	\item
		 By 2016, 5 states categorized pregnancy as a temporary disability, mandating paid time off around the birth of a child.
	\end{itemize}
	
\end{frame}

\begin{frame}{Policy Specifics and Context}
	\begin{itemize}
	\item
	 California Paid Family Leave Act
	\begin{itemize}
	\item
	 July 2004
	\item
	 55\% of wages up to \$1,067 per week up to six weeks
	\end{itemize}
	\item
	 New Jersey Family Leave Insurance
	\begin{itemize}
	\item
	 July 2009
	\item
	 66\% of wages up to \$584 per week up to six weeks
	\end{itemize}
	\item
	 Both laws supplement the temporary disability insurance which already provides 10 weeks of replaced wages
	\end{itemize}
\end{frame}

\begin{frame}{Empirical Methodology}
	\begin{itemize}		
	\item
	 Survey of Income and Program Participants (SIPP)
	\begin{itemize}
		\item
		 National 48 Month panel survey
		\item
		 1996, 2001, 2004, 2008
		\item
		 Large enough to study state level policy changes
	\end{itemize}
	\item
	  Estimate the effect of CA and NJ laws on labor force participation around the event of the birth of child
	 \end{itemize}
\end{frame}

\begin{frame}{Empirical Methodology}
	\begin{itemize}
	\item
	 Treatment group consists of California and New Jersey
	\item
	 Control group consists of New York, Texas, and Florida
	\begin{itemize}
		\item
		 Not the three other states offering temporary disability insurance for birth of a child
		 \item
		  Parallel trends assumption circumvents this issue
	\end{itemize}
	\item
	Use of panel data makes triple difference model no longer needed
	\end{itemize}
\end{frame}

\begin{frame}{Empirical Methodology}
	 $Y_{its} = \alpha_i + \lambda_t +\theta_s\times\lambda_t + \sum_{j=-24}^{24} {\delta_j B_{it}^j} + \sum_{j=-24}^{24} {\gamma_j B_{it}^j} \times\lambda_t + \sum_{j=-24}^{24} {\pi_j B_{it}^j} \times\theta_s + \sum_{j=-24}^{24}{\beta_j B_{it}^j} \times{Policy}_{ts} +\varepsilon_{its}$
\bigskip	
	\begin{itemize}
	\item
	 $Y_{its} \equiv$ labor-force outcome for woman $i$ living in state $s$ in period $t$
	\item
	 $\alpha_i \equiv$ individual fixed effects
	\item 
	 $\lambda_t \equiv$ year fixed effects
	\item
	 $\theta_s \equiv$ state fixed effects
	\item
	 $B^j_{it} \equiv$ set of event-study dummy variables indicating each observation's timing relative to a birth with $j \in [-24,24]$
	\item
	 ${Policy}_{ts} \equiv$ indicator equal to one if a paid parental leave law is in effect in period $t$ in state $s$
	\item
	 $\beta_j \equiv$ vector of coefficients providing monthly estimates of the impact of laws around and after birth in terms of changes from pre-birth levels
	\end{itemize}
\end{frame}

\begin{frame}{Results}
	\begin{center}
	\includegraphics[scale=.5]{fig1_a}
	\end{center}
\end{frame}

\begin{frame}{Results}
	\begin{center}
	\includegraphics[scale=.5]{fig1_b}
	\end{center}
\end{frame}

\begin{frame}{Results}
	\begin{center}
	\includegraphics[scale=.5]{fig2}
	\end{center}
\end{frame}

\begin{frame}{Results}
	\begin{itemize}
	\item
	 P-Values from joint test of significance for months -3 to 3
		 \begin{itemize}
		  \item
		   $p$-value for all mothers = .04
		  \item
		   $p$-value for mothers with less than a bachelor's degree = .02
		  \item
		   $p$-value for mothers with at least a bachelor's degree = .84
	 	 \end{itemize}
	 	 \item
	 P-Values from joint test of significance for months -6 to 12 months after birth
		 \begin{itemize}
		  \item
		   $p$-value for all mother looking for work = .03
	 	 \end{itemize}
	\end{itemize}
\end{frame}
\begin{frame}{Conclusions}
	\begin{itemize}
	\item
	 Short-duration paid leave can potentially, increase labor-force attachment around the birth of a child
	\item
	 Policies are particularly effective for women with comparatively less education
	\end{itemize}
\end{frame}


\end{document}


© 2019 GitHub, Inc.
Terms
Privacy
Security
Status
Help
Contact GitHub
Pricing
API
Training
Blog
About
