%Input preamble
%Style
\documentclass[12pt]{article}
\usepackage[top=1in, bottom=1in, left=1in, right=1in]{geometry}
\parindent 22pt
\usepackage{fancyhdr}

%Packages
\usepackage{adjustbox}
\usepackage{amsmath}
\usepackage{amsfonts}
\usepackage{amssymb}
\usepackage{bm}
\usepackage[table]{xcolor}
\usepackage{tabu}
\usepackage{color,soul}
\usepackage{makecell}
\usepackage{longtable}
\usepackage{multirow}
\usepackage[normalem]{ulem}
\usepackage{etoolbox}
\usepackage{graphicx}
\usepackage{tabularx}
\usepackage{ragged2e}
\usepackage{booktabs}
\usepackage{caption}
\usepackage{fixltx2e}
\usepackage[para, flushleft]{threeparttablex}
\usepackage[capposition=top,objectset=centering]{floatrow}
\usepackage{subcaption}
\usepackage{pdfpages}
\usepackage{pdflscape}
\usepackage{natbib}
\usepackage{bibunits}
\definecolor{maroon}{HTML}{990012}
\usepackage[bottom]{footmisc}
\usepackage[colorlinks=true,linkcolor=maroon,citecolor=maroon,urlcolor=maroon,anchorcolor=maroon]{hyperref}
\usepackage{marvosym}
\usepackage{makeidx}
\usepackage{tikz}
\usetikzlibrary{shapes}
\usepackage{setspace}
\usepackage{enumerate}
\usepackage{rotating}
\usepackage{tocloft}
\usepackage{epstopdf}
\usepackage[titletoc]{appendix}
\usepackage{framed}
\usepackage{comment}
\usepackage{xr}
\usepackage{titlesec}
\usepackage{footnote}
\usepackage{longtable}
\newlength{\tablewidth}
\setlength{\tablewidth}{9.3in}
\setcounter{secnumdepth}{4}
\usepackage{textgreek}

\titleformat{\paragraph}
{\normalfont\normalsize\bfseries}{\theparagraph}{1em}{}
\titlespacing*{\paragraph}
{0pt}{3.25ex plus 1ex minus .2ex}{1.5ex plus .2ex}
\makeatletter
\pretocmd\start@align
{%
  \let\everycr\CT@everycr
  \CT@start
}{}{}
\apptocmd{\endalign}{\CT@end}{}{}
\makeatother
%Watermark
\usepackage[printwatermark]{xwatermark}
\usepackage{lipsum}
\definecolor{lightgray}{RGB}{220,220,220}
%\newwatermark[allpages,color=lightgray,angle=45,scale=3,xpos=0,ypos=0]{Preliminary Draft}

%Further subsection level
\usepackage{titlesec}
\setcounter{secnumdepth}{4}
\titleformat{\paragraph}
{\normalfont\normalsize\bfseries}{\theparagraph}{1em}{}
\titlespacing*{\paragraph}
{0pt}{3.25ex plus 1ex minus .2ex}{1.5ex plus .2ex}

\setcounter{secnumdepth}{5}
\titleformat{\subparagraph}
{\normalfont\normalsize\bfseries}{\thesubparagraph}{1em}{}
\titlespacing*{\subparagraph}
{0pt}{3.25ex plus 1ex minus .2ex}{1.5ex plus .2ex}

%Functions
\DeclareMathOperator{\cov}{Cov}
\DeclareMathOperator{\corr}{Corr}
\DeclareMathOperator{\var}{Var}
\DeclareMathOperator{\plim}{plim}
\DeclareMathOperator*{\argmin}{arg\,min}
\DeclareMathOperator*{\argmax}{arg\,max}
\DeclareMathOperator{\supp}{supp}

%Math Environments
\newtheorem{theorem}{Theorem}
\newtheorem{claim}{Claim}
\newtheorem{condition}{Condition}
\renewcommand\thecondition{C--\arabic{condition}}
\newtheorem{algorithm}{Algorithm}
\newtheorem{assumption}{Assumption}
\renewcommand\theassumption{A--\arabic{assumption}}
\newtheorem{remark}{Remark}
\renewcommand\theremark{R--\arabic{remark}}
\newtheorem{definition}[theorem]{Definition}
\newtheorem{hypothesis}[theorem]{Hypothesis}
\newtheorem{property}[theorem]{Property}
\newtheorem{example}[theorem]{Example}
\newtheorem{result}[theorem]{Result}
\newenvironment{proof}{\textbf{Proof:}}{$\bullet$}

%Commands
\newcommand\independent{\protect\mathpalette{\protect\independenT}{\perp}}
\def\independenT#1#2{\mathrel{\rlap{$#1#2$}\mkern2mu{#1#2}}}
\newcommand{\overbar}[1]{\mkern 1.5mu\overline{\mkern-1.5mu#1\mkern-1.5mu}\mkern 1.5mu}
\newcommand{\equald}{\ensuremath{\overset{d}{=}}}
\captionsetup[table]{skip=10pt}
%\makeindex

\setlength\parindent{20pt}
\setlength{\parskip}{0pt}

\newcolumntype{L}[1]{>{\raggedright\let\newline\\\arraybackslash\hspace{0pt}}m{#1}}
\newcolumntype{C}[1]{>{\centering\let\newline\\\arraybackslash\hspace{0pt}}m{#1}}
\newcolumntype{R}[1]{>{\raggedleft\let\newline\\\arraybackslash\hspace{0pt}}m{#1}}



%Logo
%\AddToShipoutPictureBG{%
%  \AtPageUpperLeft{\raisebox{-\height}{\includegraphics[width=1.5cm]{uchicago.png}}}
%}

\newcolumntype{L}[1]{>{\raggedright\let\newline\\\arraybackslash\hspace{0pt}}m{#1}}
\newcolumntype{C}[1]{>{\centering\let\newline\\\arraybackslash\hspace{0pt}}m{#1}}
\newcolumntype{R}[1]{>{\raggedleft\let\newline\\\arraybackslash\hspace{0pt}}m{#1}}

\newcommand{\mr}{\multirow}
\newcommand{\mc}{\multicolumn}

%\newcommand{\comment}[1]{}

\let\counterwithout\relax
\let\counterwithin\relax
\definecolor{maroon}{HTML}{4B0082}

\begin{document}
\noindent \textbf{Ian Davis}\\
\noindent \textbf{Economics 8050}\\
\noindent \textbf{16 April 2020}\\
\\
Reading the excerpts from Gordon's text was interesting for me. I read the work when I was a freshman in college shortly after it came out but it is absolutely worth revisiting.\\
\\
The introduction lays out Gordon's thesis and begins building the lofty foundation of his defense. In his own words, his thesis is \textit{some policies are more important than others}. This phrase describes both American and worldwide economic growth beyond just the past couple decades but can be extended to nearly all of human history. Clark discussed this idea in his work "Farewell to Alms" where he primarily focused on the difference between the growth (or lackthereof) which occured before the 1800s and after in which the United Kingdom was launched into the modern age. More importantly, the economic gains which occured in light of this innovation remained for the first time in human history.\\
\\
I will admit is unfair to tie these works together becuase they do actaully differ a great deal but the overarching idea is the same in that not all growth is equal. For Clark, it is that not all growth can stay and for Gordon, its that not all growth \textit{matters}. The innovation which occured in the century after the civil war was unlike anything human history had ever seen. Changes in standard of living and quality of life which had previously taken hundred or thousands of years were occured in a matter of decades or even less. What Gordon forecasts is not a stagnation in economic growth. Capital will continue to accumulate and production will continue to move along. What believes we will lose are the gains in standard of living and quality of life.\\
\\
An excellent illustration of this comes is when he discusses the differences between the childhood of youths today v. that of their parents. Then their parent v. their grandparents. These massive jumps from generation to generation is what Gordon argues is not going to be sustained. His point largely comes from comparing the inputs of these changes. In the past 30 years, we have entered the age of computation which may have done wonders for workplace productivity and leisure, but has only marginally changed the standard of living within the US. Additionally, after the computers where integrated into society, moving them from our homes to our pockets has made life easier but not necessary better. Compare the creation of the cell phone to what I believe to be the most important innovation discussed in his work; the connected house. A house that, without requiring much effort from homeowners, brings in clean water and gets rid of dirty water. Much simpler than a smartphone but is one of the inventions which made city life possible. Consider the air conditioner as well which made entire American cities such as Austin, Texas or entire states like Arizona inhabitable.\\
\\
Gordon then goes on to question the current state of innovation and all of his critiques are fair. Specifically, the type of entrepreneurial spark which led us into the age of computers seems to have dissapeared which he defends wit statistics of new patent shares (firms v. individuals). Of note is also the decline of non-academic research firms such as those housed under IBM and Bell Labs. What Gordon fails to discuss however, is the rise of open source software. Operating systems such as Linux and programming languages such as Python have been created outside of the usual economic innovation framework. While I don't believe that github and message boards will continue our growth revolutions, it is an area where a great deal of innovation is being done but is hardly noticed much of the time. 

\end{document}
