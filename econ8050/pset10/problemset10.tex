\documentclass[11pt]{article}
\usepackage{amsmath}
\usepackage{amssymb}
\usepackage{centernot}
\usepackage{enumitem}
\usepackage{pdfpages}
\begin{document}
\includepdf{questions.pdf}
\begin{flushleft}
Ian Davis\\
Economics 8050\\
Spring 2020\\
\bigskip
\textbf{Problem Set 10}\\
\end{flushleft}
\textbf{1. Pestilence in Extended Solow}
\begin{enumerate}[label=\alph*]
    \item In the steady state, the grwoth rate of per worker values all be 0 because no factors are growing endogenously. Because human capital depreciates, we also know that GPD, H, and K will all reach a steady state quantity which will not change in the long run.
    \item The steady state values of $H$ and $T$ come when
    \begin{equation}
        H^* = (\frac{\delta_k}{s_k})^\frac{1}{\beta}K^{*\frac{1 - \alpha}{\beta}}
    \end{equation}
    \item See attached diagram
    \item The endownment of this economy would put it in the northwest quadrant of the phase diagram. From the diagram, we can see that resources will be moved away from investing in human capital and instead be invested in physical capital. This will cause lasting growth in the supply of physical capital and aggregate human capital will be allowed to depreciate. This will occur until the steady state is reached.
    \item If we assume that the labor force never rebounds and stays at the post pestilence levels, the steady state of the economy will be permenently moved the $(0,0)$ point. Additionally, we will see phycial capital depreciate as more resources are spent in investing in human capital until the new steady state is reached. We would also get $H^* > H_0$ and $K^* < K_0$
\end{enumerate}
\textbf{2. Technological Progress in Extended Solow}
\begin{enumerate}[label=\alph*]
    \item Because A will continue to grow endogenously, there will not be a unique steady state. There also won't be a steady state for any of the per worker values. Only per effective worker values will hit a steady state. Finally, total GDP, H, and K will all continue to grow.
    \item We can only derive steady states for physical capital per effective worker and human capital per effective worker. These would occur at 
    \begin{equation}
        \widetilde{k}^* = \frac{s}{n + \delta + s_k}^\frac{1}{1 - \beta - \alpha}\\
        \widetilde{h}^* = (\frac{\delta_k}{s_k})^\frac{1}{\beta}\widetilde{k}^{*\frac{1 - \alpha}{\beta}}
    \end{equation}
    \item See document attached
    \item In this case, the endowment of the economy will begin in the southwest corner of the phase diagram. To get to the steady state, investment in human capital will increase and investment in physical capital will decrease. This will cause permenent changes in the stocks of each capital. We will eventually get to the steady state $(H^*,K^*)$ with $H^* > H_)$ and $K^* < K_0$
    \item This will permenently increase both types of capital per effective worker and increase the rate at which the types of capital grow.
    \item Anticipating this change will increase both levels of investment in order to smooth consumption
\end{enumerate}
\textbf{3. Optimal Consumption in the Solow Model}
\begin{enumerate}[label=\alph*]
    \item $K^* = \frac{.01}{\delta}^\frac{1}{1 - \alpha}X_tN_t$ and $\widetilde{k}^* = \frac{.01}{\delta}^\frac{1}{1 - \alpha}$. Because $X$ and $N$ are constant, the growth paths of capital and capital per effective worker are the same and are only different in their magnitudes.
    \item See graph attached
    \item Yes, this will increase GDP becaues there will be an increase in total capital. This is represented in the Solow diagram by an upward shift in the capital path. It should be noted that this could not hold depending on the value of $\delta$
    \item Depending on the depreciation rate, it could be that $55\%$ is too high and too much will be invested per time period. This would lead to a decrease in GDP.
    \item I would explain to the leader that happiness is not a good measure of an economy's wellbeing. Additionally, household utility functions do not care about the same things as the aggregrate economy. A decision would need to be made about whether or not we want to optimize household happiness or output.
\end{enumerate}

\textbf{4. Development Aid}
\begin{enumerate}[label=\alph*]
    \item Yes, it is possible to have multiple steady states so long as the first steady state is less that $\bar{K}$. Then, the only way to get to the second steady state is to invest until the stock is greater than $\bar{K}$
    \item In the standard model, each economy will reach the same steady state in the long wrong but if the intitial conditions lie on either side of $\bar{K}$, then the economies will go to different steady states.
    \item In the standard model, the one time injection of capital will not do anything to change the steady state.
    \item If this one time capital injection pushes the stock above $\bar{K}$, the steady state will increase to the higher one.
\end{enumerate}
\textbf{5. Smarter Workers Don't Break Their Tools}
\begin{enumerate}[label=\alph*]
    \item $K_{t+1} = K_t + s_KY_t - \delta_KK_t$ and $H_{t+1} = H_t + s_HY_t - \delta_HH_t$
    \item see previous answers regarding phase spaces
    \item see attached phase space. Note that there are two nodal sinks
    \item Becausae L is constant, the new L will remain for the rest of time and cause both steady states to drop
    \item Where the economy goes depends on where in the phase space the this reduction in capital puts the ordered pair $(H,K)$
\end{enumerate}

\end{document}