%Input preamble
\input{preamble}
\let\counterwithout\relax
\let\counterwithin\relax
\definecolor{maroon}{HTML}{4B0082}

\begin{document}
\noindent \textbf{Ian Davis}\\
\noindent \textbf{Economics 8050}\\
\noindent \textbf{27 January 2020}\\
\\
"The Trouble with GDP" is a history of the Gross Domestic Product measure as well as a illuminating look into what information provides and what it lacks. The story of Keynes's original idea of what is now the modern GDP is told as it relates to the economies of the second world war. Later on, many countries and international organizations would leverage this metric as a means to understand their economies at a large scale. Specficially, the United Nations required all member states to account their GDPs as well as all nations receiveing United States aid uner the Marshall Plan.\\
\\
Most interesting are the issues taken with GDP in the article. The author notes many of the underlying assumptions that go into a GDP calculation and how their were either dubious to begin with or have not aged well with our changing economy. The GDP formula, $Y = C + I + G + (X - M)$ was created to reflect the strength of an economy based on manufacturing. As our economy has shifted to one based primarily on services, trouble has arisen regarding how to accurately measure economic action that goes beyond the selling of manufactured widgets. While it is not impossible to measure the value of services, it does lead to some metrics that are not reflective of reality. One such example occurs in the measuring of the financial sector. Becuase financial services has such an effect on the overall economy, would could not simply sum the rates being paid to all traders and believe that is the value added. Instead, national accountants will measure the spread between the risk-free interest rate and the lending rate multiplied by the stock of loans. Becuase this spread is a product of the risk environment itself, flawed assumptions regarding current risk environments can then cause drastic over or under estimations of a country's GDP.\\
\\
GDP fails to be able to accuratly represent the value of the black market economy as well. Illegal activity is not bound to any reporting standards and the assumption that anyone doing at a high value would be ccaught and accounted for is simply flawed. While it may seem like a minor issue if your a picturing one drug dealer on a street corner, adding up all of the street corners in America where drugs are present can become significant fast. Or consider the narcos of central and south America. Unbelievebly high volumes of transactions occured from cocaine and other drugs which rivaled many countries entire economies and all of which would not be accounted for in a measure of GDP.\\
\\
The most important aspect which the GDP fails to capture is a country's quality of life. Yes, GDP per capita is highly correlated with what many would consider to be a high quality of life but it is neither a necessary nor sufficient condition. This sometimes feels surprising and it takes discipline to not equate GDP to quality of life, but when we are reminded what GDP is it becomes obvious. There's a moment in nearly all undergrade economics student's life when the first learn the equation for GDP and, almost for certain, the thought appears of "So this is what politicians make a big deal out of. Why?" And it is a fair question to ask. If every time we talked about GDP or GDP per capita we simple stopped and rephrased our point as it relates to the actual equation of GDP, we would stop talking about GDP as a perfect measure of modern life. 

\end{document}
