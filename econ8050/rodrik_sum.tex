%Input preamble
%Style
\documentclass[12pt]{article}
\usepackage[top=1in, bottom=1in, left=1in, right=1in]{geometry}
\parindent 22pt
\usepackage{fancyhdr}

%Packages
\usepackage{adjustbox}
\usepackage{amsmath}
\usepackage{amsfonts}
\usepackage{amssymb}
\usepackage{bm}
\usepackage[table]{xcolor}
\usepackage{tabu}
\usepackage{color,soul}
\usepackage{makecell}
\usepackage{longtable}
\usepackage{multirow}
\usepackage[normalem]{ulem}
\usepackage{etoolbox}
\usepackage{graphicx}
\usepackage{tabularx}
\usepackage{ragged2e}
\usepackage{booktabs}
\usepackage{caption}
\usepackage{fixltx2e}
\usepackage[para, flushleft]{threeparttablex}
\usepackage[capposition=top,objectset=centering]{floatrow}
\usepackage{subcaption}
\usepackage{pdfpages}
\usepackage{pdflscape}
\usepackage{natbib}
\usepackage{bibunits}
\definecolor{maroon}{HTML}{990012}
\usepackage[bottom]{footmisc}
\usepackage[colorlinks=true,linkcolor=maroon,citecolor=maroon,urlcolor=maroon,anchorcolor=maroon]{hyperref}
\usepackage{marvosym}
\usepackage{makeidx}
\usepackage{tikz}
\usetikzlibrary{shapes}
\usepackage{setspace}
\usepackage{enumerate}
\usepackage{rotating}
\usepackage{tocloft}
\usepackage{epstopdf}
\usepackage[titletoc]{appendix}
\usepackage{framed}
\usepackage{comment}
\usepackage{xr}
\usepackage{titlesec}
\usepackage{footnote}
\usepackage{longtable}
\newlength{\tablewidth}
\setlength{\tablewidth}{9.3in}
\setcounter{secnumdepth}{4}
\usepackage{textgreek}

\titleformat{\paragraph}
{\normalfont\normalsize\bfseries}{\theparagraph}{1em}{}
\titlespacing*{\paragraph}
{0pt}{3.25ex plus 1ex minus .2ex}{1.5ex plus .2ex}
\makeatletter
\pretocmd\start@align
{%
  \let\everycr\CT@everycr
  \CT@start
}{}{}
\apptocmd{\endalign}{\CT@end}{}{}
\makeatother
%Watermark
\usepackage[printwatermark]{xwatermark}
\usepackage{lipsum}
\definecolor{lightgray}{RGB}{220,220,220}
%\newwatermark[allpages,color=lightgray,angle=45,scale=3,xpos=0,ypos=0]{Preliminary Draft}

%Further subsection level
\usepackage{titlesec}
\setcounter{secnumdepth}{4}
\titleformat{\paragraph}
{\normalfont\normalsize\bfseries}{\theparagraph}{1em}{}
\titlespacing*{\paragraph}
{0pt}{3.25ex plus 1ex minus .2ex}{1.5ex plus .2ex}

\setcounter{secnumdepth}{5}
\titleformat{\subparagraph}
{\normalfont\normalsize\bfseries}{\thesubparagraph}{1em}{}
\titlespacing*{\subparagraph}
{0pt}{3.25ex plus 1ex minus .2ex}{1.5ex plus .2ex}

%Functions
\DeclareMathOperator{\cov}{Cov}
\DeclareMathOperator{\corr}{Corr}
\DeclareMathOperator{\var}{Var}
\DeclareMathOperator{\plim}{plim}
\DeclareMathOperator*{\argmin}{arg\,min}
\DeclareMathOperator*{\argmax}{arg\,max}
\DeclareMathOperator{\supp}{supp}

%Math Environments
\newtheorem{theorem}{Theorem}
\newtheorem{claim}{Claim}
\newtheorem{condition}{Condition}
\renewcommand\thecondition{C--\arabic{condition}}
\newtheorem{algorithm}{Algorithm}
\newtheorem{assumption}{Assumption}
\renewcommand\theassumption{A--\arabic{assumption}}
\newtheorem{remark}{Remark}
\renewcommand\theremark{R--\arabic{remark}}
\newtheorem{definition}[theorem]{Definition}
\newtheorem{hypothesis}[theorem]{Hypothesis}
\newtheorem{property}[theorem]{Property}
\newtheorem{example}[theorem]{Example}
\newtheorem{result}[theorem]{Result}
\newenvironment{proof}{\textbf{Proof:}}{$\bullet$}

%Commands
\newcommand\independent{\protect\mathpalette{\protect\independenT}{\perp}}
\def\independenT#1#2{\mathrel{\rlap{$#1#2$}\mkern2mu{#1#2}}}
\newcommand{\overbar}[1]{\mkern 1.5mu\overline{\mkern-1.5mu#1\mkern-1.5mu}\mkern 1.5mu}
\newcommand{\equald}{\ensuremath{\overset{d}{=}}}
\captionsetup[table]{skip=10pt}
%\makeindex

\setlength\parindent{20pt}
\setlength{\parskip}{0pt}

\newcolumntype{L}[1]{>{\raggedright\let\newline\\\arraybackslash\hspace{0pt}}m{#1}}
\newcolumntype{C}[1]{>{\centering\let\newline\\\arraybackslash\hspace{0pt}}m{#1}}
\newcolumntype{R}[1]{>{\raggedleft\let\newline\\\arraybackslash\hspace{0pt}}m{#1}}



%Logo
%\AddToShipoutPictureBG{%
%  \AtPageUpperLeft{\raisebox{-\height}{\includegraphics[width=1.5cm]{uchicago.png}}}
%}

\newcolumntype{L}[1]{>{\raggedright\let\newline\\\arraybackslash\hspace{0pt}}m{#1}}
\newcolumntype{C}[1]{>{\centering\let\newline\\\arraybackslash\hspace{0pt}}m{#1}}
\newcolumntype{R}[1]{>{\raggedleft\let\newline\\\arraybackslash\hspace{0pt}}m{#1}}

\newcommand{\mr}{\multirow}
\newcommand{\mc}{\multicolumn}

%\newcommand{\comment}[1]{}

\let\counterwithout\relax
\let\counterwithin\relax
\definecolor{maroon}{HTML}{4B0082}

\begin{document}
\noindent \textbf{Ian Davis}\\
\noindent \textbf{Economics 8050}\\
\noindent \textbf{17 January 2020}\\
\\
\\
\indent Rodrik begins the introduction with a breif discussion on three major impacts the field of economics has made, all of which came from applications of its popular principles. The World Bank, a product of macro-stabilization efforts, is discussed first through a quick summary of Bretton Woods. Congestion pricing and conditional cash transfers are then mentioned as applications of supply and demand and incentive compatible policy. The three topics share an origin in abstract economic models that do not always have much to do with the areas in which they eventually were leveraged.\\
\\
\indent The chapter begins in earnest by presenting the current rift in economics regarding just how much models can and should be leaned upon within the profession. There is a strange hypocrisy which seems to arise in the profession where economists want to discredit mathematicians and staticians for seeing mathematical rigor as the end all be all to explain all phenomena, while at the same time wanting to discredit the rest of social science for not embracing mathematics enough. I am aware I am putting words in the mouth of the author but this feels like the conclusion of the arguement. It should be noted, however, that the author does hedge his words but explicitely stating the importance of simple models which illustrate economic intuitions which would like somewhere between the two extremems.\\
\\
\indent The author expands on how useful these simple frameworks are as jumping off points in which more complex analysis can begin. Taking the supply and demand model, for example, we can quickly see how the imposition of an ad volorem tax whill change quantity demanded or the price faced by customers. Additionally, the prisoners delimma serves as a model which can provide a great deal of insights into the behavior of firms, especially when these firms are seemingly acting irrationally. While these two models differ greatly in how they explain markets, niether is wrong but neither is right. What matters is which model is appropriate for the situation at hand.\\
\\
\indent Models are then comapred to fables in the sense that they serve as simplifications of the real world which usually teach a clear, hard to miss lesson. This view is not shared by all, however. Some say a model is more of a parable while others claim that the lessons in models are not nearly as straightforward as the analogy makes them out to be. This may just be an oversimplification of the fable itself. Fables are varied in lessons and complexity while having the singular thread that binds them being some sort of lesson is taught in the end. The fable story is immediately juxtaposed with a comaprison between models and experiments. A "thought experiment" may seem like a far cry from the work being done in laboratories by chemists and biologists, but the two require the same set up and are leveraged to achieve the same goal. Both begin with an abstraction of a phenomenon or occurance and are followed by development of a methodology regarding how to capture it. Parameters are set up to create an artificial representation of the world. Then, the artificial world is put into motion and we pull out findings to determine their real world implications. The real difference between the experiment and the model come from one living in a room (or a field) and one living abstractly on paper.\\
\\
\indent There is much debate on how the accuracy of assumption plays in to the usefulness of a model. Friedman is reported to have dismissed the importance of realistic assumptions so long as a model is accurate. This raises the question, however, of if a model can be accurate when it is built on unrealistic assumptions. Additionally, the explanatory power of the model would but hurt significantly if we cannot trace why these unrealistic assumptions are needed for a model to work but not needed in reality. Another question should be raised on what the goal of the model is. If it is only to make an accurate prediction regarding some outcome, than a "black box" model is suitable but most of economics is concerned with causality. Without being able to trace how assumptions and variables relate in the model and in the outside world, an arguement for causality quickly loses its teeth.\\
\\
\indent The chapter lands on a point which is so important it cannot be overstated. That is, a model works best when it is not forced to make a claim it was not built to make. The math, the assumptions, the extrapolations, all serve to either weaken this point or strengthen it for each model.


\end{document}
