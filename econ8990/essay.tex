%Input preamble
%Style
\documentclass[12pt]{article}
\usepackage[top=1in, bottom=1in, left=1in, right=1in]{geometry}
\parindent 22pt
\usepackage{fancyhdr}

%Packages
\usepackage{adjustbox}
\usepackage{amsmath}
\usepackage{amsfonts}
\usepackage{amssymb}
\usepackage{bm}
\usepackage[table]{xcolor}
\usepackage{tabu}
\usepackage{color,soul}
\usepackage{makecell}
\usepackage{longtable}
\usepackage{multirow}
\usepackage[normalem]{ulem}
\usepackage{etoolbox}
\usepackage{graphicx}
\usepackage{tabularx}
\usepackage{ragged2e}
\usepackage{booktabs}
\usepackage{caption}
\usepackage{fixltx2e}
\usepackage[para, flushleft]{threeparttablex}
\usepackage[capposition=top,objectset=centering]{floatrow}
\usepackage{subcaption}
\usepackage{pdfpages}
\usepackage{pdflscape}
\usepackage{natbib}
\usepackage{bibunits}
\definecolor{maroon}{HTML}{990012}
\usepackage[bottom]{footmisc}
\usepackage[colorlinks=true,linkcolor=maroon,citecolor=maroon,urlcolor=maroon,anchorcolor=maroon]{hyperref}
\usepackage{marvosym}
\usepackage{makeidx}
\usepackage{tikz}
\usetikzlibrary{shapes}
\usepackage{setspace}
\usepackage{enumerate}
\usepackage{rotating}
\usepackage{tocloft}
\usepackage{epstopdf}
\usepackage[titletoc]{appendix}
\usepackage{framed}
\usepackage{comment}
\usepackage{xr}
\usepackage{titlesec}
\usepackage{footnote}
\usepackage{longtable}
\newlength{\tablewidth}
\setlength{\tablewidth}{9.3in}
\setcounter{secnumdepth}{4}
\usepackage{textgreek}

\titleformat{\paragraph}
{\normalfont\normalsize\bfseries}{\theparagraph}{1em}{}
\titlespacing*{\paragraph}
{0pt}{3.25ex plus 1ex minus .2ex}{1.5ex plus .2ex}
\makeatletter
\pretocmd\start@align
{%
  \let\everycr\CT@everycr
  \CT@start
}{}{}
\apptocmd{\endalign}{\CT@end}{}{}
\makeatother
%Watermark
\usepackage[printwatermark]{xwatermark}
\usepackage{lipsum}
\definecolor{lightgray}{RGB}{220,220,220}
%\newwatermark[allpages,color=lightgray,angle=45,scale=3,xpos=0,ypos=0]{Preliminary Draft}

%Further subsection level
\usepackage{titlesec}
\setcounter{secnumdepth}{4}
\titleformat{\paragraph}
{\normalfont\normalsize\bfseries}{\theparagraph}{1em}{}
\titlespacing*{\paragraph}
{0pt}{3.25ex plus 1ex minus .2ex}{1.5ex plus .2ex}

\setcounter{secnumdepth}{5}
\titleformat{\subparagraph}
{\normalfont\normalsize\bfseries}{\thesubparagraph}{1em}{}
\titlespacing*{\subparagraph}
{0pt}{3.25ex plus 1ex minus .2ex}{1.5ex plus .2ex}

%Functions
\DeclareMathOperator{\cov}{Cov}
\DeclareMathOperator{\corr}{Corr}
\DeclareMathOperator{\var}{Var}
\DeclareMathOperator{\plim}{plim}
\DeclareMathOperator*{\argmin}{arg\,min}
\DeclareMathOperator*{\argmax}{arg\,max}
\DeclareMathOperator{\supp}{supp}

%Math Environments
\newtheorem{theorem}{Theorem}
\newtheorem{claim}{Claim}
\newtheorem{condition}{Condition}
\renewcommand\thecondition{C--\arabic{condition}}
\newtheorem{algorithm}{Algorithm}
\newtheorem{assumption}{Assumption}
\renewcommand\theassumption{A--\arabic{assumption}}
\newtheorem{remark}{Remark}
\renewcommand\theremark{R--\arabic{remark}}
\newtheorem{definition}[theorem]{Definition}
\newtheorem{hypothesis}[theorem]{Hypothesis}
\newtheorem{property}[theorem]{Property}
\newtheorem{example}[theorem]{Example}
\newtheorem{result}[theorem]{Result}
\newenvironment{proof}{\textbf{Proof:}}{$\bullet$}

%Commands
\newcommand\independent{\protect\mathpalette{\protect\independenT}{\perp}}
\def\independenT#1#2{\mathrel{\rlap{$#1#2$}\mkern2mu{#1#2}}}
\newcommand{\overbar}[1]{\mkern 1.5mu\overline{\mkern-1.5mu#1\mkern-1.5mu}\mkern 1.5mu}
\newcommand{\equald}{\ensuremath{\overset{d}{=}}}
\captionsetup[table]{skip=10pt}
%\makeindex

\setlength\parindent{20pt}
\setlength{\parskip}{0pt}

\newcolumntype{L}[1]{>{\raggedright\let\newline\\\arraybackslash\hspace{0pt}}m{#1}}
\newcolumntype{C}[1]{>{\centering\let\newline\\\arraybackslash\hspace{0pt}}m{#1}}
\newcolumntype{R}[1]{>{\raggedleft\let\newline\\\arraybackslash\hspace{0pt}}m{#1}}



%Logo
%\AddToShipoutPictureBG{%
%  \AtPageUpperLeft{\raisebox{-\height}{\includegraphics[width=1.5cm]{uchicago.png}}}
%}

\newcolumntype{L}[1]{>{\raggedright\let\newline\\\arraybackslash\hspace{0pt}}m{#1}}
\newcolumntype{C}[1]{>{\centering\let\newline\\\arraybackslash\hspace{0pt}}m{#1}}
\newcolumntype{R}[1]{>{\raggedleft\let\newline\\\arraybackslash\hspace{0pt}}m{#1}}

\newcommand{\mr}{\multirow}
\newcommand{\mc}{\multicolumn}

%\newcommand{\comment}[1]{}

\let\counterwithout\relax
\let\counterwithin\relax
\definecolor{maroon}{HTML}{4B0082}
\usepackage{listings}

\begin{document}
\noindent \textbf{Ian Davis}\\
\noindent \textbf{Economics 8990}\\
\noindent \textbf{Professor Harberger}\\
\noindent \textbf{9 December 2019}\\
\\
C. Essay\\
\\
\indent The primary determinants of corporation income tax incidence are the types of production functions facing industries, the elasticities of demand for the final goods, the elasticities of substitution for both the final goods and factor inputs, and the factor shares of production in each industry. Looking first at how various production functions can change the tax incidence, consider the Cobb-Douglass and Leontief, or fixed proportions, production functions. In the case of a two sector economy, one corporate and one non-corporate, where both sectors face Cobb-Douglass production functions, a tax on income from capital in the corporate sector would lead to a decrease in the price of capital, a shift in the capital stock from the taxed sector to the untaxed sector, and labor would be unaffected. The price of the product produced in the corporate sector, however, will rise and consumers will be affected. Just how much the consumers are affected is dependent on their preferences and the elasticity of substitution between the two goods. It should be noted that consumers are neither hurt nor benefited by the tax in aggregate due to offsetting loses and gains. To summarize, capital in both industries will bear the cost of the tax while labor and aggregate consumers are unaffected.\\
\\
\indent Contrast this Cobb-Douglass economy with the same economy but where the taxed industry faces Leontief production functions. In this scenario, a shift in capital must be accompanied by a shift in labor in order to keep proportions constant. Hence, if proportions are the same across sectors, then a tax on capital gains will be born by both labor and capital in both sectors based on the fraction if national income they account for. Should the untaxed sector be more labor intensive, the price of capital will fall further than that of labor and capital will bear more of the tax. The opposite will occure in the case of the untaxed sector being capital intensive. The most curious outcome occurs when the untaxed industry faces a Leontief production function. This case results in capital bearing more than 100\% of the tax while labor increases its absolute income. Speaking more generally and without assuming the functional forms of the production functions, we can say that the burden of corporate income tax is determined by the elasticities of demand faced by each sector's final goods, each sectors elasticity of substitution between labor and capital, and the elasticity of substitution faced by each sector's final goods\\
\\
\indent Finally, it should be noted that all of the above is discussed in the environment of a closed economy. Should the analysis extend to an open economy, we would see increased mobility of capital as it now can be moved between sectors domestically and internationally. This would lessen the burden placed on capital and increase the burden of the more immobile factor, labor.

\end{document}